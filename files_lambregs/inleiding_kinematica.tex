\chapter*{Inleiding}

We zijn opzoek naar de wetten die verklaren hoe de verschillende verschijnselen tot stand komen. Om die wetten te kunnen gebruiken hebben we een formalisme nodig: een wiskundige manier om de verschillende dingen te representeren, voor te stellen. Om verschijnselen en bewegingen te kunnen verklaren, moeten we die bewegingen kunnen beschrijven. Dit onderdeel krijgt een aparte naam binnen de mechanica, namelijk de kinematica. De dynamica houdt zicht enkel bezig met het beschrijven van bewegingen, niet met het onderliggende verklarende principe ervan. Het verklarende principe binnen de mechanica hoort thuis in de dynamica.

- vergelijking van de fysica met het schaakspel

- geschiedenis van de fysica?

- grote figuren?
