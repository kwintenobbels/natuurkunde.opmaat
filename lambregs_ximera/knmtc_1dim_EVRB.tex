\documentclass{ximera}
\input{../preamble}

\addPrintStyle{..}

\begin{document}
	\author{Bart Lambregs}
	\xmtitle{EVRB}{}



	\section*{C Eenparig veranderlijke rechtlijnige beweging}

	Een beweging waarvan de versnelling constant is, noemen we een \textit{eenparig veranderlijke rechtlijnige be\-we\-ging}, afgekort EVRB. Eenparig betekent gelijkmatig; de snelheidsverandering is steeds gelijk. In symbolen geldt dus dat $a(t)=a$ waarbij het maatgetal van $a$ een re\"eel getal is. We werken voor dit model de plaatsfunctie en de snelheidsfunctie uit. M.a.w. willen we het verloop van de plaats en de snelheid in functie van de tijd kennen. 
	
	Aangezien de versnelling constant is en de versnelling de afgeleide van de snelheid, moet de snelheid in functie van de tijd een lineaire functie (in de tijd) zijn.\footnote{Strikt genomen zien we hier iets over het hoofd. A priori zou het immers kunnen dat er nog andere functies dan lineaire functies zijn waarvoor de afgeleide een constante functie is. Dat is echter niet het geval. Het bewijs hiervan zie je later dit jaar in het vak wiskunde. Je bewijst dat alle mogelijke functies die in aanmerking komen slechts op een constante na aan elkaar gelijk zijn.} Nu dat we het snelheidsverloop kennen, kunnen we het verloop van de plaats afleiden. Aangezien de snelheid een lineaire functie is en de snelheid de afgeleide van de positie is, moet de positie een kwadratische functie (in de tijd) zijn. De afgeleide van een kwadratische functie is immers een lineaire functie.\footnote{Hier geldt een gelijkaardige opmerking als die in de vorige voetnoot.}
	
	Dat de positie in functie van de tijd een tweedegraadsveeltermfunctie is, geeft in symbolen:\footnote{we gebruiken de constanten $p$, $q$ en $r$ i.p.v. $a$, $b$ en $c$ om verwarring met de betekenis van $a$ te voorkomen.}
	\begin{eqnarray}
	x(t)=pt^2+qt+r\label{pqr}
	\end{eqnarray}
	Nu willen we de constanten $p$, $q$ en $r$ fysisch kunnen duiden en eventueel andere symbolen geven, zodat hun betekenis sneller af te lezen is. Dat doen we door te eisen dat de afgeleide met de snelheid overeenkomt en de tweede afgeleide met de versnelling. Ook gebruiken we de beginvoorwaarden. Dus:
	\begin{align}
	v(t)&=\frac{dx}{dt}=2pt+q\label{2pq}\\
	a(t)&=\frac{d^2x}{dt^2}=2p\nonumber
	\end{align}
	Omdat de versnelling constant is, halen we uit de laatste regel dat $a(t)=a=2p\Leftrightarrow p=\frac{a}{2}$. Als we met $v_0$ de snelheid op tijdstip $t=0$ voorstellen ($v_0=v(0)$) en $t=0$ invullen in (\ref{2pq}), dan vinden we dat $q=v_0$. De constante $q$ stelt dus de beginsnelheid voor. Als we vervolgens met $x_0$ de positie op tijdstip $t=0$ voorstellen en $t=0$ invullen in (\ref{pqr}), dan vinden we dat $r=x_0$. De constante $r$ stelt dus de beginpositie voor.
	
	Samengevat levert dat:
	
	\kader{
	%\phantom{}
	\vspace{1ex}
	De plaatsfunctie $x(t)$ en de snelheidsfunctie $v(t)$ van een EVRB met versnelling $a$ worden gegeven door:
	\begin{eqnarray}
	x(t)&=&x_0+v_0t+\frac{1}{2}at^2\label{x(t)0}\\
	v(t)&=&v_0+at\label{v(t)0}
	\end{eqnarray}
	Hierin is $x_0$ de \textit{beginpositie} en $v_0$ de \textit{beginsnelheid}. Ze worden bepaald door de \textit{beginvoorwaarden} of \textit{randvoorwaarden}.
	\vspace{1ex}
	%\phantom{.}
	}
	
	Indien we de beschrijving van de beweging niet op $t=0$ willen laten starten maar op een gegeven tijdstip $t_0$, dan moeten we in de beschrijving $t$ vervangen door $\Delta t= t-t_0$, de verstreken tijd vanaf het begintijdstip $t_0$. De functies (\ref{x(t)0}) en (\ref{v(t)0}) worden dan een klein beetje ingewikkelder:
	\begin{eqnarray}
	x(t)&=&x_0+v_0(t-t_0)+\frac{1}{2}a(t-t_0)^2\label{x(t)}\\
	v(t)&=&v_0+a(t-t_0)\label{v(t)}
	\end{eqnarray}
	
	Met de functies (\ref{x(t)}) en (\ref{v(t)}) kunnen we de volgende formule voor de gemiddelde snelheid van een EVRB aantonen:%\footnote{Het formuletje is handig te gebruiken in veel vraagstukken door gebruik te maken van $\Delta x=\overline{v}\Delta t$.} 
	\footnote{De afleiding van de gemiddelde snelheid is als volgt:
	\begin{eqnarray*}
	\overline{v}=\frac{\Delta x}{\Delta t}=\frac{x-x_0}{t-t_0}=\frac{v_0(t-t_0)+\frac{1}{2}a(t-t_0)^2}{(t-t_0)}=\frac{2v_0+a(t-t_0)}{2}=\frac{v_0+v}{2}.
	\end{eqnarray*}}
	\begin{eqnarray*}
	  \overline{v}=\frac{v_0+v}{2}
	\end{eqnarray*}
	Blijkbaar houdt het eenparig toenemen van de snelheid in dat we het rekenkundig gemiddelde kunnen gebruiken voor de gemiddelde snelheid.
	
	%%\newpage
	
	\begin{figure}[h]
	\centering
	\includegraphics[width=\textwidth]{EVRB_grafieken}
	%\includegraphics[height=\textheight]{EVRB_grafieken}
	\caption{Grafieken van een EVRB}
	\end{figure}
	
	\clearpage
	%%\newpage
	\thispagestyle{empty}
	
	
	\section*{Voorbeeldoefening}
	
	\begin{enumerate}
	
	\item[\textbf{Opgave}]\textsf{Een auto die $\SI{60}{km/h}$ rijdt, raakt een boom; de voorkant van de auto wordt in elkaar gedrukt en de bestuurder komt na $\SI{70}{cm}$ tot stilstand. Welke gemiddelde vertraging onderging de bestuurder tijdens de botsing? Druk je antwoord uit in $g$, waarbij $g=\SI{9,81}{m/s^2}$.
	\item[\textit{Gegeven}]$v_0=\SI{16,7}{m/s}$%%%\newline$x=\SI{0,70}{m}$
	\item[\textit{Gevraagd}]$a$
	\item[\textit{Oplossing}] Om de (constante) vertraging te vinden, hebben we de snelheidsverandering en de benodigde tijd nodig. De verandering in snelheid kennen we; de eindsnelheid van de auto moet nul worden maar de duur is niet onmiddellijk gegeven. Omdat de eindsnelheid nul is, kunnen we wel uit de snelheidsvergelijking van een eenparig veranderlijke beweging een \emph{uitdrukking} vinden voor die tijd die we vervolgens kunnen substitueren in de plaatsvergelijking. De enige onbekende is dan de gezochte versnelling.\footnote{M.b.v. de formule $\overline{v}=\frac{v_0+v}{2}$ voor de gemiddelde snelheid en de definitie voor de gemiddelde snelheid $\overline{v}=\frac{\Delta x}{\Delta t}$ is het antwoord sneller te vinden. Ga maar na \ldots}
	%%%\newline
	%%%\newline
	Uit $v(t)=0$ of $0=v_0+at$ halen we een uitdrukking voor de tijd die nodig is om tot stilstand te komen:
	\begin{align*}
	t&=-\frac{v_0}{a}
	\end{align*}
	Substitutie van deze tijd in de plaatsfunctie levert:
	\begin{align*}
	x&=v_0t+\frac{1}{2}at^2\\
	&=v_0\left(-\frac{v_0}{a}\right)+\frac{1}{2}a\left(-\frac{v_0}{a}\right)^2\\
	%&=&-\frac{v_0^2}{a}+\frac{v_0^2}{2a}\\
	&=-\frac{v_0^2}{2a}\\
	\end{align*}
	De versnelling is dan gelijk aan:
	\begin{align*}
	a&=-\frac{v_0^2}{2x}
	\end{align*}
	Invullen van de gegevens levert $a=\SI{-198}{m/s^2}$, wat gelijk is aan $20g$.
	}
	\end{enumerate}
	
	%%\newpage
	

	
\end{document}
