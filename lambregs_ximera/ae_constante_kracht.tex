\documentclass{ximera}
\input{../preamble}

\addPrintStyle{..}

\begin{document}
	\author{Bart Lambregs}
	\xmtitle{Arbeid geleverd door een constante kracht}{}
    \xmsource
    \xmsource


	%% \chapter{Arbeid \& Energie}

	In essentie hebben we met de wetten van Newton, de formule voor o.a. de gravitatiekracht en de kinematica voldoende om alle mechanische verschijnselen te beschrijven. Je zou dus kunnen zeggen dat we geen bijkomende elementen nodig hebben. Maar wat met het begrip energie, een fundamentele grootheid in de fysica en vaak handig bij het oplossen van probleemstellingen?  Willen we dit begrip in de mechanica gebruiken, dan zullen we het moeten defini\"eren aan de hand van begrippen die we al hebben -- zoals kracht. De wetten van Newton vormen immers het fundament van de mechanica -- alles moet hierop gestoeld zijn. 
	
	Je leerde ooit iets over energie: \textit{een lichaam bezit energie als het de mogelijkheid heeft om arbeid te verrichten.} Zo bezit een bewegende hamer \textit{kinetische energie} omdat hij als gevolg van zijn beweging arbeid kan verrichten: hij kan een nagel in de muur drijven. De opgewonden veer
	van een mechanisch horloge is een voorbeeld van \textit{potenti\"ele energie}. Als gevolg van de spanningstoestand van de veer -- die door een bepaalde plaats wordt gekarakteriseerd -- kan de veer arbeid verrichten: terwijl de veer zich geleidelijk ontspant, verricht ze arbeid door de wijzers te laten ronddraaien. %De veer verkreeg zijn potenti\"ele energie doordat degene die het horloge opwond er arbeid op leverde.
	
	Ook in de omgangstaal kennen we de termen arbeid en energie. Je mag echter niet vergeten dat deze hier een bredere betekenis dragen dan degene die we binnen de fysica voor ogen hebben. Zo moet de \textit{kwalitatieve} omschrijving van de mogelijkheid om arbeid te verrichten, vervangen worden door een \textit{kwantitatieve}; we willen weten \textit{hoeveel} arbeid er geleverd wordt, of \textit{hoeveel} energie een bepaald lichaam bezit. Als we bovendien zeggen dat energie de mogelijkheid is om arbeid te leveren, dan moeten we in eerste instantie het begrip arbeid \textit{defini\"eren}. We gebruiken hiervoor o.a. het begrip kracht. Aan de hand van het begrip arbeid, kunnen we dan vervolgens het begrip energie defini\"eren.
	
	

	
	\section{Arbeid geleverd door een constante kracht}
	
	%\subsection{Kracht en verplaatsing hebben dezelfde richting}
	De arbeid $W$ door een constante kracht $F$ geleverd op een lichaam bij een verplaatsing $\Delta x=x_b-x_a$ wordt gedefinieerd als \footnote{Besef dat dit een \textit{definitie} van een fysische grootheid is. Terecht zou ge u kunnen afvragen waarom we arbeid op deze manier defini\"eren en niet anders. Dat de grootheid kracht erin moet voorkomen, is duidelijk: anders kan je een veer van een horloge niet opwinden. Ook is een verplaatsing nodig om van arbeid te kunnen spreken. Als immers de veer niet beweegt, kan er geen sprake zijn van een verandering in de energietoestand van de veer en dus ook niet van een overdracht van energie naar de veer -- wat arbeid is. Dat er bijvoorbeeld geen snelheid in voorkomt, kan je inzien door je te realiseren dat de snelheid waarmee je een veer opwindt, niet mag uitmaken. Enkel de uiteindelijke toestand van de veer is van belang. En waarom dan het product van de kracht met de verplaatsing? Wel, omdat het de enige manier is waarop we aan een fysische grootheid zullen komen die de naam energie waardig is... (!). Zie hiervoor het arbeid-energietheorema.}
	\begin{eqnarray}
	W&=&F\Delta x
	\end{eqnarray}
	Hierbij moeten kracht en verplaatsing dezelfde richting hebben. De SI-eenheid van arbeid is de joule:
	\begin{eqnarray*}
	[W]&=&[F]\cdot[\Delta x]=\rm N\cdot m=J
	\end{eqnarray*}
	%\subsection{Kracht en verplaatsing; verschillende rich\-ting}
	Indien de beschouwde kracht en de verplaatsing niet dezelfde richting hebben, wordt alleen de component van de kracht volgens de verplaatsing in rekening gebracht. Er is namelijk geen beweging geassocieerd met de component loodrecht op de verplaatsing.
	\begin{image}
	\includegraphics[width=0.7\textwidth]{arbeid_verschillende_richting}
	\end{image}
	\captionof{figure}{kracht en verplaatsing met een verschillende richting}
	
	Als $\alpha$ de hoek tussen de kracht en de verplaatsing is, dan is de getalcomponent volgens de bewegingszin te schrijven als $F\cos{\alpha}$ en de arbeid door deze component geleverd als $W=F\cos{\alpha}\Delta x$. Dit laatste is niets anders dan het scalair product tussen de kracht en de verplaatsing.
	
	De mechanische arbeid $W$, geleverd door een constante kracht $\vec{F}$, is gelijk aan het scalair product van de kracht met de verplaatsing:
	\begin{eqnarray}
	W&=&\vec{F}\cdot\Delta\vec{x}
	\end{eqnarray}
	
	
	
	
	

\end{document}
