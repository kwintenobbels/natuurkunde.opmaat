\documentclass{ximera}
\input{../preamble}

\addPrintStyle{..}

\begin{document}
	\author{Bart Lambregs}
	\xmtitle{Referentiepunt van de potentiële energie}{}



	\subsection{Referentiepunt van de potenti\"ele energie}

	Ai, de potenti\"ele energiefunctie (\ref{Ep=-Gmm'/x}) is voor alle waardes van $x$
	\textit{negatief}. Zou potenti\"ele energie overeenkomen met de mogelijk te leveren hoeveelheid  arbeid, dan zou de massa met andere woorden minder dan geen arbeid kunnen leveren vanop een
	bepaalde afstand. Een vallende steen op de aarde lijkt toch wel het tegendeel te bewijzen\ldots. Bovendien is de hoeveelheid arbeid die de massa $m'$ vanaf een bepaald punt tot aan $m$ kan leveren \textit{oneindig} groot \footnote{Waarom?}. De massa zou dus oneindig veel energie hebben. Zeggen dat de potenti\"ele energie gelijk is aan de hoeveelheid arbeid die kan worden verricht, is dus niet mogelijk?!
	
	De potenti\"ele energie kan niet als een absolute maar enkel als een relatieve grootheid worden gedefinieerd. Het enige wat we van de potenti\"ele ener\-gie kunnen verwachten is dat het
	\textit{verschil} tussen een begin- en eindpunt overeenkomt met de hoeveelheid geleverde arbeid tussen die twee punten.
	%\begin{eqnarray*}
	%W&=&-\Delta E_p
	%\end{eqnarray*}
	Een relatieve grootheid is tot op een constante na bepaald. Het gevolg is dat als $E_p$ een potenti\"ele energiefunctie is, $E_p'=E_p+\rm cte$ dat ook is. Immers:
	\begin{eqnarray*}
	E_{p,a}'-E_{p,b}'&=&(E_{p,a}+{\rm cte})-(E_{p,b}+{\rm cte})\\
	&=&E_{p,a}-E_{p,b}\\
	&=&W
	\end{eqnarray*}
	Bij de potenti\"ele energiefuncties (\ref{Ep=1/2kx^2}), (\ref{Ep=mgh}) en (\ref{Ep=-Gmm'/x}) zoals we ze hiervoor hebben gegeven, mag dus steeds een constante worden opgeteld. Het
	\textit{re\-fe\-ren\-tiepunt} -- daar waar de potenti\"ele energie nul is -- kan vrij worden gekozen.
	


\end{document}
