%\documentclass[12pt,numbers,noauthor,nooutcomes,wordchoicegiven]{xourse}
\documentclass{xourse}

\addPrintStyle{.}

\pdfOnly{
    \renewcommand{\xmcursusnaam}{{\textsc{Natuurkunde}}}
}

\logo{xmPictures/nomlogo.png}

\begin{document}
%	\setcounter{tocdepth}{2}
    \xmtitle{Kinematica: Basisbegrippen}{}  
 

\setcounter{part}{1}

    
\part{Basisbegrippen van de kinematica}

\activitychapter{../6dejaar/kinematica/knmtc_bgrppn_intro.tex}
\activitychapter{../6dejaar/kinematica/knmtc_bgrppn_referentiestelsel.tex}
\activitychapter{../6dejaar/kinematica/knmtc_bgrppn_positie.tex}
\activitychapter{../6dejaar/kinematica/knmtc_bgrppn_snelheid.tex}
\activitychapter{../6dejaar/kinematica/knmtc_bgrppn_versnelling.tex}
\activitychapter{../6dejaar/kinematica/knmtc_bgrppn_oef.tex}



\end{document}