\documentclass{ximera}

%\addPrintStyle{..}

\begin{document}
	\author{Bart Lambregs}
	\xmtitle{Trillingen}{}
    \xmsource\xmuitleg

Om golven te kunnen beschrijven, behandelen we eerst trillingen. Dat is een iets eenvoudiger verschijnsel. Trillingen zijn heen en weer gaande bewegingen op een lijn. Zo trilt bijvoorbeeld een gitaarsnaar wanneer je hem aanslaat, trilt het rietje in een hobo bij het blazen of trillen elektronen in de antenne van je gsm.
	
\end{document}
