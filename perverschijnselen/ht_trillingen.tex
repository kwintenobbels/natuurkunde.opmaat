\documentclass{ximera}

%\addPrintStyle{..}

\begin{document}
	\author{Bart Lambregs}
	\xmtitle{Trillingen}{}
    \xmsource\xmuitleg

Om golven te kunnen beschrijven, behandelen we eerst trillingen. Dat is een iets eenvoudiger verschijnsel. 

\begin{definition}
	{\textbf{Trilling}} \nl

	Een trilling is een heen en weer gaande beweging op een lijn.
\end{definition}

Zo trilt bijvoorbeeld een gitaarsnaar wanneer je hem aanslaat, trilt het rietje in een hobo bij het blazen of trillen elektronen in de antenne van je gsm.

Om een trilling te beschrijven is het vaak gemakkelijk om de oorsprong van een referentieas te laten samenvallen met de evenwichtsstand. De positie krijgt in dat geval de naam uitwijking:

\begin{definition}
	{\textbf{uitwijking}} \nl

	De uitwijking $x(t)$ van een trillend voorwerp is de positie van dat voorwerp in functie van de tijd waarbij de oorsprong van de referentieas samenvalt met de evenwichtsstand.
\end{definition}

\begin{remark}
	Merk op dat de uitwijking zowel positief als negatief kan zijn.
\end{remark}

	
\end{document}
