\documentclass{ximera}

%\addPrintStyle{..}

\begin{document}
	\author{Bart Lambregs}
	\xmtitle{Energie}{}
    \xmsource\xmuitleg
	
Naast in detail de beweging te hebben bestudeerd, kunnen we ook naar het energieverloop kijken. Hiertoe vullen we gewoonweg de positie en de snelheid in de formules voor energie in.
\begin{eqnarray*}
    E_k&=&\frac{mv^2}{2}=\frac{1}{2}m\omega^2A^2\cos^2(\omega t+\varphi)\\
    E_p&=&\frac{1}{2}kx^2=\frac{1}{2}kA^2\sin^2(\omega t+\varphi)=\frac{1}{2}m\omega^2A^2\sin^2(\omega t+\varphi)
\end{eqnarray*}
Deze hangen duidelijk af van de tijd. Wanneer we naar de totale energie kijken, vinden we -- zoals we volgens de wet van behoud van energie mogen verwachten -- dat deze constant is.
\begin{eqnarray*}
    E=E_k+E_p=\frac{1}{2}m\omega^2A^2\left(\sin^2(\omega t+\varphi)+\cos^2(\omega t+\varphi)\right)=\frac{1}{2}m\omega^2A^2
\end{eqnarray*}
\begin{image}
    %\includegraphics[width=.7\textwidth]{EpEkE}
    \includegraphics[width=.9\textwidth]{ht_behoud_energie}
\end{image}
Merk op dat de energie van een trilling recht evenredig is met het kwadraat van de amplitude. Een golf in de zee die dus twee keer zo hoog is, draagt vier keer zoveel energie met zich mee. In de figuur zie je mooi hoe de kinetische en de potentiële energie in elkaar worden omgezet; het verlies van de ene is de winst van de andere. Let op het verschil tussen de periode van de energievormen en de periode van de eigenlijke trilling \ldots

\end{document}
