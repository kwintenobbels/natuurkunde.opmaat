\documentclass{ximera}

%\addPrintStyle{..}

\begin{document}
	\author{Bart Lambregs}
	\xmtitle{Senlheid en versnelling}{}
    \xmsource\xmuitleg
	
Nu dat we de positie in functie van de tijd kennen, vinden we door af te leiden de snelheid en de versnelling van de massa. 
\begin{eqnarray*}
	x&=&A\sin(\omega t+\varphi)\\[3mm]
	v=\frac{dx}{dt}&=&\omega A\cos(\omega t+\varphi)\\[3mm]
	a=\frac{d^2x}{dt^2}&=&-\omega^2A\sin(\omega t+\varphi)=-\omega^2x
\end{eqnarray*}
Merk op dat de versnelling op een constante ($\omega^2$) na tegengesteld is aan de positie. Dat is niet verwonderlijk aangezien we te maken hebben met de wet van Hooke en volgens de tweede wet van Newton de versnelling recht evenredig is met de kracht.

\end{document}
