\documentclass{ximera}

%\addPrintStyle{..}

\begin{document}
	\author{Bart Lambregs}
	\xmtitle{Het massa-veersysteem}{}
    \xmsource\xmuitleg
	
Als eerste voorbeeld van een trilling behandelen we een eenvoudig mechanisch systeem: een massa die aan een veer is bevestigd. We noemen dat een massa-veersysteem. En zoals we al zo vaak hebben gedaan, beschouwen we een model waar geen wrijving aanwezig is. Laten we bovendien de veer een horizontale oriëntatie geven zodat we ook de zwaartekracht buiten beschouwing kunnen laten.
% \begin{image}
% 	figuur massa-veersysteem
% 	% \includegraphics[width=0.8\textwidth]{kogeltegenveer2}
% \end{image}

Wanneer we de massa uit de positie halen waar de veer zijn rustlengte heeft en hem vervolgens loslaten, voert de massa een trilling uit. We constateren dat hij heen en weer beweegt waarbij hij steeds door de evenwichtspositie gaat. Deze beweging willen we natuurlijk uit onze natuurwetten tevoorschijn zien komen. Meer bepaald moet de tweede wet van Newton -- ons beginsel der beginselen in de klassieke mechanica -- de beweging opleveren. De trilling is een mechanisch verschijnsel en moet dus te verklaren zijn met behulp van die natuurwetten. 

Om de beweging te kunnen vinden, moeten we het systeem vrij maken -- alle krachten erop tekenen. We hebben dus de resulterende kracht nodig. Omdat het lichaam op een ondergrond rust en geen verticale versnelling heeft, heffen de normaalkracht en de zwaartekracht elkaar op. De resulterende kracht is bijgevolg de veerkracht. We gebruiken de wet van Hooke en gieten de component van de kracht volgens de bewegingsrichting in de vorm zoals we ze al kennen:\footnote{Deze beschrijving van de veerkracht is natuurlijk een benadering. Ze is maar geldig voor zolang de veer een niet te grote uitwijking kent. Het minteken zorgt voor een terugroepkracht; wanneer de coördinaat $x$ positief is, is de component van de kracht negatief en wanneer de coördinaat negatief is, is de component positief.}
\begin{eqnarray*}
	F(x)=-kx
\end{eqnarray*}
Nu dat we ons model voor het massa-veersysteem hebben, kunnen we opzoek naar de beweging die de massa uitvoert. Hoe ziet de trilling er precies uit? De tweede wet van Newton dus \ldots
\begin{eqnarray}
	F&=&ma\nonumber\\
	&\Downarrow&\nonumber\\
	-kx&=&m\frac{d^2x}{dt^2}\nonumber\\
	&\Updownarrow&\nonumber\\
	\frac{d^2x}{dt^2}+\frac{k}{m}x&=&0\label{diffvgl_ht}
\end{eqnarray}
\ldots\,en toen kwamen we uit bij een \ldots\,\emph{differentiaalvergelijking}. Om meer precies te zijn: een tweede-orde lineaire differentiaalvergelijking met constante coëfficiënten. Dat klinkt redelijk ingewikkeld. In feite is het ook niet zo simpel. We zijn opzoek naar de beweging, wat betekent dat we de positie $x$ in functie van de tijd willen vinden -- de functie $x(t)$ dus. De vergelijking die we gevonden hebben is niet zozeer een algebraïsche vergelijking in een onbekende variabele $x$ (zoals bijvoorbeeld een tweedegraadsvergelijking) dan wel \emph{een vergelijking voor een functie}!\footnote{Wanneer we de impliciete afhankelijkheid van de tijd expliciet aangeven,ziet de vergelijking er als volgt uit
\begin{eqnarray*}
	\frac{d^2x(t)}{dt^2}+\frac{k}{m}x(t)=0
\end{eqnarray*}}
We zoeken dus een functie die aan de vergelijking voldoet. Een functie die voldoet is dan een oplossing van de vergelijking en dus een mogelijke beweging die de massa kan volgen. We spreken hier in eerste instantie over \emph{een} oplossing omdat er a priori\footnote{Vooraf beschouwd. Zonder het gezien, ervaren of onderzocht te hebben.} verschillende oplossingen kunnen zijn.

Het oplossen van differentiaalvergelijkingen is bijna een stiel op zich. Want dergelijke vergelijkingen zijn er in alle maten, geuren, kleuren en vooral moeilijkheidsgraden. En aangezien we die cursus willen overlaten aan degene die zich bij verdere studies hierin wil verdiepen, hebben wij op dit moment geen directe manier om tot de oplossingen te komen. Dat betekent dat we hier iets creatiever zullen moeten zijn en de methode van het geïnspireerd gokken zullen moeten toepassen \ldots In de modus van minder hoge ambities, kunnen we in eerste instantie een functie uit onze hoed toveren en nagaan of ze een oplossing is. Dat is natuurlijk onbegonnen werk wanneer we \emph{alle} functies nagaan\footnote{Zoals het ook voor een schaakcomputer onbegonnen werk zou zijn, moest hij alle mogelijke zetten evalueren om de beste er uit te kunnen pikken.} maar haalbaar wanneer we ons fysisch inzicht erbij halen\footnote{Zoals ook een schaker alleen maar die paar zetten bestudeert waarvan hij ziet dat ze de moeite waard zijn.}. Zo moet de functie de beweging beschrijven en zullen de vergelijkingen voor een EVRB naar alle waarschijnlijkheid niet voldoen. De massa gaat heen en weer zodat we met een functie te maken moeten hebben die dat ook doet. Waarom zou dus een \emph{sinusfunctie} niet voldoen \ldots?!

Om te proberen nemen we een eenvoudige sinusfunctie $x(t)=\sin\omega t$. Bij het invullen in de vergelijking (\ref{diffvgl_ht}) moeten we voor de eerste term in het linkerlid de tweede afgeleide berekenen; $x''(t)=-\omega^2\sin\omega t$. De tweede term geeft bij het invullen $\frac{k}{m}\sin\omega t$. Wil nu de voorgestelde functie een oplossing zijn, dan moet hetgeen we verkregen hebben, gelijk zijn aan het rechterlid -- namelijk nul.
\begin{eqnarray*}
	-\omega^2\sin\omega t+\frac{k}{m}\sin\omega t=0\Leftrightarrow\left(-\omega^2+\frac{k}{m}\right)\sin\omega t=0\Leftrightarrow\omega^2=\frac{k}{m}
\end{eqnarray*}
In de laatste overgang hebben we gebruik gemaakt van het feit dat de sinus weliswaar nul kan worden maar dit nooit voor \emph{alle} tijdstippen het geval is. De vergelijking moet altijd opgaan -- niet enkel voor een paar momenten in de tijd. Wanneer we dus $\omega=\sqrt{k/m}$ nemen, is de voorgestelde functie een oplossing van de differentiaalvergelijking. We hebben zowaar een mogelijke beweging gevonden die de massa kan uitvoeren!

Natuurlijk hebben we nu slechts \emph{een} oplossing. Misschien zijn er nog. Want, waarom zou een cosinusfunctie niet voldoen? Een cosinusfunctie kunnen we beschrijven als een verschoven sinusfunctie zodat een sinusfunctie in algemene vorm $x(t)=a\sin(b(t-c))+d$ het proberen waard is. We kiezen voor het gemak $d$ gelijk aan nul\footnote{Dit komt overeen met het plaatsen van de oorsprong in de evenwichtspositie van de massa.} en geven in deze context de andere parameters andere symbolen: $a$ vervangen we door $A$, $b$ door $\omega$ en $-bc$ door $\varphi$. Die parameters hebben nl. een iets meer fysische betekenis -- waarover later meer. Ga nu na\footnote{Doe dit effectief. En nee, al liggend in je bed en kijkend naar dit blad voldoet niet aan de imperatief. Iets nagaan in deze exacte wereld van wiskunde en natuurkunde doet men met pen en papier / bord en krijt. Eventueel bijgestaan door een computer.} dat de functie 
\begin{eqnarray*}
	x(t)=A\sin(\omega t+\varphi)\quad\mathrm{met}\quad\omega=\sqrt{\frac{k}{m}}
\end{eqnarray*}
een oplossing is van de differentiaalvergelijking. In feite is dit de algemene oplossing van de differentiaalvergelijking. Hier hebben we de existentie aangetoond; het feit dat er een oplossing bestaat. De uniciteit ervan -- het feit dat er geen andere functie bestaat -- is nog een ander paar mouwen.

\begin{proposition}
	Samengevat vinden we dat voor een lineaire terugroepkracht $F=-kx$ de beweging een harmonische trilling is. Ook het omgekeerde is waar: voert een lichaam een harmonische trilling uit, dan is de kracht lineair en steeds naar de oorsprong gericht.
	\begin{eqnarray}
		\frac{d^2x}{dt^2}+\frac{k}{m}x&=&0\nonumber\\
		&\Updownarrow&\nonumber\\
		x(t)=A\sin(\omega t+\varphi)\quad&\mathrm{met}&\quad\omega=\sqrt{\frac{k}{m}}\label{opl_diffvgl_ht}
	\end{eqnarray}	
\end{proposition}

We hebben nu de wiskundige oplossing van de differentiaalvergelijking gevonden. De massa trilt volgens een sinusfunctie in de tijd. Vandaar dat we over een harmonische trilling spreken.

\end{document}
