\documentclass{ximera}

%\addPrintStyle{..}

\begin{document}
	\author{Bart Lambregs}
	\xmtitle{De zwaartekracht}{}
    \xmsource\xmuitleg





	%%%\section{De zwaartekracht}
	De zwaartekracht is een bijzonder geval van de universele gravitatiekracht: de gravitatiekracht door de aarde op een massa uitgeoefend noemen we ook de zwaartekracht.
	
	\subsection{De valversnelling}
	
	Proefondervindelijk zien we dat op de aarde alle lichamen met
	dezelfde valversnelling $g$ naar de aarde toe vallen. De tweede wet
	van Newton zegt dan dat de zwaartekracht die deze versnelling moet
	veroorzaken gelijk moet zijn aan de massa van het lichaam
	vermenigvuldigd met deze versnelling:
	%\begin{eqnarray*}
	%F_z&=&mg
	%\end{eqnarray*}
	%Deze vergelijking geeft dus aan \textit{hoe groot} de kracht moet
	%zijn. Het \textit{is} echter de universele gravitatiekracht door de
	%aarde op de massa uitgeoefend. Er moet dus gelden:
	\begin{eqnarray}
	F&=&ma\nonumber\\
	&\Downarrow&\nonumber\\
	G\frac{m_Am}{r^2}&=&mg\nonumber\\
	&\Updownarrow&\nonumber\\
	g&=&G\frac{m_A}{r^2}
	\end{eqnarray}
	De valversnelling $g$ is inderdaad onafhankelijk van de massa van
	het be\-schouw\-de lichaam.
	
	De zwaartekracht op een grotere massa is wel groter, maar doordat
	een grotere massa een grotere traagheid heeft, is het moeilijker haar
	bewegingstoestand te veranderen. Deze twee eigenschappen heffen
	elkaar dus op zodat alle massa's met een zelfde valversnelling vallen.
	
	%\subsection{$g$ is geen constante}
	
	%- factoren die $g$ be\"invloeden
	
	%- welke benadering hebben we eigenlijk gemaakt?
	
	%\clearpage
	%%%\newpage
	
\end{document}
