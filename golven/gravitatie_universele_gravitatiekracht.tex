\documentclass{ximera}

%\addPrintStyle{..}

\begin{document}
	\author{Bart Lambregs}
	\xmtitle{De universele gravitatiekracht}{}
    \xmsource\xmuitleg




	%%%\section{De universele gravitatiekracht}

	%- redenering maan
	
	%- verklaring wetten Kepler
	
	Isaac Newton (1642 - 1727) kon aantonen dat wanneer hij aannam dat planeten door de zon werden aangetrokken door een kracht die omgekeerd evenredig is met het kwadraat van de afstand, de drie wetten van Kepler hieruit af te leiden zijn. Daar waar de wetten van Kepler slechts inductief uit empirische gegevens zijn afgeleid, zijn ze bij Newton een gevolg van zijn bewegingsleer: de gravitatiekracht als oorzaak van beweging en zijn tweede wet als bewegingsvergelijking. Nu zijn de wetten een deductief gevolg uit een theoretisch model.
	
	Een ander argument dat Newton gebruikte om zijn voorstel voor de formule van de universele gravitatiekracht te corroboreren\footnote{Een moeilijk woord dat `met argumenten staven' betekent.}, was een redenering over de versnelling van de maan.\footnote{Zijn berekeningen borg hij op omdat ze niet strookte met de realiteit. Enkele jaren later bleek echter dat de door astronomen gebruikte afstand tot de maan fout was. Een nieuwe waarde toonde aan dat Newton een correcte redenering had gebruikt.} Ze gaat als volgt. De maan maakt een nagenoeg cirkelvormige beweging. Haar versnelling is dus te berekenen met de formule voor de versnelling van een ECB. We hebben dan de omlooptijd en de afstand tot de maan nodig. De straal van de cirkelbeweging van de maan is ongeveer 60 keer de straal van de aarde.
	\begin{eqnarray*}
		a&=&r\omega^2=r\left(\frac{2\pi}{T}\right)^2\\
		&=&6370\cdot10^3\rm\,m\cdot60\cdot\left(\frac{2\pi}{27,3\cdot24\cdot60\cdot60\rm\,s}\right)^2=0,0027\rm\,m/s^2
	\end{eqnarray*}
	Dit getal komt overeen met een omgekeerd kwadratische afhankelijkheid van de straal. Nemen we namelijk de valversnelling op aarde, $9,81\rm\,m/s^2$ en delen we deze door $60^2$ (de maan bevindt zich immers 60 keer zo ver), dan krijgen we hetzelfde getal.
	\begin{eqnarray*}
		a=\frac{9,81\rm\,m/s^2}{60^2}=0,0027\rm\,m/s^2
	\end{eqnarray*}
	Hiermee had Newton een argument om aan te nemen dat de gravitatiekracht omgekeerd evenredig moet zijn met het kwadraat van de onderlinge afstand tussen de massa's.
	
	De \textit{universele of algemene gravitatiekracht}, \'e\'en van de vier fundamentele natuurkrachten, wordt als volgt geformuleerd:
	\begin{center}
	\framebox{
	\begin{minipage}[t]{\textwidth}
	Twee puntmassa's $m$ en $m'$, die zich op een afstand $r$ van elkaar
	bevinden, trekken elkaar aan met een kracht die gericht is volgens
	de verbindingslijn van de twee massa's en waarvan de grootte recht
	evenredig is met het product van de twee massa's en omgekeerd
	evenredig met het kwadraat van de afstand tussen beide:
	\begin{eqnarray}
	F&=&G\frac{mm'}{r^2}
	\end{eqnarray}
	De evenredigheidscoöefficiënt $G$ wordt de
	\textit{gravitatieconstante} genoemd.
	\begin{eqnarray}\nonumber
	G&=&6,67\cdot10^{-11}\rm\,\frac{Nm^2}{kg^2}
	\end{eqnarray}
	\end{minipage}}
	\end{center}
	
	Newton bepaalde de constante $G$ niet zelf. Dat werd na zijn dood door Cavendish (1731 - 1810) gedaan.
	
	%%\newpage
	
	
	
	
	
\end{document}
