\documentclass{ximera}

%\addPrintStyle{..}

\begin{document}
	\author{Bart Lambregs}
	\xmtitle{De wetten van Kepler}{}
    \xmsource\xmuitleg

	
%%%\section{De wetten van Kepler}

Johannes Kepler (1571 - 1630) destilleerde uit de enorme hoeveelheid meetgegevens van planeetposities die Tycho Brahe (1546 - 1601) verzamelde, zijn drie wetten. Ze luiden: een planeet beweegt in een elliptische baan met de zon in een van de brandpunten, de voerstraal snijdt in gelijke tijdsintervallen gelijke oppervlakten (perken) uit (perkenwet) en de verhouding van het kwadraat van de periode tot de derde macht van de halve lange as van de ellipsbaan is voor alle planeten gelijk.
	
	

\end{document}
