\documentclass{ximera}

%\addPrintStyle{../..}

\begin{document}
	\author{Bart Lambregs}
	\xmtitle{Bewerkingen met vectoren}{}
    \xmsource\xmuitleg


% SCALAIRE VERMENIGVULDIGING 

% ph: maar met 1 shift vector doen; is veel simpelen -1*(Shift) is de andere
%  TIKZPICTURE DIE DE SCALAIRE VERMENIGVULDIGING ILLUSTREERT 

\subsection*{Scalaire vermenigvuldiging van een reëel getal met een vector}


Een vector kan 'herschaald' worden door hem te vermenigvuldigen met een reëel getal (d.w.z. een scalar). De richting blijft op die manier behouden. 
De grootte en zin kunnen veranderen. 
De scalaire verminigvuldiging wordt genoteerd als \(\vec{c} = k \cdot \vec{a}\) waarbij \(k \in \R\).

% ERUIT OP AANRADEN VAN VINCENT: 
% Als de vector \(\vec{a}\) gegeven wordt door \((a_x, a_y)\), is de scalaire vermenigvuldiging \(3\cdot\vec{a}\) gelijk aan \(3\cdot((a_x, a_y)) = (3a_x, 3a_y)\).

% Het is een soort reëel veelvoud van de vector. 

\begin{image}[0.3\textwidth]
\begin{tikzpicture}
    % \draw (-4,-4) grid (4,4);

    \pgfmathsetmacro{\ax}{2}  
    \pgfmathsetmacro{\ay}{0.5}  

    % \pgfmathsetmacro{\c}{$\sqrt{2}$}  
    \pgfmathsetmacro{\d}{2} 

    \coordinate (O) at (0,0); 
    \coordinate (A) at (\ax,\ay); 
    
    \coordinate (Shift) at (0,0.1); 

    \fill (O) circle (2pt); %node[below left]{O};

    \draw[->, very thick,  -latex] (O) -- ($(A)$) node[midway, below]{\(\vec{a}\)};
    \draw[->, very thick,  -latex, blue] ($ (O) - (Shift)$) -- ($-1*(A) - (Shift)$) node[midway, above]{\(-\vec{a}\)};

    \draw[->, very thick,  -latex] ($ (O) + (Shift)$) -- ($\d*(A) + (Shift)$) node[midway, above]{\(2 \cdot \vec{a}\)};
    \draw[->, very thick,  -latex, blue] ($ (O) - 2*(Shift)$) -- ($-0.5*(A) - 2*(Shift)$) node[midway, below]{\(-\frac{1}{2}\vec{a}\)};

\end{tikzpicture}
\end{image}
\captionof{figure}{De scalaire vermenigvuldiging van een vector \(\vec{a}\)}



\subsection*{De samenstelling of som van twee (of meer) vectoren}

Twee vectoren van dezelfde grootheid met hetzelfde aangrijpingspunt kunnen opgeteld worden met als resultaat een nieuwe vector. 
Deze vector wordt \textbf{de resultante} genoemd. 
Grafisch (kwalitatief) bekomt men de resultante via de kopstaartmethode of parallellogrammethode.  %kopstaartmethode (voor de eenvoud enkel parallellogram gebruiken? --> toch niet (vincent))

\begin{image}[0.3\textwidth]
    \begin{tikzpicture}
     % TIKZPICTURE DIE DE SOM VAN TWEE VECTOREN BEREKENT 
    % \draw (-4,-4) grid (4,4);

    \pgfmathsetmacro{\ax}{1}  
    \pgfmathsetmacro{\ay}{2}  
    \pgfmathsetmacro{\bx}{3}  
    \pgfmathsetmacro{\by}{1} 

    \coordinate (O) at (0,0); 
    \coordinate (A) at (\ax,\ay); 
    \coordinate (B) at (\bx,\by); 

    \fill (O) circle (2pt) node[below left]{O};
    % \fill (A) circle (2pt);
    % \fill (B) circle (2pt);

    \draw[->, -latex] (O) -- (A) node[midway, below]{\(\vec{a}\)};
    \draw[->, -latex] (O) -- (B) node[midway, below]{\(\vec{b}\)};
    
    \draw[->, -latex, red, thick] (O) -- ($ (A) + (B)$) node[midway, below]{\(\vec{a} + \vec{b}\)};
    \draw[->, dashed, -latex, red, thick] (A) -- ($ (A) + (B)$) node[midway, below]{\( \vec{b} \)};
    \draw[->, dashed, -latex, red, thick] (B) -- ($ (A) + (B)$) node[midway, below]{\( \vec{a} \)};

\end{tikzpicture}
\end{image}
\captionof{figure}{De optelling van twee vectoren}


De \textbf{grootte van de resultante} (kwantitatief) kan op verschillende manieren bepaald worden. 
Erg belangrijk hierbij is om meetkundige samenstelling in het oog te houden en zeker niet blindelings de groottes van de gegeven vectoren op te tellen! 
In het algemeen wordt de grootte van de resultante berekend met de cosinusregel. 
In evenwijdige of loodrechte gevallen zijn er efficiënte manieren om de resultante te bepalen (som/verschil of stelling van Pythagoras), de meest algemene methode is echter met de (aangepaste) cosinusregel voor de lengte van $\vec{c} = \vec{a} + \vec{b}$:

\[
\| \vec{a} + \vec{b} \|^2 = \| \vec{a} \|^2 + \| \vec{b} \|^2 + 2 \cdot \|\vec{a}\| \cdot \|\vec{b}\| \cdot \cos(\alpha)
\]

met \(\alpha\) de hoek tussen \(\vec{a}\) en \(\vec{b}\). 

\begin{quickquestion*}{}{}
  Hoe vereenvoudigt de cosinusregel als \(\vec{a}\) en \(\vec{b}\) loodrecht op elkaar staan? \\
  Hoe vereenvoudigt de cosinusregel als \(\vec{a}\) en \(\vec{b}\) dezelfde richting en dezelfde zin hebben? 
  Hoe vereenvoudigt de cosinusregel als \(\vec{a}\) en \(\vec{b}\) dezelfde richting en tegengestelde zin hebben? 
\end{quickquestion*}


\begin{expandable}{proposition}{Twee versies van de cosinusregel ?}


De 'klassieke' cosinusregel wordt geformuleerd voor de zijden van een driehoek, en zegt  
\[
    c^2 = a^2 + b^2 \red{ - } 2 a b \cos\varphi
\]
met \(\varphi\) de overstaande hoek van zijde \(c\). 

Merk op dat het teken van de derde term verschillend is. Waarom is dat zo, en hoe moet je dat onthouden?

In een driehoek gebruik je natuurlijkerwijze niet de hoek $\alpha$ tussen de vectoren $\vec{a}$ en $\vec{b}$, maar wel de \textit{complementaire} hoek $\varphi$ tussen de lijnstukken $a$ en $b$.

Volgende tekening en redenering geeft het verband tussen beide situaties, en de bijhorende formuleringen. 
Het enige verschil is een minteken, en dat komt omdat je afhankelijk van de situatie liever met één van de twee complementaire hoeken werkt. 
En zoals je op de goniometrische cirkel onmiddellijk kan zien, hebben complementaire hoeken tegengestelde cosinussen. 
Dat precies dat verschil wordt gecompenseerd door het extra minteken.


    \begin{image}[0.8\textwidth]

        \begin{tikzpicture}
            % TIKZPICTURE met extra \varphi aangeduid
       
           \pgfmathsetmacro{\ang}{50} 
       
           \coordinate (O) at (0,0); 
           \coordinate (A) at (\ang :2); 
           \coordinate (B) at (0:3); 
           \coordinate (C) at ($ (A) + (B)$); 
           
           %\draw[->, -latex, blue   ] (O) -- (A) node[midway, above left]{\( a \)};
           \draw[red    ]     (O) -- (B) node[midway, below     ]{\( b \phantom{\vec{b}}\)};  % \phantom to get nice alignment !
           \draw[xmgreen]     (O) -- (C) node[midway, above left]{\( c \)};
           \draw[blue, thick] (B) -- (C) node[midway, below right]{\( a\)};
           %\draw[->, dashed, -latex, red , thick] (A) -- (C) node[midway, above left ]{\( \vec{b} \)};

           %\draw pic[ pic text = \(\alpha \), draw, angle radius=0.7cm]                        {angle= B--O--A};
           %\draw pic[ pic text = \(\gamma \), draw, angle radius=1.2cm, angle eccentricity=1.2]{angle= B--O--C};
           \draw pic[ pic text = \(\varphi\), draw, angle radius=0.5cm]                        {angle= C--B--O};
           %\draw pic[ pic text = \(\varphi\), draw, angle radius=0.5cm]                        {angle= A--O--Z};

           \draw (0,-0.7) node[right,scale=0.8]{Natuurlijke hoek in een driehoek: $\varphi$} ;
       \end{tikzpicture}
       \begin{tikzpicture}
            % TIKZPICTURE met extra \varphi aangeduid
           \pgfmathsetmacro{\ang}{50} 
              
           \coordinate (O) at (0,0); 
           \coordinate (A) at (\ang :2); 
           \coordinate (B) at (0:3); 
           \coordinate (C) at ($ (A) + (B)$); 
           
           \coordinate (X) at (0:4); 

           \draw[->, -latex, blue          ] (O) -- (A) node[midway, above left]{\( \vec{a} \)};
           \draw[->, -latex, red           ] (O) -- (B) node[midway, below     ]{\( \vec{b} \)};
    %       \draw[->, -latex, xmgreen,dashed] (O) -- (C) node[pos=1, above right]{\( \vec{c} = \vec{a}+\vec{b} \)};
           
        %    \draw[->, dashed, -latex, red , thick] (A) -- (C) node[midway, above left]{\( \vec{b} \)};
        %    \draw[->, dashed, -latex, blue, thick] (B) -- (C) node[midway, above left]{\( \vec{a} \)};

           \draw pic[ pic text = \(\alpha \), draw, angle radius=0.7cm] {angle= B--O--A};
        %    \draw pic[ pic text = \(\alpha \), draw, angle radius=0.7cm] {angle= X--B--C};
        %    \draw pic[ pic text = \(\varphi\), draw, angle radius=0.7cm] {angle= C--B--O};

           \draw (0,-0.7) node[right,scale=0.8]{Natuurlijke hoek bij vectoren: $\alpha$} ;

        \end{tikzpicture}
       \begin{tikzpicture}
            % TIKZPICTURE met extra \varphi aangeduid
           \pgfmathsetmacro{\ang}{50} 
              
           \coordinate (O) at (0,0); 
           \coordinate (A) at (\ang :2); 
           \coordinate (B) at (0:3); 
           \coordinate (C) at ($ (A) + (B)$); 
           
           \coordinate (X) at (0:4); 
           \draw[dotted] (O)--(X);

           \draw[->, -latex, blue          ] (O) -- (A) node[midway, above left]{\( \vec{a} \)};
           \draw[->, -latex, red           ] (O) -- (B) node[midway, below     ]{\( \vec{b} \)};
    %       \draw[->, -latex, xmgreen,dashed] (O) -- (C) node[pos=1, above right]{\( \vec{c} = \vec{a}+\vec{b} \)};
           
           \draw[->, dashed, -latex, red , thick] (A) -- (C) node[midway, above left]{\( \vec{b} \)};
           \draw[->, dashed, -latex, blue, thick] (B) -- (C) node[midway, above left]{\( \vec{a} \)};

           \draw pic[ pic text = \(\alpha \), draw, angle radius=0.7cm] {angle= B--O--A};
           \draw pic[ pic text = \(\alpha \), draw, angle radius=0.7cm] {angle= X--B--C};
           \draw pic[ pic text = \(\varphi\), draw, angle radius=0.7cm] {angle= C--B--O};

           \draw (0,-0.7) node[right,scale=0.8]{Verband: $\alpha + \varphi = 180^\circ$} ;

        \end{tikzpicture}
    \end{image}

    \begin{image}[0.7\textwidth]
       \begin{tikzpicture}
            % TIKZPICTURE met extra \varphi aangeduid
           \pgfmathsetmacro{\ang}{50} 
              
           \coordinate (O) at (0,0); 
           \coordinate (A) at (\ang :2); 
           \coordinate (B) at (0:3); 
           \coordinate (C) at ($ (A) + (B)$); 
           

           \draw[            ->, -latex, blue,dashed ] (O) -- (A) node[midway, above left]{\( \vec{a} \)};
           \draw[very thick, ->, -latex, red,        ] (O) -- (B) node[midway, below     ]{\( \vec{b} \)};
           \draw[very thick, ->, -latex, xmgreen     ] (O) -- (C) node[pos=1, above right]{\( \vec{c} = \vec{a}+\vec{b} \)};
           
           \draw[->, dashed, -latex, red ,           ] (A) -- (C) node[midway, above left ]{\( \vec{b} \)};
           \draw[->,         -latex, blue, very thick] (B) -- (C) node[midway, below right]{\( \vec{a} \)};

           % \draw pic[ pic text = \(\alpha \), draw, angle radius=0.7cm]                        {angle= B--O--A};
           % \draw pic[ pic text = \(\gamma \), draw, angle radius=1.2cm, angle eccentricity=1.2]{angle= B--O--C};
           \draw[very thick] pic[ pic text = \(\varphi\), draw, angle radius=0.7cm]                        {angle= C--B--O};

           \draw (0,-0.7) node[right,scale=0.8]{In een driehoek: $c^2 = a^2 + b^2 \red{-} 2ab\cos\red{\varphi}$ } ;

       \end{tikzpicture}
       \begin{tikzpicture}

           \pgfmathsetmacro{\ang}{50} 
       
           \coordinate (O) at (0,0); 
           \coordinate (A) at (\ang :2); 
           \coordinate (B) at (0:3); 
           \coordinate (C) at ($ (A) + (B)$); 
           
           \coordinate (X) at (0:4); 
           \draw[dotted] (O)--(X);

           
           \draw[very thick, ->, -latex, blue   ] (O) -- (A) node[midway, above left]{\( \vec{a} \)};
           \draw[very thick, ->, -latex, red    ] (O) -- (B) node[midway, below     ]{\( \vec{b} \)};
           \draw[            ->, -latex, xmgreen,dashed] (O) -- (C) node[pos=1, above right]{\( \vec{c} = \vec{a}+\vec{b} \)};
           
           \draw[->, dashed, -latex, red ] (A) -- (C) node[midway, above left ]{\( \vec{b} \)};
           \draw[->, dashed, -latex, blue] (B) -- (C) node[midway, below right]{\( \vec{a} \)};

           \draw[very thick] pic[ pic text = \(\alpha \), draw, angle radius=0.9cm]                        {angle= B--O--A};
           \draw pic[ pic text = \(\varphi\), draw, angle radius=0.5cm,angle eccentricity=1.3] {angle= C--B--O};
           \draw pic[ pic text = \(\alpha \), draw, angle radius=0.5cm,angle eccentricity=1.3] {angle= X--B--C};

           \draw (0,-0.7) node[right,scale=0.8]{Met vectoren: $\|c\|^2 = \|a\|^2 + \|b\|^2 \red{+} 2\|a\|\|b\|\cos\red{\alpha}$} ;

       \end{tikzpicture}
    \end{image}
    \captionof{figure}{Omdat $\alpha + \varphi = 180^\circ$, is $\cos\varphi = -\cos\alpha$.}

    Het verband tussen beide versies van de cosinusregel is nu duidelijk: afhankelijk van welke hoek je neemt tussen de richtingen van $\vec{a}$ en $\vec{b}$ krijg je ofwel een plusteken, ofwel een minteken.
    Het is niet de moeite om dat teken van buiten te leren, want het teken volgt onmiddellijk uit het speciale geval wanneer $\vec{a}$ en $\vec{b}$ loodrecht op elkaar staan: dan zijn zowel $\alpha$ als $\varphi$ dus $90^\circ$, en de cosinus is dus nul.% 
    \footnote{Gelukkig, want de Stelling van Pythagoras zegt dat bij een rechte hoek $c^2 = a^2 + b^2$, en dat klopt enkel met de cosinusregel als de derde term nul is. }
    \\
    Als de hoek tussen $\vec{a}$ en $\vec{b}$ \textit{kleiner} wordt, wordt ook $\|\vec{c}\|$ \textit{kleiner} en moet de derde term in de cosinusregel \textit{negatief} zijn.\\
    Als de hoek tussen $\vec{a}$ en $\vec{b}$ \textit{groter } wordt, wordt ook $\|\vec{c}\|$ \textit{groter } en moet de derde term in de cosinusregel \textit{positief} zijn.

    Naargelang je met de hoek $\alpha$ rekent dan wel met de hoek $\varphi$ wordt het teken van de term met de cosinus dus aangepast.

% \[
% \|\vec{c}\|^2 = \|\vec{a}\|^2 + \|\vec{b}\|^2 - 2 \,\|\vec{a}\|\,\|\vec{b}\|\cos\varphi
% \]
% \[
% = \|\vec{a}\|^2 + \|\vec{b}\|^2 - 2 \,\|\vec{a}\|\,\|\vec{b}\|\cos(180^\circ - \alpha)
% \]
% \[
% = \|\vec{a}\|^2 + \|\vec{b}\|^2 - 2 \,\|\vec{a}\|\,\|\vec{b}\|(-\cos\alpha)
% \]
% \[
% = \|\vec{a}\|^2 + \|\vec{b}\|^2 + 2 \,\|\vec{a}\|\,\|\vec{b}\|\cos\alpha
% \]


\end{expandable}


\begin{example}

%% Som colineair
\begin{question}
Als in onderstaande figuur \(\|\vec{a}\| = \SI{3}{\newton}\) en \(\|\vec{b}\| = \SI{5}{\newton}\),\\
dan is hier  \(\| \vec{c} \| = \|\vec{a} + \vec{b} \| = \|\vec{a}\| + \|\vec{b}\| = \SI{5}{\newton} - \SI{3}{\newton}= \SI{8}{\newton}\). 

\begin{image}[!]
    \begin{tikzpicture}
        \coordinate (O) at (0,0);
        \coordinate (A) at (1,0);
        \coordinate (B) at (2,0);
        \coordinate (C) at (3,0);

        \coordinate (Shift) at (0,0.05);

        \fill[xmgreen] (O) circle (1pt);
        \fill[red] ($ (O) + (Shift) $) circle (1pt);
        \fill[blue] ($ (O) - (Shift) $) circle (1pt);


        \draw[->, red] ($ (O) + (Shift) $)--($ (A) + (Shift) $)  node[midway, above]{$\vec{a}$};
        \draw[->, xmgreen] (O)--(C)                                  node[pos=0.7, above]{$ \vec{c} = \vec{a} + \vec{b} $};
        \draw[->, blue] ($ (O) - (Shift) $)--($ (B) - (Shift) $)  node[midway, below]{$\vec{b}$};
    \end{tikzpicture}
\end{image}
\end{question}

%% Verschil colineair
\begin{question}
% hier staat onnauwkeurigheid; de norm moet buiten de bewerking staan; de norm van een verschil is commutatief

Als in onderstaande figuur \(\|\vec{a}\| = \SI{3}{\newton}\) en \(\|\vec{b}\| = \SI{5}{\newton}\), \\
dan is hier \(\| \vec{c} \| = \|\vec{a} + \vec{b} \| = \|\vec{a}\|  - \|\vec{b}\| =  \SI{5}{\newton} - \SI{3}{\newton}  = \SI{2}{\newton}\). 

\begin{image}[!]
    \begin{tikzpicture}
        \coordinate (O) at (0,0);
        \coordinate (A) at (-1,0);
        \coordinate (B) at (2,0);
        \coordinate (C) at (1,0);

        \coordinate (Shift) at (0,0.05);

        \fill[xmgreen] (O) circle (1pt);
        \fill[red] ($ (O) + (Shift) $) circle (1pt);
        \fill[blue] ($ (O) - (Shift) $) circle (1pt);


        \draw[->, red] ($ (O) + (Shift) $)--($ (A) + (Shift) $)  node[midway, above]{$\vec{a}$};
        \draw[->, xmgreen] (O)--(C)                                node[pos=0.7, above]{$ \vec{c} = \vec{a} + \vec{b} $};
        \draw[->, blue] ($ (O) - (Shift) $)--($ (B) - (Shift) $) node[midway, below]{$\vec{b}$};
    \end{tikzpicture}
\end{image}

\end{question}

%% Algemeen geval: cosinusregel
\begin{question}

Als in onderstaande figuur \(\|\vec{a}\| = \SI{3}{\newton}, \; \|\vec{b}\| = \SI{5}{\newton} \text{ en } \alpha = 50^\circ\),\\
dan moet  \(\| \vec{c} \| = \|\vec{a} + \vec{b}\| \) worden berekend met de cosinusregel. 


    \begin{image}[0.5\textwidth]
        \begin{tikzpicture}
            % TIKZPICTURE DIE DE SOM VAN TWEE VECTOREN BEREKENT 
           % \draw (-4,-4) grid (4,4);
       
        %    \pgfmathsetmacro{\ax}{1}  
        %    \pgfmathsetmacro{\ay}{2}  
        %    \pgfmathsetmacro{\bx}{3}  
        %    \pgfmathsetmacro{\by}{1} 

           \pgfmathsetmacro{\ang}{50} 
       
           \coordinate (O) at (0,0); 
           \coordinate (A) at (\ang :2); 
           \coordinate (B) at (0:3); 
           \coordinate (C) at ($ (A) + (B)$); 
           
           \coordinate (Z) at (180:1); 

           \draw[dotted] (O)--(Z);
       
           \draw[->, -latex, blue   ] (O) -- (A) node[midway, above left]{\( \vec{a} \)};
           \draw[->, -latex, red    ] (O) -- (B) node[midway, below     ]{\( \vec{b} \)};
           \draw[->, -latex, xmgreen] (O) -- (C) node[pos=1, above right]{\( \vec{c} = \vec{a}+\vec{b} \)};
           
           \draw[->, dashed, -latex, red , thick] (A) -- (C) node[midway, above left ]{\( \vec{b} \)};
           \draw[->, dashed, -latex, blue, thick] (B) -- (C) node[midway, below right]{\( \vec{a} \)};

           \draw pic[ pic text = \(\alpha \), draw, angle radius=0.7cm]                        {angle= B--O--A};
           \draw pic[ pic text = \(\gamma \), draw, angle radius=1.2cm, angle eccentricity=1.2]{angle= B--O--C};
           %\draw pic[ pic text = \(\varphi\), draw, angle radius=0.5cm]                        {angle= C--B--O};
           %\draw pic[ pic text = \(\varphi\), draw, angle radius=0.5cm]                        {angle= A--O--Z};
       \end{tikzpicture}
    \end{image}
    
De grootte van de resultante \(\vec{c}\) wordt bepaald met bovenstaande versie van de  cosinusregel: 

% Cosinusregel
\[
\|\vec{c}\| = \|\vec{a} / \vec{b}\|  = \sqrt{\|\vec{a}\|^2 + \|\vec{b}\|^2 + 2 \,\|\vec{a}\|\,\|\vec{b}\|\cos\alpha}
\]
\[
= \sqrt{3^2 + 5^2 + 2 \cdot 3 \cdot 5 \cdot \cos 50^\circ} \approx \SI{7.3}{N}. 
\]


\end{question}

\end{example}

\begin{remark}
In het algemeen geldt dus \textbf{niet} dat  \( \xcancel{ \|\vec{a+b}\| = \|\vec{a}\| + \|\vec{b}\| } \). 
In welk(e) geval(len) geldt de geljkheid wel? 
\end{remark}


De \textbf{richting van de resultante} (d.w.z. de hoek \(\gamma\)) kan bepaald worden met de sinusregel: 

% TODO HIER IS GEEN COMMANDO \BGSIN 

\[
\frac{\sin\gamma}{\|\vec{b}\|} = \frac{\sin\alpha}{\|\vec{c}\|}
\quad\Longrightarrow\quad
\sin\gamma = \frac{\|\vec{b}\|\sin\alpha}{\|\vec{c}\|}
\]
\[
\gamma = \bgsin\!\left(\frac{5\cdot\sin 50^\circ}{7.3}\right) \approx 18^\circ
\]



% TODO;  TIKZ VAN SOM HERNEMEN MET GROOTTE EN RICHTING RESULTANTE OP AANGEDUID 


De optelling van vectoren is \textit{associatief}, d.w.z. dat \((\vec{a} + \vec{b}) + \vec{c} = \vec{a} + (\vec{b} + \vec{c})\). 
Met deze eigenschap kan je de som bereken van meerderen vectoren. 
Indien er dus meer dan twee vectoren worden samengesteld, tel je eerst twee ervan met elkaar op en het resultaat daarvan tel je met de volgende op, enzovoort totdat alle vectoren in de som zitten (zoals ook met de optelling van getallen gebeurt)

\subsection*{Verschil van twee vectoren}

Net zoals bij getallen $5 - 3 = 5 + (-3)$, kan je ook bij vectoren een verschil schrijven als een som met de tegengestelde: 

\[
\vec{c} = \vec{a}-\vec{b} = \vec{a} + (-\vec{b})
\]

Om \(\vec{c}\) te vinden moeten \(\vec{a}\) en \(-\vec{b}\) dus worden samengesteld.
Het verschil van de getallen acht en vijf is gelijk aan drie. 
Drie is dus het getal dat je bij vijf moet optellen om acht te bekomen. 
Op dezelfde manier is het verschil van vectoren \(\vec{a}\) en \(\vec{b}\) gelijk aan de vector \(\vec{c}\) die je bij \(\vec{b}\) moet optellen om \(\vec{a}\) te bekomen.
 \(\vec{c}\)is dus inderdaad het verschil of 'onderscheid' tussen \(\vec{a}\) en \(\vec{b}\). 


% TIKZPICTURE DIE HET VERSCHIL VAN TWEE VECTOREN WEERGEEFT 
\begin{image}[!]
    \begin{tikzpicture}
        \pgfmathsetmacro{\r}{3}
        \pgfmathsetmacro{\ra}{0.5 * \r}
        \pgfmathsetmacro{\ang}{50}

        \coordinate (O) at (0:0);
        \coordinate (A) at (\ang : \ra);
        \coordinate (B) at (0:\r);
        \coordinate (minB) at (180:\r);
        
        \draw[->, -latex, blue   ] (O)--(A) node[midway, above left]{$\vec{a}$};
        \draw[->, -latex, red    ] (O)--(B) node[midway, below]{$\vec{b}$};
        \draw[->, -latex, xmgreen] (B)--(A) node[midway, above right]{$\vec{c}$};
        \draw[->, -latex, red    ] (O)--(minB) node[midway, below]{$-\vec{b}$};
        \draw[->, -latex, xmgreen] (O)--($(A) - (B)$) node[midway, above right]{$\vec{c}$};
        
        \draw[dotted] (A)--($(A) - (B)$)--(minB);
        \fill[red] (O) circle (1pt); 
    \end{tikzpicture}
\end{image}
\captionof{figure}{Het verschil van de vectoren $\vec{a}$ en $\vec{b}$}

Grafisch blijkt dat indien \(\vec{a}\) en \(\vec{b}\) in hetzelfde punt aangrijpen, \(\vec{a} - \vec{b}\) gelijk is aan de vector met als aangrijpingspunt het eindpunt van \(\vec{b}\) en als eindpunt het eindpunt van \(\vec{b}\).


% TODO: VINCENT HEEFT NOG EEN VOORBEELD VAN DE SLIDES HIER EXTRA GEGEVEN TIJDENS REVIEUW

\subsection*{De loodrechte ontbinding of projectie van een vector in componenten}


% dit zou iets uitgebreiden moeten via de omgekeerde richting (een vecor is eerder opgebouwd uit zijn componenten)

Een vector is opgebouwd als de samenstelling van zijn componten volgens de assen.  
In bepaalde contexten is het vaak erg nuttig om een vector (loodrecht) te ontbinden in zijn componenten. 
Noteer met \(\vec{a}_x\) de component volgens de \(x\)-as en met \(\vec{a}_y\) de component volgens de \(y\)-as. 
Voor elke vector geldt dan 
\[
\vec{a} = \vec{a}_x + \vec{a}_y
\]

De grootte van de componenten volgt rechtstreeks uit de goniometrische getallen: 

\[
\cos(\alpha) = \frac{\| \vec{a}_x \|}{\| \vec{a} \|} \Rightarrow \| \vec{a}_x \| = \|\vec{a}\| \cdot \cos(\alpha)
\]

\[
\sin(\alpha) = \frac{\| \vec{a}_y \|}{\| \vec{a} \|} \Rightarrow \| \vec{a}_y \| = \|\vec{a}\| \cdot \sin(\alpha)
\]

% TIKZPICTURE DIE DE ONTBINDING IN COMPONENTEN WEERGEEFT 
\begin{image}[!]
\begin{tikzpicture}
    % \draw (-4,-4) grid (4,4);

    \pgfmathsetmacro{\ax}{1}  
    \pgfmathsetmacro{\ay}{2}  

    \coordinate (O) at (0,0); 
    \coordinate (A) at (\ax,\ay); 
    \coordinate (AX) at (\ax,0); 
    \coordinate (AY) at (0,\ay); 

    \fill (O) circle (2pt) node[below left]{O};

    \draw[->, -latex] (O) -- (A) node[midway, below]{\(\vec{a}\)};
    \draw[->, dashed, -latex, red, thick] (O) -- (AX) node[midway, below]{\( \vec{a}_x \)};
    \draw[->, dashed, -latex, red, thick] (O) -- (AY) node[midway, left]{\( \vec{a}_y \)};
    \draw pic[ pic text = \(\alpha\), draw, angle radius=0.4cm]{angle= AX--O--A};

    \draw[dotted] (0,\ay)--(A)--(\ax,0);
\end{tikzpicture}
\end{image}
\captionof{figure}{De loodrechte projectie van de vector \(\vec{a}\)}

Indien de componenten worden geschreven met behulp van de basisvectoren geeft dit 


$$
\vec{a} = \vec{a}_x + \vec{a}_y = a_x \cdot \vec{e}_x + a_y \cdot \vec{e}_y 
$$

% VINCENT: KIER KAN NOG EEN EXTRA FABEELDIN G

\subsection*{Het scalair product van twee vectoren (of inwendig product)}

Twee vectoren kan men op twee verschillende manieren met elkaar vermenigvuldigen die een ander resultaat opleveren. 

Het scalair product levert een \textbf{scalar} (= getal) als resultaat op die per definitie gelijk is aan de grootte van de projectie van de ene vector op de andere vermenigvuldigd met de grootte van diezelfde andere vector. 

Het scalair product wordt alsvolgt gedefineerd: 

% todo: hier gedefineerd als? 
$$
c = \vec{a} \cdot \vec{b} = \| \vec{a}_x \| \cdot \|\vec{b}\| = \|\vec{a}\| \cdot \cos(\alpha) \cdot \|\vec{b}\| = \| \vec{a}\| \cdot \| \vec{b}\| \cdot \cos(\alpha)
$$

\begin{quickquestion*}{}{}
    Is de eerste bewerking '\(\cdot\)' dezelfde als de tweede bewerking '\(\cdot\)'? Verklaar. 
\end{quickquestion*}

% TIKZPICTURE DIE SCALAIR PRODUCT van a op b ILLUSTREERT  
\begin{image}[!]
\begin{tikzpicture}
    % \draw (-4,-4) grid (4,4);

    \pgfmathsetmacro{\ax}{1}  
    \pgfmathsetmacro{\ay}{2}  
    \pgfmathsetmacro{\bx}{3}  
    \pgfmathsetmacro{\by}{0} 

    \coordinate (O) at (0,0); 
    \coordinate (A) at (\ax,\ay); 
    \coordinate (B) at (\bx,\by); 

    \fill (O) circle (2pt) node[below left]{O};
    
    \draw[->, -latex] (O) -- (B) node[midway, below]{$\vec{b}$};

    \draw[->, -latex] (O) -- (A) node[midway, above left]{$\vec{a}$};
    \draw[->, dashed, -latex, red, thick] (O) -- (\ax, 0) node[midway, below]{$ \vec{a}_x $};
    \draw[dashed, -latex, thick] (\ax, 0) -- (A);


    \draw pic[ pic text= $\alpha$, draw,  angle radius=0.7cm]{angle = B--O--A}; 
\end{tikzpicture}
\end{image}
\captionof{figure}{De projectie van de vector $\vec{a}$ op $\vec{b}$}

\subsection*{Het vectorieel product van twee vectoren (of kruisproduct)}

Het vectorieel product levert een \textbf{vector} als resultaat op waarvan de grootte gelijk is aan de oppervlakte van de parallellogram ingesloten tussen de twee vectoren. 
De richting van het vectorproduct is loodrecht op het vlak gevormd door de twee gegeven vectoren en de zin is te bepalen met de rechterhandregel.
Het vectorieel product wordt genoteerd als 

\[
\vec{c} = \vec{a} \times \vec{b} = \| \vec{a}_y \| \cdot \|\vec{b}\| = \|\vec{a}\| \cdot \sin(\alpha) \cdot \|\vec{b}\| = \| \vec{a}\| \cdot \| \vec{b}\| \cdot \sin(\alpha)
\]

\begin{image}[!]
    \begin{tikzpicture}
        
        \pgfmathsetmacro{\ax}{1}
        \pgfmathsetmacro{\ay}{2.2}
        \pgfmathsetmacro{\bx}{1.8}
        \pgfmathsetmacro{\by}{0}

        \coordinate (O) at (0,0);
        \coordinate (A) at (\ax,\ay);
        \coordinate (B) at (\bx,\by);

        \fill[gray!50] (O)--(A)--($(A)+(B)$)--(B)--cycle;

        \draw[->, -latex, blue] (O)--(A) node[midway, above left]{$\vec{a}$};
        \draw[->, -latex, red] (O)--(B) node[midway, below]{$\vec{b}$};

        \draw[dotted, thick] (\ax,0)--(A) node[midway, right]{$\|\vec{a}_y\|$};

        \fill (O) node[]{\(\otimes\)};


    \end{tikzpicture}

    
\end{image}
\captionof{figure}{Het vectorieel product}

\begin{image}[0.2\textwidth]
\includegraphics{rechterhandregel}
\end{image}
\captionof{figure}{De rechterhandregel}


\begin{remark}
    Een vector met zin in het blad wordt genoteerd met \(\otimes\). Een vector met zin uit het blad wordt genoteerd met \(\odot\).
\end{remark}

\end{document}