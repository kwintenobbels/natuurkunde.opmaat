\documentclass{ximera}

%\addPrintStyle{../..}

\begin{document}
	\author{Bart Lambregs}
	\xmtitle{Voorstelling en notatie}{}
    \xmsource\xmuitleg


Vectoren worden grafisch voorgesteld met een pijl.  
Een vectoriële grootheid wordt genoteerd met \(\vec{z}, \vec{a}, \vec{x}, ...\) en wordt altijd bij de pijl gezet ter benoeming. 
Om duidelijk te maken dat het telkens om een vector gaat wordt een pijltje boven de letter geplaatst. 
Zonder de vector te benoemen stelt de pijl geen vector voor (en kan het dus evengoed een echte pijl afgeschoten door een boog zijn)! 
De pijl geeft alle kenmerken die een vector vastleggen weer. 



% Tikz zo proberen programeren opdat het een oefening kan zijn in lager jaar: bepaal aangrijpingspunt, richting, grootte en zin. 

\begin{image}
\begin{tikzpicture}

\begin{scope}[shift={(0,0)}]

	\pgfmathsetmacro{\ax}{1}
	\pgfmathsetmacro{\ay}{1}
	\pgfmathsetmacro{\ex}{4}
	\pgfmathsetmacro{\ey}{2}
	
	\coordinate (O) at (0,0); %OORSPRONG 
	\coordinate (A) at (\ax,\ay);  % AANGRIJPINGSPUNT 
	\coordinate (E) at (\ex, \ey); % EINDPUNT 

	\draw[->] (A)--(E) node[midway, below right]{\(\vec{z}\)};

	\fill (A) circle (2pt);
\end{scope}

\begin{scope}[shift={(5,0)}]

	\pgfmathsetmacro{\ax}{1}
	\pgfmathsetmacro{\ay}{1}
	\pgfmathsetmacro{\ex}{4}
	\pgfmathsetmacro{\ey}{2}
	
	\coordinate (O) at (0,0); %OORSPRONG 
	\coordinate (A) at (\ax,\ay);  % AANGRIJPINGSPUNT 
	\coordinate (E) at (\ex, \ey); % EINDPUNT 

	\draw[dashed] (E) -- ($(A) ! 1.5 ! (E) $) node[pos=1, above left]{richting};
	\draw[dashed] (A) -- ($(A) ! -0.5 ! (E) $);
	
	\draw[->,blue, very thick, -{latex[fill=orange]}] (A)--(E) 
		node[midway, below right, black]{\(\vec{z}\)}
		node[midway, above left, blue]{ grootte \( \lVert \vec{z} \rVert \)}
		node[pos=1, above, orange]{zin};

	\fill[red] (A) circle (2pt) node[below right]{aangrijpingspunt};
\end{scope}
\end{tikzpicture}
\end{image}
\captionof{figure}{De vector $\vec{z}$ met al zijn componenten.}


Aan de zin van een vector wordt wiskundig een teken gekoppeld dat afhangt van de gekozen referentie-as. 
Vectoren in de zin van de gekozen referentie-as worden als positief beschouwd, vectoren tegen de zin van de referentie-as als negatief. 

% De grootte van de vector wordt vaak als enkel positief aanzien (vooral in een situatie zonder referentie-as). % absolute waarde is altijd postief? Lengte van het lijnstuk is nooit negatief?
De grootte van een vector \(\vec{z}\) wordt aangeduid met de norm $ \lVert \vec{z} \rVert $  of het absolutewaardeteken $\lvert z \rvert $ en is altijd positief. 
De grootte komt immers overeen met de lengte van de vector (en een lengte is altijd positief). 
% Wanneer er echter geen twijfel is en de grootte sowieso positief is (zoals meestal bij krachten), noteert men ook gewoon $z$. % dit zou ik weglaten



% TODO: ZOU GOED ZIJN ALS HIER VOORBEELDEN UIT VORIGE JAREN WORDEN HERNOMEN (BV VECTOREN BIJ ELECTROMECHANICA)

% Enkele voorbeelden:


% INSERT TIKZ PICTURE ZWAARTEKRACHT EN ROLLENDE BAL 

De richting van een vector wordt weergegeven met een eenheidsvector $\vec{e}$ waarvoor $ \lVert \vec{e} \rVert = 1$. 
Het invoeren van een eenheidsvector blijkt erg nuttig in notaties. 
Hiermee kunnen alle kenmerken van een vector ook algebraïsch weergegeven worden. 

Er geldt in onderstaande tekeningen dat \(\vec{a} = \pm  \lVert \vec{a} \rVert \cdot \vec{e} = +2 \cdot \vec{e} \) met aangrijpingspunt \(2\). 
Voor de vector \(\vec{b}\) geldt \(\vec{b} = \pm  \lVert \vec{b} \rVert \cdot \vec{e} = -1 \cdot \vec{e} \) met aangrijpingspunt \(6\). 

% PH DIT BESCHRIJVEN IS VEEL EENVOUDIGER MET POOLCOORDINATEN.  
% TIKZPICTURE DIE DE EENHEIDSVECTOR ILLUSTREERT 



\begin{image}
\begin{tikzpicture}

	\pgfmathsetmacro{\ang}{15}

	\coordinate (O) at (0,0); 
	\coordinate (E) at ({\ang} : 1); 
	\coordinate (A) at ({\ang} : 2); 
	\coordinate (B) at ({\ang} : 6); 

	\draw[dashed] ({\ang} : -1.5)--({\ang} : 7.5); 
	
	\foreach \i in {-1,...,7}{
		\draw[dashed] ({\ang}:\i) -- ++({\ang+90}:0.1);
		\draw[dashed] ({\ang}:\i) -- ++({\ang-90}:0.1);

		\node[font=\tiny, shift={({\ang}:\i)}] at ({\ang+90}:0.3){\i};
	}

	\draw[->, very thick] (O)--(E) node[midway, below right]{\(\vec{e}\)};
	\draw[->, very thick, red] (A)--($ (A) + ({\ang}:2)$) node[midway, below right]{\(\vec{a}\)};
	\draw[->, very thick, blue] (B)--($ (B) + ({\ang}:-1)$) node[midway, below right]{\(\vec{b}\)};

	\fill (O) circle (2pt);
	\fill[red] (A) circle (2pt);
	\fill[blue] (B) circle (2pt);

\end{tikzpicture}
\end{image}



%  INSERT TIKZ NORM RICHTING ZIN ALGEBRAISCH 

\begin{remark}
Als de grootte van een vector $\vec{c}$ gelijk is aan nul, noemt men dit ook de \textbf{nulvector}. 
Men noteert dit als: $\vec{c} = \vec{0}$   of  $ \lVert \vec{c} \rVert $  of   $ c = 0 $
Men mag niet noteren dat: $ \vec{c} = 0$  
Linkerlid en rechterlid moeten immers beiden een scalar of beiden een vector zijn!
\end{remark}

\begin{quickquestion*}{}{}
Waarom mag je \textbf{niet} noteren dat \(\vec{c} = 0\)? 
\end{quickquestion*}
	
\end{document}
