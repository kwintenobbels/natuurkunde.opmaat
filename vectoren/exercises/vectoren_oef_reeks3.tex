\documentclass{ximera}

%\addPrintStyle{../..}

\begin{document}
	\author{Bart Lambregs en Vincent Gellens}
	\xmtitle{Oefeningen vectoren reeks 3}{}
    \xmsource\xmuitleg



\begin{exercise}
Bij de opzet van een aanval loopt een voetballer eerst \SI{15}{\meter} evenwijdig met de zijlijn om vervolgens onder een hoek van \(45^\circ\) met de zijlijn \SI{18}{\meter} naar binnen te snijden. 
Hoe ver van het vertrekpunt komt hij uit? Maak een schets met vectoren en voer ook hiermee je berekening uit.
\end{exercise}

\begin{exercise}
Vanop dezelfde middenstip vertrekken twee spelers, één wandelt \SI{9}{\meter} evenwijdig met de zijlijn naar het ene doel en de ander wandelt \SI{17}{\meter} in een richting die een hoek van \(35^\circ\) maakt met de middellijn, naar het andere doel toe. 
Hoe ver komen de spelers van elkaar uit? Maak een schets met vectoren en voer ook hiermee je berekening uit.
\end{exercise}

\sisetup{per-mode=symbol, per-symbol=/}

\begin{exercise}
	Twee treinen vertrekken gelijktijdig uit Leuven station met constante snelheden van \SI{10}{\meter\per\second} en \SI{20}{\meter\per\second}.
	De trage trein rijdt recht naar Mechelen en de andere recht naar Aarschot. 
	
	\begin{image}[0.4\textwidth]
	\begin{tikzpicture}
		\pgfmathsetmacro{\ang}{80}
	
		\coordinate (L) at (0,0); 
		\coordinate (A) at (70:1); 
		\coordinate (M) at (70 + \ang : 2); 
	
		\fill (L) circle (1pt);
		\fill (A) circle (1pt);
		\fill (M) circle (1pt);
	
		\draw[dotted, very thick] (L)--(A) node[pos=0, below]{\small{Leuven}} node[pos=1, right]{\small{Aarschot}} --(M) node[pos=1, below left]{\small{Mechelen}} --cycle;
	
		\draw pic[ pic text= \(80^\circ\), draw,  angle radius=0.7cm]{angle = A--L--M};
		
	\end{tikzpicture}
	\end{image}

	\begin{question}
		Bepaal de snelheid van de trage trein ten op zichte van de snelle trein. Werk met vectoren!
	\end{question}

	\begin{question}
		Heeft de snelheid van de snelle t.o.v. de trage trein dezelfde grootte, richting en/of zin?
	\end{question}
	\end{exercise}

\begin{exercise}
	Toon aan dat \( \| \vec{a} - \vec{b} \| = \|\vec{a}\|^2 + \|\vec{b}\|^2 - 2 \cdot \|\vec{a}\| \cdot \|\vec{b}\| \cdot \cos(\alpha)\) met \(\alpha\) de hoek tussen \(\vec{a}\) en \(\vec{b}\). 
\end{exercise}

\begin{exercise}
	Geldt er algemeen dat \(\vec{a} \cdot \vec{b} = \vec{b} \cdot \vec{a}\) \,? 
	Geldt er dat \(\vec{a} \times \vec{b} = \vec{b} \times \vec{a}\) \,? 
	Verklaar kort.
\end{exercise}

\begin{exercise}
	Kan er gelden dat \(\vec{a} \cdot \vec{b} = \| \vec{a} \times \vec{b} \|\) \,? 
	Zoja, geef de nodige voorwaarden en zoniet, verklaar. 
\end{exercise}
\end{document}
