\documentclass{ximera}

%\addPrintStyle{../..}

\begin{document}
	\author{Bart Lambregs en Vincent Gellens}
	\xmtitle{Oefeningen vectoren reeks 1}{}
    \xmsource\xmuitleg


% Bepaal grafisch en kwantitatief de resultante van de gegeven vectoren. 
\begin{exercise}

Bepaal grafisch en kwantitatief de resultante van de gegeven vectoren. 

\begin{question}

\(\|\vec{a} \| = \SI{5}{N}\),\; \(\|\vec{b} \| = \SI{4}{N}\), \(\alpha = 40^\circ\)\;

\begin{image}[0.2\textwidth]
\begin{tikzpicture}
	\pgfmathsetmacro{\ang}{40}

	\coordinate (O) at (0,0); 
	\coordinate (X) at (2,0); 
	\coordinate (A) at (90:3); 
	\coordinate (B) at (-\ang :2); 


	\draw[dotted, thick] (O)--(4,0);
	\draw[->, -latex, very thick, red] (O)--(A) node[midway, left]{$\vec{a}$};
	\draw[->, -latex, very thick, xmgreen] (O)--(B) node[midway, below left]{$\vec{b}$};

	\draw pic[ pic text= \(\alpha\), draw,  angle radius=0.7cm]{angle = B--O--X};
	\draw pic[ angle radius=0.7cm]{right angle = X--O--A};
	
\end{tikzpicture}
\end{image}
\end{question}

\begin{question}
\(\|\vec{a} \| = \SI{4}{N}\),\; \(\|\vec{b} \| = \SI{2.5}{N}\) \; \(\|\vec{c} \| = \SI{3}{N}\), \; \(\alpha = 15^\circ\)\;

\begin{image}[0.2\textwidth]
	\begin{tikzpicture}
		\pgfmathsetmacro{\ang}{15}
	
		\coordinate (O) at (0,0); 
		\coordinate (A) at (180:3); 
		\coordinate (B) at (90-\ang :2); 
		\coordinate (C) at (90:2.5); 
	
		\draw[->, -latex, very thick, red] (O)--(A) node[midway, below]{$\vec{a}$};
		\draw[->, -latex, very thick, xmgreen] (O)--(B) node[midway, below right]{$\vec{b}$};
		\draw[->, -latex, very thick, blue] (O)--(C) node[pos=0.8, left]{$\vec{c}$};
	
		\draw pic[ pic text= \(\alpha\), draw,  angle radius=0.7cm, angle eccentricity=1.5]{angle = B--O--C};
		\draw pic[ draw, angle radius=0.4cm]{right angle = C--O--A};
		% \draw pic[ draw, pic text= \LARGE{$\llcorner$},  angle radius=0.4cm]{right angle = C--O--A};
		
	\end{tikzpicture}
	\end{image}
\end{question}

\begin{question}
\(\|\vec{a} \| = \SI{5}{N}\),\; \(\|\vec{b} \| = \SI{8}{N}\) \; \(\|\vec{c} \| = \SI{15}{N}\), \; \(\|\vec{d} \| = \SI{3}{N}\), \; \(\alpha = 60^\circ\)\;

	\begin{image}[0.2\textwidth]
		\begin{tikzpicture}
			\pgfmathsetmacro{\ang}{60}
		
			\coordinate (O) at (0,0); 
			\coordinate (A) at (0:1); 
			\coordinate (B) at (90-\ang :2); 
			\coordinate (C) at (90:2.5); 
			\coordinate (D) at (270:1.5); 
		
		
			\draw[->, -latex, very thick, blue] (O)--(A) node[midway, below]{$\vec{a}$};
			\draw[->, -latex, very thick, xmgreen] (O)--(B) node[pos=0.7, below right]{$\vec{b}$};
			\draw[->, -latex, very thick, red] (O)--(C) node[midway, left]{$\vec{c}$};
			\draw[->, -latex, very thick, cyan] (O)--(D) node[midway, left]{$\vec{d}$};
		
			\draw pic[ pic text= \(\alpha\), draw,  angle radius=0.5cm]{angle = B--O--C};
			\draw pic[ draw, angle radius=0.2cm]{right angle = D--O--A};
			\draw pic[ pic text= \(\pi\), draw, angle radius=0.3cm]{ angle = C--O--D};
			
		\end{tikzpicture}
		\end{image}
\end{question}
\end{exercise}


\begin{exercise}

Gegeven zijn de vectoren \(\vec{F}\) en \(\vec{x}\) waarvan geweten is dat \(\| \vec{F} \| = \SI{3}{N}, \; \|\vec{x}\| = \SI{4}{m} \text{ en } \alpha = 50^\circ \).
Construeer indien mogelijk en bereken de grootte van: 

	\begin{image}[0.2\textwidth]
		\begin{tikzpicture}
			\pgfmathsetmacro{\ang}{50}
		
			\coordinate (O) at (0,0); 
			\coordinate (X) at (0:2); 
			\coordinate (F) at (\ang :1); 
		
			\draw[->, -latex, very thick, blue] (O)--(F) node[midway, above left]{$\vec{F}$};
			\draw[->, -latex, very thick, red] (O)--(X) node[midway, below]{$\vec{x}$};
		
			\draw pic[ pic text= \(\alpha\), draw,  angle radius=0.7cm]{angle = X--O--F};
		\end{tikzpicture}
	\end{image}


\begin{question}
	\(\vec{F}_x\), zijnde de component van \(\vec{F}\) evenwijdig met \(\vec{x}\). 
\end{question}

\begin{question}
	\(\vec{F}_y\), zijnde de component van \(\vec{F}\) loodrecht op \(\vec{x}\). 
\end{question}

\begin{question}
	\(\vec{F} \cdot \vec{x}\)
\end{question}

\begin{question}
	\(\vec{F} \times \vec{x}\)
\end{question}

\end{exercise}

\begin{exercise}

Gegeven zijn de vectoren \(\vec{F}\) en \(\vec{x}\) waarvan geweten is dat \(\| \vec{F} \| = \SI{7}{N}, \; \|\vec{x}\| = \SI{2}{m} \text{ en } \alpha = 125^\circ \).
Construeer indien mogelijk en bereken de grootte van: 

	\begin{image}[0.2\textwidth]
		\begin{tikzpicture}
			\pgfmathsetmacro{\ang}{125}
			\pgfmathsetmacro{\angg}{30}
		
			\coordinate (O) at (0,0); 
			\coordinate (S) at (\ang + \angg :1); 
			\coordinate (F) at (\angg :2); 
		
			\draw[->, -latex, very thick, blue] (O)--(F) node[midway, below right]{$\vec{F}$};
			\draw[->, -latex, very thick, red] (O)--(S) node[pos=1, above]{$\vec{x}$};
		
			\draw pic[ pic text= \(\alpha\), draw,  angle radius=0.7cm]{angle = F--O--S};
			
		\end{tikzpicture}
		\end{image}

\begin{question}
	\(\vec{F}_x\), zijnde de component van \(\vec{F}\) evenwijdig met \(\vec{x}\). 
\end{question}

\begin{question}
	\(\vec{F}_y\), zijnde de component van \(\vec{F}\) loodrecht op \(\vec{x}\). 
\end{question}

\begin{question}
	\(\vec{F} \cdot \vec{x}\)
\end{question}

\begin{question}
	\(\vec{F} \times \vec{x}\)
\end{question}
\end{exercise}


\end{document}
