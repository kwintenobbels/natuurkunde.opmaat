\documentclass{ximera}

%\addPrintStyle{..}

\begin{document}
	\author{Bart Lambregs}
	\xmtitle{Het begrip vector}{}
    \xmsource\xmuitleg


De natuurkunde beschrijft de niet levende natuur met grootheden die worden opsplitst in twee categorieën: scalaire grootheden (scalars) en vectoriële grootheden (vectoren). 
Grootheden die de vraag kunnen oproepen: “Naar waar gericht?” zijn vectoren, grootheden waarbij die vraag geen antwoord heeft, zijn scalars. 
Dit onderscheid en een correcte omgang met beiden zijn ontzettend belangrijk in fysica.


% het referentiestelsel in onderstaand voorbeeld ligt in de helikopter --> horizontaal is richting; zuiden is zin 
Stel dat \textit{een helikopter vliegt met een snelheid van \SI{40}{\kilo\meter\per\hour}.} 
Vraag: “Naar waar?” Antwoord: “Naar het zuiden, naar Brussel, naar omhoog, schuin naar onderen, …” 
Er zijn vele betekenisvolle antwoorden mogelijk. 
Snelheid is een vector. 
Als \textit{het zwembadwater een temperatuur van  \(\SI{27}{\celsius}\) heeft}, is er geen zinnig antwoord op de vraag \textit{naar waar?}. 
Temperatuur is een scalar.


% AANGEPAST DOOR PARAGRAAF HIEROP VOLGEND 
% Een vectoriële grootheid heeft vier kenmerken: grootte, richting, zin en een aangrijpingspunt. 
% Zo kan de helikopter aan 40 km/h \textit{horizontaal en richting het zuiden} vliegen. % horizontaal?
% Een scalaire grootheid heeft enkel een grootte met teken, zo kan je diepvriezer een temperatuur hebben van \(\SI{-10}{\celsius}\).

Een vectoriële grootheid heeft drie variabele kenmerken: grootte, richting en zin. Voorbeeld: de helikopter vliegt aan \SI{40}{\kilo\meter\per\hour} , horizontaal en naar het zuiden. 
Een scalaire grootheid heeft slechts één kenmerk: de grootte (waarin soms ook een teken vervat zit). 
Voorbeeld: een sneeuwbal heeft een temperatuur van -10°C.

De plaats waarop de vector van toepassing is, noemt men het aangrijpingspunt van de vector.



\begin{expandable}{xmyoutube}{Een universiteitscollege over dit hoofdstuk}
    \youtube{0na1JdPE_JY}
\end{expandable}

% Opmerking eruit gelaten wegens verwarring bij de leerlingen (vincent)
% \begin{remark}
% Het onderscheid tussen scalar en vector is in de eerste plaats een verschil in \textit{naamgeving}.
% Je kan zeggen dat de temperatuur in graden gelijk is aan de scalar \(27\) of gegeven wordt door de vector \(\vec{temp} = (27) \) met slechts één component. 
% Zoals elke \textit{scalar} een vector is, zal op de universiteit blijken dat elke \textit{vector} een \textit{tensor} is. Dus ook elke \textit{scaler} is eigenlijk een \textit{tensor}. 
% \end{remark}


% TODO: MOET ER NOG IETS GEZEGD WORDEN OVER VRIJE VECTOREN? 


	
\end{document}
