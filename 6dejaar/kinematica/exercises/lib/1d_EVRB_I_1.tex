
\begin{exercise}
% !TEX root = ../main.tex




 Een auto vertrekt vanuit rust en bereikt na \SI{3,0}{km} een snelheid van \SI{450}{km/h} We onderstellen de versnelling constant en de baan recht. Bereken de versnelling en de tijd, nodig om die \SI{3,0}{km} af te leggen.
\begin{oplossing}
    % \item[gegeven]$x=3000\rm\,m$\newline $v=125\rm\,m/s$
    % \item[gevraagd]$a$, $t$
    % \item[oplossing]
    Omdat voor een EVRB de gemiddelde snelheid gegeven wordt door $\overline{v}=\frac{v_0+v}{2}$ en we de afgelegde afstand kennen, kunnen we de benodigde tijd gemakkelijk vinden. We kiezen $t_0=0$, $x_0=0$. De beginsnelheid is nul zodat:
    \begin{eqnarray*}
        \Delta x &=& \overline{v}\Delta t \\
        &\Downarrow & \\
        t &=& \frac{x}{\left(\frac{v}{2}\right)} = \frac{2x}{v}
    \end{eqnarray*}
    Invullen van de gegevens levert een tijd van \SI{48}{s}. Met de formule $v=v_0+at$ voor de snelheid vinden we de versnelling door de tijd erin te substitueren, en de beginsnelheid nul te nemen:
    \begin{eqnarray*}
        % v &=& at \\
        % &\Updownarrow&\\
        a &=& \frac{v}{t}=\frac{v}{\left(\frac{2x}{v}\right)}\\
        &=& \frac{v^2}{2x}
    \end{eqnarray*}
    Invullen van de gegeven grootheden levert een versnelling van \SI{2,6}{m/s^2}.
\end{oplossing}

\end{exercise}
