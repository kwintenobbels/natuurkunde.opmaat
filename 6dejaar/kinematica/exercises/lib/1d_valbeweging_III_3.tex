
\begin{exercise}
% !TEX root = ../main.tex


\pt{10}Een ziek man zit voor een raam dat \SI{1,20}{m} hoog is. Een steen wordt vanop de grond opgeworpen en passeert het raam een keer opwaarts en een keer neerwaarts. De man ziet de steen in totaal voor \'e\'en seconde.% (Een halve bij het opgaan en een halve bij het terug naar beneden gaan.)
\begin{enumerate}
\item Bepaal de snelheid waarmee de steen de onderkant van het raam bereikt.% (die is op het teken na hetzelfde in zowel de opwaartse als de neerwaartse beweging).
\item Toon aan dat het met deze gegevens niet mogelijk is te berekenen hoe hoog het raam boven de grond is gelegen.
\end{enumerate}

\begin{oplossing}
	(a) $\Delta x= v_1\Delta t-\frac{1}{2}g\Delta t^2$ zodat $v_1=\bar{v}+g\frac{\Delta t}{2}=\SI{4,85}{m/s}$ ($\Delta t = \SI{0,5}{s}$)
	
	(b) Naast de hoogte van het raam boven de grond, zijn ook de benodigde tijd en de beginsnelheid onbekende grootheden. Met maar twee vergelijkingen die een EVRB beschrijven ($x=x_0+v_0t+\frac{1}{2}at^2$ en $v=v_0+at$), zijn deze onbekenden niet vast te leggen. 
	
	In een meer fysische uitleg kan je je realiseren dat eenzelfde snelheid aan de onderkant van het raam voor een grotere hoogte boven de grond te realiseren is met een grotere snelheid waarmee de steen opgeworpen wordt.
\end{oplossing}





\end{exercise}
