\documentclass{ximera}
\input{../../preamble}

\addPrintStyle{../..}

\begin{document}
	\author{Bart Lambregs,Vincent Gellens}
	\xmtitle{De snelheid}{}
    \xmsource\xmuitleg


\sisetup{per-mode=symbol} %TEST VOOR DE SI EENHEDEN 

\NewDocumentCommand{\maakachtvar}{m m m m m m m m}{

\pgfmathsetmacro{\at}{#1}    \pgfmathsetmacro{\ax}{#2}
\pgfmathsetmacro{\bt}{#3}    \pgfmathsetmacro{\bx}{#4}
\pgfmathsetmacro{\ct}{#5}    \pgfmathsetmacro{\cx}{#6}
\pgfmathsetmacro{\dt}{#7}    \pgfmathsetmacro{\dx}{#8}

\pgfmathsetmacro{\tmin}{min(\at,\bt,\ct,\dt)}
\pgfmathsetmacro{\tmax}{max(\at,\bt,\ct,\dt)}
\pgfmathsetmacro{\xmin}{min(\ax,\bx,\cx,\dx)}
\pgfmathsetmacro{\xmax}{max(\ax,\bx,\cx,\dx)}

\pgfmathsetmacro{\deltat}{\tmax-\tmin}
\pgfmathsetmacro{\deltax}{\xmax-\xmin}

\pgfmathsetmacro{\xmargin}{0.08*\deltat}   
\pgfmathsetmacro{\ymargin}{0.3*\deltax} 

\pgfmathsetmacro{\xrangemin}{\tmin-\xmargin}
\pgfmathsetmacro{\xrangemax}{\tmax+\xmargin}
\pgfmathsetmacro{\yrangemin}{\xmin-\ymargin}
\pgfmathsetmacro{\yrangemax}{\xmax+\ymargin}
}


Een voorwerp in beweging heeft een snelheid. 
De ervaring leert dat hoe groter de snelheid (in de auto, op je fiets, ...) hoe groter de verplaatsing in een bepaald tijdsinterval. 
Als je fietst aan \SI{30}{\kilo\meter\per\hour}, leg je op één uur tijd \SI{30}{km} af. 
Als je wandelt aan \SI{5}{\kilo\meter\per\hour} leg je op één uur tijd slechts \SI{5}{km}. 
Met behulp van de plaatsfunctie kan de snelheid van een voorwerp volledig worden bepaald. 


% Een voorwerp heeft snelheid als het beweegt, er is een verandering van de positie in de tijd.
% De snelheidsvector grijpt aan op het bewegend voorwerp en wijst in de zin van de ogenblikkelijke beweging, dat is rakend aan de baan dat het voorwerp maakt. % bewijs hiervoor van Bart kan toegevoegd worden
% In animaties lijkt het alsof de snelheidsvector aan de positievector 'trekt'. Bezie hiervoor animaties te vinden op https://www.hansbekaert.be/fysica/new/applets6.html.



\subsection*{Snelheid bij ééndimensionale bewegingen}

% gemiddelde snelheid als je een rondje loopt is nu nul; dat moet miss benoemd worden 

% Als een voorwerp op een rechte lijn beweegt (1D), dan valt de beweegrichting en dus ook die van de snelheidsvector samen met de richting van die lijn.  % dit weten ze nog niet...? 
% Bovenstaande zin is erg moeilijk; gemiddelde vector is eerder een scalar dan een vector. 

De tijd nodig voor een bepaalde verplaatsing geeft aanleiding tot de \textit{gemiddelde snelheid}. 

% hieronder gedefinieerd als een scalar --> niet helemeel eenduidig dus momenteel. 

\begin{definition}
	
De gemiddelde snelheid $\overline{v}$ van een voorwerp tussen twee tijdstippen wordt gedefiniëerd als
\[
\overline{v}=\frac{\Delta x}{\Delta t}=\frac{x_2-x_1}{t_2-t_1}
\]
De eenheid van snelheid is meter per seconde $[v]=\rm\,m/s$. 
\end{definition}


In het traject van de zeilboot is de gemiddelde snelheid van de boot tussen de tijdstippen $t_1$ en $t_2$ gelijk aan 
$\overline{v}=\frac{x_2-x_1}{t_2-t_1}=\frac{\SI{10}{\meter} - \SI{40}{\meter}}{\SI{15}{\second} - \SI{5}{\second}}= \frac{\SI{-30}{\meter}}{\SI{10}{\second}} = \SI{-3}{\meter\per\second}$. 
Deze gemiddelde snelheid is negatief, wat betekent dat de zeilboot tegen de positieve richting-as bewogen is. 


\begin{quickquestion*}
Kan je op de plaatsgrafiek van de zeilboot de gemiddelde snelheid tussen \(t_2\) en \(t_0\) aanduiden? 
\end{quickquestion*}



\subsection*{Ogenblikkelijke snelheid bij ééndimensionale bewegingen}

De gemiddelde snelheid \(\overline{v}\) is enkel gedefinieerd \textit{tussen} twee posities \(x_1\) en \(x_2\). 
De plaatsfunctie \(x(t)\) kent voor elk tijdstip \(t\) aan een puntmassa een positie \(x\) toe. Op dezelfde manier zou je op elk tijdstip de snelheid willen kennen. 

De eenheid van snelheid is \SI{}{\meter\per\second}, het gaat dus over het aantal meter dat in een bepaald tijds\-in\-ter\-val wordt afgelegd. 
Op één bepaald moment \(t\), één ogenblik, is er helemaal geen tijdsverloop en bijgevolg kan er ook geen verplaatsing zijn \ldots! Er is immers geen tijd verstreken om afstand te kunnen afleggen.

\begin{denkvraag*}{}
Als een raket opstijgt vanuit stilstand is de snelheid duidelijk groter dan nul. \textbf{Wat is volgens jou de snelheid op het eerste moment dat een opstijgende raket de grond niet meer raakt?}
\end{denkvraag*}


`Ja, maar', ga je zeggen, `de snelheidsmeter van mijn fiets zegt toch hoe hard ik ga?!' 
Dat \textit{lijkt} inderdaad een ogenblikkelijke snelheid te zijn maar in werkelijkheid is dat steeds de gemiddelde snelheid over het tijdsinterval dat het sensortje op je wiel nodig heeft om één omwenteling te maken. 
Je snelheidsmeter berekent dus de gemiddelde snelheid door je wielomtrek\footnote{Die je hebt moeten ingeven\ldots} te delen door de tijd van één omwenteling. 
Als je plots remt gaat je snelheid afnemen maar je metertje gaat dit niet ogenblikkelijk kunnen aangeven. 
Het moet wachten totdat het sensortje weer rond is geweest om de tijd te kennen en zo de snelheidsverandering te kunnen registreren.



% DIT GAAT WEL ALS DE TEKST INORDE IS
% % ------------------% % ------------------% % ------------------% % ------------------% % ------------------
% \begin{image}
% \includegraphics[width=0.8\textwidth]{stuiterendetennisbal}
% \end{image}
% \captionof{figure}{Stuiterende tennisbal}
% Hoe kan dit probleem opgelost worden? Op de stroboscopische foto van de stuiterende tennisbal is te zien dat de bal bovenaan trager beweegt dan wanneer hij de grond nadert. 
% Bovenaan liggen de beelden immers dichter bij elkaar zodat de tennisbal minder afstand aflegt in de tijdsspanne tussen twee opeenvolgende opnames. 
% Deze kwantitatieve\footnote{Kwantitatief wil zeggen dat het over een hoeveelheid of een grootte gaat.} informatie die levert echter opnieuw gemiddelde snelheid en niet zomaar de ogenblikkelijke snelheid. 
% De tennisbal verandert immers nog van snelheid tussen twee opeenvolgende opnames. 
% Door de frequentie\footnote{Frequentie is een grootheid die aangeeft hoeveel cyclussen er per seconde worden doorlopen. 
% Hier gaat het dus over het aantal beelden dat per seconde wordt gemaakt. 
% De eenheid van frequentie is $\rm\,s^{-1}$ oftewel Hz (de Hertz).} waarmee de foto's worden genomen op te drijven, krijgen we een accurater beeld van de snelheid die de tennisbal op een gegeven moment heeft. 
% De tijdsintervallen zijn nu immers korter zodat de bal minder van snelheid kan veranderen gedurende de intervallen en zodoende de gemiddelde snelheid een indicatie wordt van de ogenblikkelijke snelheid. 
% De ogenblikkelijke snelheid wordt dus beter en beter benaderd door het tijdsinterval kleiner en kleiner en kleiner en kleiner\ldots te nemen. Echter, hoe kort het tijdsinterval ook is, de snelheid zal veranderen gedurende dat hele kleine tijdsinterval.
% Daarom, je raadde het misschien al, wordt de ogenblikkelijke snelheid gedefinieerd als de \textit{limiet} van de gemiddelde snelheid over een tijdsinterval waarbij we dat interval naar nul laten gaan. 
% % ------------------% % ------------------% % ------------------% % ------------------% % ------------------


Hoe kan dit probleem opgelost worden? 
Om de ogenblikkelijke snelheid op \(t_1 = \SI{5}{\second}\) te kennen, lijk het eerste (en misschien wel enige...) idee om te vertrekken van de gemiddelde snelheid tussen \(t_1\) en \(t_2\) die hierboven werd berekend. 
Deze gemiddelde snelheid kan je beschouwen als een erg ruwe schatting van de snelheid op \(t_1 = 5\). 

\[
\overline{v}=\frac{\Delta x}{\Delta t}=\frac{x_2-x_1}{t_2-t_1} = \frac{10-40}{10-5} = \frac{-30}{5} = \SI{-6}{\meter\per\second}
\]


\maakachtvar{0}{30}{5}{40}{15}{10}{35}{-10}

\begin{image}[0.8\textwidth]
  \begin{tikzpicture}
  \begin{axis}[
	  axis x line=bottom,
	  axis y line=none,   
	  axis line style={->},
	  xmin= \yrangemin, xmax=\yrangemax,
	  ymin=-1, ymax=3,  %plek voor schepen en tijdlabels 
	  xlabel={Positie--as (in meter)},
	  xtick distance=5,
	  xticklabel={\tiny\SI[round-precision=0, round-mode=places]{\tick}{m}},
	  width=14cm, height=4cm,
  ]

  % Schip niet tonen-->opacity op nul
  \node[opacity=0,scale=1,xscale=-1]  (schip0) at (axis cs:\ax,0.5)  {\includegraphics[width=1cm]{schip_icoon}};
  \node[opacity=0,draw, rectangle, inner sep=1pt] (t1) at (axis cs:\ax,2.2) {\small $t_0=\SI{\at}{\second}$};

  \node[opacity=1,scale=1,xscale=1] (schip1) at (axis cs:\bx,0.5)  {\includegraphics[width=1cm]{schip_icoon}};
  \node[draw, rectangle, inner sep=1pt] (t2) at (axis cs:\bx,2.2) {\small $t_1=\SI{\bt}{\second}$};

  \node[opacity=1]                     (schip2) at (axis cs:\cx,0.5) {\includegraphics[width=1cm]{schip_icoon}};
  \node[draw, rectangle, inner sep=1pt] (t3) at (axis cs:\cx,2.2) {\small $t_2=\SI{\ct}{\second}$};

  % Schip niet tonen-->opacity op nul
  \node[opacity=0]                                (schip3) at (axis cs:\dx,0.5) {\includegraphics[width=1cm]{schip_icoon}};
  \node[opacity=0, draw, rectangle, inner sep=1pt] (t4) at (axis cs:\dx,2.2) {\small $t_3=\SI{\dt}{\second}$};

  \draw[<->, dotted, red] (schip1) -- (schip2) node[midway, fill=white]{\(\Delta x\)};

  
  \end{axis}
  \end{tikzpicture}

  \begin{tikzpicture}
  \begin{axis}[
	  axis x line=bottom,
	  axis y line=left,
	  axis line style={->},
	  xlabel={tijd (in seconden)},
	  ylabel={positie (in meter)},
	  xmin= \xrangemin, xmax=\xrangemax,
	  ymin= \yrangemin, ymax=\yrangemax,
	  xtick distance=5,
	  ytick distance=10,
	  xticklabel={\SI[round-precision=0, round-mode=places]{\tick}{s}},
	  yticklabel={\SI[round-precision=0, round-mode=places]{\tick}{m}},
	  grid=both,
	  width=15cm, height=10cm,
  ]

  \addplot[blue, thick, smooth, tension=0.5] coordinates{
	  (\at,\ax)
	  (\bt,\bx)
	  (\ct,\cx)
	  (\dt,\dx)
  };
  
  % SCHIP NIET AFBEELDEN --> OPACITY 0 
  \node[opacity=0] (schip0) at (axis cs:\at,\ax) {\includegraphics[width=1cm]{schip_icoon}};
  \node[opacity=0, draw, rectangle, inner sep=1pt] at (axis cs:\at,\ax+10) {\small $t_0=\SI{\at}{\second}$};

  \node[] (schip1) at (axis cs:\bt,\bx) {\includegraphics[width=1cm]{schip_icoon}};
  \node[draw, rectangle, inner sep=1pt] at (axis cs:\bt,\bx+10) {\small $t_1=\SI{\bt}{\second}$};

  \node[] (schip2) at (axis cs:\ct,\cx) {\includegraphics[width=1cm]{schip_icoon}};
  \node[draw, rectangle, inner sep=1pt] at (axis cs:\ct,\cx+10) {\small $t_2=\SI{\ct}{\second}$};

  % SCHIP NIET AFBEELDEN --> OPACITY 0 
  \node[opacity=0] (schip3) at (axis cs:\dt,\dx) {\includegraphics[width=1cm]{schip_icoon}};
  \node[opacity=0, draw, rectangle, inner sep=1pt] at (axis cs:\dt,\dx+10) {\small $t_3=\SI{\dt}{\second}$};

  % --- DeltaX accolade ---
  \draw[decorate, decoration={brace, amplitude=5pt, mirror}, red, thick] 
  (axis cs:\bt,\bx) -- (axis cs:\bt,\cx) 
  node[midway, right, xshift=2pt] {\small $\Delta x$};

  % --- DeltaT accolade ---
  \draw[decorate, decoration={brace, amplitude=5pt, mirror}, xmgreen, thick] 
	  (axis cs:\bt,\cx) -- (axis cs:\ct,\cx) 
	  node[midway, below, yshift=-2pt] {\small $\Delta t$};

  \end{axis}
  \end{tikzpicture}
\end{image}
\captionof{figure}{De positie van de zeilboot met een tijd-as}

Indien niet met \(t_2 = 15\) de gemiddelde snelheid wordt berekent, maar bijvoorbeeld met \(t_l = \SI{10}{\second} \), zal deze gemiddelde snelheid beter de ogenblikkelijke snelheid op \(t_1\) benaderen. 
Op de grafiek zien we dat \(x_l = 22\) en een rechtsreekse berekening levert dan dat de gemiddelde snelheid tussen \(t_1=5\) en \(t_l=8\) gegeven wordt door 

\[
\overline{v}=\frac{x_l-x_1}{t_l-t_1}=\frac{\SI{22}{\meter} - \SI{40}{\meter}}{\SI{10}{\second} - \SI{5}{\second}}=  \frac{\SI{-18}{\meter}}{\SI{5}{\second}} = \SI{-3.6}{\meter\per\second}
\]

De gemiddelde snelheid is kleiner geworden en op de plaatsgrafiek schuift de zeilboot in de richting van \(t_1\). 
De gemiddelde snelheid \(\bar{v} = \frac{x_l - x_1}{t_l-t_1}\) is een \textit{betere} benadering voor de ogenblikkelijke snelheid op \(t_a\). 
Het is nog steeds een gemiddelde snelheid! 
De lezer raadde het waarschijnlijk al...

Door de gemiddelde snelheid te berekenen voor \(t_m = 7\) waarvoor \(x_m = 31\) wordt de benadering voor de ogenblikkelijke snelheid op \(t_1\) nog beter... 
Op de grafiek komen de twee schepen nu wel erg dicht bij elkaar... 

\[
\overline{v}=\frac{x_2-x_m}{t_2-t_m}=\frac{\SI{31}{\meter} - \SI{40}{\meter}}{\SI{7}{\second} - \SI{5}{\second}}=  \frac{\SI{9}{\meter}}{\SI{2}{\meter}} = \SI{4.5}{\meter\per\second}
\]

% HIER ZIJN DE NUMMER NOG FOUT --> WORDT GROTER DOOR AFRONDINGSFOUTEN... 

\maakachtvar{0}{30}{5}{40}{15}{10}{35}{-10}

\begin{image}[0.8\textwidth]
  \begin{tikzpicture}
  \begin{axis}[
	  axis x line=bottom,
	  axis y line=none,   
	  axis line style={->},
	  xmin= \yrangemin, xmax=\yrangemax,
	  ymin=-1, ymax=3,  %plek voor schepen en tijdlabels 
	  xlabel={Positie--as (in meter)},
	  xtick distance=5,
	  xticklabel={\tiny\SI[round-precision=0, round-mode=places]{\tick}{m}},
	  width=14cm, height=4cm,
  ]

  % Schip niet tonen-->opacity op nul
  \node[opacity=0,scale=1,xscale=-1]  (schip0) at (axis cs:\ax,0.5)  {\includegraphics[width=1cm]{schip_icoon}};
  \node[opacity=0,draw, rectangle, inner sep=1pt] (t1) at (axis cs:\ax,2.2) {\small $t_0=\SI{\at}{\second}$};

  \node[opacity=1,scale=1,xscale=1] (schip1) at (axis cs:\bx,0.5)  {\includegraphics[width=1cm]{schip_icoon}};
  \node[draw, rectangle, inner sep=1pt] (t2) at (axis cs:\bx,2.2) {\small $t_1=\SI{\bt}{\second}$};

  \node[opacity=1]                     (schip2) at (axis cs:\cx,0.5) {\includegraphics[width=1cm]{schip_icoon}};
  \node[draw, rectangle, inner sep=1pt] (t3) at (axis cs:\cx,2.2) {\small $t_2=\SI{\ct}{\second}$};

  \node[opacity=1]                     (schipl) at (axis cs:22,0.5) {\includegraphics[width=1cm]{schip_icoon}};
  \node[draw, rectangle, inner sep=1pt] (t3) at (axis cs:22,2.2) {\small $t_l=\SI{10}{\second}$};

  \node[opacity=1]                     (schipm) at (axis cs:31,0.5) {\includegraphics[width=1cm]{schip_icoon}};
  \node[draw, rectangle, inner sep=1pt] (t3) at (axis cs:31,2.2) {\small $t_m=\SI{7}{\second}$};

  % Schip niet tonen-->opacity op nul
  \node[opacity=0]                                (schip3) at (axis cs:\dx,0.5) {\includegraphics[width=1cm]{schip_icoon}};
  \node[opacity=0, draw, rectangle, inner sep=1pt] (t4) at (axis cs:\dx,2.2) {\small $t_3=\SI{\dt}{\second}$};

  \draw[<->, dotted, red] (schip1) -- (schipm) node[above, fill=white]{\(\Delta x\)};

  
  \end{axis}
  \end{tikzpicture}

  \begin{tikzpicture}
  \begin{axis}[
	  axis x line=bottom,
	  axis y line=left,
	  axis line style={->},
	  xlabel={tijd (in seconden)},
	  ylabel={positie (in meter)},
	  xmin= \xrangemin, xmax=\xrangemax,
	  ymin= \yrangemin, ymax=\yrangemax,
	  xtick distance=5,
	  ytick distance=10,
	  xticklabel={\SI[round-precision=0, round-mode=places]{\tick}{s}},
	  yticklabel={\SI[round-precision=0, round-mode=places]{\tick}{m}},
	  grid=both,
	  width=15cm, height=10cm,
  ]

  \addplot[blue, thick, smooth, tension=0.5] coordinates{
	  (\at,\ax)
	  (\bt,\bx)
	  (\ct,\cx)
	  (\dt,\dx)
  };
  
  % SCHIP NIET AFBEELDEN --> OPACITY 0 
  \node[opacity=0] (schip0) at (axis cs:\at,\ax) {\includegraphics[width=1cm]{schip_icoon}};
  \node[opacity=0, draw, rectangle, inner sep=1pt] at (axis cs:\at,\ax+10) {\small $t_0=\SI{\at}{\second}$};

  \node[] (schip1) at (axis cs:\bt,\bx) {\includegraphics[width=1cm]{schip_icoon}};
  \node[draw, rectangle, inner sep=1pt] at (axis cs:\bt,\bx+10) {\small $t_1=\SI{\bt}{\second}$};

  \node[] (schip2) at (axis cs:\ct,\cx) {\includegraphics[width=1cm]{schip_icoon}};
  \node[draw, rectangle, inner sep=1pt] at (axis cs:\ct,\cx+10) {\small $t_2=\SI{\ct}{\second}$};

  \node[] (schipl) at (axis cs:10,22) {\includegraphics[width=1cm]{schip_icoon}};
  \node[draw, rectangle, inner sep=1pt] at (axis cs:10,22+10) {\small $t_l=\SI{10}{\second}$};

  \node[] (schipm) at (axis cs:7,31) {\includegraphics[width=1cm]{schip_icoon}};
  \node[draw, rectangle, inner sep=1pt] at (axis cs:7,31+10) {\small $t_m=\SI{7}{\second}$};

  % SCHIP NIET AFBEELDEN --> OPACITY 0 
  \node[opacity=0] (schip3) at (axis cs:\dt,\dx) {\includegraphics[width=1cm]{schip_icoon}};
  \node[opacity=0, draw, rectangle, inner sep=1pt] at (axis cs:\dt,\dx+10) {\small $t_3=\SI{\dt}{\second}$};

  % --- DeltaX accolade ---
  \draw[decorate, decoration={brace, amplitude=5pt, mirror}, red, thick] 
  (axis cs:\bt,\bx) -- (axis cs:\bt,22) 
  node[midway, left, xshift=2pt] {\small $\Delta x$};

  % --- DeltaT accolade ---
  \draw[decorate, decoration={brace, amplitude=5pt, mirror}, xmgreen, thick] 
	  (axis cs:\bt,22) -- (axis cs:10,22) 
	  node[midway, below, yshift=-2pt] {\small $\Delta t$};

  \end{axis}
  \end{tikzpicture}
\end{image}
\captionof{figure}{De positie van de zeilboot met een tijd-as}







De ogenblikkelijke snelheid wordt dus beter en beter benaderd door het tijdsinterval kleiner en kleiner en kleiner en kleiner\ldots te nemen. 
Echter, hoe kort het tijdsinterval ook is, de snelheid zal veranderen gedurende dat hele kleine tijdsinterval.
Daarom, je raadde het misschien al, wordt de ogenblikkelijke snelheid gedefinieerd als de \textbf{limiet van de gemiddelde snelheid} over een tijdsinterval waarbij we dat interval naar nul laten gaan. 

\begin{definition}
	
De \textbf{ogenblikkelijke snelheid}  is de afgeleide van de plaatsfunctie:
\[
v=\lim_{\Delta t\to 0}\frac{\Delta x}{\Delta t}=\lim_{t\to t_0}\frac{x(t)-x(t_0)}{t-t_0}=\frac{dx}{dt}
\]
De notatie met een accent $v(t)=x'(t)$ of $v=x'$ wordt op dezelfde manier als in de wiskunde gebruikt. 
De functie $v(t)$ geeft op elk moment $t$ de snelheid $v(t)$. 
\end{definition}


Grafisch kan je de afgeleide terugvinden als de richtingscoöefficiënt van de raaklijn. In een $x-t$ grafiek (de grafiek van de functie $x(t)$, $x$ in functie van $t$) vind je  de snelheid als de richtingscoöefficiënt van de raaklijn in het beschouwde punt. 

\begin{image}
\includegraphics[width=0.8\textwidth]{snelheid1D}

\end{image}

\begin{image}
\includegraphics[width=0.8\textwidth]{overzichtsnelheid1D}

\end{image}

\begin{remark}
	Het woord 'ogenblikkelijk' mag je weglaten. Wanneer we het over snelheid hebben, bedoelen we vanaf nu steeds ogenblikkelijke snelheid.
\end{remark}



\begin{exercise}
Hieronder staat opnieuw de plaatsfunctie \(x(t)\) van de zeilboot. Bepaal enkel met de grafiek en zonder te rekenen: 
\begin{itemize}
	\item Waar ligt de zeilboot stil? 
	\item Waar heeft de zeilboot een positieve snelheid?
	\item waar is de snelheid negatief? 
	\item Op welk moment bewoog de zeilboot het snelst? 
\end{itemize}

\begin{image}[0.5\textwidth]
	\begin{tikzpicture}
		\begin{axis}[
			axis x line=bottom,
			axis y line=left,
			axis line style={->},
			xlabel={tijd (in seconden)},
			ylabel={positie (in meter)},
			xmin= \xrangemin, xmax=\xrangemax,
			ymin= \yrangemin, ymax=\yrangemax,
			xtick distance=5,
			ytick distance=10,
			xticklabel={\SI[round-precision=0, round-mode=places]{\tick}{s}},
			yticklabel={\SI[round-precision=0, round-mode=places]{\tick}{m}},
			grid=both,
			width=15cm, height=10cm,
		]

		\addplot[blue, thick, smooth, tension=0.5] coordinates{
			(\at,\ax)
			(\bt,\bx)
			(\ct,\cx)
			(\dt,\dx)
		};
		
		\node[] at (axis cs:\at,\ax) {\includegraphics[width=1cm]{schip_icoon}};
		\node[draw, rectangle, inner sep=1pt] at (axis cs:\at,\ax+10) {\small $t_0=\SI{\at}{\second}$};
	
		\node[] at (axis cs:\bt,\bx) {\includegraphics[width=1cm]{schip_icoon}};
		\node[draw, rectangle, inner sep=1pt] at (axis cs:\bt,\bx+10) {\small $t_1=\SI{\bt}{\second}$};
	
		\node[] at (axis cs:\ct,\cx) {\includegraphics[width=1cm]{schip_icoon}};
		\node[draw, rectangle, inner sep=1pt] at (axis cs:\ct,\cx+10) {\small $t_2=\SI{\ct}{\second}$};
	
		\node[] at (axis cs:\dt,\dx) {\includegraphics[width=1cm]{schip_icoon}};
		\node[draw, rectangle, inner sep=1pt] at (axis cs:\dt,\dx+10) {\small $t_3=\SI{\dt}{\second}$};
	
		\end{axis}
		\end{tikzpicture}
	\end{image}
	\captionof{figure}{De positie van de zeilboot met een tijd-as}
	
\end{exercise}

\subsection*{Snelheid bij tweedimensionale bewegingen}

Bij voorwerpen die in twee dimensies bewegen, splitst men de beweging doorgaans op in loodrechte x- en y-componenten. Daarmee kan men analoge redeneringen en gelijkaardige formules maken zoals in het 1D geval.

\begin{image}
\includegraphics[width=0.8\textwidth]{snelheid2D}

\end{image}

	
\end{document}