\documentclass{ximera}
\input{../../preamble}

\addPrintStyle{../..}

\begin{document}
	\author{Bart Lambregs,Vincent Gellens}
	\xmtitle{De snelheid}{}
    \xmsource\xmuitleg

Een voorwerp heeft snelheid als het beweegt, er is een verandering van de positie in de tijd.
De snelheidsvector grijpt altijd aan op het voorwerp en wijst in de zin van de ogenblikkelijke beweging, rakend aan de baan dus.
In animaties lijkt het alsof de snelheidsvector aan de positievector "trekt". Bezie hiervoor animaties te vinden op https://www.hansbekaert.be/fysica/new/applets6.html.


\subsection*{Gemiddelde snelheid bij ééndimensionale bewegingen}

% gemiddelde snelheid als je een rondje loopt is nu nul; dat moet miss benoemd worden 

De tijd nodig voor een bepaalde verplaatsing geeft aanleiding tot de \textit{gemiddelde snelheid}. 


\begin{definition}
	
De gemiddelde snelheid $\overline{v}$ van een voorwerp tussen twee tijdstippen wordt gedefiniëerd als
\[
\overline{v}=\frac{\Delta x}{\Delta t}=\frac{x_2-x_1}{t_2-t_1}
\]
De eenheid van snelheid is meter per seconde $[v]=\rm\,m/s$. 
\end{definition}

In de figuur is de gemiddelde snelheid van de auto tussen de tijdstippen $t_1$ en $t_2$ gelijk aan $\overline{v}=\frac{x_2-x_1}{t_2-t_1}=\frac{40\rm\,m-50\rm\,m}{20\rm\,s-10\rm\,s}=-1\rm\,m/s$. Als de snelheid negatief is betekent dit dat de auto achteruit rijdt.



\subsection*{Ogenblikkelijke snelheid bij ééndimensionale bewegingen}

De gemiddelde snelheid \(\overline{v}\) is enkel gedefinieerd \textit{tussen} twee posities \(x_1\) en \(x_2\). De plaatsfunctie \(x(t)\) kent voor elk tijdstip \(t\) aan een puntmassa een positie \(x\) toe. Op dezelfde manier zou je op elk tijdstip de snelheid willen kennen. 

De eenheid van snelheid is \(m/s\), het gaat dus over het aantal meter dat in een bepaald tijds\-in\-ter\-val wordt afgelegd. Op één bepaald moment \(t\), één ogenblik, is er helemaal geen tijdsverloop en bijgevolg kan er ook geen verplaatsing zijn \ldots! Er is immers geen tijd verstreken om afstand te kunnen afleggen.

\begin{denkvraag*}{}
Als een raket opstijgt vanuit stilstand is de snelheid duidelijk groter dan nul. \textbf{Wat is volgens jou de snelheid op het eerste moment dat een opstijgende raket de grond niet meer raakt?}
\end{denkvraag*}


`Ja, maar', ga je zeggen, `de snelheidsmeter van mijn fiets zegt toch hoe hard ik ga?!' Dat \textit{lijkt} inderdaad een ogenblikkelijke snelheid te zijn maar in werkelijkheid is dat steeds de gemiddelde snelheid over het tijdsinterval dat het sensortje op je wiel nodig heeft om één omwenteling te maken. Je snelheidsmeter berekent dus de gemiddelde snelheid door je wielomtrek\footnote{Die je hebt moeten ingeven\ldots} te delen door de tijd van één omwenteling. Als je plots remt gaat je snelheid afnemen maar je metertje gaat dit niet ogenblikkelijk kunnen aangeven. Het moet wachten totdat het sensortje weer rond is geweest om de tijd te kennen en zo de snelheidsverandering te kunnen registreren.
\begin{image}

\includegraphics[width=0.8\textwidth]{stuiterendetennisbal}
\end{image}
\captionof{figure}{Stuiterende tennisbal}

Hoe kan dit probleem opgelost worden? Op de stroboscopische foto van de stuiterende tennisbal is te zien dat de bal bovenaan trager beweegt dan wanneer hij de grond nadert. Bovenaan liggen de beelden immers dichter bij elkaar zodat de tennisbal minder afstand aflegt in de tijdsspanne tussen twee opeenvolgende opnames. 
Deze kwantitatieve\footnote{Kwantitatief wil zeggen dat het over een hoeveelheid of een grootte gaat.} informatie die levert echter opnieuw gemiddelde snelheid en niet zomaar de ogenblikkelijke snelheid. De tennisbal verandert immers nog van snelheid tussen twee opeenvolgende opnames. Door de frequentie\footnote{Frequentie is een grootheid die aangeeft hoeveel cyclussen er per seconde worden doorlopen. Hier gaat het dus over het aantal beelden dat per seconde wordt gemaakt. De eenheid van frequentie is $\rm\,s^{-1}$ oftewel Hz (de Hertz).} waarmee de foto's worden genomen op te drijven, krijgen we een accurater beeld van de snelheid die de tennisbal op een gegeven moment heeft. De tijdsintervallen zijn nu immers korter zodat de bal minder van snelheid kan veranderen gedurende de intervallen en zodoende de gemiddelde snelheid een indicatie wordt van de ogenblikkelijke snelheid. De ogenblikkelijke snelheid wordt dus beter en beter benaderd door het tijdsinterval kleiner en kleiner en kleiner en kleiner\ldots te nemen. Echter, hoe kort het tijdsinterval ook is, de snelheid zal veranderen gedurende dat hele kleine tijdsinterval. Daarom, je raadde het misschien al, wordt de ogenblikkelijke snelheid gedefinieerd als de \emph{limiet} van de gemiddelde snelheid over een tijdsinterval waarbij we dat interval naar nul laten gaan. 

\begin{definition}
	
De \textbf{ogenblikkelijke snelheid}  is de afgeleide van de plaatsfunctie:
\[
v=\lim_{\Delta t\to 0}\frac{\Delta x}{\Delta t}=\lim_{t\to t_0}\frac{x(t)-x(t_0)}{t-t_0}=\frac{dx}{dt}
\]
De notatie met een accent $v(t)=x'(t)$ of $v=x'$ wordt op dezelfde manier als in de wiskunde gebruikt. 
De functie $v(t)$ geeft op elk moment $t$ de snelheid $v(t)$. 
\end{definition}


Grafisch kan je de afgeleide terugvinden als de richtingscoöefficiënt van de raaklijn. In een $x-t$ grafiek (de grafiek van de functie $x(t)$, $x$ in functie van $t$) vind je  de snelheid als de richtingscoöefficiënt van de raaklijn in het beschouwde punt. 

\begin{image}
\includegraphics[width=0.8\textwidth]{snelheid1D}

\end{image}

\begin{image}
\includegraphics[width=0.8\textwidth]{overzichtsnelheid1D}

\end{image}

\begin{remark}
	Het woord 'ogenblikkelijk' mag je weglaten. Wanneer we het over snelheid hebben, bedoelen we vanaf nu steeds ogenblikkelijke snelheid.
\end{remark}

% TODO de plaatsfunctie kan beter voor deze vraag --> eventueel naar tikz iets? 
\begin{exercise}
Hieronder staat opnieuw de plaatsfunctie \(x(t)\) van het autotje. Bepaal enkel met de grafiek en zonder te rekenen: 
\begin{itemize}
	\item Waar staat de auto stil? 
	\item Waar heeft de auto een positieve snelheid?
	\item waar is de snelheid negatief? 
	\item Op welk moment bewoog de auto het snelst? 
\end{itemize}

\begin{image}
	\includegraphics[width=0.45\textwidth]{Serway2p1(1)}
	$\qquad$   % hack 
	\includegraphics[width=0.45\textwidth]{Serway2p1(2)}
	\end{image}
	\captionof{figure}{Verschillende posities en de grafiek van de plaatsfunctie}
	
\end{exercise}

	
\end{document}