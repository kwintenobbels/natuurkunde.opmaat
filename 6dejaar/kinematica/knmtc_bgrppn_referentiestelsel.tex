\documentclass{ximera}
\input{../preamble}

\addPrintStyle{..}

\begin{document}
	\author{Bart Lambregs}
	\xmtitle{Het referentiestelsel}{}
    \xmsource\xmuitleg


Elke beweging wordt beschreven ten opzichte van een \textbf{referentiestelsel}, hierbij wordt de ruimte zo ingedeeld dat elke plaats in de ruimte eenduidig wordt bepaald door een aantal getallen die \textbf{coördinaten} genoemd worden. 

De keuze van het referentiestelsel is altijd relatief. Toch is het erg belangrijk om telkens duidelijk van waaruit een beweging beschreven wordt. Stel je voor dat je aan het studeren abent an een bureau en je pen op \textit{ooghoogte} voor je houdt, hoe 'hoog' bevindt je pen zich dan? Meet je dit vanaf je tafelblad, de vloer, het straatniveau, het aantal meters boven de zeespiegel, ...? 

Als je een vogel ziet vliegen kan je deze beweging op verschillende manieren beschrijven: de vogel kan \textit{stijgen} of een \textit{duikvlucht} nemen. De vogel kan \textit{omdraaien} of -indien het een kolibri is- misschien zelfs blijven hangen. Om deze bewegingen kwantitatief en nauwkeurig te bespreken kies je een referentiestelsel en coördinaatassen. Op die manier krijgen objecten in de ruimte een positie, snelheid, ... 

\begin{exercise}
De kolibri in onderstaande foto blijft ter plekke in de lucht hangen onder de bloem. Geen twee referentiestelsels waarin deze vogel \textbf{niet} stilstaat. 

\begin{image}
	\includegraphics[width=0.6\textwidth]{kolibri}
% Bron: https://unsplash.com/photos/green-and-black-humming-bird-flying-p-DDK9lOmmE 
\end{image}

\begin{oplossing}
Het referentiestelsel met de kern van de aarde als oorsprong. (De kolibri draait nu rond de as van de aarde...)
Het referentiestelsel met de zon als middelpunt (De kolibri draait nu ook rond de zon...)
\end{oplossing}

\end{exercise}


\end{document}
