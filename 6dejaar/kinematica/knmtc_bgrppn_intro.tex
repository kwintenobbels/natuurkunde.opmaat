\documentclass{ximera}
\input{../preamble}

\addPrintStyle{..}

\begin{document}
	\author{Bart Lambregs}
	\xmtitle{Inleiding kinematica}{}
    \xmsource\xmuitleg


De kinematica \textbf{beschrijft} bewegingen zonder een verklaring (bv. zwaarte\textit{kracht}) te geven die de oorzaak van de beweging verklaart. 
Als een appel van een boom valt, kan je veel vragen over deze valbeweging beantwoorden: 

\begin{itemize}
	\item Hoe lang duurt het voor de appel de grond raakt? 
	\item Wat is de snelheid van de appel na 1 seconde? 
	\item Wat is de snelheid van de appel op het moment dat hij de grond raakt? 
	\item ...
\end{itemize}

In dit hoofdstuk worden de basisbegrippen uit de kinematica geïntroduceerd. 

% Als je de valbeweging van een appel wilt beschrijven, heb je eerst een \textbf{referentiestelsel} nodig van waaruit je dit zal doen. In dit referentiestelsel een plaatsvector, snelheidsvector, versnellingsvector, ... toekennen en de grootte van deze vectoren beschrijven met vergelijkingen. 

% deze zin is enkel goed als ze het al kennen. 

% wat we niet doen: 

	
\end{document}
