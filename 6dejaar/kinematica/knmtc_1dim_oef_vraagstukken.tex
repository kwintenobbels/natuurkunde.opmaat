\documentclass{ximera}
\input{../../preamble}

\addPrintStyle{..}

\begin{document}
	\author{Bart Lambregs}
	\xmtitle{Oefeningen}{}
    \xmsource\xmuitleg

\begin{exercise}
    Een automobilist rijdt gedurende $1,5\rm\,u$ tegen $80\rm\,km/h$ en daarna gedurende dezelfde tijdsduur tegen
    $70\rm\,km/h$.
    \begin{enumerate}
        \item Wat is zijn gemiddelde snelheid?
        \item Met welke snelheid had hij moeten rijden om met een constante snelheid hetzelfde traject in dezelfde tijd af te leggen?
\end{enumerate}
\end{exercise}

\begin{exercise}
    Een automobilist legt $120\rm\,km$ af. De eerste helft van de weg legt hij af tegen $90\rm\,km/h$, de tweede helft tegen $120\rm\,km/h$. Wat is zijn gemiddelde snelheid?
\end{exercise}

\begin{exercise}
    Een fietser legt een bepaalde afstand af over een zekere tijd. Gedurende de eerste helft van de tijd houdt hij constant een snelheid $v_1$ aan, gedurende de tweede helft een snelheid $v_2$. Wat is zijn gemiddelde snelheid over het totale tijdsinterval?
    \begin{oplossing}
        $\overline{v}=\frac{v_1+v_2}{2}$
    \end{oplossing}
\end{exercise}

\begin{exercise}
    Een fietser legt een bepaalde afstand af over een zekere tijd. Gedurende de eerste helft van de af te leggen afstand houdt hij constant een snelheid $v_1$ aan, gedurende de tweede helft een snelheid $v_2$. Wat is zijn gemiddelde snelheid over het totale tijdsinterval? 
    \begin{oplossing}
        $\overline{v}=\frac{2v_1v_2}{v_1+v_2}$
    \end{oplossing}
\end{exercise}

\begin{exercise}
    Als je met de fiets heen en terug naar school rijdt en in het heengaan tegenwind en in het terugkeren rugwind hebt, compenseert dat dan mekaar precies?

    Stel om dit op te lossen dat de weg rechtlijnig is. Bereken je gemiddelde snelheid over het traject heen en terug en vergelijk die met de snelheid die je zonder wind zou halen. Neem aan dat je normaal $10~\rm km/h$ zou fietsen, maar door de wind win of verlies je $2~\rm km/h$.
    \begin{multipleChoice}
        \choice[correct]{Nee, je hebt netto een nadeel vanwege de tegenwind.}
        \choice{Nee, je hebt netto een voordeel vanwege de rugwind.}
        \choice{Ja, de afstand heen is de afstand terug, dus het is net alsof je helemaal geen wind had.}
    \end{multipleChoice}
\end{exercise}

\begin{exercise}
    Een bowlingbal die met een constante snelheid voort rolt, raakt de kegels aan het einde van een kegelbaan van $16,5\rm\,m$ lengte. De werper hoorde het geluid waarmee de bal op de kegels botst $2,5\rm\,s$ nadat hij de bal losliet. Welke snelheid had de bal? De snelheid van het geluid is $343\rm\,m/s$. 
    \begin{oplossing}
    $v_1=\frac{x_1}{t_2-\frac{x_1}{v_2}}=6,73\rm\,m/s$
    \end{oplossing}
\end{exercise}

\begin{exercise}
    Twee personen A en B voeren op dezelfde rechte en vanuit dezelfde beginstand een eenparige beweging uit. A vertrekt $100\rm\,s$ eerder dan B. Met een snelheid die dubbel zo groot is als die van A haalt B, op $400\rm\,m$ van het vertrekpunt, A in. Bereken beide snelheden en stel ze grafisch voor.
    %\begin{oplossing}
    %\item[gegeven]$x_0=1,00\cdot10^3\rm\,m$\newline$t_0=100\rm\,s$\newline$v_b=-2v_a$\newline$x=400\rm\,m$
    %\item[gevraagd]$v_a$, $v_b$
    %\item[oplossing]De bewegingsvergelijkingen voor A en B worden gegeven door:
    %\begin{eqnarray}
    %A:\qquad x&=&v_at \label{verg A}\\
    %B:\qquad x&=&x_0+v_b(t-t_0)\nonumber\\
    %&=&x_0-2v_a(t-t_0) \label{verg B}
    %\end{eqnarray}
    %De grafiek van beide functies ziet er als volgt uit:
    %\begin{image}}[h]
    %\centering\includegraphics[width=0.6\textwidth]{38p40}
    %\end{image}}
    %\newline
    %Als we voor $x$ de ontmoetingsplaats van $400\rm\,m$ nemen, hebben we twee vergelijkingen (\ref{verg A}), (\ref{verg B}) en twee onbekenden $t$, $v_a$. Dit kunnen we oplossen door een variabele te substitueren. We nemen de tijd, $(\ref{verg A})\Leftrightarrow t=\frac{x}{v_a}$ en substitueren deze in vergelijking (\ref{verg B}):
    %\begin{eqnarray*}
    %x&=&x_0-2v_a(t-t_0)\\
    %&=&x_0-2v_a\left(\frac{x}{v_a}-t_0\right)\\
    %&\Updownarrow&\\
    %v_a&=&\frac{3x-x_0}{2t_0}=1,0\rm\,m/s
    %\end{eqnarray*}
    %En de snelheid van B:
    %\begin{equation}
    % v_b=-2v_a=\frac{x_0-3x}{t_0}=-2,0\rm\,m/s
    %\end{equation} 
    %\end{oplossing}
\end{exercise}

\begin{exercise}
    Een vliegtuig moet minstens een snelheid van $108\rm\,km/h$ hebben om te kunnen opstijgen. Indien de schroeven aan het toestel een versnelling van $1,50\rm\,m/s^2$ geven, hoe lang moet de startbaan dan minstens zijn? 
    \begin{oplossing}
    % \item[gegeven]$v=30,0\rm\,m/s$\newline$a=1,50\rm\,m/s^2$
    % \item[gevraagd]$x$
    % \item[oplossing]
    Doordat we de versnelling van het vliegtuig kennen en de snelheid die het moet bereiken, kunnen we de tijd die het vliegtuig hiervoor nodig heeft, gemakkelijke berekenen met de formule $v=v_0+at$ voor de snelheid van een EVRB:
    \begin{eqnarray*}
        t=\frac{v}{a}
    \end{eqnarray*}
    De afstand die in deze tijd wordt afgelegd, kunnen we berekenen doordat we de gemiddelde snelheid kennen\footnote{De benodigde afstand kunnen we evenzeer berekenen met de formule $x=x_0+v_0+\frac{1}{2}at^2$ door de tijd in te vullen.}:
    \begin{eqnarray*}
    x&=&\frac{v_0+v}{2}\cdot t\\
    &=&\frac{v}{2}\cdot\frac{v}{a}\\
    &=&\frac{v^2}{2a}
    \end{eqnarray*}
    De startbaan moet dus minstens $300\rm\,m$ lang zijn.
    \end{oplossing}
\end{exercise}

\begin{exercise}
    Op een bevroren meer komt een glijdende hockeyschijf na $200\rm\,m$ tot stilstand. Als zijn initi\"ele snelheid $3,00\rm\,m/s$ was, bepaal dan
    \begin{enumerate}
        \item de versnelling in de veronderstelling dat deze constant is,
        \item de tijd die de schijf nodig heeft om tot stilstand te komen.
    \end{enumerate}
    \begin{oplossing}
        $a=\frac{v_0^2}{2x}=0,0225\rm\,m/s^2$; $t=\frac{2x}{v_0}=133,33\rm\,s$
    \end{oplossing}
\end{exercise}

\begin{exercise}
    Een bootje vaart met een snelheid van $36,0\rm\,km/h$ een eerste tijdopnemer voorbij en drijft daarna eenparig zijn snelheid op. Na $20,0\rm\,s$ komt het voorbij een tweede tijdopnemer met een snelheid van $90,0\rm\,km/h$. Bereken de versnelling van het bootje en de afstand tussen beide tijdopnemers.
    \begin{oplossing}
        $a=\frac{v-v_0}{t-t_0}=0,750\rm\,m/s^2$, $x-x_0=\left(\frac{v_0+v}{2}\right)(t-t_0)=350\rm\,m$
    \end{oplossing}
\end{exercise}

\begin{exercise}
    Een auto begint te remmen als hij zich $35\rm\,m$ van een hindernis bevindt. Zijn snelheid op dat moment is $54\rm\,km/h$. Na $4,0\rm\,s$ botst hij tegen de hindernis. Bereken de snelheid waarmee hij de hindernis raakt en zijn constante versnelling gedurende de remweg.
\begin{oplossing}
$a=\frac{2(x-v_0t)}{t^2}=-3,125\rm\,m/s^2$, $v=\frac{2x}{t}-v_0=2,5\rm\,m/s$
\end{oplossing}
\begin{oplossing}
Uit de plaatsfunctie kunnen we de versnelling halen:
\begin{eqnarray*}
x&=&v_0t+\frac{1}{2}at^2\\
&\Updownarrow&\\
a&=&\frac{2x-2v_0t}{t^2}
\end{eqnarray*}
Substitutie van de versnelling in de snelheidsfunctie levert:\footnote{Een andere (snellere) mogelijkheid is de snelheid uit $x=\frac{v_0+v}{2}t$ halen.}
\begin{eqnarray*}
v&=&v_0+at\\
&=&v_0+\left(\frac{2x-2v_0t}{t^2}\right)t\\
&=&\frac{2x}{t}-v_0\\
&=&2,5\rm\,m/s
\end{eqnarray*}
\end{oplossing}
\end{exercise}

\begin{exercise}
    Twee fietsers vertrekken gelijktijdig om
een afstand van $200\rm\,m$ af te leggen. De eerste rijdt met een
constante snelheid van $4,0\rm\,m/s$, terwijl de tweede vertrekt met
een snelheid van $1,00\rm\,m/s$ en de afstand van $200\rm\,m$ met
een EVRB met een versnelling van $0,20\rm\,m/s^2$ aflegt. Waar zal
de tweede fietser de eerste inhalen en wanneer?
\begin{oplossing}
    $t=\frac{2(v_a-v_{b,0})}{a}=30\rm\,s$,
    $x=v_at=\frac{2v_a(v_a-v_{b,0})}{a}=120\rm\,m$
\end{oplossing}
\end{exercise}

\begin{exercise}
        Een trein verlaat het station a en rijdt
naar het station b, op $15,0~\rm km$ van a gelegen. De eerste
$1000~\rm m$ worden afgelegd met een EVRB en de verkregen snelheid
is $72,0~\rm km/h$. Die snelheid blijft constant tot op $250~\rm m$
van b. Hier begint de trein te vertragen. Wanneer komt hij in
station b toe? Maak de $v(t)$-grafiek.
\end{exercise}

\begin{exercise}
    Een deeltje beschrijft een eendimensionale beweging op de $x$-as. De positie als functie van de tijd is hiernaast weergegeven in een $x(t)$-diagram. Duid de onderstaande grafiek aan die het best het verloop weergeeft van de snelheidscomponent $v$ van dat deeltje als functie van de tijd. 
    \begin{image}
        \includegraphics[width=\textwidth]{snelheidsverloop_o}
    \end{image}
    \begin{image}
        \includegraphics[width=0.93\textwidth]{snelheidsverloop}
    \end{image}
\end{exercise}

\begin{exercise}
Een deeltje beweegt in de zin van de $x$-as. De nevenstaande grafiek geeft aan hoe de grootte van de snelheid verandert als functie van de tijd.
\begin{enumerate}
\item De afstand afgelegd na $15\rm\,s$ bedraagt:
\wordChoice{
    \choice{$30\rm\,m$}
    \choice{$120\rm\,m$}
    \choice{$150\rm\,m$}
    \choice{$240\rm\,m$}
}

\item Na $30\rm\,s$ heeft het deeltje een welbepaalde afstand afgelegd. Hoe groot zou de constante snelheid van het deeltje moeten zijn om in $30\rm\,s$ dezelfde afstand af te leggen?

\wordChoice{
    \choice{$0,0\rm\,m/s$}
    \choice{$6,0\rm\,m/s$}
    \choice{$8,0\rm\,m/s$}
    \choice{$12\rm\,m/s$}
}
\end{enumerate}
\begin{image}
    \includegraphics[width=\textwidth]{snelheidsverloop_2_o}
\end{image}
\begin{enumerate}
\setcounter{enumii}{2}
\item Het verloop van de versnellingscomponent van het deeltje wordt kwalitatief voorgesteld op figuur:
\begin{image}
    \includegraphics[width=0.93\textwidth]{snelheidsverloop_2}
\end{image}
\end{enumerate}
\end{exercise}



\end{document}
