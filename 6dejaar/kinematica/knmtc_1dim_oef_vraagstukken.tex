\documentclass{ximera}
\input{../../preamble}

\addPrintStyle{..}

\begin{document}
	\author{Bart Lambregs}
	\xmtitle{Vraagstukken}{}
    \xmsource\xmuitleg

\begin{exercise}
    Een automobilist rijdt gedurende \SI{1,5}{uur} tegen \SI{80}{km/h} en daarna gedurende dezelfde tijdsduur tegen
    \SI{70}{km/h}.
    \begin{enumerate}
        \item Wat is zijn gemiddelde snelheid?
        \item Met welke snelheid had hij moeten rijden om met een constante snelheid hetzelfde traject in dezelfde tijd af te leggen?
    \end{enumerate}
    \begin{exercise}
        Een fietser legt een bepaalde afstand af over een zekere tijd. Gedurende de eerste helft van de tijd houdt hij constant een snelheid $v_1$ aan, gedurende de tweede helft een snelheid $v_2$. Wat is zijn gemiddelde snelheid over het totale tijdsinterval?
        \begin{oplossing}
            $\overline{v}=\frac{v_1+v_2}{2}$
        \end{oplossing}
\end{exercise}
\end{exercise}

\begin{exercise}
    Een automobilist legt \SI{120}{km} af. De eerste helft van de weg legt hij af tegen \SI{90}{km/h}, de tweede helft tegen \SI{120}{km/h}. Wat is zijn gemiddelde snelheid?
    \begin{exercise}
        Een fietser legt een bepaalde afstand af over een zekere tijd. Gedurende de eerste helft van de af te leggen afstand houdt hij constant een snelheid $v_1$ aan, gedurende de tweede helft een snelheid $v_2$. Wat is zijn gemiddelde snelheid over het totale tijdsinterval? 
    \begin{oplossing}
        $\overline{v}=\frac{2v_1v_2}{v_1+v_2}$
    \end{oplossing}
\end{exercise}
\end{exercise}

\begin{exercise}
    Als je met de fiets heen en terug naar school rijdt en in het heengaan tegenwind en in het terugkeren rugwind hebt, compenseert dat dan mekaar precies?

    Stel om dit op te lossen dat de weg rechtlijnig is. Bereken je gemiddelde snelheid over het traject heen en terug en vergelijk die met de snelheid die je zonder wind zou halen. Neem aan dat je normaal \SI{10}{km/h} zou fietsen, maar door de wind win of verlies je \SI{2}{km/h}.
    \begin{multipleChoice}
        \choice[correct]{Nee, je hebt netto een nadeel vanwege de tegenwind.}
        \choice{Nee, je hebt netto een voordeel vanwege de rugwind.}
        \choice{Ja, de afstand heen is de afstand terug, dus het is net alsof je helemaal geen wind had.}
    \end{multipleChoice}
\end{exercise}

\begin{exercise}
    Een bowlingbal die met een constante snelheid voort rolt, raakt de kegels aan het einde van een kegelbaan van \SI{16,5}{m} lengte. De werper hoorde het geluid waarmee de bal op de kegels botst \SI{2,5}{s} nadat hij de bal losliet. Welke snelheid had de bal? De snelheid van het geluid is \SI{343}{m/s}. 
    \begin{oplossing}
        $v_1=\frac{x_1}{t_2-\frac{x_1}{v_2}}=\SI{6,73}{m/s}$
    \end{oplossing}
\end{exercise}

\begin{exercise}
    Een vliegtuig moet minstens een snelheid van \SI{108}{km/h} hebben om te kunnen opstijgen. Indien de schroeven aan het toestel een versnelling van \SI{1,50}{m/s^2} geven, hoe lang moet de startbaan dan minstens zijn? 
    \begin{oplossing}
    % \item[gegeven]$v=30,0\rm\,m/s$\newline$a=1,50\rm\,m/s^2$
    % \item[gevraagd]$x$
    % \item[oplossing]
    Doordat we de versnelling van het vliegtuig kennen en de snelheid die het moet bereiken, kunnen we de tijd die het vliegtuig hiervoor nodig heeft, gemakkelijke berekenen met de formule $v=v_0+at$ voor de snelheid van een EVRB:
    \begin{eqnarray*}
        t=\frac{v}{a}
    \end{eqnarray*}
    De afstand die in deze tijd wordt afgelegd, kunnen we berekenen doordat we de gemiddelde snelheid kennen\footnote{De benodigde afstand kunnen we evenzeer berekenen met de formule $x=x_0+v_0+\frac{1}{2}at^2$ door de tijd in te vullen.}:
    \begin{eqnarray*}
    x&=&\frac{v_0+v}{2}\cdot t\\
    &=&\frac{v}{2}\cdot\frac{v}{a}\\
    &=&\frac{v^2}{2a}
    \end{eqnarray*}
    De startbaan moet dus minstens \SI{300}{m} lang zijn.
    \end{oplossing}
\end{exercise}

\begin{exercise}
    Op een bevroren meer komt een glijdende hockeyschijf na \SI{200}{m} tot stilstand. Als zijn initi\"ele snelheid \SI{3,00}{m/s} was, bepaal dan
    \begin{enumerate}
        \item de versnelling in de veronderstelling dat deze constant is,
        \item de tijd die de schijf nodig heeft om tot stilstand te komen.
    \end{enumerate}
    \begin{oplossing}
        $a=\frac{v_0^2}{2x}=0,0225\rm\,m/s^2$; $t=\frac{2x}{v_0}=\SI{133,33}{s}$
    \end{oplossing}
\end{exercise}

\begin{exercise}
    Een bootje vaart met een snelheid van \SI{36,0}{km/h} een eerste tijdopnemer voorbij en drijft daarna eenparig zijn snelheid op. Na \SI{20,0}{s} komt het voorbij een tweede tijdopnemer met een snelheid van \SI{90,0}{km/h}. Bereken de versnelling van het bootje en de afstand tussen beide tijdopnemers.
    \begin{oplossing}
        $a=\frac{v-v_0}{t-t_0}=0,750\rm\,m/s^2$, $x-x_0=\left(\frac{v_0+v}{2}\right)(t-t_0)=\SI{350}{m}$
    \end{oplossing}
\end{exercise}

\begin{exercise}
    Een auto begint te remmen als hij zich \SI{35}{m} van een hindernis bevindt. Zijn snelheid op dat moment is \SI{54}{km/h}. Na \SI{4,0}{s} botst hij tegen de hindernis. Bereken de snelheid waarmee hij de hindernis raakt en zijn constante versnelling gedurende de remweg.
    \begin{oplossing}
    Uit de plaatsfunctie $x=v_0t+\frac{1}{2}at^2$ kunnen we de versnelling halen:
    \begin{equation*}
        a=\frac{2x-2v_0t}{t^2}=\SI{-3,125}{m/s^2}
    \end{equation*}
    Substitutie van de versnelling in de snelheidsfunctie levert:
    \begin{eqnarray*}
        v&=&v_0+at\\
        &=&v_0+\left(\frac{2x-2v_0t}{t^2}\right)t\\
        &=&\frac{2x}{t}-v_0\\
        &=&\SI{2,5}{m/s}
    \end{eqnarray*}
    Een andere (snellere) mogelijkheid om de snelheid te vinden is die te halen uit $x=\frac{v_0+v}{2}t$.
    \end{oplossing}
\end{exercise}

\begin{exercise}
    Twee fietsers vertrekken gelijktijdig om een afstand van \SI{200}{m} af te leggen. De eerste rijdt met een constante snelheid van \SI{4,0}{m/s}, terwijl de tweede vertrekt met een snelheid van \SI{1,00}{m/s} en de afstand van \SI{200}{m} met een EVRB met een versnelling van \SI{0,20}{m/s^2} aflegt. Waar zal de tweede fietser de eerste inhalen en wanneer?
    \begin{oplossing}
        $t=\frac{2(v_a-v_{b,0})}{a}=\SI{30}{s}$,
        $x=v_at=\frac{2v_a(v_a-v_{b,0})}{a}=\SI{120}{m}$
    \end{oplossing}
\end{exercise}

\begin{exercise}
        Een trein verlaat het station a en rijdt naar het station b, op \SI{15}{km} van a gelegen. De eerste \SI{1000}{m} worden afgelegd met een EVRB en de verkregen snelheid is $72,0~\rm km/h$. Die snelheid blijft constant tot op $250~\rm m$ van b. Hier begint de trein te vertragen. Wanneer komt hij in station b toe? Maak de $v(t)$-grafiek.
\end{exercise}

\begin{exercise}
    Een deeltje beschrijft een eendimensionale beweging op de $x$-as. De positie als functie van de tijd is hiernaast weergegeven in een $x(t)$-diagram. Duid de onderstaande grafiek aan die het best het verloop weergeeft van de snelheidscomponent $v$ van dat deeltje als functie van de tijd. 
    \begin{image}
        \includegraphics[width=\textwidth]{snelheidsverloop_o}
    \end{image}
    \begin{image}
        \includegraphics[width=0.93\textwidth]{snelheidsverloop}
    \end{image}
\end{exercise}

\begin{exercise}
Een deeltje beweegt in de zin van de $x$-as. De nevenstaande grafiek geeft aan hoe de grootte van de snelheid verandert als functie van de tijd.
\begin{enumerate}
\item De afstand afgelegd na \SI{15}{s} bedraagt:
\wordChoice{
    \choice{\SI{30}{m}}
    \choice{\SI{120}{m}}
    \choice{\SI{150}{m}}
    \choice{\SI{240}{m}}
}

\item Na \SI{30}{s} heeft het deeltje een welbepaalde afstand afgelegd. Hoe groot zou de constante snelheid van het deeltje moeten zijn om in \SI{30}{s} dezelfde afstand af te leggen?

\wordChoice{
    \choice{\SI{0,0}{m/s}}
    \choice{\SI{6,0}{m/s}}
    \choice{\SI{8,0}{m/s}}
    \choice{\SI{12}{m/s}}
}
\end{enumerate}
\begin{image}
    \includegraphics[width=\textwidth]{snelheidsverloop_2_o}
\end{image}
\begin{enumerate}
\setcounter{enumii}{2}
\item Het verloop van de versnellingscomponent van het deeltje wordt kwalitatief voorgesteld op figuur:
\begin{image}
    \includegraphics[width=0.93\textwidth]{snelheidsverloop_2}
\end{image}
\end{enumerate}
\end{exercise}

% Bron Serway 4yh edition oefening 77 chapter 2
\begin{exercise}
    (III) Maggie en Jennifer lopen de \SI{100}{m}. Beiden doen ze er exact 10,2 se\-conden over. Met een eenparige versnelling bereikt Maggie na \SI{2}{s} haar maximale snelheid, Jennifer doet dat na \SI{3}{s}. Hun maximale snelheden houden ze aan voor de rest van de wedstrijd.
    \begin{enumerate}
        \item Wat zijn hun maximale snelheden?
        \item Wat is de versnelling van iedere sprinter?
        \item Wie heeft er voorsprong na \SI{6}{s}, en hoeveel?
    \end{enumerate}
    \begin{oplossing}
        $v_1=\frac{2x_2}{2t_2-t_1}$;
        $a=\frac{2x_2}{(2t_2-t_1)t_1}$;
        $x_M-x_J=\frac{2x_2}{2t_2-t_{1,M}}(t-\frac{t_{1,M}}{2})-\frac{2x_2}{2t_2-t_{1,J}}(t-\frac{t_{1,J}}{2})$
        
        Snelheidsgrafiek van Jennifer:
        \begin{image}
            \includegraphics[width=0.5\textwidth]{sprinter}
        \end{image}
    \end{oplossing}
\end{exercise}

\begin{exercise}
    Welke afstand wordt er door een bungeejumper na een vrije val van \SI{2,5}{s} afgelegd?
    \begin{oplossing}
        $x=\frac{1}{2}gt^2=31\rm\,m$
    \end{oplossing}
\end{exercise}

\begin{exercise}
    Een auto die \SI{90}{km/h} rijdt, ligt \SI{100}{m} achter op een vrachtwagen die \SI{75}{km/h} rijdt. Hoeveel tijd kost het de auto om de vrachtwagen in te halen? 
    \begin{oplossing}
        $t=\frac{x_0}{v_a-v_v}=24\rm\,s$
    \end{oplossing}
\end{exercise}

\begin{exercise}
    De snelheid van een trein verandert eenparig in 2 minuten van \SI{20}{km/h} tot \SI{30}{km/h}. De trein rijdt gedurende die tijd over een rechte spoorlijn.
    \begin{enumerate}
        \item Bepaal de versnelling.
        \item Bepaal de afstand die de trein heeft afgelegd gedurende deze 2 minuten.
    \end{enumerate}
\end{exercise}

\begin{exercise}
    Een auto trekt in \SI{5,0}{s} op van \SI{10}{m/s} naar \SI{25}{m/s}. Wat was de versnelling in de veronderstelling dat de auto een EVRB ondergaat? Welke afstand legde de auto in deze periode af?
    \begin{oplossing}
        $a=\frac{v-v_0}{t-t_0}=3\rm\,m/s$,
        $x-x_0=\left(\frac{v_0+v}{2}\right)(t-t_0)=87,5\rm\,m$
    \end{oplossing}
\end{exercise}

\begin{exercise}
    Bij het katapulteren van vliegtuigen wordt een startbaan van \SI{25,0}{m} gebruikt, die door het vliegtuig eenparig versneld in \SI{1,00}{s} wordt doorlopen. Zoek zijn versnelling en de snelheid waarmee het de baan verlaat.
\end{exercise}

\begin{exercise}
    Een auto trekt op tot \SI{100}{km/h} in \SI{6,0}{s}. Als hij dat doet op een rechte baan met constante versnelling, welke afstand is er dan hiervoor nodig?
    \begin{oplossing}
        $a=\frac{v}{t}=4,63\rm\,m/s^2$, $x=\frac{vt}{2}=83,3\rm\,m$
    \end{oplossing}
\end{exercise}

\begin{exercise}
    Een auto vertrekt vanuit rust en bereikt na \SI{3,0}{km} een snelheid van \SI{450}{km/h} We onderstellen de versnelling constant en de baan recht. Bereken de versnelling en de tijd, nodig om die \SI{3,0}{km} af te leggen.
    \begin{oplossing}
    % \footnote{$t=\frac{2x}{v}=48\rm\,s$, $a=\frac{v^2}{2x}=2,6\rm\,m/s^2$}
    % \item[gegeven]$x=3000\rm\,m$\newline $v=125\rm\,m/s$
    % \item[gevraagd]$a$, $t$
    % \item[oplossing]
    Omdat voor een EVRB de gemiddelde snelheid gegeven wordt door $\overline{v}=\frac{v_0+v}{2}$ en we de afgelegde afstand kennen, kunnen we de benodigde tijd gemakkelijk vinden. We kiezen $t_0=0$, $x_0=0$. De beginsnelheid is nul zodat:
    \begin{eqnarray*}
    \Delta x &=& \overline{v}\Delta t \\
    &\Downarrow & \\
    t &=& \frac{x}{\left(\frac{v}{2}\right)} = \frac{2x}{v}
    \end{eqnarray*}
    Invullen van de gegevens levert een tijd van \SI{48}{s}. Met het formuletje voor de snelheid vinden we de versnelling door de tijd te substitueren:
    \begin{eqnarray*}
    v &=& at \\
    &\Updownarrow&\\
    a &=& \frac{v}{t}=\frac{v}{\left(\frac{2x}{v}\right)}\\
    &=& \frac{v^2}{2x}
    \end{eqnarray*}
    Invullen van de gegeven grootheden levert een versnelling van \SI{2,6}{m/s^2}.
    \end{oplossing}
\end{exercise}

\begin{exercise}
    Een vliegtuig landt met een snelheid van \SI{100}{m/s}. Op de ladingsbaan heeft het een vertraging van \SI{5,0}{m/s^2}. Welke afstand heeft het vliegtuig nodig om tot stilstand te komen?
\end{exercise}

\begin{exercise}
    Een trein vertrekt uit een station en rijdt met een eenparig versnelde beweging waarvan de versnelling \SI{0,50}{m/s^2} bedraagt. Hoe groot is de afstand die de trein heeft afgelegd als zijn snelheid \SI{72,0}{km/h} bedraagt?
\end{exercise}

\begin{exercise}
    Twee personen A en B voeren op dezelfde rechte en vanuit dezelfde beginstand een eenparige beweging uit. A vertrekt \SI{100}{s} eerder dan B. Met een snelheid die dubbel zo groot is als die van A haalt B, op \SI{400}{m} van het vertrekpunt, A in. Bereken beide snelheden en stel ze grafisch voor.
    %\begin{oplossing}
    %\item[gegeven]$x_0=1,00\cdot10^3\rm\,m$\newline$t_0=100\rm\,s$\newline$v_b=-2v_a$\newline$x=400\rm\,m$
    %\item[gevraagd]$v_a$, $v_b$
    %\item[oplossing]De bewegingsvergelijkingen voor A en B worden gegeven door:
    %\begin{eqnarray}
    %A:\qquad x&=&v_at \label{verg A}\\
    %B:\qquad x&=&x_0+v_b(t-t_0)\nonumber\\
    %&=&x_0-2v_a(t-t_0) \label{verg B}
    %\end{eqnarray}
    %De grafiek van beide functies ziet er als volgt uit:
    %\begin{image}}[h]
    %\centering\includegraphics[width=0.6\textwidth]{38p40}
    %\end{image}}
    %\newline
    %Als we voor $x$ de ontmoetingsplaats van \SI{400}{m} nemen, hebben we twee vergelijkingen (\ref{verg A}), (\ref{verg B}) en twee onbekenden $t$, $v_a$. Dit kunnen we oplossen door een variabele te substitueren. We nemen de tijd, $(\ref{verg A})\Leftrightarrow t=\frac{x}{v_a}$ en substitueren deze in vergelijking (\ref{verg B}):
    %\begin{eqnarray*}
    %x&=&x_0-2v_a(t-t_0)\\
    %&=&x_0-2v_a\left(\frac{x}{v_a}-t_0\right)\\
    %&\Updownarrow&\\
    %v_a&=&\frac{3x-x_0}{2t_0}=1,0\rm\,m/s
    %\end{eqnarray*}
    %En de snelheid van B:
    %\begin{equation}
    % v_b=-2v_a=\frac{x_0-3x}{t_0}=-2,0\rm\,m/s
    %\end{equation} 
    %\end{oplossing}
\end{exercise}




\end{document}
