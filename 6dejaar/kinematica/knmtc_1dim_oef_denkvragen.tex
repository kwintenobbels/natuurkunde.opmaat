\documentclass{ximera}
\input{../../preamble}

\addPrintStyle{..}

\begin{document}
	\author{Bart Lambregs}
	\xmtitle{Denkvragen}{}
    \xmsource\xmuitleg

\begin{exercise}
	Als de grootte van de snelheid van een voorwerp toeneemt, neemt de versnelling dan noodzakelijkerwijs ook toe? Motiveer je antwoord.
\end{exercise}

\begin{exercise}
	Kan de gemiddelde snelheid van een deeltje over een gegeven tijdsinterval gelijk zijn aan nul, terwijl de grootte van de snelheid over een kortere tijdsduur verschillend is van nul? Verklaar je antwoord.
\end{exercise}

\begin{exercise}
	Geef een voorbeeld waarin zowel de snelheid als de versnelling negatief zijn.
\end{exercise}

\begin{exercise}
	Wanneer een voorwerp zich met een constante snelheid verplaatst, verschilt de gemiddelde snelheid over een willekeurig tijdsinterval dan van de momentane snelheid op een willekeurig moment?
\end{exercise}

\begin{exercise}
	Vanaf een klif laat men vanop dezelfde hoogte twee identieke bollen vallen. Men laat de tweede bol \'e\'en seconde later vallen dan de eerste. De luchtwrijving is \textit{niet} te verwaarlozen. Dan
	\begin{multipleChoice}
		\choice{zal de tweede bol iets later dan \'e\'en seconde na de eerste neerkomen.}
		\choice{zal de tweede bol iets vroeger dan \'e\'en seconde na de eerste neerkomen.}
		\choice[correct]{zal de tweede bol exact \'e\'en seconde na de eerste neerkomen.}
		\choice{kunnen we hieromtrent geen uitspraak doen bij gebrek aan gegevens.}
	\end{multipleChoice}
	\begin{oplossing}
		Voor beide bollen is de omstandigheid waarin ze vallen gelijk.
	\end{oplossing}
\end{exercise}

\begin{exercise}
	Een puntmassa beweegt volgens de plaatsfunctie
	\[
	x(t)=t^3-3t^2-10t
	\]
	Bereken haar snelheidscomponent telkens als ze het vertrekpunt passeert. Hoe groot is dan de versnellingscomponent?
	\begin{oplossing}
		$x=t(t-5)(t+2)$
	\end{oplossing}
\end{exercise}

\begin{exercise}
    Een puntmassa voert een eenparig veranderlijke rechtlijnige beweging uit over het tijdsinterval [\SI{0}{s},\SI{2}{s}] met beginsnelheid en beginpositie van respectievelijk \SI{0}{m/s} en \SI{0}{m}. De versnelling is \SI{3}{m/s^2}. Waarom kan je onmiddellijk stellen dat als in het tijdsinterval de tijd half om is, de puntmassa nog niet halfweg is?
    \begin{enumerate}
        %\item Breng, zoals de opgelegde werkwijze bij het oplossen van vraagstukken vereist, zowel het gegeven als het gevraagde (hier het te bewijzen) in symbolen.
        \item Leg in woorden uit op welke manier je onmiddellijk kan inzien dat het te bewijzen juist is.
        \item Controleer het te bewijzen via de numerieke waarden van dit vraagstuk.
        \item Geef nu het bewijs.
    \end{enumerate}
\end{exercise}

\end{document}
