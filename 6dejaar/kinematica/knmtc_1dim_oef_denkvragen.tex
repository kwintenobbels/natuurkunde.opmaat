\documentclass{ximera}
\input{../../preamble}

\addPrintStyle{..}

\begin{document}
	\author{Bart Lambregs}
	\xmtitle{Denkvragen}{}
    \xmsource\xmuitleg

\begin{exercise}
	Als de grootte van de snelheid van een voorwerp toeneemt, neemt de versnelling dan noodzakelijkerwijs ook toe? Motiveer je antwoord.
\end{exercise}

\begin{exercise}
	Kan de gemiddelde snelheid van een deeltje over een gegeven tijdsinterval gelijk zijn aan nul, terwijl de grootte van de snelheid over een kortere tijdsduur verschillend is van nul? Verklaar je antwoord.
\end{exercise}

\begin{exercise}
	Geef een voorbeeld waarin zowel de snelheids- als de versnellingscomponent negatief zijn.
\end{exercise}

\begin{exercise}
	Wanneer een voorwerp zich met een constante snelheid verplaatst, verschilt de gemiddelde snelheid over een willekeurig tijdsinterval dan van de momentane snelheid op een willekeurig moment?
\end{exercise}

\begin{exercise}
	Een puntmassa beweegt volgens de plaatsfunctie
	\[
	x(t)=t^3-3t^2-10t
	\]
	Bereken haar snelheidscomponent telkens als ze het vertrekpunt passeert. Hoe groot is dan de versnellingscomponent?
	\begin{oplossing}
		$x=t(t-5)(t+2)$
	\end{oplossing}
\end{exercise}

\begin{exercise}
    Een puntmassa voert een eenparig veranderlijke rechtlijnige beweging uit over het tijdsinterval [\SI{0}{s},\SI{2}{s}] met beginsnelheid en beginpositie van respectievelijk \SI{0}{m/s} en \SI{0}{m}. De versnelling is \SI{3}{m/s^2}. Waarom kan je onmiddellijk stellen dat als in het tijdsinterval de tijd half om is, de puntmassa nog niet halfweg is?
    \begin{enumerate}
        %\item Breng, zoals de opgelegde werkwijze bij het oplossen van vraagstukken vereist, zowel het gegeven als het gevraagde (hier het te bewijzen) in symbolen.
        \item Leg in woorden uit op welke manier je onmiddellijk kan inzien dat het te bewijzen juist is.
        \item Controleer het te bewijzen via de numerieke waarden van dit vraagstuk.
        \item Geef nu het bewijs.
    \end{enumerate}
\end{exercise}

% Denkvragen over grafische voorstellingen

\begin{exercise}
	Teken de overeenkomstige $x(t)$- en $a(t)$-grafiek bij de gegeven $v(t)$-grafiek. Ga ervan uit dat $x_0=\SI{0}{m}$.
	\begin{image}[0.3\textwidth]
		\includegraphics[width=\textwidth]{v(t)}
	\end{image}
\end{exercise}

\begin{exercise}
	De snelheid van een deeltje voldoet aan $v=at$ waarin $a$ constant en negatief is. De plaats van het deeltje wordt voorgesteld door $x$. Aangenomen wordt dat $x=0\rm\,m$ op het ogenblik $t=0\rm\,s$. Welke grafiek geeft het juiste verloop van $x(t)$?
	\begin{multipleChoice}
		\choice{% !TEX root = ../oefeningen_fys6.tex
\begin{tikzpicture}[line cap=round,line join=round,>=triangle 45,x=1.0cm,y=1.0cm, scale=0.8]
\draw[->,color=black] (-0.26126256871071957,0.) -- (3.6896534027248262,0.);
\foreach \x in {,0.5,1.,1.5,2.,2.5,3.,3.5}
\draw[shift={(\x,0)},color=black] (0pt,-2pt);
\draw[->,color=black] (0.,-0.2694355115461891) -- (0.,3.9273529130473617);
\foreach \y in {,0.5,1.,1.5,2.,2.5,3.,3.5}
%\draw[shift={(0,\y)},color=black] (2pt,0pt) -- (-2pt,0pt);
\clip(-0.26126256871071957,-0.2694355115461891) rectangle (3.6896534027248262,3.9273529130473617);
\draw[line width=1.1pt] (0.006830113675660519,2.0136135768984986) -- (0.013660227351321039,2.0271338528913523);
\draw[line width=1.1pt] (0.013660227351321039,2.0271338528913523) -- (0.020490341026981558,2.0405608279785614);
\draw[line width=1.1pt] (0.020490341026981558,2.0405608279785614) -- (0.027320454702642077,2.0538945021601247);
\draw[line width=1.1pt] (0.027320454702642077,2.0538945021601247) -- (0.0341505683783026,2.067134875436044);
\draw[line width=1.1pt] (0.0341505683783026,2.067134875436044) -- (0.040980682053963116,2.080281947806318);
\draw[line width=1.1pt] (0.040980682053963116,2.080281947806318) -- (0.047810795729623635,2.0933357192709474);
\draw[line width=1.1pt] (0.047810795729623635,2.0933357192709474) -- (0.054640909405284155,2.106296189829932);
\draw[line width=1.1pt] (0.054640909405284155,2.106296189829932) -- (0.061471023080944674,2.1191633594832715);
\draw[line width=1.1pt] (0.061471023080944674,2.1191633594832715) -- (0.0683011367566052,2.131937228230966);
\draw[line width=1.1pt] (0.0683011367566052,2.131937228230966) -- (0.0751312504322657,2.1446177960730153);
\draw[line width=1.1pt] (0.0751312504322657,2.1446177960730153) -- (0.08196136410792623,2.15720506300942);
\draw[line width=1.1pt] (0.08196136410792623,2.15720506300942) -- (0.08879147778358676,2.1696990290401805);
\draw[line width=1.1pt] (0.08879147778358676,2.1696990290401805) -- (0.09562159145924728,2.1820996941652955);
\draw[line width=1.1pt] (0.09562159145924728,2.1820996941652955) -- (0.10245170513490781,2.1944070583847655);
\draw[line width=1.1pt] (0.10245170513490781,2.1944070583847655) -- (0.10928181881056834,2.206621121698591);
\draw[line width=1.1pt] (0.10928181881056834,2.206621121698591) -- (0.11611193248622886,2.218741884106771);
\draw[line width=1.1pt] (0.11611193248622886,2.218741884106771) -- (0.12294204616188939,2.2307693456093065);
\draw[line width=1.1pt] (0.12294204616188939,2.2307693456093065) -- (0.12977215983754992,2.242703506206197);
\draw[line width=1.1pt] (0.12977215983754992,2.242703506206197) -- (0.13660227351321044,2.2545443658974427);
\draw[line width=1.1pt] (0.13660227351321044,2.2545443658974427) -- (0.14343238718887097,2.2662919246830437);
\draw[line width=1.1pt] (0.14343238718887097,2.2662919246830437) -- (0.1502625008645315,2.277946182563);
\draw[line width=1.1pt] (0.1502625008645315,2.277946182563) -- (0.15709261454019202,2.289507139537311);
\draw[line width=1.1pt] (0.15709261454019202,2.289507139537311) -- (0.16392272821585255,2.300974795605977);
\draw[line width=1.1pt] (0.16392272821585255,2.300974795605977) -- (0.17075284189151307,2.3123491507689984);
\draw[line width=1.1pt] (0.17075284189151307,2.3123491507689984) -- (0.1775829555671736,2.323630205026374);
\draw[line width=1.1pt] (0.1775829555671736,2.323630205026374) -- (0.18441306924283413,2.334817958378106);
\draw[line width=1.1pt] (0.18441306924283413,2.334817958378106) -- (0.19124318291849465,2.3459124108241927);
\draw[line width=1.1pt] (0.19124318291849465,2.3459124108241927) -- (0.19807329659415518,2.356913562364634);
\draw[line width=1.1pt] (0.19807329659415518,2.356913562364634) -- (0.2049034102698157,2.3678214129994313);
\draw[line width=1.1pt] (0.2049034102698157,2.3678214129994313) -- (0.21173352394547623,2.378635962728583);
\draw[line width=1.1pt] (0.21173352394547623,2.378635962728583) -- (0.21856363762113676,2.38935721155209);
\draw[line width=1.1pt] (0.21856363762113676,2.38935721155209) -- (0.22539375129679728,2.399985159469952);
\draw[line width=1.1pt] (0.22539375129679728,2.399985159469952) -- (0.2322238649724578,2.4105198064821693);
\draw[line width=1.1pt] (0.2322238649724578,2.4105198064821693) -- (0.23905397864811834,2.4209611525887413);
\draw[line width=1.1pt] (0.23905397864811834,2.4209611525887413) -- (0.24588409232377886,2.4313091977896693);
\draw[line width=1.1pt] (0.24588409232377886,2.4313091977896693) -- (0.2527142059994394,2.4415639420849518);
\draw[line width=1.1pt] (0.2527142059994394,2.4415639420849518) -- (0.2595443196750999,2.4517253854745893);
\draw[line width=1.1pt] (0.2595443196750999,2.4517253854745893) -- (0.2663744333507604,2.4617935279585823);
\draw[line width=1.1pt] (0.2663744333507604,2.4617935279585823) -- (0.2732045470264209,2.47176836953693);
\draw[line width=1.1pt] (0.2732045470264209,2.47176836953693) -- (0.2800346607020814,2.481649910209633);
\draw[line width=1.1pt] (0.2800346607020814,2.481649910209633) -- (0.2868647743777419,2.491438149976691);
\draw[line width=1.1pt] (0.2868647743777419,2.491438149976691) -- (0.2936948880534024,2.501133088838104);
\draw[line width=1.1pt] (0.2936948880534024,2.501133088838104) -- (0.3005250017290629,2.5107347267938724);
\draw[line width=1.1pt] (0.3005250017290629,2.5107347267938724) -- (0.3073551154047234,2.520243063843996);
\draw[line width=1.1pt] (0.3073551154047234,2.520243063843996) -- (0.3141852290803839,2.529658099988475);
\draw[line width=1.1pt] (0.3141852290803839,2.529658099988475) -- (0.3210153427560444,2.5389798352273085);
\draw[line width=1.1pt] (0.3210153427560444,2.5389798352273085) -- (0.3278454564317049,2.548208269560497);
\draw[line width=1.1pt] (0.3278454564317049,2.548208269560497) -- (0.33467557010736537,2.5573434029880406);
\draw[line width=1.1pt] (0.33467557010736537,2.5573434029880406) -- (0.34150568378302587,2.56638523550994);
\draw[line width=1.1pt] (0.34150568378302587,2.56638523550994) -- (0.34833579745868637,2.5753337671261933);
\draw[line width=1.1pt] (0.34833579745868637,2.5753337671261933) -- (0.35516591113434687,2.5841889978368027);
\draw[line width=1.1pt] (0.35516591113434687,2.5841889978368027) -- (0.36199602481000737,2.592950927641767);
\draw[line width=1.1pt] (0.36199602481000737,2.592950927641767) -- (0.36882613848566786,2.601619556541087);
\draw[line width=1.1pt] (0.36882613848566786,2.601619556541087) -- (0.37565625216132836,2.6101948845347613);
\draw[line width=1.1pt] (0.37565625216132836,2.6101948845347613) -- (0.38248636583698886,2.618676911622791);
\draw[line width=1.1pt] (0.38248636583698886,2.618676911622791) -- (0.38931647951264936,2.6270656378051758);
\draw[line width=1.1pt] (0.38931647951264936,2.6270656378051758) -- (0.39614659318830986,2.6353610630819158);
\draw[line width=1.1pt] (0.39614659318830986,2.6353610630819158) -- (0.40297670686397036,2.6435631874530103);
\draw[line width=1.1pt] (0.40297670686397036,2.6435631874530103) -- (0.40980682053963086,2.6516720109184604);
\draw[line width=1.1pt] (0.40980682053963086,2.6516720109184604) -- (0.41663693421529135,2.659687533478266);
\draw[line width=1.1pt] (0.41663693421529135,2.659687533478266) -- (0.42346704789095185,2.667609755132426);
\draw[line width=1.1pt] (0.42346704789095185,2.667609755132426) -- (0.43029716156661235,2.6754386758809416);
\draw[line width=1.1pt] (0.43029716156661235,2.6754386758809416) -- (0.43712727524227285,2.683174295723812);
\draw[line width=1.1pt] (0.43712727524227285,2.683174295723812) -- (0.44395738891793335,2.6908166146610375);
\draw[line width=1.1pt] (0.44395738891793335,2.6908166146610375) -- (0.45078750259359385,2.698365632692618);
\draw[line width=1.1pt] (0.45078750259359385,2.698365632692618) -- (0.45761761626925435,2.705821349818554);
\draw[line width=1.1pt] (0.45761761626925435,2.705821349818554) -- (0.46444772994491484,2.713183766038845);
\draw[line width=1.1pt] (0.46444772994491484,2.713183766038845) -- (0.47127784362057534,2.7204528813534914);
\draw[line width=1.1pt] (0.47127784362057534,2.7204528813534914) -- (0.47810795729623584,2.7276286957624927);
\draw[line width=1.1pt] (0.47810795729623584,2.7276286957624927) -- (0.48493807097189634,2.734711209265849);
\draw[line width=1.1pt] (0.48493807097189634,2.734711209265849) -- (0.49176818464755684,2.74170042186356);
\draw[line width=1.1pt] (0.49176818464755684,2.74170042186356) -- (0.49859829832321734,2.7485963335556267);
\draw[line width=1.1pt] (0.49859829832321734,2.7485963335556267) -- (0.5054284119988779,2.7553989443420486);
\draw[line width=1.1pt] (0.5054284119988779,2.7553989443420486) -- (0.5122585256745384,2.762108254222825);
\draw[line width=1.1pt] (0.5122585256745384,2.762108254222825) -- (0.519088639350199,2.768724263197957);
\draw[line width=1.1pt] (0.519088639350199,2.768724263197957) -- (0.5259187530258596,2.775246971267444);
\draw[line width=1.1pt] (0.5259187530258596,2.775246971267444) -- (0.5327488667015201,2.7816763784312863);
\draw[line width=1.1pt] (0.5327488667015201,2.7816763784312863) -- (0.5395789803771807,2.7880124846894834);
\draw[line width=1.1pt] (0.5395789803771807,2.7880124846894834) -- (0.5464090940528412,2.794255290042036);
\draw[line width=1.1pt] (0.5464090940528412,2.794255290042036) -- (0.5532392077285018,2.8004047944889434);
\draw[line width=1.1pt] (0.5532392077285018,2.8004047944889434) -- (0.5600693214041623,2.8064609980302055);
\draw[line width=1.1pt] (0.5600693214041623,2.8064609980302055) -- (0.5668994350798229,2.8124239006658236);
\draw[line width=1.1pt] (0.5668994350798229,2.8124239006658236) -- (0.5737295487554834,2.8182935023957962);
\draw[line width=1.1pt] (0.5737295487554834,2.8182935023957962) -- (0.580559662431144,2.824069803220124);
\draw[line width=1.1pt] (0.580559662431144,2.824069803220124) -- (0.5873897761068045,2.829752803138807);
\draw[line width=1.1pt] (0.5873897761068045,2.829752803138807) -- (0.5942198897824651,2.8353425021518452);
\draw[line width=1.1pt] (0.5942198897824651,2.8353425021518452) -- (0.6010500034581256,2.8408389002592385);
\draw[line width=1.1pt] (0.6010500034581256,2.8408389002592385) -- (0.6078801171337862,2.8462419974609867);
\draw[line width=1.1pt] (0.6078801171337862,2.8462419974609867) -- (0.6147102308094468,2.85155179375709);
\draw[line width=1.1pt] (0.6147102308094468,2.85155179375709) -- (0.6215403444851073,2.8567682891475488);
\draw[line width=1.1pt] (0.6215403444851073,2.8567682891475488) -- (0.6283704581607679,2.8618914836323626);
\draw[line width=1.1pt] (0.6283704581607679,2.8618914836323626) -- (0.6352005718364284,2.866921377211531);
\draw[line width=1.1pt] (0.6352005718364284,2.866921377211531) -- (0.642030685512089,2.8718579698850553);
\draw[line width=1.1pt] (0.642030685512089,2.8718579698850553) -- (0.6488607991877495,2.876701261652934);
\draw[line width=1.1pt] (0.6488607991877495,2.876701261652934) -- (0.6556909128634101,2.881451252515168);
\draw[line width=1.1pt] (0.6556909128634101,2.881451252515168) -- (0.6625210265390706,2.8861079424717575);
\draw[line width=1.1pt] (0.6625210265390706,2.8861079424717575) -- (0.6693511402147312,2.8906713315227015);
\draw[line width=1.1pt] (0.6693511402147312,2.8906713315227015) -- (0.6761812538903917,2.895141419668001);
\draw[line width=1.1pt] (0.6761812538903917,2.895141419668001) -- (0.6830113675660523,2.8995182069076555);
\draw[line width=1.1pt] (0.6830113675660523,2.8995182069076555) -- (0.6898414812417129,2.9038016932416655);
\draw[line width=1.1pt] (0.6898414812417129,2.9038016932416655) -- (0.6966715949173734,2.90799187867003);
\draw[line width=1.1pt] (0.6966715949173734,2.90799187867003) -- (0.703501708593034,2.9120887631927497);
\draw[line width=1.1pt] (0.703501708593034,2.9120887631927497) -- (0.7103318222686945,2.9160923468098248);
\draw[line width=1.1pt] (0.7103318222686945,2.9160923468098248) -- (0.7171619359443551,2.920002629521255);
\draw[line width=1.1pt] (0.7171619359443551,2.920002629521255) -- (0.7239920496200156,2.92381961132704);
\draw[line width=1.1pt] (0.7239920496200156,2.92381961132704) -- (0.7308221632956762,2.9275432922271802);
\draw[line width=1.1pt] (0.7308221632956762,2.9275432922271802) -- (0.7376522769713367,2.931173672221676);
\draw[line width=1.1pt] (0.7376522769713367,2.931173672221676) -- (0.7444823906469973,2.934710751310526);
\draw[line width=1.1pt] (0.7444823906469973,2.934710751310526) -- (0.7513125043226578,2.938154529493732);
\draw[line width=1.1pt] (0.7513125043226578,2.938154529493732) -- (0.7581426179983184,2.9415050067712927);
\draw[line width=1.1pt] (0.7581426179983184,2.9415050067712927) -- (0.764972731673979,2.9447621831432085);
\draw[line width=1.1pt] (0.764972731673979,2.9447621831432085) -- (0.7718028453496395,2.9479260586094793);
\draw[line width=1.1pt] (0.7718028453496395,2.9479260586094793) -- (0.7786329590253,2.9509966331701056);
\draw[line width=1.1pt] (0.7786329590253,2.9509966331701056) -- (0.7854630727009606,2.9539739068250865);
\draw[line width=1.1pt] (0.7854630727009606,2.9539739068250865) -- (0.7922931863766212,2.956857879574423);
\draw[line width=1.1pt] (0.7922931863766212,2.956857879574423) -- (0.7991233000522817,2.9596485514181143);
\draw[line width=1.1pt] (0.7991233000522817,2.9596485514181143) -- (0.8059534137279423,2.962345922356161);
\draw[line width=1.1pt] (0.8059534137279423,2.962345922356161) -- (0.8127835274036028,2.9649499923885623);
\draw[line width=1.1pt] (0.8127835274036028,2.9649499923885623) -- (0.8196136410792634,2.9674607615153192);
\draw[line width=1.1pt] (0.8196136410792634,2.9674607615153192) -- (0.8264437547549239,2.9698782297364312);
\draw[line width=1.1pt] (0.8264437547549239,2.9698782297364312) -- (0.8332738684305845,2.972202397051898);
\draw[line width=1.1pt] (0.8332738684305845,2.972202397051898) -- (0.840103982106245,2.97443326346172);
\draw[line width=1.1pt] (0.840103982106245,2.97443326346172) -- (0.8469340957819056,2.9765708289658974);
\draw[line width=1.1pt] (0.8469340957819056,2.9765708289658974) -- (0.8537642094575661,2.9786150935644296);
\draw[line width=1.1pt] (0.8537642094575661,2.9786150935644296) -- (0.8605943231332267,2.9805660572573167);
\draw[line width=1.1pt] (0.8605943231332267,2.9805660572573167) -- (0.8674244368088873,2.9824237200445594);
\draw[line width=1.1pt] (0.8674244368088873,2.9824237200445594) -- (0.8742545504845478,2.984188081926157);
\draw[line width=1.1pt] (0.8742545504845478,2.984188081926157) -- (0.8810846641602084,2.9858591429021093);
\draw[line width=1.1pt] (0.8810846641602084,2.9858591429021093) -- (0.8879147778358689,2.9874369029724175);
\draw[line width=1.1pt] (0.8879147778358689,2.9874369029724175) -- (0.8947448915115295,2.9889213621370803);
\draw[line width=1.1pt] (0.8947448915115295,2.9889213621370803) -- (0.90157500518719,2.990312520396098);
\draw[line width=1.1pt] (0.90157500518719,2.990312520396098) -- (0.9084051188628506,2.9916103777494714);
\draw[line width=1.1pt] (0.9084051188628506,2.9916103777494714) -- (0.9152352325385111,2.9928149341971997);
\draw[line width=1.1pt] (0.9152352325385111,2.9928149341971997) -- (0.9220653462141717,2.993926189739283);
\draw[line width=1.1pt] (0.9220653462141717,2.993926189739283) -- (0.9288954598898322,2.9949441443757214);
\draw[line width=1.1pt] (0.9288954598898322,2.9949441443757214) -- (0.9357255735654928,2.995868798106515);
\draw[line width=1.1pt] (0.9357255735654928,2.995868798106515) -- (0.9425556872411534,2.9967001509316638);
\draw[line width=1.1pt] (0.9425556872411534,2.9967001509316638) -- (0.9493858009168139,2.9974382028511677);
\draw[line width=1.1pt] (0.9493858009168139,2.9974382028511677) -- (0.9562159145924745,2.9980829538650267);
\draw[line width=1.1pt] (0.9562159145924745,2.9980829538650267) -- (0.963046028268135,2.9986344039732407);
\draw[line width=1.1pt] (0.963046028268135,2.9986344039732407) -- (0.9698761419437956,2.9990925531758097);
\draw[line width=1.1pt] (0.9698761419437956,2.9990925531758097) -- (0.9767062556194561,2.9994574014727338);
\draw[line width=1.1pt] (0.9767062556194561,2.9994574014727338) -- (0.9835363692951167,2.9997289488640133);
\draw[line width=1.1pt] (0.9835363692951167,2.9997289488640133) -- (0.9903664829707772,2.999907195349648);
\draw[line width=1.1pt] (0.9903664829707772,2.999907195349648) -- (0.9971965966464378,2.9999921409296375);
\draw[line width=1.1pt] (0.9971965966464378,2.9999921409296375) -- (1.0040267103220983,2.999983785603982);
\draw[line width=1.1pt] (1.0040267103220983,2.999983785603982) -- (1.010856823997759,2.999882129372682);
\draw[line width=1.1pt] (1.010856823997759,2.999882129372682) -- (1.0176869376734194,2.999687172235736);
\draw[line width=1.1pt] (1.0176869376734194,2.999687172235736) -- (1.02451705134908,2.9993989141931463);
\draw[line width=1.1pt] (1.02451705134908,2.9993989141931463) -- (1.0313471650247406,2.999017355244912);
\draw[line width=1.1pt] (1.0313471650247406,2.999017355244912) -- (1.038177278700401,2.998542495391032);
\draw[line width=1.1pt] (1.038177278700401,2.998542495391032) -- (1.0450073923760617,2.9979743346315066);
\draw[line width=1.1pt] (1.0450073923760617,2.9979743346315066) -- (1.0518375060517222,2.9973128729663374);
\draw[line width=1.1pt] (1.0518375060517222,2.9973128729663374) -- (1.0586676197273828,2.9965581103955237);
\draw[line width=1.1pt] (1.0586676197273828,2.9965581103955237) -- (1.0654977334030433,2.995710046919064);
\draw[line width=1.1pt] (1.0654977334030433,2.995710046919064) -- (1.0723278470787039,2.994768682536959);
\draw[line width=1.1pt] (1.0723278470787039,2.994768682536959) -- (1.0791579607543644,2.9937340172492104);
\draw[line width=1.1pt] (1.0791579607543644,2.9937340172492104) -- (1.085988074430025,2.992606051055817);
\draw[line width=1.1pt] (1.085988074430025,2.992606051055817) -- (1.0928181881056855,2.9913847839567778);
\draw[line width=1.1pt] (1.0928181881056855,2.9913847839567778) -- (1.099648301781346,2.9900702159520933);
\draw[line width=1.1pt] (1.099648301781346,2.9900702159520933) -- (1.1064784154570066,2.988662347041765);
\draw[line width=1.1pt] (1.1064784154570066,2.988662347041765) -- (1.1133085291326672,2.9871611772257918);
\draw[line width=1.1pt] (1.1133085291326672,2.9871611772257918) -- (1.1201386428083278,2.985566706504173);
\draw[line width=1.1pt] (1.1201386428083278,2.985566706504173) -- (1.1269687564839883,2.983878934876909);
\draw[line width=1.1pt] (1.1269687564839883,2.983878934876909) -- (1.1337988701596489,2.982097862344001);
\draw[line width=1.1pt] (1.1337988701596489,2.982097862344001) -- (1.1406289838353094,2.9802234889054486);
\draw[line width=1.1pt] (1.1406289838353094,2.9802234889054486) -- (1.14745909751097,2.9782558145612503);
\draw[line width=1.1pt] (1.14745909751097,2.9782558145612503) -- (1.1542892111866305,2.9761948393114066);
\draw[line width=1.1pt] (1.1542892111866305,2.9761948393114066) -- (1.161119324862291,2.9740405631559197);
\draw[line width=1.1pt] (1.161119324862291,2.9740405631559197) -- (1.1679494385379516,2.971792986094787);
\draw[line width=1.1pt] (1.1679494385379516,2.971792986094787) -- (1.1747795522136122,2.969452108128009);
\draw[line width=1.1pt] (1.1747795522136122,2.969452108128009) -- (1.1816096658892727,2.9670179292555865);
\draw[line width=1.1pt] (1.1816096658892727,2.9670179292555865) -- (1.1884397795649333,2.9644904494775193);
\draw[line width=1.1pt] (1.1884397795649333,2.9644904494775193) -- (1.1952698932405939,2.9618696687938075);
\draw[line width=1.1pt] (1.1952698932405939,2.9618696687938075) -- (1.2021000069162544,2.95915558720445);
\draw[line width=1.1pt] (1.2021000069162544,2.95915558720445) -- (1.208930120591915,2.9563482047094474);
\draw[line width=1.1pt] (1.208930120591915,2.9563482047094474) -- (1.2157602342675755,2.953447521308801);
\draw[line width=1.1pt] (1.2157602342675755,2.953447521308801) -- (1.222590347943236,2.9504535370025096);
\draw[line width=1.1pt] (1.222590347943236,2.9504535370025096) -- (1.2294204616188966,2.947366251790572);
\draw[line width=1.1pt] (1.2294204616188966,2.947366251790572) -- (1.2362505752945572,2.9441856656729906);
\draw[line width=1.1pt] (1.2362505752945572,2.9441856656729906) -- (1.2430806889702177,2.940911778649764);
\draw[line width=1.1pt] (1.2430806889702177,2.940911778649764) -- (1.2499108026458783,2.9375445907208935);
\draw[line width=1.1pt] (1.2499108026458783,2.9375445907208935) -- (1.2567409163215388,2.9340841018863766);
\draw[line width=1.1pt] (1.2567409163215388,2.9340841018863766) -- (1.2635710299971994,2.9305303121462147);
\draw[line width=1.1pt] (1.2635710299971994,2.9305303121462147) -- (1.27040114367286,2.9268832215004092);
\draw[line width=1.1pt] (1.27040114367286,2.9268832215004092) -- (1.2772312573485205,2.923142829948959);
\draw[line width=1.1pt] (1.2772312573485205,2.923142829948959) -- (1.284061371024181,2.9193091374918625);
\draw[line width=1.1pt] (1.284061371024181,2.9193091374918625) -- (1.2908914846998416,2.9153821441291212);
\draw[line width=1.1pt] (1.2908914846998416,2.9153821441291212) -- (1.2977215983755022,2.911361849860736);
\draw[line width=1.1pt] (1.2977215983755022,2.911361849860736) -- (1.3045517120511627,2.9072482546867064);
\draw[line width=1.1pt] (1.3045517120511627,2.9072482546867064) -- (1.3113818257268233,2.9030413586070303);
\draw[line width=1.1pt] (1.3113818257268233,2.9030413586070303) -- (1.3182119394024838,2.8987411616217096);
\draw[line width=1.1pt] (1.3182119394024838,2.8987411616217096) -- (1.3250420530781444,2.894347663730745);
\draw[line width=1.1pt] (1.3250420530781444,2.894347663730745) -- (1.331872166753805,2.889860864934135);
\draw[line width=1.1pt] (1.331872166753805,2.889860864934135) -- (1.3387022804294655,2.8852807652318795);
\draw[line width=1.1pt] (1.3387022804294655,2.8852807652318795) -- (1.345532394105126,2.8806073646239794);
\draw[line width=1.1pt] (1.345532394105126,2.8806073646239794) -- (1.3523625077807866,2.875840663110435);
\draw[line width=1.1pt] (1.3523625077807866,2.875840663110435) -- (1.3591926214564471,2.870980660691246);
\draw[line width=1.1pt] (1.3591926214564471,2.870980660691246) -- (1.3660227351321077,2.866027357366411);
\draw[line width=1.1pt] (1.3660227351321077,2.866027357366411) -- (1.3728528488077683,2.860980753135931);
\draw[line width=1.1pt] (1.3728528488077683,2.860980753135931) -- (1.3796829624834288,2.855840847999807);
\draw[line width=1.1pt] (1.3796829624834288,2.855840847999807) -- (1.3865130761590894,2.8506076419580384);
\draw[line width=1.1pt] (1.3865130761590894,2.8506076419580384) -- (1.39334318983475,2.845281135010624);
\draw[line width=1.1pt] (1.39334318983475,2.845281135010624) -- (1.4001733035104105,2.8398613271575646);
\draw[line width=1.1pt] (1.4001733035104105,2.8398613271575646) -- (1.407003417186071,2.834348218398861);
\draw[line width=1.1pt] (1.407003417186071,2.834348218398861) -- (1.4138335308617316,2.8287418087345126);
\draw[line width=1.1pt] (1.4138335308617316,2.8287418087345126) -- (1.4206636445373921,2.823042098164519);
\draw[line width=1.1pt] (1.4206636445373921,2.823042098164519) -- (1.4274937582130527,2.8172490866888795);
\draw[line width=1.1pt] (1.4274937582130527,2.8172490866888795) -- (1.4343238718887132,2.8113627743075966);
\draw[line width=1.1pt] (1.4343238718887132,2.8113627743075966) -- (1.4411539855643738,2.8053831610206688);
\draw[line width=1.1pt] (1.4411539855643738,2.8053831610206688) -- (1.4479840992400344,2.799310246828095);
\draw[line width=1.1pt] (1.4479840992400344,2.799310246828095) -- (1.454814212915695,2.7931440317298764);
\draw[line width=1.1pt] (1.454814212915695,2.7931440317298764) -- (1.4616443265913555,2.786884515726014);
\draw[line width=1.1pt] (1.4616443265913555,2.786884515726014) -- (1.468474440267016,2.7805316988165063);
\draw[line width=1.1pt] (1.468474440267016,2.7805316988165063) -- (1.4753045539426766,2.774085581001353);
\draw[line width=1.1pt] (1.4753045539426766,2.774085581001353) -- (1.4821346676183371,2.767546162280555);
\draw[line width=1.1pt] (1.4821346676183371,2.767546162280555) -- (1.4889647812939977,2.760913442654113);
\draw[line width=1.1pt] (1.4889647812939977,2.760913442654113) -- (1.4957948949696582,2.754187422122026);
\draw[line width=1.1pt] (1.4957948949696582,2.754187422122026) -- (1.5026250086453188,2.747368100684293);
\draw[line width=1.1pt] (1.5026250086453188,2.747368100684293) -- (1.5094551223209793,2.7404554783409156);
\draw[line width=1.1pt] (1.5094551223209793,2.7404554783409156) -- (1.51628523599664,2.733449555091894);
\draw[line width=1.1pt] (1.51628523599664,2.733449555091894) -- (1.5231153496723004,2.7263503309372275);
\draw[line width=1.1pt] (1.5231153496723004,2.7263503309372275) -- (1.529945463347961,2.719157805876915);
\draw[line width=1.1pt] (1.529945463347961,2.719157805876915) -- (1.5367755770236216,2.7118719799109576);
\draw[line width=1.1pt] (1.5367755770236216,2.7118719799109576) -- (1.543605690699282,2.7044928530393566);
\draw[line width=1.1pt] (1.543605690699282,2.7044928530393566) -- (1.5504358043749427,2.69702042526211);
\draw[line width=1.1pt] (1.5504358043749427,2.69702042526211) -- (1.5572659180506032,2.6894546965792183);
\draw[line width=1.1pt] (1.5572659180506032,2.6894546965792183) -- (1.5640960317262638,2.6817956669906815);
\draw[line width=1.1pt] (1.5640960317262638,2.6817956669906815) -- (1.5709261454019243,2.6740433364965006);
\draw[line width=1.1pt] (1.5709261454019243,2.6740433364965006) -- (1.5777562590775849,2.666197705096675);
\draw[line width=1.1pt] (1.5777562590775849,2.666197705096675) -- (1.5845863727532454,2.6582587727912035);
\draw[line width=1.1pt] (1.5845863727532454,2.6582587727912035) -- (1.591416486428906,2.6502265395800872);
\draw[line width=1.1pt] (1.591416486428906,2.6502265395800872) -- (1.5982466001045665,2.642101005463327);
\draw[line width=1.1pt] (1.5982466001045665,2.642101005463327) -- (1.605076713780227,2.6338821704409217);
\draw[line width=1.1pt] (1.605076713780227,2.6338821704409217) -- (1.6119068274558876,2.6255700345128705);
\draw[line width=1.1pt] (1.6119068274558876,2.6255700345128705) -- (1.6187369411315482,2.6171645976791744);
\draw[line width=1.1pt] (1.6187369411315482,2.6171645976791744) -- (1.6255670548072088,2.6086658599398347);
\draw[line width=1.1pt] (1.6255670548072088,2.6086658599398347) -- (1.6323971684828693,2.60007382129485);
\draw[line width=1.1pt] (1.6323971684828693,2.60007382129485) -- (1.6392272821585299,2.5913884817442194);
\draw[line width=1.1pt] (1.6392272821585299,2.5913884817442194) -- (1.6460573958341904,2.582609841287944);
\draw[line width=1.1pt] (1.6460573958341904,2.582609841287944) -- (1.652887509509851,2.5737378999260243);
\draw[line width=1.1pt] (1.652887509509851,2.5737378999260243) -- (1.6597176231855115,2.5647726576584597);
\draw[line width=1.1pt] (1.6597176231855115,2.5647726576584597) -- (1.666547736861172,2.5557141144852498);
\draw[line width=1.1pt] (1.666547736861172,2.5557141144852498) -- (1.6733778505368326,2.546562270406395);
\draw[line width=1.1pt] (1.6733778505368326,2.546562270406395) -- (1.6802079642124932,2.537317125421896);
\draw[line width=1.1pt] (1.6802079642124932,2.537317125421896) -- (1.6870380778881537,2.527978679531752);
\draw[line width=1.1pt] (1.6870380778881537,2.527978679531752) -- (1.6938681915638143,2.518546932735962);
\draw[line width=1.1pt] (1.6938681915638143,2.518546932735962) -- (1.7006983052394749,2.509021885034527);
\draw[line width=1.1pt] (1.7006983052394749,2.509021885034527) -- (1.7075284189151354,2.4994035364274487);
\draw[line width=1.1pt] (1.7075284189151354,2.4994035364274487) -- (1.714358532590796,2.4896918869147253);
\draw[line width=1.1pt] (1.714358532590796,2.4896918869147253) -- (1.7211886462664565,2.479886936496356);
\draw[line width=1.1pt] (1.7211886462664565,2.479886936496356) -- (1.728018759942117,2.469988685172342);
\draw[line width=1.1pt] (1.728018759942117,2.469988685172342) -- (1.7348488736177776,2.4599971329426835);
\draw[line width=1.1pt] (1.7348488736177776,2.4599971329426835) -- (1.7416789872934382,2.4499122798073802);
\draw[line width=1.1pt] (1.7416789872934382,2.4499122798073802) -- (1.7485091009690987,2.4397341257664316);
\draw[line width=1.1pt] (1.7485091009690987,2.4397341257664316) -- (1.7553392146447593,2.429462670819838);
\draw[line width=1.1pt] (1.7553392146447593,2.429462670819838) -- (1.7621693283204198,2.4190979149676);
\draw[line width=1.1pt] (1.7621693283204198,2.4190979149676) -- (1.7689994419960804,2.4086398582097175);
\draw[line width=1.1pt] (1.7689994419960804,2.4086398582097175) -- (1.775829555671741,2.398088500546189);
\draw[line width=1.1pt] (1.775829555671741,2.398088500546189) -- (1.7826596693474015,2.387443841977016);
\draw[line width=1.1pt] (1.7826596693474015,2.387443841977016) -- (1.789489783023062,2.3767058825021983);
\draw[line width=1.1pt] (1.789489783023062,2.3767058825021983) -- (1.7963198966987226,2.365874622121736);
\draw[line width=1.1pt] (1.7963198966987226,2.365874622121736) -- (1.8031500103743832,2.354950060835628);
\draw[line width=1.1pt] (1.8031500103743832,2.354950060835628) -- (1.8099801240500437,2.3439321986438753);
\draw[line width=1.1pt] (1.8099801240500437,2.3439321986438753) -- (1.8168102377257043,2.3328210355464787);
\draw[line width=1.1pt] (1.8168102377257043,2.3328210355464787) -- (1.8236403514013648,2.3216165715434367);
\draw[line width=1.1pt] (1.8236403514013648,2.3216165715434367) -- (1.8304704650770254,2.3103188066347493);
\draw[line width=1.1pt] (1.8304704650770254,2.3103188066347493) -- (1.837300578752686,2.298927740820417);
\draw[line width=1.1pt] (1.837300578752686,2.298927740820417) -- (1.8441306924283465,2.2874433741004405);
\draw[line width=1.1pt] (1.8441306924283465,2.2874433741004405) -- (1.850960806104007,2.275865706474819);
\draw[line width=1.1pt] (1.850960806104007,2.275865706474819) -- (1.8577909197796676,2.264194737943552);
\draw[line width=1.1pt] (1.8577909197796676,2.264194737943552) -- (1.8646210334553281,2.25243046850664);
\draw[line width=1.1pt] (1.8646210334553281,2.25243046850664) -- (1.8714511471309887,2.240572898164084);
\draw[line width=1.1pt] (1.8714511471309887,2.240572898164084) -- (1.8782812608066493,2.228622026915883);
\draw[line width=1.1pt] (1.8782812608066493,2.228622026915883) -- (1.8851113744823098,2.2165778547620363);
\draw[line width=1.1pt] (1.8851113744823098,2.2165778547620363) -- (1.8919414881579704,2.2044403817025446);
\draw[line width=1.1pt] (1.8919414881579704,2.2044403817025446) -- (1.898771601833631,2.1922096077374094);
\draw[line width=1.1pt] (1.898771601833631,2.1922096077374094) -- (1.9056017155092915,2.1798855328666287);
\draw[line width=1.1pt] (1.9056017155092915,2.1798855328666287) -- (1.912431829184952,2.1674681570902026);
\draw[line width=1.1pt] (1.912431829184952,2.1674681570902026) -- (1.9192619428606126,2.1549574804081315);
\draw[line width=1.1pt] (1.9192619428606126,2.1549574804081315) -- (1.9260920565362731,2.1423535028204164);
\draw[line width=1.1pt] (1.9260920565362731,2.1423535028204164) -- (1.9329221702119337,2.1296562243270563);
\draw[line width=1.1pt] (1.9329221702119337,2.1296562243270563) -- (1.9397522838875942,2.1168656449280503);
\draw[line width=1.1pt] (1.9397522838875942,2.1168656449280503) -- (1.9465823975632548,2.1039817646234);
\draw[line width=1.1pt] (1.9465823975632548,2.1039817646234) -- (1.9534125112389154,2.0910045834131052);
\draw[line width=1.1pt] (1.9534125112389154,2.0910045834131052) -- (1.960242624914576,2.0779341012971657);
\draw[line width=1.1pt] (1.960242624914576,2.0779341012971657) -- (1.9670727385902365,2.0647703182755803);
\draw[line width=1.1pt] (1.9670727385902365,2.0647703182755803) -- (1.973902852265897,2.05151323434835);
\draw[line width=1.1pt] (1.973902852265897,2.05151323434835) -- (1.9807329659415576,2.0381628495154755);
\draw[line width=1.1pt] (1.9807329659415576,2.0381628495154755) -- (1.9875630796172181,2.0247191637769566);
\draw[line width=1.1pt] (1.9875630796172181,2.0247191637769566) -- (1.9943931932928787,2.011182177132792);
\draw[line width=1.1pt] (1.9943931932928787,2.011182177132792) -- (2.001223306968539,1.997551889582983);
\draw[line width=1.1pt] (2.001223306968539,1.997551889582983) -- (2.0080534206441993,1.983828301127529);
\draw[line width=1.1pt] (2.0080534206441993,1.983828301127529) -- (2.0148835343198597,1.9700114117664302);
\draw[line width=1.1pt] (2.0148835343198597,1.9700114117664302) -- (2.02171364799552,1.9561012214996865);
\draw[line width=1.1pt] (2.02171364799552,1.9561012214996865) -- (2.0285437616711803,1.9420977303272986);
\draw[line width=1.1pt] (2.0285437616711803,1.9420977303272986) -- (2.0353738753468407,1.9280009382492649);
\draw[line width=1.1pt] (2.0353738753468407,1.9280009382492649) -- (2.042203989022501,1.9138108452655862);
\draw[line width=1.1pt] (2.042203989022501,1.9138108452655862) -- (2.0490341026981613,1.8995274513762634);
\draw[line width=1.1pt] (2.0490341026981613,1.8995274513762634) -- (2.0558642163738217,1.8851507565812957);
\draw[line width=1.1pt] (2.0558642163738217,1.8851507565812957) -- (2.062694330049482,1.870680760880683);
\draw[line width=1.1pt] (2.062694330049482,1.870680760880683) -- (2.0695244437251423,1.8561174642744245);
\draw[line width=1.1pt] (2.0695244437251423,1.8561174642744245) -- (2.0763545574008027,1.8414608667625219);
\draw[line width=1.1pt] (2.0763545574008027,1.8414608667625219) -- (2.083184671076463,1.8267109683449743);
\draw[line width=1.1pt] (2.083184671076463,1.8267109683449743) -- (2.0900147847521233,1.8118677690217826);
\draw[line width=1.1pt] (2.0900147847521233,1.8118677690217826) -- (2.0968448984277837,1.796931268792945);
\draw[line width=1.1pt] (2.0968448984277837,1.796931268792945) -- (2.103675012103444,1.7819014676584626);
\draw[line width=1.1pt] (2.103675012103444,1.7819014676584626) -- (2.1105051257791043,1.766778365618336);
\draw[line width=1.1pt] (2.1105051257791043,1.766778365618336) -- (2.1173352394547647,1.7515619626725636);
\draw[line width=1.1pt] (2.1173352394547647,1.7515619626725636) -- (2.124165353130425,1.736252258821147);
\draw[line width=1.1pt] (2.124165353130425,1.736252258821147) -- (2.1309954668060853,1.7208492540640847);
\draw[line width=1.1pt] (2.1309954668060853,1.7208492540640847) -- (2.1378255804817456,1.7053529484013783);
\draw[line width=1.1pt] (2.1378255804817456,1.7053529484013783) -- (2.144655694157406,1.689763341833027);
\draw[line width=1.1pt] (2.144655694157406,1.689763341833027) -- (2.1514858078330663,1.6740804343590305);
\draw[line width=1.1pt] (2.1514858078330663,1.6740804343590305) -- (2.1583159215087266,1.6583042259793892);
\draw[line width=1.1pt] (2.1583159215087266,1.6583042259793892) -- (2.165146035184387,1.642434716694103);
\draw[line width=1.1pt] (2.165146035184387,1.642434716694103) -- (2.1719761488600473,1.6264719065031725);
\draw[line width=1.1pt] (2.1719761488600473,1.6264719065031725) -- (2.1788062625357076,1.6104157954065963);
\draw[line width=1.1pt] (2.1788062625357076,1.6104157954065963) -- (2.185636376211368,1.594266383404375);
\draw[line width=1.1pt] (2.185636376211368,1.594266383404375) -- (2.1924664898870283,1.5780236704965098);
\draw[line width=1.1pt] (2.1924664898870283,1.5780236704965098) -- (2.1992966035626886,1.5616876566829996);
\draw[line width=1.1pt] (2.1992966035626886,1.5616876566829996) -- (2.206126717238349,1.5452583419638435);
\draw[line width=1.1pt] (2.206126717238349,1.5452583419638435) -- (2.2129568309140093,1.5287357263390433);
\draw[line width=1.1pt] (2.2129568309140093,1.5287357263390433) -- (2.2197869445896696,1.5121198098085982);
\draw[line width=1.1pt] (2.2197869445896696,1.5121198098085982) -- (2.22661705826533,1.495410592372508);
\draw[line width=1.1pt] (2.22661705826533,1.495410592372508) -- (2.2334471719409903,1.478608074030773);
\draw[line width=1.1pt] (2.2334471719409903,1.478608074030773) -- (2.2402772856166506,1.461712254783393);
\draw[line width=1.1pt] (2.2402772856166506,1.461712254783393) -- (2.247107399292311,1.4447231346303688);
\draw[line width=1.1pt] (2.247107399292311,1.4447231346303688) -- (2.2539375129679713,1.4276407135716989);
\draw[line width=1.1pt] (2.2539375129679713,1.4276407135716989) -- (2.2607676266436316,1.410464991607384);
\draw[line width=1.1pt] (2.2607676266436316,1.410464991607384) -- (2.267597740319292,1.393195968737425);
\draw[line width=1.1pt] (2.267597740319292,1.393195968737425) -- (2.2744278539949523,1.375833644961821);
\draw[line width=1.1pt] (2.2744278539949523,1.375833644961821) -- (2.2812579676706126,1.358378020280571);
\draw[line width=1.1pt] (2.2812579676706126,1.358378020280571) -- (2.288088081346273,1.3408290946936772);
\draw[line width=1.1pt] (2.288088081346273,1.3408290946936772) -- (2.2949181950219333,1.3231868682011383);
\draw[line width=1.1pt] (2.2949181950219333,1.3231868682011383) -- (2.3017483086975936,1.3054513408029544);
\draw[line width=1.1pt] (2.3017483086975936,1.3054513408029544) -- (2.308578422373254,1.2876225124991256);
\draw[line width=1.1pt] (2.308578422373254,1.2876225124991256) -- (2.3154085360489143,1.2697003832896518);
\draw[line width=1.1pt] (2.3154085360489143,1.2697003832896518) -- (2.3222386497245746,1.251684953174534);
\draw[line width=1.1pt] (2.3222386497245746,1.251684953174534) -- (2.329068763400235,1.2335762221537703);
\draw[line width=1.1pt] (2.329068763400235,1.2335762221537703) -- (2.3358988770758953,1.2153741902273625);
\draw[line width=1.1pt] (2.3358988770758953,1.2153741902273625) -- (2.3427289907515556,1.1970788573953088);
\draw[line width=1.1pt] (2.3427289907515556,1.1970788573953088) -- (2.349559104427216,1.1786902236576111);
\draw[line width=1.1pt] (2.349559104427216,1.1786902236576111) -- (2.3563892181028763,1.1602082890142675);
\draw[line width=1.1pt] (2.3563892181028763,1.1602082890142675) -- (2.3632193317785366,1.1416330534652799);
\draw[line width=1.1pt] (2.3632193317785366,1.1416330534652799) -- (2.370049445454197,1.1229645170106473);
\draw[line width=1.1pt] (2.370049445454197,1.1229645170106473) -- (2.3768795591298573,1.1042026796503697);
\draw[line width=1.1pt] (2.3768795591298573,1.1042026796503697) -- (2.3837096728055176,1.0853475413844471);
\draw[line width=1.1pt] (2.3837096728055176,1.0853475413844471) -- (2.390539786481178,1.0663991022128805);
\draw[line width=1.1pt] (2.390539786481178,1.0663991022128805) -- (2.3973699001568383,1.047357362135668);
\draw[line width=1.1pt] (2.3973699001568383,1.047357362135668) -- (2.4042000138324986,1.0282223211528105);
\draw[line width=1.1pt] (2.4042000138324986,1.0282223211528105) -- (2.411030127508159,1.008993979264309);
\draw[line width=1.1pt] (2.411030127508159,1.008993979264309) -- (2.4178602411838193,0.9896723364701616);
\draw[line width=1.1pt] (2.4178602411838193,0.9896723364701616) -- (2.4246903548594796,0.9702573927703702);
\draw[line width=1.1pt] (2.4246903548594796,0.9702573927703702) -- (2.43152046853514,0.9507491481649337);
\draw[line width=1.1pt] (2.43152046853514,0.9507491481649337) -- (2.4383505822108003,0.9311476026538523);
\draw[line width=1.1pt] (2.4383505822108003,0.9311476026538523) -- (2.4451806958864606,0.911452756237126);
\draw[line width=1.1pt] (2.4451806958864606,0.911452756237126) -- (2.452010809562121,0.8916646089147546);
\draw[line width=1.1pt] (2.452010809562121,0.8916646089147546) -- (2.4588409232377813,0.8717831606867383);
\draw[line width=1.1pt] (2.4588409232377813,0.8717831606867383) -- (2.4656710369134416,0.8518084115530771);
\draw[line width=1.1pt] (2.4656710369134416,0.8518084115530771) -- (2.472501150589102,0.8317403615137708);
\draw[line width=1.1pt] (2.472501150589102,0.8317403615137708) -- (2.4793312642647622,0.8115790105688205);
\draw[line width=1.1pt] (2.4793312642647622,0.8115790105688205) -- (2.4861613779404226,0.7913243587182244);
\draw[line width=1.1pt] (2.4861613779404226,0.7913243587182244) -- (2.492991491616083,0.7709764059619841);
\draw[line width=1.1pt] (2.492991491616083,0.7709764059619841) -- (2.4998216052917432,0.750535152300098);
\draw[line width=1.1pt] (2.4998216052917432,0.750535152300098) -- (2.5066517189674036,0.7300005977325679);
\draw[line width=1.1pt] (2.5066517189674036,0.7300005977325679) -- (2.513481832643064,0.7093727422593927);
\draw[line width=1.1pt] (2.513481832643064,0.7093727422593927) -- (2.5203119463187242,0.6886515858805726);
\draw[line width=1.1pt] (2.5203119463187242,0.6886515858805726) -- (2.5271420599943846,0.6678371285961076);
\draw[line width=1.1pt] (2.5271420599943846,0.6678371285961076) -- (2.533972173670045,0.6469293704059975);
\draw[line width=1.1pt] (2.533972173670045,0.6469293704059975) -- (2.5408022873457052,0.6259283113102425);
\draw[line width=1.1pt] (2.5408022873457052,0.6259283113102425) -- (2.5476324010213656,0.6048339513088434);
\draw[line width=1.1pt] (2.5476324010213656,0.6048339513088434) -- (2.554462514697026,0.5836462904017985);
\draw[line width=1.1pt] (2.554462514697026,0.5836462904017985) -- (2.5612926283726862,0.5623653285891095);
\draw[line width=1.1pt] (2.5612926283726862,0.5623653285891095) -- (2.5681227420483466,0.5409910658707746);
\draw[line width=1.1pt] (2.5681227420483466,0.5409910658707746) -- (2.574952855724007,0.5195235022467957);
\draw[line width=1.1pt] (2.574952855724007,0.5195235022467957) -- (2.5817829693996672,0.49796263771717175);
\draw[line width=1.1pt] (2.5817829693996672,0.49796263771717175) -- (2.5886130830753276,0.476308472281902);
\draw[line width=1.1pt] (2.5886130830753276,0.476308472281902) -- (2.595443196750988,0.45456100594098814);
\draw[line width=1.1pt] (2.595443196750988,0.45456100594098814) -- (2.6022733104266482,0.4327202386944302);
\draw[line width=1.1pt] (2.6022733104266482,0.4327202386944302) -- (2.6091034241023086,0.41078617054222644);
\draw[line width=1.1pt] (2.6091034241023086,0.41078617054222644) -- (2.615933537777969,0.3887588014843777);
\draw[line width=1.1pt] (2.615933537777969,0.3887588014843777) -- (2.622763651453629,0.366638131520884);
\draw[line width=1.1pt] (2.622763651453629,0.366638131520884) -- (2.6295937651292896,0.3444241606517462);
\draw[line width=1.1pt] (2.6295937651292896,0.3444241606517462) -- (2.63642387880495,0.32211688887696255);
\draw[line width=1.1pt] (2.63642387880495,0.32211688887696255) -- (2.64325399248061,0.2997163161965348);
\draw[line width=1.1pt] (2.64325399248061,0.2997163161965348) -- (2.6500841061562705,0.27722244261046125);
\draw[line width=1.1pt] (2.6500841061562705,0.27722244261046125) -- (2.656914219831931,0.2546352681187436);
\draw[line width=1.1pt] (2.656914219831931,0.2546352681187436) -- (2.663744333507591,0.23195479272138098);
\draw[line width=1.1pt] (2.663744333507591,0.23195479272138098) -- (2.6705744471832515,0.2091810164183734);
\draw[line width=1.1pt] (2.6705744471832515,0.2091810164183734) -- (2.677404560858912,0.18631393920972084);
\draw[line width=1.1pt] (2.677404560858912,0.18631393920972084) -- (2.684234674534572,0.16335356109542332);
\draw[line width=1.1pt] (2.684234674534572,0.16335356109542332) -- (2.6910647882102325,0.14029988207548172);
\draw[line width=1.1pt] (2.6910647882102325,0.14029988207548172) -- (2.697894901885893,0.11715290214989427);
\draw[line width=1.1pt] (2.697894901885893,0.11715290214989427) -- (2.704725015561553,0.09391262131866185);
\draw[line width=1.1pt] (2.704725015561553,0.09391262131866185) -- (2.7115551292372135,0.07057903958178535);
\draw[line width=1.1pt] (2.7115551292372135,0.07057903958178535) -- (2.718385242912874,0.04715215693926389);
\draw[line width=1.1pt] (2.718385242912874,0.04715215693926389) -- (2.725215356588534,0.023631973391096572);
\draw[line width=1.1pt] (2.725215356588534,0.023631973391096572) -- (2.7320454702641945,0.0);
\draw (0.10330693079942738,3.8849611107787396) node[anchor=north west] {$x$};
\draw (3.0537763686955,0.5) node[anchor=north west] {$t$};
\draw (0.02,0.5) node[anchor=north west] {$O$};
\begin{scriptsize}
\draw [fill=black] (0.,0.) circle (1.1pt);
\end{scriptsize}
\end{tikzpicture}}
		\choice{\begin{tikzpicture}[line cap=round,line join=round,>=triangle 45,x=1.0cm,y=1.0cm, scale=1]
\draw[->,color=black] (-0.26126256871071957,0.) -- (3.6896534027248262,0.);
\foreach \x in {,0.5,1.,1.5,2.,2.5,3.,3.5}
\draw[shift={(\x,0)},color=black] (0pt,-2pt);
\draw[->,color=black] (0.,-0.2694355115461891) -- (0.,3.9273529130473617);
%\foreach \y in {,0.5,1.,1.5,2.,2.5,3.,3.5}
%\draw[shift={(0,\y)},color=black] (2pt,0pt) -- (-2pt,0pt);
\clip(-0.26126256871071957,-0.2694355115461891) rectangle (3.6896534027248262,3.9273529130473617);
\draw (0.10330693079942738,3.8849611107787396) node[anchor=north west] {$x$};
\draw (3.0537763686955,0.5105736501964502) node[anchor=north west] {$t$};
%\draw (0.0015666053547352067,0.0) node[anchor=north west] {$O$};
\draw[line width=1.1pt] (0.00922412450019114,0.0028097795864548553) -- (0.01844824900038228,0.005704643645704737);
\draw[line width=1.1pt] (0.01844824900038228,0.005704643645704737) -- (0.027672373500573423,0.008684592177749646);
\draw[line width=1.1pt] (0.027672373500573423,0.008684592177749646) -- (0.03689649800076456,0.01174962518258958);
\draw[line width=1.1pt] (0.03689649800076456,0.01174962518258958) -- (0.0461206225009557,0.01489974266022454);
\draw[line width=1.1pt] (0.0461206225009557,0.01489974266022454) -- (0.05534474700114684,0.018134944610654527);
\draw[line width=1.1pt] (0.05534474700114684,0.018134944610654527) -- (0.06456887150133798,0.021455231033879543);
\draw[line width=1.1pt] (0.06456887150133798,0.021455231033879543) -- (0.07379299600152912,0.024860601929899584);
\draw[line width=1.1pt] (0.07379299600152912,0.024860601929899584) -- (0.08301712050172026,0.02835105729871465);
\draw[line width=1.1pt] (0.08301712050172026,0.02835105729871465) -- (0.0922412450019114,0.03192659714032474);
\draw[line width=1.1pt] (0.0922412450019114,0.03192659714032474) -- (0.10146536950210254,0.03558722145472986);
\draw[line width=1.1pt] (0.10146536950210254,0.03558722145472986) -- (0.11068949400229368,0.039332930241930006);
\draw[line width=1.1pt] (0.11068949400229368,0.039332930241930006) -- (0.11991361850248482,0.04316372350192518);
\draw[line width=1.1pt] (0.11991361850248482,0.04316372350192518) -- (0.12913774300267597,0.047079601234715385);
\draw[line width=1.1pt] (0.12913774300267597,0.047079601234715385) -- (0.1383618675028671,0.05108056344030061);
\draw[line width=1.1pt] (0.1383618675028671,0.05108056344030061) -- (0.14758599200305825,0.05516661011868086);
\draw[line width=1.1pt] (0.14758599200305825,0.05516661011868086) -- (0.15681011650324939,0.05933774126985614);
\draw[line width=1.1pt] (0.15681011650324939,0.05933774126985614) -- (0.16603424100344052,0.06359395689382644);
\draw[line width=1.1pt] (0.16603424100344052,0.06359395689382644) -- (0.17525836550363166,0.06793525699059177);
\draw[line width=1.1pt] (0.17525836550363166,0.06793525699059177) -- (0.1844824900038228,0.07236164156015212);
\draw[line width=1.1pt] (0.1844824900038228,0.07236164156015212) -- (0.19370661450401394,0.07687311060250751);
\draw[line width=1.1pt] (0.19370661450401394,0.07687311060250751) -- (0.20293073900420508,0.08146966411765792);
\draw[line width=1.1pt] (0.20293073900420508,0.08146966411765792) -- (0.21215486350439622,0.08615130210560336);
\draw[line width=1.1pt] (0.21215486350439622,0.08615130210560336) -- (0.22137898800458736,0.09091802456634382);
\draw[line width=1.1pt] (0.22137898800458736,0.09091802456634382) -- (0.2306031125047785,0.0957698314998793);
\draw[line width=1.1pt] (0.2306031125047785,0.0957698314998793) -- (0.23982723700496963,0.10070672290620983);
\draw[line width=1.1pt] (0.23982723700496963,0.10070672290620983) -- (0.24905136150516077,0.10572869878533536);
\draw[line width=1.1pt] (0.24905136150516077,0.10572869878533536) -- (0.25827548600535194,0.11083575913725595);
\draw[line width=1.1pt] (0.25827548600535194,0.11083575913725595) -- (0.2674996105055431,0.11602790396197156);
\draw[line width=1.1pt] (0.2674996105055431,0.11602790396197156) -- (0.2767237350057343,0.1213051332594822);
\draw[line width=1.1pt] (0.2767237350057343,0.1213051332594822) -- (0.28594785950592544,0.12666744702978786);
\draw[line width=1.1pt] (0.28594785950592544,0.12666744702978786) -- (0.2951719840061166,0.13211484527288855);
\draw[line width=1.1pt] (0.2951719840061166,0.13211484527288855) -- (0.30439610850630777,0.13764732798878426);
\draw[line width=1.1pt] (0.30439610850630777,0.13764732798878426) -- (0.31362023300649894,0.143264895177475);
\draw[line width=1.1pt] (0.31362023300649894,0.143264895177475) -- (0.3228443575066901,0.1489675468389608);
\draw[line width=1.1pt] (0.3228443575066901,0.1489675468389608) -- (0.33206848200688127,0.1547552829732416);
\draw[line width=1.1pt] (0.33206848200688127,0.1547552829732416) -- (0.34129260650707244,0.16062810358031743);
\draw[line width=1.1pt] (0.34129260650707244,0.16062810358031743) -- (0.3505167310072636,0.16658600866018827);
\draw[line width=1.1pt] (0.3505167310072636,0.16658600866018827) -- (0.35974085550745477,0.17262899821285416);
\draw[line width=1.1pt] (0.35974085550745477,0.17262899821285416) -- (0.36896498000764594,0.17875707223831505);
\draw[line width=1.1pt] (0.36896498000764594,0.17875707223831505) -- (0.3781891045078371,0.18497023073657098);
\draw[line width=1.1pt] (0.3781891045078371,0.18497023073657098) -- (0.38741322900802827,0.19126847370762196);
\draw[line width=1.1pt] (0.38741322900802827,0.19126847370762196) -- (0.39663735350821944,0.19765180115146794);
\draw[line width=1.1pt] (0.39663735350821944,0.19765180115146794) -- (0.4058614780084106,0.20412021306810896);
\draw[line width=1.1pt] (0.4058614780084106,0.20412021306810896) -- (0.41508560250860177,0.210673709457545);
\draw[line width=1.1pt] (0.41508560250860177,0.210673709457545) -- (0.42430972700879294,0.21731229031977606);
\draw[line width=1.1pt] (0.42430972700879294,0.21731229031977606) -- (0.4335338515089841,0.22403595565480217);
\draw[line width=1.1pt] (0.4335338515089841,0.22403595565480217) -- (0.44275797600917527,0.23084470546262328);
\draw[line width=1.1pt] (0.44275797600917527,0.23084470546262328) -- (0.45198210050936644,0.23773853974323944);
\draw[line width=1.1pt] (0.45198210050936644,0.23773853974323944) -- (0.4612062250095576,0.2447174584966506);
\draw[line width=1.1pt] (0.4612062250095576,0.2447174584966506) -- (0.47043034950974877,0.2517814617228568);
\draw[line width=1.1pt] (0.47043034950974877,0.2517814617228568) -- (0.47965447400993994,0.25893054942185806);
\draw[line width=1.1pt] (0.47965447400993994,0.25893054942185806) -- (0.4888785985101311,0.2661647215936543);
\draw[line width=1.1pt] (0.4888785985101311,0.2661647215936543) -- (0.49810272301032227,0.2734839782382456);
\draw[line width=1.1pt] (0.49810272301032227,0.2734839782382456) -- (0.5073268475105134,0.28088831935563185);
\draw[line width=1.1pt] (0.5073268475105134,0.28088831935563185) -- (0.5165509720107045,0.28837774494581314);
\draw[line width=1.1pt] (0.5165509720107045,0.28837774494581314) -- (0.5257750965108957,0.2959522550087895);
\draw[line width=1.1pt] (0.5257750965108957,0.2959522550087895) -- (0.5349992210110869,0.30361184954456094);
\draw[line width=1.1pt] (0.5349992210110869,0.30361184954456094) -- (0.544223345511278,0.31135652855312734);
\draw[line width=1.1pt] (0.544223345511278,0.31135652855312734) -- (0.5534474700114692,0.3191862920344888);
\draw[line width=1.1pt] (0.5534474700114692,0.3191862920344888) -- (0.5626715945116604,0.32710113998864526);
\draw[line width=1.1pt] (0.5626715945116604,0.32710113998864526) -- (0.5718957190118515,0.3351010724155968);
\draw[line width=1.1pt] (0.5718957190118515,0.3351010724155968) -- (0.5811198435120427,0.34318608931534333);
\draw[line width=1.1pt] (0.5811198435120427,0.34318608931534333) -- (0.5903439680122339,0.3513561906878848);
\draw[line width=1.1pt] (0.5903439680122339,0.3513561906878848) -- (0.599568092512425,0.3596113765332214);
\draw[line width=1.1pt] (0.599568092512425,0.3596113765332214) -- (0.6087922170126162,0.36795164685135306);
\draw[line width=1.1pt] (0.6087922170126162,0.36795164685135306) -- (0.6180163415128074,0.37637700164227966);
\draw[line width=1.1pt] (0.6180163415128074,0.37637700164227966) -- (0.6272404660129985,0.38488744090600135);
\draw[line width=1.1pt] (0.6272404660129985,0.38488744090600135) -- (0.6364645905131897,0.39348296464251803);
\draw[line width=1.1pt] (0.6364645905131897,0.39348296464251803) -- (0.6456887150133809,0.40216357285182974);
\draw[line width=1.1pt] (0.6456887150133809,0.40216357285182974) -- (0.654912839513572,0.4109292655339365);
\draw[line width=1.1pt] (0.654912839513572,0.4109292655339365) -- (0.6641369640137632,0.4197800426888383);
\draw[line width=1.1pt] (0.6641369640137632,0.4197800426888383) -- (0.6733610885139544,0.42871590431653506);
\draw[line width=1.1pt] (0.6733610885139544,0.42871590431653506) -- (0.6825852130141455,0.43773685041702687);
\draw[line width=1.1pt] (0.6825852130141455,0.43773685041702687) -- (0.6918093375143367,0.4468428809903137);
\draw[line width=1.1pt] (0.6918093375143367,0.4468428809903137) -- (0.7010334620145279,0.4560339960363956);
\draw[line width=1.1pt] (0.7010334620145279,0.4560339960363956) -- (0.710257586514719,0.46531019555527253);
\draw[line width=1.1pt] (0.710257586514719,0.46531019555527253) -- (0.7194817110149102,0.47467147954694444);
\draw[line width=1.1pt] (0.7194817110149102,0.47467147954694444) -- (0.7287058355151014,0.48411784801141133);
\draw[line width=1.1pt] (0.7287058355151014,0.48411784801141133) -- (0.7379299600152925,0.4936493009486733);
\draw[line width=1.1pt] (0.7379299600152925,0.4936493009486733) -- (0.7471540845154837,0.5032658383587304);
\draw[line width=1.1pt] (0.7471540845154837,0.5032658383587304) -- (0.7563782090156749,0.5129674602415825);
\draw[line width=1.1pt] (0.7563782090156749,0.5129674602415825) -- (0.765602333515866,0.5227541665972295);
\draw[line width=1.1pt] (0.765602333515866,0.5227541665972295) -- (0.7748264580160572,0.5326259574256715);
\draw[line width=1.1pt] (0.7748264580160572,0.5326259574256715) -- (0.7840505825162484,0.5425828327269087);
\draw[line width=1.1pt] (0.7840505825162484,0.5425828327269087) -- (0.7932747070164395,0.5526247925009409);
\draw[line width=1.1pt] (0.7932747070164395,0.5526247925009409) -- (0.8024988315166307,0.562751836747768);
\draw[line width=1.1pt] (0.8024988315166307,0.562751836747768) -- (0.8117229560168219,0.5729639654673903);
\draw[line width=1.1pt] (0.8117229560168219,0.5729639654673903) -- (0.820947080517013,0.5832611786598074);
\draw[line width=1.1pt] (0.820947080517013,0.5832611786598074) -- (0.8301712050172042,0.5936434763250197);
\draw[line width=1.1pt] (0.8301712050172042,0.5936434763250197) -- (0.8393953295173954,0.604110858463027);
\draw[line width=1.1pt] (0.8393953295173954,0.604110858463027) -- (0.8486194540175865,0.6146633250738294);
\draw[line width=1.1pt] (0.8486194540175865,0.6146633250738294) -- (0.8578435785177777,0.6253008761574266);
\draw[line width=1.1pt] (0.8578435785177777,0.6253008761574266) -- (0.8670677030179689,0.636023511713819);
\draw[line width=1.1pt] (0.8670677030179689,0.636023511713819) -- (0.87629182751816,0.6468312317430064);
\draw[line width=1.1pt] (0.87629182751816,0.6468312317430064) -- (0.8855159520183512,0.6577240362449888);
\draw[line width=1.1pt] (0.8855159520183512,0.6577240362449888) -- (0.8947400765185424,0.6687019252197662);
\draw[line width=1.1pt] (0.8947400765185424,0.6687019252197662) -- (0.9039642010187335,0.6797648986673388);
\draw[line width=1.1pt] (0.9039642010187335,0.6797648986673388) -- (0.9131883255189247,0.6909129565877062);
\draw[line width=1.1pt] (0.9131883255189247,0.6909129565877062) -- (0.9224124500191159,0.7021460989808688);
\draw[line width=1.1pt] (0.9224124500191159,0.7021460989808688) -- (0.931636574519307,0.7134643258468263);
\draw[line width=1.1pt] (0.931636574519307,0.7134643258468263) -- (0.9408606990194982,0.7248676371855789);
\draw[line width=1.1pt] (0.9408606990194982,0.7248676371855789) -- (0.9500848235196894,0.7363560329971265);
\draw[line width=1.1pt] (0.9500848235196894,0.7363560329971265) -- (0.9593089480198805,0.7479295132814692);
\draw[line width=1.1pt] (0.9593089480198805,0.7479295132814692) -- (0.9685330725200717,0.7595880780386068);
\draw[line width=1.1pt] (0.9685330725200717,0.7595880780386068) -- (0.9777571970202629,0.7713317272685394);
\draw[line width=1.1pt] (0.9777571970202629,0.7713317272685394) -- (0.986981321520454,0.7831604609712672);
\draw[line width=1.1pt] (0.986981321520454,0.7831604609712672) -- (0.9962054460206452,0.79507427914679);
\draw[line width=1.1pt] (0.9962054460206452,0.79507427914679) -- (1.0054295705208363,0.8070731817951076);
\draw[line width=1.1pt] (1.0054295705208363,0.8070731817951076) -- (1.0146536950210274,0.8191571689162203);
\draw[line width=1.1pt] (1.0146536950210274,0.8191571689162203) -- (1.0238778195212186,0.831326240510128);
\draw[line width=1.1pt] (1.0238778195212186,0.831326240510128) -- (1.0331019440214098,0.8435803965768309);
\draw[line width=1.1pt] (1.0331019440214098,0.8435803965768309) -- (1.042326068521601,0.8559196371163289);
\draw[line width=1.1pt] (1.042326068521601,0.8559196371163289) -- (1.051550193021792,0.8683439621286217);
\draw[line width=1.1pt] (1.051550193021792,0.8683439621286217) -- (1.0607743175219833,0.8808533716137097);
\draw[line width=1.1pt] (1.0607743175219833,0.8808533716137097) -- (1.0699984420221744,0.8934478655715925);
\draw[line width=1.1pt] (1.0699984420221744,0.8934478655715925) -- (1.0792225665223656,0.9061274440022706);
\draw[line width=1.1pt] (1.0792225665223656,0.9061274440022706) -- (1.0884466910225568,0.9188921069057436);
\draw[line width=1.1pt] (1.0884466910225568,0.9188921069057436) -- (1.097670815522748,0.9317418542820115);
\draw[line width=1.1pt] (1.097670815522748,0.9317418542820115) -- (1.106894940022939,0.9446766861310747);
\draw[line width=1.1pt] (1.106894940022939,0.9446766861310747) -- (1.1161190645231303,0.9576966024529328);
\draw[line width=1.1pt] (1.1161190645231303,0.9576966024529328) -- (1.1253431890233214,0.9708016032475859);
\draw[line width=1.1pt] (1.1253431890233214,0.9708016032475859) -- (1.1345673135235126,0.9839916885150339);
\draw[line width=1.1pt] (1.1345673135235126,0.9839916885150339) -- (1.1437914380237038,0.9972668582552773);
\draw[line width=1.1pt] (1.1437914380237038,0.9972668582552773) -- (1.153015562523895,1.0106271124683155);
\draw[line width=1.1pt] (1.153015562523895,1.0106271124683155) -- (1.162239687024086,1.0240724511541486);
\draw[line width=1.1pt] (1.162239687024086,1.0240724511541486) -- (1.1714638115242773,1.0376028743127768);
\draw[line width=1.1pt] (1.1714638115242773,1.0376028743127768) -- (1.1806879360244684,1.0512183819442);
\draw[line width=1.1pt] (1.1806879360244684,1.0512183819442) -- (1.1899120605246596,1.0649189740484184);
\draw[line width=1.1pt] (1.1899120605246596,1.0649189740484184) -- (1.1991361850248508,1.0787046506254319);
\draw[line width=1.1pt] (1.1991361850248508,1.0787046506254319) -- (1.208360309525042,1.0925754116752402);
\draw[line width=1.1pt] (1.208360309525042,1.0925754116752402) -- (1.217584434025233,1.1065312571978434);
\draw[line width=1.1pt] (1.217584434025233,1.1065312571978434) -- (1.2268085585254243,1.120572187193242);
\draw[line width=1.1pt] (1.2268085585254243,1.120572187193242) -- (1.2360326830256154,1.1346982016614353);
\draw[line width=1.1pt] (1.2360326830256154,1.1346982016614353) -- (1.2452568075258066,1.1489093006024238);
\draw[line width=1.1pt] (1.2452568075258066,1.1489093006024238) -- (1.2544809320259978,1.1632054840162074);
\draw[line width=1.1pt] (1.2544809320259978,1.1632054840162074) -- (1.263705056526189,1.1775867519027858);
\draw[line width=1.1pt] (1.263705056526189,1.1775867519027858) -- (1.27292918102638,1.1920531042621594);
\draw[line width=1.1pt] (1.27292918102638,1.1920531042621594) -- (1.2821533055265713,1.206604541094328);
\draw[line width=1.1pt] (1.2821533055265713,1.206604541094328) -- (1.2913774300267624,1.2212410623992915);
\draw[line width=1.1pt] (1.2913774300267624,1.2212410623992915) -- (1.3006015545269536,1.2359626681770501);
\draw[line width=1.1pt] (1.3006015545269536,1.2359626681770501) -- (1.3098256790271448,1.2507693584276038);
\draw[line width=1.1pt] (1.3098256790271448,1.2507693584276038) -- (1.319049803527336,1.2656611331509524);
\draw[line width=1.1pt] (1.319049803527336,1.2656611331509524) -- (1.328273928027527,1.2806379923470963);
\draw[line width=1.1pt] (1.328273928027527,1.2806379923470963) -- (1.3374980525277183,1.295699936016035);
\draw[line width=1.1pt] (1.3374980525277183,1.295699936016035) -- (1.3467221770279094,1.3108469641577687);
\draw[line width=1.1pt] (1.3467221770279094,1.3108469641577687) -- (1.3559463015281006,1.3260790767722974);
\draw[line width=1.1pt] (1.3559463015281006,1.3260790767722974) -- (1.3651704260282918,1.3413962738596212);
\draw[line width=1.1pt] (1.3651704260282918,1.3413962738596212) -- (1.374394550528483,1.3567985554197401);
\draw[line width=1.1pt] (1.374394550528483,1.3567985554197401) -- (1.383618675028674,1.372285921452654);
\draw[line width=1.1pt] (1.383618675028674,1.372285921452654) -- (1.3928427995288652,1.3878583719583628);
\draw[line width=1.1pt] (1.3928427995288652,1.3878583719583628) -- (1.4020669240290564,1.4035159069368668);
\draw[line width=1.1pt] (1.4020669240290564,1.4035159069368668) -- (1.4112910485292476,1.4192585263881659);
\draw[line width=1.1pt] (1.4112910485292476,1.4192585263881659) -- (1.4205151730294387,1.4350862303122598);
\draw[line width=1.1pt] (1.4205151730294387,1.4350862303122598) -- (1.42973929752963,1.4509990187091488);
\draw[line width=1.1pt] (1.42973929752963,1.4509990187091488) -- (1.438963422029821,1.466996891578833);
\draw[line width=1.1pt] (1.438963422029821,1.466996891578833) -- (1.4481875465300122,1.483079848921312);
\draw[line width=1.1pt] (1.4481875465300122,1.483079848921312) -- (1.4574116710302034,1.499247890736586);
\draw[line width=1.1pt] (1.4574116710302034,1.499247890736586) -- (1.4666357955303946,1.515501017024655);
\draw[line width=1.1pt] (1.4666357955303946,1.515501017024655) -- (1.4758599200305857,1.5318392277855193);
\draw[line width=1.1pt] (1.4758599200305857,1.5318392277855193) -- (1.485084044530777,1.5482625230191784);
\draw[line width=1.1pt] (1.485084044530777,1.5482625230191784) -- (1.494308169030968,1.5647709027256327);
\draw[line width=1.1pt] (1.494308169030968,1.5647709027256327) -- (1.5035322935311592,1.5813643669048818);
\draw[line width=1.1pt] (1.5035322935311592,1.5813643669048818) -- (1.5127564180313504,1.598042915556926);
\draw[line width=1.1pt] (1.5127564180313504,1.598042915556926) -- (1.5219805425315416,1.6148065486817653);
\draw[line width=1.1pt] (1.5219805425315416,1.6148065486817653) -- (1.5312046670317327,1.6316552662793997);
\draw[line width=1.1pt] (1.5312046670317327,1.6316552662793997) -- (1.540428791531924,1.648589068349829);
\draw[line width=1.1pt] (1.540428791531924,1.648589068349829) -- (1.549652916032115,1.6656079548930534);
\draw[line width=1.1pt] (1.549652916032115,1.6656079548930534) -- (1.5588770405323062,1.6827119259090728);
\draw[line width=1.1pt] (1.5588770405323062,1.6827119259090728) -- (1.5681011650324974,1.6999009813978871);
\draw[line width=1.1pt] (1.5681011650324974,1.6999009813978871) -- (1.5773252895326886,1.7171751213594966);
\draw[line width=1.1pt] (1.5773252895326886,1.7171751213594966) -- (1.5865494140328797,1.734534345793901);
\draw[line width=1.1pt] (1.5865494140328797,1.734534345793901) -- (1.595773538533071,1.7519786547011005);
\draw[line width=1.1pt] (1.595773538533071,1.7519786547011005) -- (1.604997663033262,1.769508048081095);
\draw[line width=1.1pt] (1.604997663033262,1.769508048081095) -- (1.6142217875334532,1.7871225259338845);
\draw[line width=1.1pt] (1.6142217875334532,1.7871225259338845) -- (1.6234459120336444,1.8048220882594692);
\draw[line width=1.1pt] (1.6234459120336444,1.8048220882594692) -- (1.6326700365338356,1.8226067350578488);
\draw[line width=1.1pt] (1.6326700365338356,1.8226067350578488) -- (1.6418941610340267,1.8404764663290234);
\draw[line width=1.1pt] (1.6418941610340267,1.8404764663290234) -- (1.651118285534218,1.858431282072993);
\draw[line width=1.1pt] (1.651118285534218,1.858431282072993) -- (1.660342410034409,1.8764711822897577);
\draw[line width=1.1pt] (1.660342410034409,1.8764711822897577) -- (1.6695665345346002,1.8945961669793174);
\draw[line width=1.1pt] (1.6695665345346002,1.8945961669793174) -- (1.6787906590347914,1.9128062361416718);
\draw[line width=1.1pt] (1.6787906590347914,1.9128062361416718) -- (1.6880147835349826,1.9311013897768219);
\draw[line width=1.1pt] (1.6880147835349826,1.9311013897768219) -- (1.6972389080351737,1.9494816278847664);
\draw[line width=1.1pt] (1.6972389080351737,1.9494816278847664) -- (1.706463032535365,1.9679469504655063);
\draw[line width=1.1pt] (1.706463032535365,1.9679469504655063) -- (1.715687157035556,1.986497357519041);
\draw[line width=1.1pt] (1.715687157035556,1.986497357519041) -- (1.7249112815357472,2.005132849045371);
\draw[line width=1.1pt] (1.7249112815357472,2.005132849045371) -- (1.7341354060359384,2.023853425044496);
\draw[line width=1.1pt] (1.7341354060359384,2.023853425044496) -- (1.7433595305361296,2.0426590855164157);
\draw[line width=1.1pt] (1.7433595305361296,2.0426590855164157) -- (1.7525836550363207,2.061549830461131);
\draw[line width=1.1pt] (1.7525836550363207,2.061549830461131) -- (1.761807779536512,2.080525659878641);
\draw[line width=1.1pt] (1.761807779536512,2.080525659878641) -- (1.771031904036703,2.0995865737689456);
\draw[line width=1.1pt] (1.771031904036703,2.0995865737689456) -- (1.7802560285368942,2.118732572132046);
\draw[line width=1.1pt] (1.7802560285368942,2.118732572132046) -- (1.7894801530370854,2.137963654967941);
\draw[line width=1.1pt] (1.7894801530370854,2.137963654967941) -- (1.7987042775372766,2.1572798222766307);
\draw[line width=1.1pt] (1.7987042775372766,2.1572798222766307) -- (1.8079284020374677,2.176681074058116);
\draw[line width=1.1pt] (1.8079284020374677,2.176681074058116) -- (1.817152526537659,2.196167410312396);
\draw[line width=1.1pt] (1.817152526537659,2.196167410312396) -- (1.82637665103785,2.2157388310394714);
\draw[line width=1.1pt] (1.82637665103785,2.2157388310394714) -- (1.8356007755380412,2.2353953362393417);
\draw[line width=1.1pt] (1.8356007755380412,2.2353953362393417) -- (1.8448249000382324,2.255136925912007);
\draw[line width=1.1pt] (1.8448249000382324,2.255136925912007) -- (1.8540490245384236,2.274963600057467);
\draw[line width=1.1pt] (1.8540490245384236,2.274963600057467) -- (1.8632731490386147,2.2948753586757222);
\draw[line width=1.1pt] (1.8632731490386147,2.2948753586757222) -- (1.872497273538806,2.3148722017667724);
\draw[line width=1.1pt] (1.872497273538806,2.3148722017667724) -- (1.881721398038997,2.334954129330618);
\draw[line width=1.1pt] (1.881721398038997,2.334954129330618) -- (1.8909455225391882,2.3551211413672584);
\draw[line width=1.1pt] (1.8909455225391882,2.3551211413672584) -- (1.9001696470393794,2.3753732378766936);
\draw[line width=1.1pt] (1.9001696470393794,2.3753732378766936) -- (1.9093937715395706,2.395710418858924);
\draw[line width=1.1pt] (1.9093937715395706,2.395710418858924) -- (1.9186178960397617,2.4161326843139492);
\draw[line width=1.1pt] (1.9186178960397617,2.4161326843139492) -- (1.927842020539953,2.43664003424177);
\draw[line width=1.1pt] (1.927842020539953,2.43664003424177) -- (1.937066145040144,2.457232468642385);
\draw[line width=1.1pt] (1.937066145040144,2.457232468642385) -- (1.9462902695403352,2.4779099875157957);
\draw[line width=1.1pt] (1.9462902695403352,2.4779099875157957) -- (1.9555143940405264,2.4986725908620016);
\draw[line width=1.1pt] (1.9555143940405264,2.4986725908620016) -- (1.9647385185407176,2.519520278681002);
\draw[line width=1.1pt] (1.9647385185407176,2.519520278681002) -- (1.9739626430409087,2.5404530509727974);
\draw[line width=1.1pt] (1.9739626430409087,2.5404530509727974) -- (1.9831867675411,2.5614709077373883);
\draw[line width=1.1pt] (1.9831867675411,2.5614709077373883) -- (1.992410892041291,2.582573848974774);
\draw[line width=1.1pt] (1.992410892041291,2.582573848974774) -- (2.0016350165414822,2.6037618746849542);
\draw[line width=1.1pt] (2.0016350165414822,2.6037618746849542) -- (2.0108591410416734,2.62503498486793);
\draw[line width=1.1pt] (2.0108591410416734,2.62503498486793) -- (2.0200832655418646,2.646393179523701);
\draw[line width=1.1pt] (2.0200832655418646,2.646393179523701) -- (2.0293073900420557,2.6678364586522667);
\draw[line width=1.1pt] (2.0293073900420557,2.6678364586522667) -- (2.038531514542247,2.6893648222536273);
\draw[line width=1.1pt] (2.038531514542247,2.6893648222536273) -- (2.047755639042438,2.710978270327783);
\draw[line width=1.1pt] (2.047755639042438,2.710978270327783) -- (2.0569797635426292,2.732676802874734);
\draw[line width=1.1pt] (2.0569797635426292,2.732676802874734) -- (2.0662038880428204,2.75446041989448);
\draw[line width=1.1pt] (2.0662038880428204,2.75446041989448) -- (2.0754280125430116,2.776329121387021);
\draw[line width=1.1pt] (2.0754280125430116,2.776329121387021) -- (2.0846521370432027,2.7982829073523567);
\draw[line width=1.1pt] (2.0846521370432027,2.7982829073523567) -- (2.093876261543394,2.8203217777904874);
\draw[line width=1.1pt] (2.093876261543394,2.8203217777904874) -- (2.103100386043585,2.8424457327014134);
\draw[line width=1.1pt] (2.103100386043585,2.8424457327014134) -- (2.1123245105437762,2.8646547720851347);
\draw[line width=1.1pt] (2.1123245105437762,2.8646547720851347) -- (2.1215486350439674,2.8869488959416505);
\draw[line width=1.1pt] (2.1215486350439674,2.8869488959416505) -- (2.1307727595441586,2.9093281042709616);
\draw[line width=1.1pt] (2.1307727595441586,2.9093281042709616) -- (2.1399968840443497,2.931792397073068);
\draw[line width=1.1pt] (2.1399968840443497,2.931792397073068) -- (2.149221008544541,2.9543417743479687);
\draw[line width=1.1pt] (2.149221008544541,2.9543417743479687) -- (2.158445133044732,2.976976236095665);
\draw[line width=1.1pt] (2.158445133044732,2.976976236095665) -- (2.1676692575449232,2.9996957823161563);
\draw[line width=1.1pt] (2.1676692575449232,2.9996957823161563) -- (2.1768933820451144,3.022500413009442);
\draw[line width=1.1pt] (2.1768933820451144,3.022500413009442) -- (2.1861175065453056,3.0453901281755233);
\draw[line width=1.1pt] (2.1861175065453056,3.0453901281755233) -- (2.1953416310454967,3.0683649278144);
\draw[line width=1.1pt] (2.1953416310454967,3.0683649278144) -- (2.204565755545688,3.091424811926071);
\draw[line width=1.1pt] (2.204565755545688,3.091424811926071) -- (2.213789880045879,3.1145697805105375);
\draw[line width=1.1pt] (2.213789880045879,3.1145697805105375) -- (2.2230140045460702,3.1377998335677986);
\draw[line width=1.1pt] (2.2230140045460702,3.1377998335677986) -- (2.2322381290462614,3.161114971097855);
\draw[line width=1.1pt] (2.2322381290462614,3.161114971097855) -- (2.2414622535464526,3.1845151931007063);
\draw[line width=1.1pt] (2.2414622535464526,3.1845151931007063) -- (2.2506863780466437,3.208000499576353);
\draw[line width=1.1pt] (2.2506863780466437,3.208000499576353) -- (2.259910502546835,3.2315708905247944);
\draw[line width=1.1pt] (2.259910502546835,3.2315708905247944) -- (2.269134627047026,3.2552263659460308);
\draw[line width=1.1pt] (2.269134627047026,3.2552263659460308) -- (2.2783587515472172,3.278966925840062);
\draw[line width=1.1pt] (2.2783587515472172,3.278966925840062) -- (2.2875828760474084,3.3027925702068885);
\draw[line width=1.1pt] (2.2875828760474084,3.3027925702068885) -- (2.2968070005475996,3.3267032990465104);
\draw[line width=1.1pt] (2.2968070005475996,3.3267032990465104) -- (2.3060311250477907,3.3506991123589267);
\draw[line width=1.1pt] (2.3060311250477907,3.3506991123589267) -- (2.315255249547982,3.3747800101441383);
\draw[line width=1.1pt] (2.315255249547982,3.3747800101441383) -- (2.324479374048173,3.398945992402145);
\draw[line width=1.1pt] (2.324479374048173,3.398945992402145) -- (2.3337034985483642,3.4231970591329466);
\draw[line width=1.1pt] (2.3337034985483642,3.4231970591329466) -- (2.3429276230485554,3.4475332103365433);
\draw[line width=1.1pt] (2.3429276230485554,3.4475332103365433) -- (2.3521517475487466,3.471954446012935);
\draw[line width=1.1pt] (2.3521517475487466,3.471954446012935) -- (2.3613758720489377,3.4964607661621216);
\draw[line width=1.1pt] (2.3613758720489377,3.4964607661621216) -- (2.370599996549129,3.5210521707841034);
\draw[line width=1.1pt] (2.370599996549129,3.5210521707841034) -- (2.37982412104932,3.5457286598788804);
\draw[line width=1.1pt] (2.37982412104932,3.5457286598788804) -- (2.3890482455495112,3.570490233446452);
\draw[line width=1.1pt] (2.3890482455495112,3.570490233446452) -- (2.3982723700497024,3.5953368914868187);
\draw[line width=1.1pt] (2.3982723700497024,3.5953368914868187) -- (2.4074964945498936,3.620268633999981);
\draw[line width=1.1pt] (2.4074964945498936,3.620268633999981) -- (2.4167206190500847,3.6452854609859378);
\draw[line width=1.1pt] (2.4167206190500847,3.6452854609859378) -- (2.425944743550276,3.6703873724446896);
\draw[line width=1.1pt] (2.425944743550276,3.6703873724446896) -- (2.435168868050467,3.6955743683762363);
\draw[line width=1.1pt] (2.435168868050467,3.6955743683762363) -- (2.4443929925506582,3.7208464487805784);
\draw[line width=1.1pt] (2.4443929925506582,3.7208464487805784) -- (2.4536171170508494,3.7462036136577153);
\draw[line width=1.1pt] (2.4536171170508494,3.7462036136577153) -- (2.4628412415510406,3.7716458630076475);
\draw[line width=1.1pt] (2.4628412415510406,3.7716458630076475) -- (2.4720653660512317,3.7971731968303746);
\draw[line width=1.1pt] (2.4720653660512317,3.7971731968303746) -- (2.481289490551423,3.8227856151258965);
\draw[line width=1.1pt] (2.481289490551423,3.8227856151258965) -- (2.490513615051614,3.848483117894214);
\draw[line width=1.1pt] (2.490513615051614,3.848483117894214) -- (2.4997377395518052,3.874265705135326);
\draw[line width=1.1pt] (2.4997377395518052,3.874265705135326) -- (2.5089618640519964,3.900133376849233);
\draw[line width=1.1pt] (2.5089618640519964,3.900133376849233) -- (2.5181859885521876,3.9260861330359353);
\draw[line width=1.1pt] (2.5181859885521876,3.9260861330359353) -- (2.5274101130523787,3.9521239736954326);
\draw[line width=1.1pt] (2.5274101130523787,3.9521239736954326) -- (2.53663423755257,3.9782468988277246);
\draw[line width=1.1pt] (2.53663423755257,3.9782468988277246) -- (2.545858362052761,4.0044549084328125);
\draw[line width=1.1pt] (2.545858362052761,4.0044549084328125) -- (2.5550824865529522,4.030748002510695);
\draw[line width=1.1pt] (2.5550824865529522,4.030748002510695) -- (2.5643066110531434,4.0571261810613715);
\draw[line width=1.1pt] (2.5643066110531434,4.0571261810613715) -- (2.5735307355533346,4.083589444084844);
\draw[line width=1.1pt] (2.5735307355533346,4.083589444084844) -- (2.5827548600535257,4.110137791581112);
\draw[line width=1.1pt] (2.5827548600535257,4.110137791581112) -- (2.591978984553717,4.1367712235501735);
\draw[line width=1.1pt] (2.591978984553717,4.1367712235501735) -- (2.601203109053908,4.163489739992031);
\draw[line width=1.1pt] (2.601203109053908,4.163489739992031) -- (2.6104272335540992,4.190293340906684);
\draw[line width=1.1pt] (2.6104272335540992,4.190293340906684) -- (2.6196513580542904,4.2171820262941315);
\draw[line width=1.1pt] (2.6196513580542904,4.2171820262941315) -- (2.6288754825544816,4.244155796154374);
\draw[line width=1.1pt] (2.6288754825544816,4.244155796154374) -- (2.6380996070546727,4.271214650487411);
\draw[line width=1.1pt] (2.6380996070546727,4.271214650487411) -- (2.647323731554864,4.298358589293244);
\draw[line width=1.1pt] (2.647323731554864,4.298358589293244) -- (2.656547856055055,4.325587612571871);
\draw[line width=1.1pt] (2.656547856055055,4.325587612571871) -- (2.6657719805552462,4.352901720323294);
\draw[line width=1.1pt] (2.6657719805552462,4.352901720323294) -- (2.6749961050554374,4.380300912547511);
\draw[line width=1.1pt] (2.6749961050554374,4.380300912547511) -- (2.6842202295556286,4.407785189244525);
\draw[line width=1.1pt] (2.6842202295556286,4.407785189244525) -- (2.6934443540558197,4.435354550414332);
\draw[line width=1.1pt] (2.6934443540558197,4.435354550414332) -- (2.702668478556011,4.463008996056935);
\draw[line width=1.1pt] (2.702668478556011,4.463008996056935) -- (2.711892603056202,4.490748526172332);
\draw[line width=1.1pt] (2.711892603056202,4.490748526172332) -- (2.7211167275563932,4.518573140760525);
\draw[line width=1.1pt] (2.7211167275563932,4.518573140760525) -- (2.7303408520565844,4.546482839821513);
\draw[line width=1.1pt] (2.7303408520565844,4.546482839821513) -- (2.7395649765567756,4.574477623355296);
\draw[line width=1.1pt] (2.7395649765567756,4.574477623355296) -- (2.7487891010569667,4.602557491361874);
\draw[line width=1.1pt] (2.7487891010569667,4.602557491361874) -- (2.758013225557158,4.630722443841247);
\draw[line width=1.1pt] (2.758013225557158,4.630722443841247) -- (2.767237350057349,4.658972480793414);
\draw[line width=1.1pt] (2.767237350057349,4.658972480793414) -- (2.77646147455754,4.687307602218377);
\draw[line width=1.1pt] (2.77646147455754,4.687307602218377) -- (2.7856855990577314,4.715727808116135);
\draw[line width=1.1pt] (2.7856855990577314,4.715727808116135) -- (2.7949097235579226,4.7442330984866885);
\draw[line width=1.1pt] (2.7949097235579226,4.7442330984866885) -- (2.8041338480581137,4.772823473330036);
\draw[line width=1.1pt] (2.8041338480581137,4.772823473330036) -- (2.813357972558305,4.801498932646179);
\draw[line width=1.1pt] (2.813357972558305,4.801498932646179) -- (2.822582097058496,4.830259476435118);
\draw[line width=1.1pt] (2.822582097058496,4.830259476435118) -- (2.831806221558687,4.859105104696851);
\draw[line width=1.1pt] (2.831806221558687,4.859105104696851) -- (2.8410303460588784,4.888035817431379);
\draw[line width=1.1pt] (2.8410303460588784,4.888035817431379) -- (2.8502544705590696,4.917051614638702);
\draw[line width=1.1pt] (2.8502544705590696,4.917051614638702) -- (2.8594785950592607,4.94615249631882);
\draw[line width=1.1pt] (2.8594785950592607,4.94615249631882) -- (2.868702719559452,4.975338462471734);
\draw[line width=1.1pt] (2.868702719559452,4.975338462471734) -- (2.877926844059643,5.004609513097441);
\draw[line width=1.1pt] (2.877926844059643,5.004609513097441) -- (2.887150968559834,5.033965648195945);
\draw[line width=1.1pt] (2.887150968559834,5.033965648195945) -- (2.8963750930600254,5.0634068677672435);
\draw[line width=1.1pt] (2.8963750930600254,5.0634068677672435) -- (2.9055992175602166,5.092933171811336);
\draw[line width=1.1pt] (2.9055992175602166,5.092933171811336) -- (2.9148233420604077,5.122544560328225);
\draw[line width=1.1pt] (2.9148233420604077,5.122544560328225) -- (2.924047466560599,5.1522410333179085);
\draw[line width=1.1pt] (2.924047466560599,5.1522410333179085) -- (2.93327159106079,5.182022590780386);
\draw[line width=1.1pt] (2.93327159106079,5.182022590780386) -- (2.942495715560981,5.21188923271566);
\draw[line width=1.1pt] (2.942495715560981,5.21188923271566) -- (2.9517198400611724,5.241840959123729);
\draw[line width=1.1pt] (2.9517198400611724,5.241840959123729) -- (2.9609439645613636,5.271877770004592);
\draw[line width=1.1pt] (2.9609439645613636,5.271877770004592) -- (2.9701680890615547,5.301999665358251);
\draw[line width=1.1pt] (2.9701680890615547,5.301999665358251) -- (2.979392213561746,5.332206645184704);
\draw[line width=1.1pt] (2.979392213561746,5.332206645184704) -- (2.988616338061937,5.362498709483953);
\draw[line width=1.1pt] (2.988616338061937,5.362498709483953) -- (2.997840462562128,5.392875858255996);
\draw[line width=1.1pt] (2.997840462562128,5.392875858255996) -- (3.0070645870623194,5.423338091500835);
\draw[line width=1.1pt] (3.0070645870623194,5.423338091500835) -- (3.0162887115625105,5.4538854092184685);
\draw[line width=1.1pt] (3.0162887115625105,5.4538854092184685) -- (3.0255128360627017,5.484517811408897);
\draw[line width=1.1pt] (3.0255128360627017,5.484517811408897) -- (3.034736960562893,5.51523529807212);
\draw[line width=1.1pt] (3.034736960562893,5.51523529807212) -- (3.043961085063084,5.546037869208139);
\draw[line width=1.1pt] (3.043961085063084,5.546037869208139) -- (3.053185209563275,5.576925524816954);
\draw[line width=1.1pt] (3.053185209563275,5.576925524816954) -- (3.0624093340634664,5.607898264898562);
\draw[line width=1.1pt] (3.0624093340634664,5.607898264898562) -- (3.0716334585636575,5.638956089452966);
\draw[line width=1.1pt] (3.0716334585636575,5.638956089452966) -- (3.0808575830638487,5.670098998480165);
\draw[line width=1.1pt] (3.0808575830638487,5.670098998480165) -- (3.09008170756404,5.701326991980158);
\draw[line width=1.1pt] (3.09008170756404,5.701326991980158) -- (3.099305832064231,5.732640069952947);
\draw[line width=1.1pt] (3.099305832064231,5.732640069952947) -- (3.108529956564422,5.764038232398531);
\draw[line width=1.1pt] (3.108529956564422,5.764038232398531) -- (3.1177540810646134,5.7955214793169105);
\draw[line width=1.1pt] (3.1177540810646134,5.7955214793169105) -- (3.1269782055648045,5.827089810708084);
\draw[line width=1.1pt] (3.1269782055648045,5.827089810708084) -- (3.1362023300649957,5.858743226572053);
\draw[line width=1.1pt] (3.1362023300649957,5.858743226572053) -- (3.145426454565187,5.890481726908817);
\draw[line width=1.1pt] (3.145426454565187,5.890481726908817) -- (3.154650579065378,5.922305311718376);
\draw[line width=1.1pt] (3.154650579065378,5.922305311718376) -- (3.163874703565569,5.954213981000731);
\draw[line width=1.1pt] (3.163874703565569,5.954213981000731) -- (3.1730988280657604,5.9862077347558795);
\draw[line width=1.1pt] (3.1730988280657604,5.9862077347558795) -- (3.1823229525659515,6.018286572983824);
\draw[line width=1.1pt] (3.1823229525659515,6.018286572983824) -- (3.1915470770661427,6.050450495684562);
\draw[line width=1.1pt] (3.1915470770661427,6.050450495684562) -- (3.200771201566334,6.082699502858097);
\draw[line width=1.1pt] (3.200771201566334,6.082699502858097) -- (3.209995326066525,6.115033594504426);
\draw[line width=1.1pt] (3.209995326066525,6.115033594504426) -- (3.219219450566716,6.14745277062355);
\draw[line width=1.1pt] (3.219219450566716,6.14745277062355) -- (3.2284435750669074,6.179957031215469);
\draw[line width=1.1pt] (3.2284435750669074,6.179957031215469) -- (3.2376676995670985,6.212546376280184);
\draw[line width=1.1pt] (3.2376676995670985,6.212546376280184) -- (3.2468918240672897,6.245220805817693);
\draw[line width=1.1pt] (3.2468918240672897,6.245220805817693) -- (3.256115948567481,6.277980319827997);
\draw[line width=1.1pt] (3.256115948567481,6.277980319827997) -- (3.265340073067672,6.310824918311097);
\draw[line width=1.1pt] (3.265340073067672,6.310824918311097) -- (3.274564197567863,6.343754601266991);
\draw[line width=1.1pt] (3.274564197567863,6.343754601266991) -- (3.2837883220680544,6.376769368695681);
\draw[line width=1.1pt] (3.2837883220680544,6.376769368695681) -- (3.2930124465682455,6.409869220597165);
\draw[line width=1.1pt] (3.2930124465682455,6.409869220597165) -- (3.3022365710684367,6.443054156971445);
\draw[line width=1.1pt] (3.3022365710684367,6.443054156971445) -- (3.311460695568628,6.476324177818519);
\draw[line width=1.1pt] (3.311460695568628,6.476324177818519) -- (3.320684820068819,6.5096792831383885);
\draw[line width=1.1pt] (3.320684820068819,6.5096792831383885) -- (3.32990894456901,6.543119472931053);
\draw[line width=1.1pt] (3.32990894456901,6.543119472931053) -- (3.3391330690692014,6.576644747196513);
\draw[line width=1.1pt] (3.3391330690692014,6.576644747196513) -- (3.3483571935693925,6.610255105934767);
\draw[line width=1.1pt] (3.3483571935693925,6.610255105934767) -- (3.3575813180695837,6.643950549145817);
\draw[line width=1.1pt] (3.3575813180695837,6.643950549145817) -- (3.366805442569775,6.677731076829661);
\draw[line width=1.1pt] (3.366805442569775,6.677731076829661) -- (3.376029567069966,6.711596688986301);
\draw[line width=1.1pt] (3.376029567069966,6.711596688986301) -- (3.385253691570157,6.745547385615736);
\draw[line width=1.1pt] (3.385253691570157,6.745547385615736) -- (3.3944778160703484,6.779583166717965);
\draw[line width=1.1pt] (3.3944778160703484,6.779583166717965) -- (3.4037019405705395,6.81370403229299);
\draw[line width=1.1pt] (3.4037019405705395,6.81370403229299) -- (3.4129260650707307,6.84790998234081);
\draw[line width=1.1pt] (3.4129260650707307,6.84790998234081) -- (3.422150189570922,6.882201016861425);
\draw[line width=1.1pt] (3.422150189570922,6.882201016861425) -- (3.431374314071113,6.916577135854835);
\draw[line width=1.1pt] (3.431374314071113,6.916577135854835) -- (3.440598438571304,6.95103833932104);
\draw[line width=1.1pt] (3.440598438571304,6.95103833932104) -- (3.4498225630714954,6.98558462726004);
\draw[line width=1.1pt] (3.4498225630714954,6.98558462726004) -- (3.4590466875716865,7.0202159996718345);
\draw[line width=1.1pt] (3.4590466875716865,7.0202159996718345) -- (3.4682708120718777,7.054932456556425);
\draw[line width=1.1pt] (3.4682708120718777,7.054932456556425) -- (3.477494936572069,7.089733997913809);
\draw[line width=1.1pt] (3.477494936572069,7.089733997913809) -- (3.48671906107226,7.12462062374399);
\draw[line width=1.1pt] (3.48671906107226,7.12462062374399) -- (3.495943185572451,7.159592334046964);
\draw[line width=1.1pt] (3.495943185572451,7.159592334046964) -- (3.5051673100726424,7.194649128822735);
\draw[line width=1.1pt] (3.5051673100726424,7.194649128822735) -- (3.5143914345728335,7.2297910080713);
\draw[line width=1.1pt] (3.5143914345728335,7.2297910080713) -- (3.5236155590730247,7.26501797179266);
\draw[line width=1.1pt] (3.5236155590730247,7.26501797179266) -- (3.532839683573216,7.300330019986815);
\draw[line width=1.1pt] (3.532839683573216,7.300330019986815) -- (3.542063808073407,7.3357271526537655);
\draw[line width=1.1pt] (3.542063808073407,7.3357271526537655) -- (3.551287932573598,7.371209369793511);
\draw[line width=1.1pt] (3.551287932573598,7.371209369793511) -- (3.5605120570737894,7.406776671406051);
\draw[line width=1.1pt] (3.5605120570737894,7.406776671406051) -- (3.5697361815739805,7.442429057491386);
\draw[line width=1.1pt] (3.5697361815739805,7.442429057491386) -- (3.5789603060741717,7.478166528049516);
\draw[line width=1.1pt] (3.5789603060741717,7.478166528049516) -- (3.588184430574363,7.513989083080442);
\draw[line width=1.1pt] (3.588184430574363,7.513989083080442) -- (3.597408555074554,7.549896722584162);
\draw[line width=1.1pt] (3.597408555074554,7.549896722584162) -- (3.606632679574745,7.585889446560677);
\draw[line width=1.1pt] (3.606632679574745,7.585889446560677) -- (3.6158568040749364,7.621967255009988);
\draw[line width=1.1pt] (3.6158568040749364,7.621967255009988) -- (3.6250809285751275,7.658130147932093);
\draw[line width=1.1pt] (3.6250809285751275,7.658130147932093) -- (3.6343050530753187,7.694378125326994);
\draw[line width=1.1pt] (3.6343050530753187,7.694378125326994) -- (3.64352917757551,7.7307111871946885);
\draw[line width=1.1pt] (3.64352917757551,7.7307111871946885) -- (3.652753302075701,7.76712933353518);
\draw[line width=1.1pt] (3.652753302075701,7.76712933353518) -- (3.661977426575892,7.803632564348465);
\draw[line width=1.1pt] (3.661977426575892,7.803632564348465) -- (3.6712015510760834,7.840220879634545);
\draw[line width=1.1pt] (3.6712015510760834,7.840220879634545) -- (3.6804256755762745,7.876894279393421);
\draw[line width=1.1pt] (3.6804256755762745,7.876894279393421) -- (3.6896498000764657,7.913652763625092);
\begin{scriptsize}
\draw [fill=black] (0.,0.) circle (1.5pt);
\end{scriptsize}
\end{tikzpicture}}
		\choice[correct]{% !TEX root = ../oefeningen_fys6.tex
\begin{tikzpicture}[line cap=round,line join=round,>=triangle 45,x=1.0cm,y=1.0cm, scale=0.8]
\draw[->,color=black] (-0.17647896417347608,0.) -- (3.77443700726207,0.);
\foreach \x in {,0.5,1.,1.5,2.,2.5,3.,3.5}
\draw[shift={(\x,0)},color=black] (0pt,-2pt);
\draw[->,color=black] (0.,-2.838378729024666) -- (0.,1.3584096955688847);
%\foreach \y in {-2.5,-2.,-1.5,-1.,-0.5,0.5,1.}
%\draw[shift={(0,\y)},color=black] (2pt,0pt) -- (-2pt,0pt);
\clip(-0.17647896417347608,-2.838378729024666) rectangle (3.77443700726207,1.3584096955688847);
\draw (0.10330693079942738,1.4686283814673011) node[anchor=north west] {$x$};
\draw (3.0537763686955,0.5105736501964502) node[anchor=north west] {$t$};
\draw (0.02,0.6) node[anchor=north west] {$O$};
\draw[line width=1.2pt] (0.009436091142058657,0.0) -- (0.018872182284117314,0.0);
\draw[line width=1.2pt] (0.018872182284117314,0.0) -- (0.02830827342617597,0.0);
\draw[line width=1.2pt] (0.02830827342617597,0.0) -- (0.03774436456823463,0.0);
\draw[line width=1.2pt] (0.03774436456823463,0.0) -- (0.047180455710293286,0.0);
\draw[line width=1.2pt] (0.047180455710293286,0.0) -- (0.05661654685235194,0.0);
\draw[line width=1.2pt] (0.05661654685235194,0.0) -- (0.0660526379944106,0.0);
\draw[line width=1.2pt] (0.0660526379944106,0.0) -- (0.07548872913646926,0.0);
\draw[line width=1.2pt] (0.07548872913646926,0.0) -- (0.08492482027852791,-0.0010818337649010398);
\draw[line width=1.2pt] (0.08492482027852791,-0.0010818337649010398) -- (0.09436091142058656,-0.0013355972406185674);
\draw[line width=1.2pt] (0.09436091142058656,-0.0013355972406185674) -- (0.10379700256264521,-0.0016160726611484664);
\draw[line width=1.2pt] (0.10379700256264521,-0.0016160726611484664) -- (0.11323309370470386,-0.0019232600264907367);
\draw[line width=1.2pt] (0.11323309370470386,-0.0019232600264907367) -- (0.12266918484676251,-0.002257159336645378);
\draw[line width=1.2pt] (0.12266918484676251,-0.002257159336645378) -- (0.13210527598882116,-0.002617770591612391);
\draw[line width=1.2pt] (0.13210527598882116,-0.002617770591612391) -- (0.1415413671308798,-0.0030050937913917754);
\draw[line width=1.2pt] (0.1415413671308798,-0.0030050937913917754) -- (0.15097745827293846,-0.0034191289359835307);
\draw[line width=1.2pt] (0.15097745827293846,-0.0034191289359835307) -- (0.1604135494149971,-0.0038598760253876576);
\draw[line width=1.2pt] (0.1604135494149971,-0.0038598760253876576) -- (0.16984964055705576,-0.004327335059604156);
\draw[line width=1.2pt] (0.16984964055705576,-0.004327335059604156) -- (0.1792857316991144,-0.004821506038633025);
\draw[line width=1.2pt] (0.1792857316991144,-0.004821506038633025) -- (0.18872182284117306,-0.005342388962474266);
\draw[line width=1.2pt] (0.18872182284117306,-0.005342388962474266) -- (0.1981579139832317,-0.005889983831127878);
\draw[line width=1.2pt] (0.1981579139832317,-0.005889983831127878) -- (0.20759400512529036,-0.006464290644593861);
\draw[line width=1.2pt] (0.20759400512529036,-0.006464290644593861) -- (0.217030096267349,-0.007065309402872217);
\draw[line width=1.2pt] (0.217030096267349,-0.007065309402872217) -- (0.22646618740940766,-0.007693040105962942);
\draw[line width=1.2pt] (0.22646618740940766,-0.007693040105962942) -- (0.2359022785514663,-0.00834748275386604);
\draw[line width=1.2pt] (0.2359022785514663,-0.00834748275386604) -- (0.24533836969352496,-0.009028637346581509);
\draw[line width=1.2pt] (0.24533836969352496,-0.009028637346581509) -- (0.25477446083558364,-0.00973650388410935);
\draw[line width=1.2pt] (0.25477446083558364,-0.00973650388410935) -- (0.2642105519776423,-0.010471082366449565);
\draw[line width=1.2pt] (0.2642105519776423,-0.010471082366449565) -- (0.273646643119701,-0.01123237279360215);
\draw[line width=1.2pt] (0.273646643119701,-0.01123237279360215) -- (0.28308273426175967,-0.012020375165567107);
\draw[line width=1.2pt] (0.28308273426175967,-0.012020375165567107) -- (0.29251882540381835,-0.012835089482344434);
\draw[line width=1.2pt] (0.29251882540381835,-0.012835089482344434) -- (0.30195491654587703,-0.013676515743934133);
\draw[line width=1.2pt] (0.30195491654587703,-0.013676515743934133) -- (0.3113910076879357,-0.014544653950336205);
\draw[line width=1.2pt] (0.3113910076879357,-0.014544653950336205) -- (0.3208270988299944,-0.015439504101550648);
\draw[line width=1.2pt] (0.3208270988299944,-0.015439504101550648) -- (0.33026318997205306,-0.016361066197577462);
\draw[line width=1.2pt] (0.33026318997205306,-0.016361066197577462) -- (0.33969928111411174,-0.017309340238416647);
\draw[line width=1.2pt] (0.33969928111411174,-0.017309340238416647) -- (0.3491353722561704,-0.0182843262240682);
\draw[line width=1.2pt] (0.3491353722561704,-0.0182843262240682) -- (0.3585714633982291,-0.01928602415453213);
\draw[line width=1.2pt] (0.3585714633982291,-0.01928602415453213) -- (0.3680075545402878,-0.02031443402980843);
\draw[line width=1.2pt] (0.3680075545402878,-0.02031443402980843) -- (0.37744364568234645,-0.021369555849897102);
\draw[line width=1.2pt] (0.37744364568234645,-0.021369555849897102) -- (0.38687973682440513,-0.02245138961479815);
\draw[line width=1.2pt] (0.38687973682440513,-0.02245138961479815) -- (0.3963158279664638,-0.023559935324511557);
\draw[line width=1.2pt] (0.3963158279664638,-0.023559935324511557) -- (0.4057519191085225,-0.024695192979037345);
\draw[line width=1.2pt] (0.4057519191085225,-0.024695192979037345) -- (0.41518801025058116,-0.025857162578375503);
\draw[line width=1.2pt] (0.41518801025058116,-0.025857162578375503) -- (0.42462410139263984,-0.02704584412252603);
\draw[line width=1.2pt] (0.42462410139263984,-0.02704584412252603) -- (0.4340601925346985,-0.02826123761148893);
\draw[line width=1.2pt] (0.4340601925346985,-0.02826123761148893) -- (0.4434962836767572,-0.029503343045264203);
\draw[line width=1.2pt] (0.4434962836767572,-0.029503343045264203) -- (0.4529323748188159,-0.030772160423851842);
\draw[line width=1.2pt] (0.4529323748188159,-0.030772160423851842) -- (0.46236846596087455,-0.03206768974725186);
\draw[line width=1.2pt] (0.46236846596087455,-0.03206768974725186) -- (0.47180455710293323,-0.033389931015464246);
\draw[line width=1.2pt] (0.47180455710293323,-0.033389931015464246) -- (0.4812406482449919,-0.034738884228489);
\draw[line width=1.2pt] (0.4812406482449919,-0.034738884228489) -- (0.4906767393870506,-0.036114549386326134);
\draw[line width=1.2pt] (0.4906767393870506,-0.036114549386326134) -- (0.5001128305291093,-0.03751692648897564);
\draw[line width=1.2pt] (0.5001128305291093,-0.03751692648897564) -- (0.5095489216711679,-0.038946015536437506);
\draw[line width=1.2pt] (0.5095489216711679,-0.038946015536437506) -- (0.5189850128132266,-0.04040181652871175);
\draw[line width=1.2pt] (0.5189850128132266,-0.04040181652871175) -- (0.5284211039552853,-0.04188432946579836);
\draw[line width=1.2pt] (0.5284211039552853,-0.04188432946579836) -- (0.537857195097344,-0.04339355434769735);
\draw[line width=1.2pt] (0.537857195097344,-0.04339355434769735) -- (0.5472932862394027,-0.04492949117440871);
\draw[line width=1.2pt] (0.5472932862394027,-0.04492949117440871) -- (0.5567293773814613,-0.04649213994593244);
\draw[line width=1.2pt] (0.5567293773814613,-0.04649213994593244) -- (0.56616546852352,-0.04808150066226854);
\draw[line width=1.2pt] (0.56616546852352,-0.04808150066226854) -- (0.5756015596655787,-0.04969757332341701);
\draw[line width=1.2pt] (0.5756015596655787,-0.04969757332341701) -- (0.5850376508076374,-0.05134035792937785);
\draw[line width=1.2pt] (0.5850376508076374,-0.05134035792937785) -- (0.594473741949696,-0.05300985448015107);
\draw[line width=1.2pt] (0.594473741949696,-0.05300985448015107) -- (0.6039098330917547,-0.05470606297573666);
\draw[line width=1.2pt] (0.6039098330917547,-0.05470606297573666) -- (0.6133459242338134,-0.056428983416134615);
\draw[line width=1.2pt] (0.6133459242338134,-0.056428983416134615) -- (0.6227820153758721,-0.058178615801344945);
\draw[line width=1.2pt] (0.6227820153758721,-0.058178615801344945) -- (0.6322181065179308,-0.05995496013136764);
\draw[line width=1.2pt] (0.6322181065179308,-0.05995496013136764) -- (0.6416541976599894,-0.061758016406202716);
\draw[line width=1.2pt] (0.6416541976599894,-0.061758016406202716) -- (0.6510902888020481,-0.06358778462585016);
\draw[line width=1.2pt] (0.6510902888020481,-0.06358778462585016) -- (0.6605263799441068,-0.06544426479030997);
\draw[line width=1.2pt] (0.6605263799441068,-0.06544426479030997) -- (0.6699624710861655,-0.06732745689958215);
\draw[line width=1.2pt] (0.6699624710861655,-0.06732745689958215) -- (0.6793985622282241,-0.06923736095366671);
\draw[line width=1.2pt] (0.6793985622282241,-0.06923736095366671) -- (0.6888346533702828,-0.07117397695256365);
\draw[line width=1.2pt] (0.6888346533702828,-0.07117397695256365) -- (0.6982707445123415,-0.07313730489627296);
\draw[line width=1.2pt] (0.6982707445123415,-0.07313730489627296) -- (0.7077068356544002,-0.07512734478479463);
\draw[line width=1.2pt] (0.7077068356544002,-0.07512734478479463) -- (0.7171429267964589,-0.07714409661812867);
\draw[line width=1.2pt] (0.7171429267964589,-0.07714409661812867) -- (0.7265790179385175,-0.07918756039627509);
\draw[line width=1.2pt] (0.7265790179385175,-0.07918756039627509) -- (0.7360151090805762,-0.08125773611923387);
\draw[line width=1.2pt] (0.7360151090805762,-0.08125773611923387) -- (0.7454512002226349,-0.08335462378700503);
\draw[line width=1.2pt] (0.7454512002226349,-0.08335462378700503) -- (0.7548872913646936,-0.08547822339958856);
\draw[line width=1.2pt] (0.7548872913646936,-0.08547822339958856) -- (0.7643233825067522,-0.08762853495698447);
\draw[line width=1.2pt] (0.7643233825067522,-0.08762853495698447) -- (0.7737594736488109,-0.08980555845919273);
\draw[line width=1.2pt] (0.7737594736488109,-0.08980555845919273) -- (0.7831955647908696,-0.09200929390621339);
\draw[line width=1.2pt] (0.7831955647908696,-0.09200929390621339) -- (0.7926316559329283,-0.0942397412980464);
\draw[line width=1.2pt] (0.7926316559329283,-0.0942397412980464) -- (0.802067747074987,-0.0964969006346918);
\draw[line width=1.2pt] (0.802067747074987,-0.0964969006346918) -- (0.8115038382170456,-0.09878077191614955);
\draw[line width=1.2pt] (0.8115038382170456,-0.09878077191614955) -- (0.8209399293591043,-0.10109135514241967);
\draw[line width=1.2pt] (0.8209399293591043,-0.10109135514241967) -- (0.830376020501163,-0.10342865031350218);
\draw[line width=1.2pt] (0.830376020501163,-0.10342865031350218) -- (0.8398121116432217,-0.10579265742939706);
\draw[line width=1.2pt] (0.8398121116432217,-0.10579265742939706) -- (0.8492482027852803,-0.1081833764901043);
\draw[line width=1.2pt] (0.8492482027852803,-0.1081833764901043) -- (0.858684293927339,-0.11060080749562391);
\draw[line width=1.2pt] (0.858684293927339,-0.11060080749562391) -- (0.8681203850693977,-0.1130449504459559);
\draw[line width=1.2pt] (0.8681203850693977,-0.1130449504459559) -- (0.8775564762114564,-0.11551580534110026);
\draw[line width=1.2pt] (0.8775564762114564,-0.11551580534110026) -- (0.8869925673535151,-0.11801337218105698);
\draw[line width=1.2pt] (0.8869925673535151,-0.11801337218105698) -- (0.8964286584955737,-0.12053765096582608);
\draw[line width=1.2pt] (0.8964286584955737,-0.12053765096582608) -- (0.9058647496376324,-0.12308864169540756);
\draw[line width=1.2pt] (0.9058647496376324,-0.12308864169540756) -- (0.9153008407796911,-0.1256663443698014);
\draw[line width=1.2pt] (0.9153008407796911,-0.1256663443698014) -- (0.9247369319217498,-0.12827075898900764);
\draw[line width=1.2pt] (0.9247369319217498,-0.12827075898900764) -- (0.9341730230638084,-0.1309018855530262);
\draw[line width=1.2pt] (0.9341730230638084,-0.1309018855530262) -- (0.9436091142058671,-0.13355972406185718);
\draw[line width=1.2pt] (0.9436091142058671,-0.13355972406185718) -- (0.9530452053479258,-0.1362442745155005);
\draw[line width=1.2pt] (0.9530452053479258,-0.1362442745155005) -- (0.9624812964899845,-0.1389555369139562);
\draw[line width=1.2pt] (0.9624812964899845,-0.1389555369139562) -- (0.9719173876320432,-0.1416935112572243);
\draw[line width=1.2pt] (0.9719173876320432,-0.1416935112572243) -- (0.9813534787741018,-0.14445819754530473);
\draw[line width=1.2pt] (0.9813534787741018,-0.14445819754530473) -- (0.9907895699161605,-0.14724959577819755);
\draw[line width=1.2pt] (0.9907895699161605,-0.14724959577819755) -- (1.0002256610582192,-0.15006770595590274);
\draw[line width=1.2pt] (1.0002256610582192,-0.15006770595590274) -- (1.0096617522002778,-0.15291252807842023);
\draw[line width=1.2pt] (1.0096617522002778,-0.15291252807842023) -- (1.0190978433423363,-0.15578406214575016);
\draw[line width=1.2pt] (1.0190978433423363,-0.15578406214575016) -- (1.028533934484395,-0.15868230815789242);
\draw[line width=1.2pt] (1.028533934484395,-0.15868230815789242) -- (1.0379700256264535,-0.16160726611484705);
\draw[line width=1.2pt] (1.0379700256264535,-0.16160726611484705) -- (1.047406116768512,-0.16455893601661406);
\draw[line width=1.2pt] (1.047406116768512,-0.16455893601661406) -- (1.0568422079105706,-0.16753731786319345);
\draw[line width=1.2pt] (1.0568422079105706,-0.16753731786319345) -- (1.0662782990526292,-0.1705424116545852);
\draw[line width=1.2pt] (1.0662782990526292,-0.1705424116545852) -- (1.0757143901946877,-0.17357421739078932);
\draw[line width=1.2pt] (1.0757143901946877,-0.17357421739078932) -- (1.0851504813367463,-0.17663273507180582);
\draw[line width=1.2pt] (1.0851504813367463,-0.17663273507180582) -- (1.0945865724788049,-0.17971796469763468);
\draw[line width=1.2pt] (1.0945865724788049,-0.17971796469763468) -- (1.1040226636208634,-0.18282990626827592);
\draw[line width=1.2pt] (1.1040226636208634,-0.18282990626827592) -- (1.113458754762922,-0.18596855978372953);
\draw[line width=1.2pt] (1.113458754762922,-0.18596855978372953) -- (1.1228948459049806,-0.1891339252439955);
\draw[line width=1.2pt] (1.1228948459049806,-0.1891339252439955) -- (1.1323309370470391,-0.19232600264907387);
\draw[line width=1.2pt] (1.1323309370470391,-0.19232600264907387) -- (1.1417670281890977,-0.19554479199896457);
\draw[line width=1.2pt] (1.1417670281890977,-0.19554479199896457) -- (1.1512031193311563,-0.19879029329366768);
\draw[line width=1.2pt] (1.1512031193311563,-0.19879029329366768) -- (1.1606392104732148,-0.2020625065331831);
\draw[line width=1.2pt] (1.1606392104732148,-0.2020625065331831) -- (1.1700753016152734,-0.20536143171751095);
\draw[line width=1.2pt] (1.1700753016152734,-0.20536143171751095) -- (1.179511392757332,-0.20868706884665114);
\draw[line width=1.2pt] (1.179511392757332,-0.20868706884665114) -- (1.1889474838993905,-0.2120394179206037);
\draw[line width=1.2pt] (1.1889474838993905,-0.2120394179206037) -- (1.198383575041449,-0.21541847893936866);
\draw[line width=1.2pt] (1.198383575041449,-0.21541847893936866) -- (1.2078196661835077,-0.21882425190294597);
\draw[line width=1.2pt] (1.2078196661835077,-0.21882425190294597) -- (1.2172557573255662,-0.22225673681133565);
\draw[line width=1.2pt] (1.2172557573255662,-0.22225673681133565) -- (1.2266918484676248,-0.22571593366453774);
\draw[line width=1.2pt] (1.2266918484676248,-0.22571593366453774) -- (1.2361279396096834,-0.22920184246255215);
\draw[line width=1.2pt] (1.2361279396096834,-0.22920184246255215) -- (1.245564030751742,-0.23271446320537895);
\draw[line width=1.2pt] (1.245564030751742,-0.23271446320537895) -- (1.2550001218938005,-0.2362537958930181);
\draw[line width=1.2pt] (1.2550001218938005,-0.2362537958930181) -- (1.264436213035859,-0.23981984052546965);
\draw[line width=1.2pt] (1.264436213035859,-0.23981984052546965) -- (1.2738723041779176,-0.24341259710273355);
\draw[line width=1.2pt] (1.2738723041779176,-0.24341259710273355) -- (1.2833083953199762,-0.24703206562480984);
\draw[line width=1.2pt] (1.2833083953199762,-0.24703206562480984) -- (1.2927444864620348,-0.2506782460916985);
\draw[line width=1.2pt] (1.2927444864620348,-0.2506782460916985) -- (1.3021805776040933,-0.2543511385033995);
\draw[line width=1.2pt] (1.3021805776040933,-0.2543511385033995) -- (1.311616668746152,-0.2580507428599129);
\draw[line width=1.2pt] (1.311616668746152,-0.2580507428599129) -- (1.3210527598882105,-0.26177705916123867);
\draw[line width=1.2pt] (1.3210527598882105,-0.26177705916123867) -- (1.330488851030269,-0.26553008740737677);
\draw[line width=1.2pt] (1.330488851030269,-0.26553008740737677) -- (1.3399249421723276,-0.2693098275983273);
\draw[line width=1.2pt] (1.3399249421723276,-0.2693098275983273) -- (1.3493610333143862,-0.2731162797340902);
\draw[line width=1.2pt] (1.3493610333143862,-0.2731162797340902) -- (1.3587971244564447,-0.27694944381466546);
\draw[line width=1.2pt] (1.3587971244564447,-0.27694944381466546) -- (1.3682332155985033,-0.28080931984005303);
\draw[line width=1.2pt] (1.3682332155985033,-0.28080931984005303) -- (1.3776693067405619,-0.28469590781025306);
\draw[line width=1.2pt] (1.3776693067405619,-0.28469590781025306) -- (1.3871053978826204,-0.2886092077252654);
\draw[line width=1.2pt] (1.3871053978826204,-0.2886092077252654) -- (1.396541489024679,-0.29254921958509017);
\draw[line width=1.2pt] (1.396541489024679,-0.29254921958509017) -- (1.4059775801667376,-0.29651594338972725);
\draw[line width=1.2pt] (1.4059775801667376,-0.29651594338972725) -- (1.4154136713087961,-0.3005093791391767);
\draw[line width=1.2pt] (1.4154136713087961,-0.3005093791391767) -- (1.4248497624508547,-0.30452952683343854);
\draw[line width=1.2pt] (1.4248497624508547,-0.30452952683343854) -- (1.4342858535929133,-0.30857638647251273);
\draw[line width=1.2pt] (1.4342858535929133,-0.30857638647251273) -- (1.4437219447349718,-0.31264995805639934);
\draw[line width=1.2pt] (1.4437219447349718,-0.31264995805639934) -- (1.4531580358770304,-0.31675024158509835);
\draw[line width=1.2pt] (1.4531580358770304,-0.31675024158509835) -- (1.462594127019089,-0.3208772370586096);
\draw[line width=1.2pt] (1.462594127019089,-0.3208772370586096) -- (1.4720302181611475,-0.32503094447693337);
\draw[line width=1.2pt] (1.4720302181611475,-0.32503094447693337) -- (1.481466309303206,-0.3292113638400694);
\draw[line width=1.2pt] (1.481466309303206,-0.3292113638400694) -- (1.4909024004452647,-0.3334184951480178);
\draw[line width=1.2pt] (1.4909024004452647,-0.3334184951480178) -- (1.5003384915873232,-0.33765233840077863);
\draw[line width=1.2pt] (1.5003384915873232,-0.33765233840077863) -- (1.5097745827293818,-0.3419128935983518);
\draw[line width=1.2pt] (1.5097745827293818,-0.3419128935983518) -- (1.5192106738714404,-0.3462001607407374);
\draw[line width=1.2pt] (1.5192106738714404,-0.3462001607407374) -- (1.528646765013499,-0.3505141398279353);
\draw[line width=1.2pt] (1.528646765013499,-0.3505141398279353) -- (1.5380828561555575,-0.3548548308599456);
\draw[line width=1.2pt] (1.5380828561555575,-0.3548548308599456) -- (1.547518947297616,-0.35922223383676827);
\draw[line width=1.2pt] (1.547518947297616,-0.35922223383676827) -- (1.5569550384396746,-0.3636163487584033);
\draw[line width=1.2pt] (1.5569550384396746,-0.3636163487584033) -- (1.5663911295817332,-0.3680371756248507);
\draw[line width=1.2pt] (1.5663911295817332,-0.3680371756248507) -- (1.5758272207237918,-0.3724847144361105);
\draw[line width=1.2pt] (1.5758272207237918,-0.3724847144361105) -- (1.5852633118658503,-0.3769589651921826);
\draw[line width=1.2pt] (1.5852633118658503,-0.3769589651921826) -- (1.594699403007909,-0.3814599278930671);
\draw[line width=1.2pt] (1.594699403007909,-0.3814599278930671) -- (1.6041354941499675,-0.385987602538764);
\draw[line width=1.2pt] (1.6041354941499675,-0.385987602538764) -- (1.613571585292026,-0.3905419891292733);
\draw[line width=1.2pt] (1.613571585292026,-0.3905419891292733) -- (1.6230076764340846,-0.3951230876645949);
\draw[line width=1.2pt] (1.6230076764340846,-0.3951230876645949) -- (1.6324437675761432,-0.3997308981447289);
\draw[line width=1.2pt] (1.6324437675761432,-0.3997308981447289) -- (1.6418798587182017,-0.40436542056967534);
\draw[line width=1.2pt] (1.6418798587182017,-0.40436542056967534) -- (1.6513159498602603,-0.409026654939434);
\draw[line width=1.2pt] (1.6513159498602603,-0.409026654939434) -- (1.6607520410023189,-0.4137146012540052);
\draw[line width=1.2pt] (1.6607520410023189,-0.4137146012540052) -- (1.6701881321443774,-0.41842925951338866);
\draw[line width=1.2pt] (1.6701881321443774,-0.41842925951338866) -- (1.679624223286436,-0.4231706297175845);
\draw[line width=1.2pt] (1.679624223286436,-0.4231706297175845) -- (1.6890603144284946,-0.42793871186659277);
\draw[line width=1.2pt] (1.6890603144284946,-0.42793871186659277) -- (1.6984964055705531,-0.4327335059604133);
\draw[line width=1.2pt] (1.6984964055705531,-0.4327335059604133) -- (1.7079324967126117,-0.4375550119990463);
\draw[line width=1.2pt] (1.7079324967126117,-0.4375550119990463) -- (1.7173685878546703,-0.44240322998249165);
\draw[line width=1.2pt] (1.7173685878546703,-0.44240322998249165) -- (1.7268046789967288,-0.4472781599107494);
\draw[line width=1.2pt] (1.7268046789967288,-0.4472781599107494) -- (1.7362407701387874,-0.4521798017838194);
\draw[line width=1.2pt] (1.7362407701387874,-0.4521798017838194) -- (1.745676861280846,-0.4571081556017019);
\draw[line width=1.2pt] (1.745676861280846,-0.4571081556017019) -- (1.7551129524229045,-0.4620632213643967);
\draw[line width=1.2pt] (1.7551129524229045,-0.4620632213643967) -- (1.7645490435649631,-0.4670449990719039);
\draw[line width=1.2pt] (1.7645490435649631,-0.4670449990719039) -- (1.7739851347070217,-0.4720534887242235);
\draw[line width=1.2pt] (1.7739851347070217,-0.4720534887242235) -- (1.7834212258490802,-0.47708869032135537);
\draw[line width=1.2pt] (1.7834212258490802,-0.47708869032135537) -- (1.7928573169911388,-0.4821506038632997);
\draw[line width=1.2pt] (1.7928573169911388,-0.4821506038632997) -- (1.8022934081331974,-0.48723922935005637);
\draw[line width=1.2pt] (1.8022934081331974,-0.48723922935005637) -- (1.811729499275256,-0.4923545667816254);
\draw[line width=1.2pt] (1.811729499275256,-0.4923545667816254) -- (1.8211655904173145,-0.4974966161580069);
\draw[line width=1.2pt] (1.8211655904173145,-0.4974966161580069) -- (1.830601681559373,-0.5026653774792006);
\draw[line width=1.2pt] (1.830601681559373,-0.5026653774792006) -- (1.8400377727014317,-0.5078608507452068);
\draw[line width=1.2pt] (1.8400377727014317,-0.5078608507452068) -- (1.8494738638434902,-0.5130830359560253);
\draw[line width=1.2pt] (1.8494738638434902,-0.5130830359560253) -- (1.8589099549855488,-0.5183319331116563);
\draw[line width=1.2pt] (1.8589099549855488,-0.5183319331116563) -- (1.8683460461276074,-0.5236075422120995);
\draw[line width=1.2pt] (1.8683460461276074,-0.5236075422120995) -- (1.877782137269666,-0.5289098632573551);
\draw[line width=1.2pt] (1.877782137269666,-0.5289098632573551) -- (1.8872182284117245,-0.5342388962474232);
\draw[line width=1.2pt] (1.8872182284117245,-0.5342388962474232) -- (1.896654319553783,-0.5395946411823036);
\draw[line width=1.2pt] (1.896654319553783,-0.5395946411823036) -- (1.9060904106958416,-0.5449770980619962);
\draw[line width=1.2pt] (1.9060904106958416,-0.5449770980619962) -- (1.9155265018379002,-0.5503862668865015);
\draw[line width=1.2pt] (1.9155265018379002,-0.5503862668865015) -- (1.9249625929799588,-0.5558221476558189);
\draw[line width=1.2pt] (1.9249625929799588,-0.5558221476558189) -- (1.9343986841220173,-0.5612847403699488);
\draw[line width=1.2pt] (1.9343986841220173,-0.5612847403699488) -- (1.9438347752640759,-0.5667740450288911);
\draw[line width=1.2pt] (1.9438347752640759,-0.5667740450288911) -- (1.9532708664061345,-0.5722900616326456);
\draw[line width=1.2pt] (1.9532708664061345,-0.5722900616326456) -- (1.962706957548193,-0.5778327901812126);
\draw[line width=1.2pt] (1.962706957548193,-0.5778327901812126) -- (1.9721430486902516,-0.583402230674592);
\draw[line width=1.2pt] (1.9721430486902516,-0.583402230674592) -- (1.9815791398323102,-0.5889983831127837);
\draw[line width=1.2pt] (1.9815791398323102,-0.5889983831127837) -- (1.9910152309743687,-0.5946212474957878);
\draw[line width=1.2pt] (1.9910152309743687,-0.5946212474957878) -- (2.0004513221164273,-0.6002708238236043);
\draw[line width=1.2pt] (2.0004513221164273,-0.6002708238236043) -- (2.009887413258486,-0.6059471120962333);
\draw[line width=1.2pt] (2.009887413258486,-0.6059471120962333) -- (2.019323504400545,-0.6116501123136746);
\draw[line width=1.2pt] (2.019323504400545,-0.6116501123136746) -- (2.0287595955426037,-0.6173798244759282);
\draw[line width=1.2pt] (2.0287595955426037,-0.6173798244759282) -- (2.0381956866846624,-0.6231362485829944);
\draw[line width=1.2pt] (2.0381956866846624,-0.6231362485829944) -- (2.0476317778267212,-0.6289193846348728);
\draw[line width=1.2pt] (2.0476317778267212,-0.6289193846348728) -- (2.05706786896878,-0.6347292326315637);
\draw[line width=1.2pt] (2.05706786896878,-0.6347292326315637) -- (2.066503960110839,-0.6405657925730669);
\draw[line width=1.2pt] (2.066503960110839,-0.6405657925730669) -- (2.0759400512528976,-0.6464290644593825);
\draw[line width=1.2pt] (2.0759400512528976,-0.6464290644593825) -- (2.0853761423949564,-0.6523190482905104);
\draw[line width=1.2pt] (2.0853761423949564,-0.6523190482905104) -- (2.094812233537015,-0.6582357440664507);
\draw[line width=1.2pt] (2.094812233537015,-0.6582357440664507) -- (2.104248324679074,-0.6641791517872034);
\draw[line width=1.2pt] (2.104248324679074,-0.6641791517872034) -- (2.1136844158211328,-0.6701492714527685);
\draw[line width=1.2pt] (2.1136844158211328,-0.6701492714527685) -- (2.1231205069631915,-0.6761461030631459);
\draw[line width=1.2pt] (2.1231205069631915,-0.6761461030631459) -- (2.1325565981052503,-0.6821696466183357);
\draw[line width=1.2pt] (2.1325565981052503,-0.6821696466183357) -- (2.141992689247309,-0.6882199021183378);
\draw[line width=1.2pt] (2.141992689247309,-0.6882199021183378) -- (2.151428780389368,-0.6942968695631524);
\draw[line width=1.2pt] (2.151428780389368,-0.6942968695631524) -- (2.1608648715314267,-0.7004005489527794);
\draw[line width=1.2pt] (2.1608648715314267,-0.7004005489527794) -- (2.1703009626734855,-0.7065309402872186);
\draw[line width=1.2pt] (2.1703009626734855,-0.7065309402872186) -- (2.1797370538155443,-0.7126880435664703);
\draw[line width=1.2pt] (2.1797370538155443,-0.7126880435664703) -- (2.189173144957603,-0.7188718587905343);
\draw[line width=1.2pt] (2.189173144957603,-0.7188718587905343) -- (2.198609236099662,-0.7250823859594108);
\draw[line width=1.2pt] (2.198609236099662,-0.7250823859594108) -- (2.2080453272417206,-0.7313196250730996);
\draw[line width=1.2pt] (2.2080453272417206,-0.7313196250730996) -- (2.2174814183837794,-0.7375835761316007);
\draw[line width=1.2pt] (2.2174814183837794,-0.7375835761316007) -- (2.2269175095258382,-0.7438742391349142);
\draw[line width=1.2pt] (2.2269175095258382,-0.7438742391349142) -- (2.236353600667897,-0.7501916140830401);
\draw[line width=1.2pt] (2.236353600667897,-0.7501916140830401) -- (2.245789691809956,-0.7565357009759783);
\draw[line width=1.2pt] (2.245789691809956,-0.7565357009759783) -- (2.2552257829520146,-0.762906499813729);
\draw[line width=1.2pt] (2.2552257829520146,-0.762906499813729) -- (2.2646618740940734,-0.7693040105962922);
\draw[line width=1.2pt] (2.2646618740940734,-0.7693040105962922) -- (2.274097965236132,-0.7757282333236675);
\draw[line width=1.2pt] (2.274097965236132,-0.7757282333236675) -- (2.283534056378191,-0.7821791679958553);
\draw[line width=1.2pt] (2.283534056378191,-0.7821791679958553) -- (2.2929701475202497,-0.7886568146128554);
\draw[line width=1.2pt] (2.2929701475202497,-0.7886568146128554) -- (2.3024062386623085,-0.7951611731746678);
\draw[line width=1.2pt] (2.3024062386623085,-0.7951611731746678) -- (2.3118423298043673,-0.8016922436812927);
\draw[line width=1.2pt] (2.3118423298043673,-0.8016922436812927) -- (2.321278420946426,-0.80825002613273);
\draw[line width=1.2pt] (2.321278420946426,-0.80825002613273) -- (2.330714512088485,-0.8148345205289796);
\draw[line width=1.2pt] (2.330714512088485,-0.8148345205289796) -- (2.3401506032305437,-0.8214457268700416);
\draw[line width=1.2pt] (2.3401506032305437,-0.8214457268700416) -- (2.3495866943726025,-0.8280836451559159);
\draw[line width=1.2pt] (2.3495866943726025,-0.8280836451559159) -- (2.3590227855146613,-0.8347482753866028);
\draw[line width=1.2pt] (2.3590227855146613,-0.8347482753866028) -- (2.36845887665672,-0.8414396175621018);
\draw[line width=1.2pt] (2.36845887665672,-0.8414396175621018) -- (2.377894967798779,-0.8481576716824134);
\draw[line width=1.2pt] (2.377894967798779,-0.8481576716824134) -- (2.3873310589408376,-0.8549024377475372);
\draw[line width=1.2pt] (2.3873310589408376,-0.8549024377475372) -- (2.3967671500828964,-0.8616739157574733);
\draw[line width=1.2pt] (2.3967671500828964,-0.8616739157574733) -- (2.406203241224955,-0.868472105712222);
\draw[line width=1.2pt] (2.406203241224955,-0.868472105712222) -- (2.415639332367014,-0.875297007611783);
\draw[line width=1.2pt] (2.415639332367014,-0.875297007611783) -- (2.425075423509073,-0.8821486214561562);
\draw[line width=1.2pt] (2.425075423509073,-0.8821486214561562) -- (2.4345115146511316,-0.889026947245342);
\draw[line width=1.2pt] (2.4345115146511316,-0.889026947245342) -- (2.4439476057931904,-0.89593198497934);
\draw[line width=1.2pt] (2.4439476057931904,-0.89593198497934) -- (2.453383696935249,-0.9028637346581505);
\draw[line width=1.2pt] (2.453383696935249,-0.9028637346581505) -- (2.462819788077308,-0.9098221962817733);
\draw[line width=1.2pt] (2.462819788077308,-0.9098221962817733) -- (2.4722558792193667,-0.9168073698502086);
\draw[line width=1.2pt] (2.4722558792193667,-0.9168073698502086) -- (2.4816919703614255,-0.9238192553634561);
\draw[line width=1.2pt] (2.4816919703614255,-0.9238192553634561) -- (2.4911280615034843,-0.9308578528215161);
\draw[line width=1.2pt] (2.4911280615034843,-0.9308578528215161) -- (2.500564152645543,-0.9379231622243884);
\draw[line width=1.2pt] (2.500564152645543,-0.9379231622243884) -- (2.510000243787602,-0.945015183572073);
\draw[line width=1.2pt] (2.510000243787602,-0.945015183572073) -- (2.5194363349296607,-0.9521339168645702);
\draw[line width=1.2pt] (2.5194363349296607,-0.9521339168645702) -- (2.5288724260717195,-0.9592793621018796);
\draw[line width=1.2pt] (2.5288724260717195,-0.9592793621018796) -- (2.5383085172137783,-0.9664515192840014);
\draw[line width=1.2pt] (2.5383085172137783,-0.9664515192840014) -- (2.547744608355837,-0.9736503884109355);
\draw[line width=1.2pt] (2.547744608355837,-0.9736503884109355) -- (2.557180699497896,-0.9808759694826821);
\draw[line width=1.2pt] (2.557180699497896,-0.9808759694826821) -- (2.5666167906399546,-0.988128262499241);
\draw[line width=1.2pt] (2.5666167906399546,-0.988128262499241) -- (2.5760528817820134,-0.9954072674606124);
\draw[line width=1.2pt] (2.5760528817820134,-0.9954072674606124) -- (2.585488972924072,-1.002712984366796);
\draw[line width=1.2pt] (2.585488972924072,-1.002712984366796) -- (2.594925064066131,-1.010045413217792);
\draw[line width=1.2pt] (2.594925064066131,-1.010045413217792) -- (2.60436115520819,-1.0174045540136005);
\draw[line width=1.2pt] (2.60436115520819,-1.0174045540136005) -- (2.6137972463502486,-1.0247904067542213);
\draw[line width=1.2pt] (2.6137972463502486,-1.0247904067542213) -- (2.6232333374923074,-1.0322029714396543);
\draw[line width=1.2pt] (2.6232333374923074,-1.0322029714396543) -- (2.632669428634366,-1.0396422480699);
\draw[line width=1.2pt] (2.632669428634366,-1.0396422480699) -- (2.642105519776425,-1.0471082366449578);
\draw[line width=1.2pt] (2.642105519776425,-1.0471082366449578) -- (2.6515416109184837,-1.0546009371648282);
\draw[line width=1.2pt] (2.6515416109184837,-1.0546009371648282) -- (2.6609777020605425,-1.0621203496295109);
\draw[line width=1.2pt] (2.6609777020605425,-1.0621203496295109) -- (2.6704137932026013,-1.0696664740390058);
\draw[line width=1.2pt] (2.6704137932026013,-1.0696664740390058) -- (2.67984988434466,-1.077239310393313);
\draw[line width=1.2pt] (2.67984988434466,-1.077239310393313) -- (2.689285975486719,-1.084838858692433);
\draw[line width=1.2pt] (2.689285975486719,-1.084838858692433) -- (2.6987220666287777,-1.092465118936365);
\draw[line width=1.2pt] (2.6987220666287777,-1.092465118936365) -- (2.7081581577708365,-1.1001180911251096);
\draw[line width=1.2pt] (2.7081581577708365,-1.1001180911251096) -- (2.7175942489128952,-1.1077977752586665);
\draw[line width=1.2pt] (2.7175942489128952,-1.1077977752586665) -- (2.727030340054954,-1.1155041713370357);
\draw[line width=1.2pt] (2.727030340054954,-1.1155041713370357) -- (2.736466431197013,-1.1232372793602174);
\draw[line width=1.2pt] (2.736466431197013,-1.1232372793602174) -- (2.7459025223390716,-1.1309970993282112);
\draw[line width=1.2pt] (2.7459025223390716,-1.1309970993282112) -- (2.7553386134811304,-1.1387836312410176);
\draw[line width=1.2pt] (2.7553386134811304,-1.1387836312410176) -- (2.764774704623189,-1.1465968750986364);
\draw[line width=1.2pt] (2.764774704623189,-1.1465968750986364) -- (2.774210795765248,-1.1544368309010675);
\draw[line width=1.2pt] (2.774210795765248,-1.1544368309010675) -- (2.7836468869073068,-1.162303498648311);
\draw[line width=1.2pt] (2.7836468869073068,-1.162303498648311) -- (2.7930829780493656,-1.170196878340367);
\draw[line width=1.2pt] (2.7930829780493656,-1.170196878340367) -- (2.8025190691914244,-1.1781169699772351);
\draw[line width=1.2pt] (2.8025190691914244,-1.1781169699772351) -- (2.811955160333483,-1.1860637735589157);
\draw[line width=1.2pt] (2.811955160333483,-1.1860637735589157) -- (2.821391251475542,-1.1940372890854085);
\draw[line width=1.2pt] (2.821391251475542,-1.1940372890854085) -- (2.8308273426176007,-1.202037516556714);
\draw[line width=1.2pt] (2.8308273426176007,-1.202037516556714) -- (2.8402634337596595,-1.2100644559728317);
\draw[line width=1.2pt] (2.8402634337596595,-1.2100644559728317) -- (2.8496995249017183,-1.2181181073337617);
\draw[line width=1.2pt] (2.8496995249017183,-1.2181181073337617) -- (2.859135616043777,-1.2261984706395042);
\draw[line width=1.2pt] (2.859135616043777,-1.2261984706395042) -- (2.868571707185836,-1.234305545890059);
\draw[line width=1.2pt] (2.868571707185836,-1.234305545890059) -- (2.8780077983278947,-1.2424393330854264);
\draw[line width=1.2pt] (2.8780077983278947,-1.2424393330854264) -- (2.8874438894699535,-1.2505998322256058);
\draw[line width=1.2pt] (2.8874438894699535,-1.2505998322256058) -- (2.8968799806120122,-1.2587870433105977);
\draw[line width=1.2pt] (2.8968799806120122,-1.2587870433105977) -- (2.906316071754071,-1.267000966340402);
\draw[line width=1.2pt] (2.906316071754071,-1.267000966340402) -- (2.91575216289613,-1.275241601315019);
\draw[line width=1.2pt] (2.91575216289613,-1.275241601315019) -- (2.9251882540381886,-1.2835089482344477);
\draw[line width=1.2pt] (2.9251882540381886,-1.2835089482344477) -- (2.9346243451802474,-1.2918030070986892);
\draw[line width=1.2pt] (2.9346243451802474,-1.2918030070986892) -- (2.944060436322306,-1.300123777907743);
\draw[line width=1.2pt] (2.944060436322306,-1.300123777907743) -- (2.953496527464365,-1.3084712606616093);
\draw[line width=1.2pt] (2.953496527464365,-1.3084712606616093) -- (2.9629326186064238,-1.316845455360288);
\draw[line width=1.2pt] (2.9629326186064238,-1.316845455360288) -- (2.9723687097484826,-1.3252463620037787);
\draw[line width=1.2pt] (2.9723687097484826,-1.3252463620037787) -- (2.9818048008905413,-1.333673980592082);
\draw[line width=1.2pt] (2.9818048008905413,-1.333673980592082) -- (2.9912408920326,-1.3421283111251976);
\draw[line width=1.2pt] (2.9912408920326,-1.3421283111251976) -- (3.000676983174659,-1.3506093536031258);
\draw[line width=1.2pt] (3.000676983174659,-1.3506093536031258) -- (3.0101130743167177,-1.3591171080258662);
\draw[line width=1.2pt] (3.0101130743167177,-1.3591171080258662) -- (3.0195491654587765,-1.367651574393419);
\draw[line width=1.2pt] (3.0195491654587765,-1.367651574393419) -- (3.0289852566008353,-1.3762127527057841);
\draw[line width=1.2pt] (3.0289852566008353,-1.3762127527057841) -- (3.038421347742894,-1.3848006429629616);
\draw[line width=1.2pt] (3.038421347742894,-1.3848006429629616) -- (3.047857438884953,-1.3934152451649515);
\draw[line width=1.2pt] (3.047857438884953,-1.3934152451649515) -- (3.0572935300270117,-1.402056559311754);
\draw[line width=1.2pt] (3.0572935300270117,-1.402056559311754) -- (3.0667296211690704,-1.4107245854033683);
\draw[line width=1.2pt] (3.0667296211690704,-1.4107245854033683) -- (3.0761657123111292,-1.4194193234397956);
\draw[line width=1.2pt] (3.0761657123111292,-1.4194193234397956) -- (3.085601803453188,-1.428140773421035);
\draw[line width=1.2pt] (3.085601803453188,-1.428140773421035) -- (3.095037894595247,-1.4368889353470868);
\draw[line width=1.2pt] (3.095037894595247,-1.4368889353470868) -- (3.1044739857373056,-1.4456638092179508);
\draw[line width=1.2pt] (3.1044739857373056,-1.4456638092179508) -- (3.1139100768793644,-1.4544653950336273);
\draw[line width=1.2pt] (3.1139100768793644,-1.4544653950336273) -- (3.123346168021423,-1.4632936927941163);
\draw[line width=1.2pt] (3.123346168021423,-1.4632936927941163) -- (3.132782259163482,-1.4721487024994173);
\draw[line width=1.2pt] (3.132782259163482,-1.4721487024994173) -- (3.1422183503055408,-1.4810304241495311);
\draw[line width=1.2pt] (3.1422183503055408,-1.4810304241495311) -- (3.1516544414475995,-1.4899388577444572);
\draw[line width=1.2pt] (3.1516544414475995,-1.4899388577444572) -- (3.1610905325896583,-1.4988740032841954);
\draw[line width=1.2pt] (3.1610905325896583,-1.4988740032841954) -- (3.170526623731717,-1.507835860768746);
\draw[line width=1.2pt] (3.170526623731717,-1.507835860768746) -- (3.179962714873776,-1.5168244301981093);
\draw[line width=1.2pt] (3.179962714873776,-1.5168244301981093) -- (3.1893988060158347,-1.5258397115722848);
\draw[line width=1.2pt] (3.1893988060158347,-1.5258397115722848) -- (3.1988348971578935,-1.5348817048912726);
\draw[line width=1.2pt] (3.1988348971578935,-1.5348817048912726) -- (3.2082709882999523,-1.543950410155073);
\draw[line width=1.2pt] (3.2082709882999523,-1.543950410155073) -- (3.217707079442011,-1.5530458273636853);
\draw[line width=1.2pt] (3.217707079442011,-1.5530458273636853) -- (3.22714317058407,-1.5621679565171105);
\draw[line width=1.2pt] (3.22714317058407,-1.5621679565171105) -- (3.2365792617261286,-1.5713167976153477);
\draw[line width=1.2pt] (3.2365792617261286,-1.5713167976153477) -- (3.2460153528681874,-1.5804923506583974);
\draw[line width=1.2pt] (3.2460153528681874,-1.5804923506583974) -- (3.255451444010246,-1.5896946156462597);
\draw[line width=1.2pt] (3.255451444010246,-1.5896946156462597) -- (3.264887535152305,-1.598923592578934);
\draw[line width=1.2pt] (3.264887535152305,-1.598923592578934) -- (3.274323626294364,-1.6081792814564209);
\draw[line width=1.2pt] (3.274323626294364,-1.6081792814564209) -- (3.2837597174364226,-1.61746168227872);
\draw[line width=1.2pt] (3.2837597174364226,-1.61746168227872) -- (3.2931958085784814,-1.6267707950458314);
\draw[line width=1.2pt] (3.2931958085784814,-1.6267707950458314) -- (3.30263189972054,-1.6361066197577556);
\draw[line width=1.2pt] (3.30263189972054,-1.6361066197577556) -- (3.312067990862599,-1.645469156414492);
\draw[line width=1.2pt] (3.312067990862599,-1.645469156414492) -- (3.3215040820046577,-1.6548584050160404);
\draw[line width=1.2pt] (3.3215040820046577,-1.6548584050160404) -- (3.3309401731467165,-1.6642743655624017);
\draw[line width=1.2pt] (3.3309401731467165,-1.6642743655624017) -- (3.3403762642887753,-1.673717038053575);
\draw[line width=1.2pt] (3.3403762642887753,-1.673717038053575) -- (3.349812355430834,-1.683186422489561);
\draw[line width=1.2pt] (3.349812355430834,-1.683186422489561) -- (3.359248446572893,-1.692682518870359);
\draw[line width=1.2pt] (3.359248446572893,-1.692682518870359) -- (3.3686845377149517,-1.7022053271959694);
\draw[line width=1.2pt] (3.3686845377149517,-1.7022053271959694) -- (3.3781206288570105,-1.7117548474663926);
\draw[line width=1.2pt] (3.3781206288570105,-1.7117548474663926) -- (3.3875567199990693,-1.7213310796816277);
\draw[line width=1.2pt] (3.3875567199990693,-1.7213310796816277) -- (3.396992811141128,-1.7309340238416755);
\draw[line width=1.2pt] (3.396992811141128,-1.7309340238416755) -- (3.406428902283187,-1.7405636799465356);
\draw[line width=1.2pt] (3.406428902283187,-1.7405636799465356) -- (3.4158649934252456,-1.750220047996208);
\draw[line width=1.2pt] (3.4158649934252456,-1.750220047996208) -- (3.4253010845673044,-1.759903127990693);
\draw[line width=1.2pt] (3.4253010845673044,-1.759903127990693) -- (3.434737175709363,-1.76961291992999);
\draw[line width=1.2pt] (3.434737175709363,-1.76961291992999) -- (3.444173266851422,-1.7793494238140994);
\draw[line width=1.2pt] (3.444173266851422,-1.7793494238140994) -- (3.453609357993481,-1.7891126396430213);
\draw[line width=1.2pt] (3.453609357993481,-1.7891126396430213) -- (3.4630454491355396,-1.7989025674167556);
\draw[line width=1.2pt] (3.4630454491355396,-1.7989025674167556) -- (3.4724815402775984,-1.8087192071353022);
\draw[line width=1.2pt] (3.4724815402775984,-1.8087192071353022) -- (3.481917631419657,-1.8185625587986611);
\draw[line width=1.2pt] (3.481917631419657,-1.8185625587986611) -- (3.491353722561716,-1.8284326224068326);
\draw[line width=1.2pt] (3.491353722561716,-1.8284326224068326) -- (3.5007898137037747,-1.8383293979598165);
\draw[line width=1.2pt] (3.5007898137037747,-1.8383293979598165) -- (3.5102259048458335,-1.8482528854576126);
\draw[line width=1.2pt] (3.5102259048458335,-1.8482528854576126) -- (3.5196619959878923,-1.8582030849002211);
\draw[line width=1.2pt] (3.5196619959878923,-1.8582030849002211) -- (3.529098087129951,-1.8681799962876418);
\draw[line width=1.2pt] (3.529098087129951,-1.8681799962876418) -- (3.53853417827201,-1.8781836196198753);
\draw[line width=1.2pt] (3.53853417827201,-1.8781836196198753) -- (3.5479702694140687,-1.8882139548969208);
\draw[line width=1.2pt] (3.5479702694140687,-1.8882139548969208) -- (3.5574063605561275,-1.8982710021187787);
\draw[line width=1.2pt] (3.5574063605561275,-1.8982710021187787) -- (3.5668424516981863,-1.908354761285449);
\draw[line width=1.2pt] (3.5668424516981863,-1.908354761285449) -- (3.576278542840245,-1.9184652323969318);
\draw[line width=1.2pt] (3.576278542840245,-1.9184652323969318) -- (3.585714633982304,-1.9286024154532269);
\draw[line width=1.2pt] (3.585714633982304,-1.9286024154532269) -- (3.5951507251243626,-1.9387663104543345);
\draw[line width=1.2pt] (3.5951507251243626,-1.9387663104543345) -- (3.6045868162664214,-1.9489569174002543);
\draw[line width=1.2pt] (3.6045868162664214,-1.9489569174002543) -- (3.61402290740848,-1.9591742362909865);
\draw[line width=1.2pt] (3.61402290740848,-1.9591742362909865) -- (3.623458998550539,-1.969418267126531);
\draw[line width=1.2pt] (3.623458998550539,-1.969418267126531) -- (3.632895089692598,-1.9796890099068882);
\draw[line width=1.2pt] (3.632895089692598,-1.9796890099068882) -- (3.6423311808346566,-1.9899864646320575);
\draw[line width=1.2pt] (3.6423311808346566,-1.9899864646320575) -- (3.6517672719767154,-2.000310631302039);
\draw[line width=1.2pt] (3.6517672719767154,-2.000310631302039) -- (3.661203363118774,-2.010661509916833);
\draw[line width=1.2pt] (3.661203363118774,-2.010661509916833) -- (3.670639454260833,-2.0210391004764396);
\draw[line width=1.2pt] (3.670639454260833,-2.0210391004764396) -- (3.6800755454028917,-2.0314434029808583);
\draw[line width=1.2pt] (3.6800755454028917,-2.0314434029808583) -- (3.6895116365449505,-2.04187441743009);
\draw[line width=1.2pt] (3.6895116365449505,-2.04187441743009) -- (3.6989477276870093,-2.0523321438241333);
\draw[line width=1.2pt] (3.6989477276870093,-2.0523321438241333) -- (3.708383818829068,-2.0628165821629896);
\draw[line width=1.2pt] (3.708383818829068,-2.0628165821629896) -- (3.717819909971127,-2.0733277324466575);
\draw[line width=1.2pt] (3.717819909971127,-2.0733277324466575) -- (3.7272560011131857,-2.0838655946751383);
\draw[line width=1.2pt] (3.7272560011131857,-2.0838655946751383) -- (3.7366920922552445,-2.0944301688484313);
\draw[line width=1.2pt] (3.7366920922552445,-2.0944301688484313) -- (3.7461281833973032,-2.1050214549665367);
\draw[line width=1.2pt] (3.7461281833973032,-2.1050214549665367) -- (3.755564274539362,-2.1156394530294547);
\draw[line width=1.2pt] (3.755564274539362,-2.1156394530294547) -- (3.765000365681421,-2.1262841630371847);
\draw[line width=1.2pt] (3.765000365681421,-2.1262841630371847) -- (3.7744364568234796,-2.1369555849897273);
\begin{scriptsize}
\draw [fill=black] (0.,0.) circle (1.5pt);
\end{scriptsize}
\end{tikzpicture}}
		\choice{% !TEX root = ../oefeningen_fys6.tex
\begin{tikzpicture}[line cap=round,line join=round,>=triangle 45,x=1.0cm,y=1.0cm, scale=0.8]
\draw[->,color=black] (-0.26126256871071957,0.) -- (3.6896534027248262,0.);
\foreach \x in {,0.5,1.,1.5,2.,2.5,3.,3.5}
\draw[shift={(\x,0)},color=black] (0pt,-2pt);
\draw[->,color=black] (0.,-0.2694355115461896) -- (0.,3.9273529130473612);
%\foreach \y in {,0.5,1.,1.5,2.,2.5,3.,3.5}
%\draw[shift={(0,\y)},color=black] (2pt,0pt) -- (-2pt,0pt);
\clip(-0.26126256871071957,-0.2694355115461896) rectangle (3.6896534027248262,3.9273529130473612);
\draw (0.10330693079942738,3.8849611107787396) node[anchor=north west] {$x$};
\draw (3.0537763686955,0.5105736501964502) node[anchor=north west] {$t$};
\draw (0.11178529125315172,0.40883332475175804) node[anchor=north west] {$O$};
\draw[line width=1.2pt] (0.00922412450019114,0.018426977882183525) -- (0.01844824900038228,0.03681141352796954);
\draw[line width=1.2pt] (0.01844824900038228,0.03681141352796954) -- (0.027672373500573423,0.05515330693735804);
\draw[line width=1.2pt] (0.027672373500573423,0.05515330693735804) -- (0.03689649800076456,0.07345265811034901);
\draw[line width=1.2pt] (0.03689649800076456,0.07345265811034901) -- (0.0461206225009557,0.09170946704694248);
\draw[line width=1.2pt] (0.0461206225009557,0.09170946704694248) -- (0.05534474700114684,0.10992373374713844);
\draw[line width=1.2pt] (0.05534474700114684,0.10992373374713844) -- (0.06456887150133798,0.1280954582109369);
\draw[line width=1.2pt] (0.06456887150133798,0.1280954582109369) -- (0.07379299600152912,0.14622464043833783);
\draw[line width=1.2pt] (0.07379299600152912,0.14622464043833783) -- (0.08301712050172026,0.16431128042934123);
\draw[line width=1.2pt] (0.08301712050172026,0.16431128042934123) -- (0.0922412450019114,0.18235537818394715);
\draw[line width=1.2pt] (0.0922412450019114,0.18235537818394715) -- (0.10146536950210254,0.20035693370215554);
\draw[line width=1.2pt] (0.10146536950210254,0.20035693370215554) -- (0.11068949400229368,0.2183159469839664);
\draw[line width=1.2pt] (0.11068949400229368,0.2183159469839664) -- (0.11991361850248482,0.23623241802937978);
\draw[line width=1.2pt] (0.11991361850248482,0.23623241802937978) -- (0.12913774300267597,0.25410634683839567);
\draw[line width=1.2pt] (0.12913774300267597,0.25410634683839567) -- (0.1383618675028671,0.27193773341101396);
\draw[line width=1.2pt] (0.1383618675028671,0.27193773341101396) -- (0.14758599200305825,0.2897265777472348);
\draw[line width=1.2pt] (0.14758599200305825,0.2897265777472348) -- (0.15681011650324939,0.3074728798470581);
\draw[line width=1.2pt] (0.15681011650324939,0.3074728798470581) -- (0.16603424100344052,0.3251766397104839);
\draw[line width=1.2pt] (0.16603424100344052,0.3251766397104839) -- (0.17525836550363166,0.3428378573375122);
\draw[line width=1.2pt] (0.17525836550363166,0.3428378573375122) -- (0.1844824900038228,0.36045653272814293);
\draw[line width=1.2pt] (0.1844824900038228,0.36045653272814293) -- (0.19370661450401394,0.3780326658823762);
\draw[line width=1.2pt] (0.19370661450401394,0.3780326658823762) -- (0.20293073900420508,0.39556625680021196);
\draw[line width=1.2pt] (0.20293073900420508,0.39556625680021196) -- (0.21215486350439622,0.4130573054816502);
\draw[line width=1.2pt] (0.21215486350439622,0.4130573054816502) -- (0.22137898800458736,0.4305058119266909);
\draw[line width=1.2pt] (0.22137898800458736,0.4305058119266909) -- (0.2306031125047785,0.4479117761353341);
\draw[line width=1.2pt] (0.2306031125047785,0.4479117761353341) -- (0.23982723700496963,0.4652751981075798);
\draw[line width=1.2pt] (0.23982723700496963,0.4652751981075798) -- (0.24905136150516077,0.48259607784342795);
\draw[line width=1.2pt] (0.24905136150516077,0.48259607784342795) -- (0.25827548600535194,0.4998744153428787);
\draw[line width=1.2pt] (0.25827548600535194,0.4998744153428787) -- (0.2674996105055431,0.5171102106059319);
\draw[line width=1.2pt] (0.2674996105055431,0.5171102106059319) -- (0.2767237350057343,0.5343034636325876);
\draw[line width=1.2pt] (0.2767237350057343,0.5343034636325876) -- (0.28594785950592544,0.5514541744228457);
\draw[line width=1.2pt] (0.28594785950592544,0.5514541744228457) -- (0.2951719840061166,0.5685623429767064);
\draw[line width=1.2pt] (0.2951719840061166,0.5685623429767064) -- (0.30439610850630777,0.5856279692941696);
\draw[line width=1.2pt] (0.30439610850630777,0.5856279692941696) -- (0.31362023300649894,0.6026510533752352);
\draw[line width=1.2pt] (0.31362023300649894,0.6026510533752352) -- (0.3228443575066901,0.6196315952199033);
\draw[line width=1.2pt] (0.3228443575066901,0.6196315952199033) -- (0.33206848200688127,0.636569594828174);
\draw[line width=1.2pt] (0.33206848200688127,0.636569594828174) -- (0.34129260650707244,0.653465052200047);
\draw[line width=1.2pt] (0.34129260650707244,0.653465052200047) -- (0.3505167310072636,0.6703179673355226);
\draw[line width=1.2pt] (0.3505167310072636,0.6703179673355226) -- (0.35974085550745477,0.6871283402346007);
\draw[line width=1.2pt] (0.35974085550745477,0.6871283402346007) -- (0.36896498000764594,0.7038961708972812);
\draw[line width=1.2pt] (0.36896498000764594,0.7038961708972812) -- (0.3781891045078371,0.7206214593235643);
\draw[line width=1.2pt] (0.3781891045078371,0.7206214593235643) -- (0.38741322900802827,0.7373042055134498);
\draw[line width=1.2pt] (0.38741322900802827,0.7373042055134498) -- (0.39663735350821944,0.7539444094669379);
\draw[line width=1.2pt] (0.39663735350821944,0.7539444094669379) -- (0.4058614780084106,0.7705420711840283);
\draw[line width=1.2pt] (0.4058614780084106,0.7705420711840283) -- (0.41508560250860177,0.7870971906647213);
\draw[line width=1.2pt] (0.41508560250860177,0.7870971906647213) -- (0.42430972700879294,0.8036097679090167);
\draw[line width=1.2pt] (0.42430972700879294,0.8036097679090167) -- (0.4335338515089841,0.8200798029169147);
\draw[line width=1.2pt] (0.4335338515089841,0.8200798029169147) -- (0.44275797600917527,0.8365072956884152);
\draw[line width=1.2pt] (0.44275797600917527,0.8365072956884152) -- (0.45198210050936644,0.8528922462235181);
\draw[line width=1.2pt] (0.45198210050936644,0.8528922462235181) -- (0.4612062250095576,0.8692346545222236);
\draw[line width=1.2pt] (0.4612062250095576,0.8692346545222236) -- (0.47043034950974877,0.8855345205845314);
\draw[line width=1.2pt] (0.47043034950974877,0.8855345205845314) -- (0.47965447400993994,0.9017918444104418);
\draw[line width=1.2pt] (0.47965447400993994,0.9017918444104418) -- (0.4888785985101311,0.9180066259999548);
\draw[line width=1.2pt] (0.4888785985101311,0.9180066259999548) -- (0.49810272301032227,0.93417886535307);
\draw[line width=1.2pt] (0.49810272301032227,0.93417886535307) -- (0.5073268475105134,0.9503085624697878);
\draw[line width=1.2pt] (0.5073268475105134,0.9503085624697878) -- (0.5165509720107045,0.9663957173501082);
\draw[line width=1.2pt] (0.5165509720107045,0.9663957173501082) -- (0.5257750965108957,0.982440329994031);
\draw[line width=1.2pt] (0.5257750965108957,0.982440329994031) -- (0.5349992210110869,0.9984424004015563);
\draw[line width=1.2pt] (0.5349992210110869,0.9984424004015563) -- (0.544223345511278,1.0144019285726842);
\draw[line width=1.2pt] (0.544223345511278,1.0144019285726842) -- (0.5534474700114692,1.0303189145074143);
\draw[line width=1.2pt] (0.5534474700114692,1.0303189145074143) -- (0.5626715945116604,1.0461933582057472);
\draw[line width=1.2pt] (0.5626715945116604,1.0461933582057472) -- (0.5718957190118515,1.0620252596676825);
\draw[line width=1.2pt] (0.5718957190118515,1.0620252596676825) -- (0.5811198435120427,1.0778146188932203);
\draw[line width=1.2pt] (0.5811198435120427,1.0778146188932203) -- (0.5903439680122339,1.0935614358823604);
\draw[line width=1.2pt] (0.5903439680122339,1.0935614358823604) -- (0.599568092512425,1.1092657106351032);
\draw[line width=1.2pt] (0.599568092512425,1.1092657106351032) -- (0.6087922170126162,1.1249274431514484);
\draw[line width=1.2pt] (0.6087922170126162,1.1249274431514484) -- (0.6180163415128074,1.140546633431396);
\draw[line width=1.2pt] (0.6180163415128074,1.140546633431396) -- (0.6272404660129985,1.1561232814749463);
\draw[line width=1.2pt] (0.6272404660129985,1.1561232814749463) -- (0.6364645905131897,1.1716573872820988);
\draw[line width=1.2pt] (0.6364645905131897,1.1716573872820988) -- (0.6456887150133809,1.187148950852854);
\draw[line width=1.2pt] (0.6456887150133809,1.187148950852854) -- (0.654912839513572,1.2025979721872115);
\draw[line width=1.2pt] (0.654912839513572,1.2025979721872115) -- (0.6641369640137632,1.2180044512851718);
\draw[line width=1.2pt] (0.6641369640137632,1.2180044512851718) -- (0.6733610885139544,1.2333683881467343);
\draw[line width=1.2pt] (0.6733610885139544,1.2333683881467343) -- (0.6825852130141455,1.2486897827718995);
\draw[line width=1.2pt] (0.6825852130141455,1.2486897827718995) -- (0.6918093375143367,1.2639686351606672);
\draw[line width=1.2pt] (0.6918093375143367,1.2639686351606672) -- (0.7010334620145279,1.2792049453130372);
\draw[line width=1.2pt] (0.7010334620145279,1.2792049453130372) -- (0.710257586514719,1.2943987132290098);
\draw[line width=1.2pt] (0.710257586514719,1.2943987132290098) -- (0.7194817110149102,1.3095499389085847);
\draw[line width=1.2pt] (0.7194817110149102,1.3095499389085847) -- (0.7287058355151014,1.3246586223517622);
\draw[line width=1.2pt] (0.7287058355151014,1.3246586223517622) -- (0.7379299600152925,1.3397247635585423);
\draw[line width=1.2pt] (0.7379299600152925,1.3397247635585423) -- (0.7471540845154837,1.3547483625289247);
\draw[line width=1.2pt] (0.7471540845154837,1.3547483625289247) -- (0.7563782090156749,1.3697294192629097);
\draw[line width=1.2pt] (0.7563782090156749,1.3697294192629097) -- (0.765602333515866,1.3846679337604972);
\draw[line width=1.2pt] (0.765602333515866,1.3846679337604972) -- (0.7748264580160572,1.3995639060216871);
\draw[line width=1.2pt] (0.7748264580160572,1.3995639060216871) -- (0.7840505825162484,1.4144173360464796);
\draw[line width=1.2pt] (0.7840505825162484,1.4144173360464796) -- (0.7932747070164395,1.4292282238348746);
\draw[line width=1.2pt] (0.7932747070164395,1.4292282238348746) -- (0.8024988315166307,1.443996569386872);
\draw[line width=1.2pt] (0.8024988315166307,1.443996569386872) -- (0.8117229560168219,1.4587223727024718);
\draw[line width=1.2pt] (0.8117229560168219,1.4587223727024718) -- (0.820947080517013,1.4734056337816743);
\draw[line width=1.2pt] (0.820947080517013,1.4734056337816743) -- (0.8301712050172042,1.4880463526244792);
\draw[line width=1.2pt] (0.8301712050172042,1.4880463526244792) -- (0.8393953295173954,1.5026445292308865);
\draw[line width=1.2pt] (0.8393953295173954,1.5026445292308865) -- (0.8486194540175865,1.5172001636008965);
\draw[line width=1.2pt] (0.8486194540175865,1.5172001636008965) -- (0.8578435785177777,1.5317132557345088);
\draw[line width=1.2pt] (0.8578435785177777,1.5317132557345088) -- (0.8670677030179689,1.5461838056317236);
\draw[line width=1.2pt] (0.8670677030179689,1.5461838056317236) -- (0.87629182751816,1.5606118132925408);
\draw[line width=1.2pt] (0.87629182751816,1.5606118132925408) -- (0.8855159520183512,1.5749972787169606);
\draw[line width=1.2pt] (0.8855159520183512,1.5749972787169606) -- (0.8947400765185424,1.589340201904983);
\draw[line width=1.2pt] (0.8947400765185424,1.589340201904983) -- (0.9039642010187335,1.6036405828566078);
\draw[line width=1.2pt] (0.9039642010187335,1.6036405828566078) -- (0.9131883255189247,1.617898421571835);
\draw[line width=1.2pt] (0.9131883255189247,1.617898421571835) -- (0.9224124500191159,1.6321137180506649);
\draw[line width=1.2pt] (0.9224124500191159,1.6321137180506649) -- (0.931636574519307,1.646286472293097);
\draw[line width=1.2pt] (0.931636574519307,1.646286472293097) -- (0.9408606990194982,1.6604166842991317);
\draw[line width=1.2pt] (0.9408606990194982,1.6604166842991317) -- (0.9500848235196894,1.674504354068769);
\draw[line width=1.2pt] (0.9500848235196894,1.674504354068769) -- (0.9593089480198805,1.6885494816020086);
\draw[line width=1.2pt] (0.9593089480198805,1.6885494816020086) -- (0.9685330725200717,1.7025520668988507);
\draw[line width=1.2pt] (0.9685330725200717,1.7025520668988507) -- (0.9777571970202629,1.7165121099592955);
\draw[line width=1.2pt] (0.9777571970202629,1.7165121099592955) -- (0.986981321520454,1.7304296107833426);
\draw[line width=1.2pt] (0.986981321520454,1.7304296107833426) -- (0.9962054460206452,1.7443045693709922);
\draw[line width=1.2pt] (0.9962054460206452,1.7443045693709922) -- (1.0054295705208363,1.7581369857222442);
\draw[line width=1.2pt] (1.0054295705208363,1.7581369857222442) -- (1.0146536950210274,1.7719268598370987);
\draw[line width=1.2pt] (1.0146536950210274,1.7719268598370987) -- (1.0238778195212186,1.7856741917155559);
\draw[line width=1.2pt] (1.0238778195212186,1.7856741917155559) -- (1.0331019440214098,1.7993789813576155);
\draw[line width=1.2pt] (1.0331019440214098,1.7993789813576155) -- (1.042326068521601,1.8130412287632776);
\draw[line width=1.2pt] (1.042326068521601,1.8130412287632776) -- (1.051550193021792,1.8266609339325421);
\draw[line width=1.2pt] (1.051550193021792,1.8266609339325421) -- (1.0607743175219833,1.8402380968654093);
\draw[line width=1.2pt] (1.0607743175219833,1.8402380968654093) -- (1.0699984420221744,1.8537727175618788);
\draw[line width=1.2pt] (1.0699984420221744,1.8537727175618788) -- (1.0792225665223656,1.8672647960219506);
\draw[line width=1.2pt] (1.0792225665223656,1.8672647960219506) -- (1.0884466910225568,1.8807143322456252);
\draw[line width=1.2pt] (1.0884466910225568,1.8807143322456252) -- (1.097670815522748,1.8941213262329022);
\draw[line width=1.2pt] (1.097670815522748,1.8941213262329022) -- (1.106894940022939,1.9074857779837817);
\draw[line width=1.2pt] (1.106894940022939,1.9074857779837817) -- (1.1161190645231303,1.9208076874982636);
\draw[line width=1.2pt] (1.1161190645231303,1.9208076874982636) -- (1.1253431890233214,1.9340870547763482);
\draw[line width=1.2pt] (1.1253431890233214,1.9340870547763482) -- (1.1345673135235126,1.947323879818035);
\draw[line width=1.2pt] (1.1345673135235126,1.947323879818035) -- (1.1437914380237038,1.9605181626233246);
\draw[line width=1.2pt] (1.1437914380237038,1.9605181626233246) -- (1.153015562523895,1.9736699031922162);
\draw[line width=1.2pt] (1.153015562523895,1.9736699031922162) -- (1.162239687024086,1.9867791015247107);
\draw[line width=1.2pt] (1.162239687024086,1.9867791015247107) -- (1.1714638115242773,1.9998457576208077);
\draw[line width=1.2pt] (1.1714638115242773,1.9998457576208077) -- (1.1806879360244684,2.012869871480507);
\draw[line width=1.2pt] (1.1806879360244684,2.012869871480507) -- (1.1899120605246596,2.025851443103809);
\draw[line width=1.2pt] (1.1899120605246596,2.025851443103809) -- (1.1991361850248508,2.0387904724907133);
\draw[line width=1.2pt] (1.1991361850248508,2.0387904724907133) -- (1.208360309525042,2.05168695964122);
\draw[line width=1.2pt] (1.208360309525042,2.05168695964122) -- (1.217584434025233,2.0645409045553293);
\draw[line width=1.2pt] (1.217584434025233,2.0645409045553293) -- (1.2268085585254243,2.077352307233041);
\draw[line width=1.2pt] (1.2268085585254243,2.077352307233041) -- (1.2360326830256154,2.0901211676743556);
\draw[line width=1.2pt] (1.2360326830256154,2.0901211676743556) -- (1.2452568075258066,2.102847485879272);
\draw[line width=1.2pt] (1.2452568075258066,2.102847485879272) -- (1.2544809320259978,2.1155312618477913);
\draw[line width=1.2pt] (1.2544809320259978,2.1155312618477913) -- (1.263705056526189,2.1281724955799133);
\draw[line width=1.2pt] (1.263705056526189,2.1281724955799133) -- (1.27292918102638,2.1407711870756376);
\draw[line width=1.2pt] (1.27292918102638,2.1407711870756376) -- (1.2821533055265713,2.1533273363349643);
\draw[line width=1.2pt] (1.2821533055265713,2.1533273363349643) -- (1.2913774300267624,2.1658409433578933);
\draw[line width=1.2pt] (1.2913774300267624,2.1658409433578933) -- (1.3006015545269536,2.178312008144425);
\draw[line width=1.2pt] (1.3006015545269536,2.178312008144425) -- (1.3098256790271448,2.1907405306945593);
\draw[line width=1.2pt] (1.3098256790271448,2.1907405306945593) -- (1.319049803527336,2.203126511008296);
\draw[line width=1.2pt] (1.319049803527336,2.203126511008296) -- (1.328273928027527,2.215469949085635);
\draw[line width=1.2pt] (1.328273928027527,2.215469949085635) -- (1.3374980525277183,2.2277708449265767);
\draw[line width=1.2pt] (1.3374980525277183,2.2277708449265767) -- (1.3467221770279094,2.2400291985311207);
\draw[line width=1.2pt] (1.3467221770279094,2.2400291985311207) -- (1.3559463015281006,2.2522450098992675);
\draw[line width=1.2pt] (1.3559463015281006,2.2522450098992675) -- (1.3651704260282918,2.2644182790310166);
\draw[line width=1.2pt] (1.3651704260282918,2.2644182790310166) -- (1.374394550528483,2.276549005926368);
\draw[line width=1.2pt] (1.374394550528483,2.276549005926368) -- (1.383618675028674,2.2886371905853222);
\draw[line width=1.2pt] (1.383618675028674,2.2886371905853222) -- (1.3928427995288652,2.300682833007879);
\draw[line width=1.2pt] (1.3928427995288652,2.300682833007879) -- (1.4020669240290564,2.3126859331940377);
\draw[line width=1.2pt] (1.4020669240290564,2.3126859331940377) -- (1.4112910485292476,2.3246464911437994);
\draw[line width=1.2pt] (1.4112910485292476,2.3246464911437994) -- (1.4205151730294387,2.3365645068571634);
\draw[line width=1.2pt] (1.4205151730294387,2.3365645068571634) -- (1.42973929752963,2.34843998033413);
\draw[line width=1.2pt] (1.42973929752963,2.34843998033413) -- (1.438963422029821,2.360272911574699);
\draw[line width=1.2pt] (1.438963422029821,2.360272911574699) -- (1.4481875465300122,2.3720633005788705);
\draw[line width=1.2pt] (1.4481875465300122,2.3720633005788705) -- (1.4574116710302034,2.3838111473466443);
\draw[line width=1.2pt] (1.4574116710302034,2.3838111473466443) -- (1.4666357955303946,2.395516451878021);
\draw[line width=1.2pt] (1.4666357955303946,2.395516451878021) -- (1.4758599200305857,2.4071792141729995);
\draw[line width=1.2pt] (1.4758599200305857,2.4071792141729995) -- (1.485084044530777,2.4187994342315813);
\draw[line width=1.2pt] (1.485084044530777,2.4187994342315813) -- (1.494308169030968,2.430377112053765);
\draw[line width=1.2pt] (1.494308169030968,2.430377112053765) -- (1.5035322935311592,2.4419122476395514);
\draw[line width=1.2pt] (1.5035322935311592,2.4419122476395514) -- (1.5127564180313504,2.4534048409889406);
\draw[line width=1.2pt] (1.5127564180313504,2.4534048409889406) -- (1.5219805425315416,2.4648548921019318);
\draw[line width=1.2pt] (1.5219805425315416,2.4648548921019318) -- (1.5312046670317327,2.4762624009785257);
\draw[line width=1.2pt] (1.5312046670317327,2.4762624009785257) -- (1.540428791531924,2.487627367618722);
\draw[line width=1.2pt] (1.540428791531924,2.487627367618722) -- (1.549652916032115,2.498949792022521);
\draw[line width=1.2pt] (1.549652916032115,2.498949792022521) -- (1.5588770405323062,2.510229674189922);
\draw[line width=1.2pt] (1.5588770405323062,2.510229674189922) -- (1.5681011650324974,2.521467014120926);
\draw[line width=1.2pt] (1.5681011650324974,2.521467014120926) -- (1.5773252895326886,2.532661811815532);
\draw[line width=1.2pt] (1.5773252895326886,2.532661811815532) -- (1.5865494140328797,2.543814067273741);
\draw[line width=1.2pt] (1.5865494140328797,2.543814067273741) -- (1.595773538533071,2.5549237804955522);
\draw[line width=1.2pt] (1.595773538533071,2.5549237804955522) -- (1.604997663033262,2.5659909514809662);
\draw[line width=1.2pt] (1.604997663033262,2.5659909514809662) -- (1.6142217875334532,2.577015580229982);
\draw[line width=1.2pt] (1.6142217875334532,2.577015580229982) -- (1.6234459120336444,2.587997666742601);
\draw[line width=1.2pt] (1.6234459120336444,2.587997666742601) -- (1.6326700365338356,2.5989372110188222);
\draw[line width=1.2pt] (1.6326700365338356,2.5989372110188222) -- (1.6418941610340267,2.6098342130586456);
\draw[line width=1.2pt] (1.6418941610340267,2.6098342130586456) -- (1.651118285534218,2.620688672862072);
\draw[line width=1.2pt] (1.651118285534218,2.620688672862072) -- (1.660342410034409,2.6315005904291007);
\draw[line width=1.2pt] (1.660342410034409,2.6315005904291007) -- (1.6695665345346002,2.642269965759732);
\draw[line width=1.2pt] (1.6695665345346002,2.642269965759732) -- (1.6787906590347914,2.6529967988539656);
\draw[line width=1.2pt] (1.6787906590347914,2.6529967988539656) -- (1.6880147835349826,2.6636810897118015);
\draw[line width=1.2pt] (1.6880147835349826,2.6636810897118015) -- (1.6972389080351737,2.6743228383332402);
\draw[line width=1.2pt] (1.6972389080351737,2.6743228383332402) -- (1.706463032535365,2.6849220447182813);
\draw[line width=1.2pt] (1.706463032535365,2.6849220447182813) -- (1.715687157035556,2.6954787088669248);
\draw[line width=1.2pt] (1.715687157035556,2.6954787088669248) -- (1.7249112815357472,2.705992830779171);
\draw[line width=1.2pt] (1.7249112815357472,2.705992830779171) -- (1.7341354060359384,2.7164644104550195);
\draw[line width=1.2pt] (1.7341354060359384,2.7164644104550195) -- (1.7433595305361296,2.7268934478944704);
\draw[line width=1.2pt] (1.7433595305361296,2.7268934478944704) -- (1.7525836550363207,2.737279943097524);
\draw[line width=1.2pt] (1.7525836550363207,2.737279943097524) -- (1.761807779536512,2.74762389606418);
\draw[line width=1.2pt] (1.761807779536512,2.74762389606418) -- (1.771031904036703,2.757925306794439);
\draw[line width=1.2pt] (1.771031904036703,2.757925306794439) -- (1.7802560285368942,2.7681841752882996);
\draw[line width=1.2pt] (1.7802560285368942,2.7681841752882996) -- (1.7894801530370854,2.778400501545763);
\draw[line width=1.2pt] (1.7894801530370854,2.778400501545763) -- (1.7987042775372766,2.7885742855668294);
\draw[line width=1.2pt] (1.7987042775372766,2.7885742855668294) -- (1.8079284020374677,2.7987055273514976);
\draw[line width=1.2pt] (1.8079284020374677,2.7987055273514976) -- (1.817152526537659,2.8087942268997685);
\draw[line width=1.2pt] (1.817152526537659,2.8087942268997685) -- (1.82637665103785,2.818840384211642);
\draw[line width=1.2pt] (1.82637665103785,2.818840384211642) -- (1.8356007755380412,2.828843999287118);
\draw[line width=1.2pt] (1.8356007755380412,2.828843999287118) -- (1.8448249000382324,2.8388050721261964);
\draw[line width=1.2pt] (1.8448249000382324,2.8388050721261964) -- (1.8540490245384236,2.848723602728877);
\draw[line width=1.2pt] (1.8540490245384236,2.848723602728877) -- (1.8632731490386147,2.8585995910951603);
\draw[line width=1.2pt] (1.8632731490386147,2.8585995910951603) -- (1.872497273538806,2.8684330372250466);
\draw[line width=1.2pt] (1.872497273538806,2.8684330372250466) -- (1.881721398038997,2.878223941118535);
\draw[line width=1.2pt] (1.881721398038997,2.878223941118535) -- (1.8909455225391882,2.8879723027756254);
\draw[line width=1.2pt] (1.8909455225391882,2.8879723027756254) -- (1.9001696470393794,2.897678122196319);
\draw[line width=1.2pt] (1.9001696470393794,2.897678122196319) -- (1.9093937715395706,2.9073413993806145);
\draw[line width=1.2pt] (1.9093937715395706,2.9073413993806145) -- (1.9186178960397617,2.916962134328513);
\draw[line width=1.2pt] (1.9186178960397617,2.916962134328513) -- (1.927842020539953,2.926540327040014);
\draw[line width=1.2pt] (1.927842020539953,2.926540327040014) -- (1.937066145040144,2.936075977515117);
\draw[line width=1.2pt] (1.937066145040144,2.936075977515117) -- (1.9462902695403352,2.9455690857538226);
\draw[line width=1.2pt] (1.9462902695403352,2.9455690857538226) -- (1.9555143940405264,2.955019651756131);
\draw[line width=1.2pt] (1.9555143940405264,2.955019651756131) -- (1.9647385185407176,2.9644276755220416);
\draw[line width=1.2pt] (1.9647385185407176,2.9644276755220416) -- (1.9739626430409087,2.973793157051555);
\draw[line width=1.2pt] (1.9739626430409087,2.973793157051555) -- (1.9831867675411,2.983116096344671);
\draw[line width=1.2pt] (1.9831867675411,2.983116096344671) -- (1.992410892041291,2.992396493401389);
\draw[line width=1.2pt] (1.992410892041291,2.992396493401389) -- (2.0016350165414822,3.0016343482217094);
\draw[line width=1.2pt] (2.0016350165414822,3.0016343482217094) -- (2.0108591410416734,3.0108296608056326);
\draw[line width=1.2pt] (2.0108591410416734,3.0108296608056326) -- (2.0200832655418646,3.0199824311531582);
\draw[line width=1.2pt] (2.0200832655418646,3.0199824311531582) -- (2.0293073900420557,3.029092659264286);
\draw[line width=1.2pt] (2.0293073900420557,3.029092659264286) -- (2.038531514542247,3.0381603451390173);
\draw[line width=1.2pt] (2.038531514542247,3.0381603451390173) -- (2.047755639042438,3.04718548877735);
\draw[line width=1.2pt] (2.047755639042438,3.04718548877735) -- (2.0569797635426292,3.056168090179286);
\draw[line width=1.2pt] (2.0569797635426292,3.056168090179286) -- (2.0662038880428204,3.065108149344824);
\draw[line width=1.2pt] (2.0662038880428204,3.065108149344824) -- (2.0754280125430116,3.0740056662739645);
\draw[line width=1.2pt] (2.0754280125430116,3.0740056662739645) -- (2.0846521370432027,3.0828606409667074);
\draw[line width=1.2pt] (2.0846521370432027,3.0828606409667074) -- (2.093876261543394,3.091673073423053);
\draw[line width=1.2pt] (2.093876261543394,3.091673073423053) -- (2.103100386043585,3.100442963643001);
\draw[line width=1.2pt] (2.103100386043585,3.100442963643001) -- (2.1123245105437762,3.1091703116265514);
\draw[line width=1.2pt] (2.1123245105437762,3.1091703116265514) -- (2.1215486350439674,3.1178551173737046);
\draw[line width=1.2pt] (2.1215486350439674,3.1178551173737046) -- (2.1307727595441586,3.12649738088446);
\draw[line width=1.2pt] (2.1307727595441586,3.12649738088446) -- (2.1399968840443497,3.135097102158818);
\draw[line width=1.2pt] (2.1399968840443497,3.135097102158818) -- (2.149221008544541,3.1436542811967785);
\draw[line width=1.2pt] (2.149221008544541,3.1436542811967785) -- (2.158445133044732,3.1521689179983414);
\draw[line width=1.2pt] (2.158445133044732,3.1521689179983414) -- (2.1676692575449232,3.1606410125635067);
\draw[line width=1.2pt] (2.1676692575449232,3.1606410125635067) -- (2.1768933820451144,3.169070564892275);
\draw[line width=1.2pt] (2.1768933820451144,3.169070564892275) -- (2.1861175065453056,3.177457574984645);
\draw[line width=1.2pt] (2.1861175065453056,3.177457574984645) -- (2.1953416310454967,3.185802042840618);
\draw[line width=1.2pt] (2.1953416310454967,3.185802042840618) -- (2.204565755545688,3.194103968460193);
\draw[line width=1.2pt] (2.204565755545688,3.194103968460193) -- (2.213789880045879,3.2023633518433714);
\draw[line width=1.2pt] (2.213789880045879,3.2023633518433714) -- (2.2230140045460702,3.2105801929901516);
\draw[line width=1.2pt] (2.2230140045460702,3.2105801929901516) -- (2.2322381290462614,3.218754491900534);
\draw[line width=1.2pt] (2.2322381290462614,3.218754491900534) -- (2.2414622535464526,3.2268862485745196);
\draw[line width=1.2pt] (2.2414622535464526,3.2268862485745196) -- (2.2506863780466437,3.2349754630121073);
\draw[line width=1.2pt] (2.2506863780466437,3.2349754630121073) -- (2.259910502546835,3.2430221352132977);
\draw[line width=1.2pt] (2.259910502546835,3.2430221352132977) -- (2.269134627047026,3.2510262651780906);
\draw[line width=1.2pt] (2.269134627047026,3.2510262651780906) -- (2.2783587515472172,3.258987852906486);
\draw[line width=1.2pt] (2.2783587515472172,3.258987852906486) -- (2.2875828760474084,3.2669068983984837);
\draw[line width=1.2pt] (2.2875828760474084,3.2669068983984837) -- (2.2968070005475996,3.2747834016540835);
\draw[line width=1.2pt] (2.2968070005475996,3.2747834016540835) -- (2.3060311250477907,3.2826173626732866);
\draw[line width=1.2pt] (2.3060311250477907,3.2826173626732866) -- (2.315255249547982,3.290408781456092);
\draw[line width=1.2pt] (2.315255249547982,3.290408781456092) -- (2.324479374048173,3.2981576580024994);
\draw[line width=1.2pt] (2.324479374048173,3.2981576580024994) -- (2.3337034985483642,3.30586399231251);
\draw[line width=1.2pt] (2.3337034985483642,3.30586399231251) -- (2.3429276230485554,3.3135277843861224);
\draw[line width=1.2pt] (2.3429276230485554,3.3135277843861224) -- (2.3521517475487466,3.3211490342233376);
\draw[line width=1.2pt] (2.3521517475487466,3.3211490342233376) -- (2.3613758720489377,3.328727741824155);
\draw[line width=1.2pt] (2.3613758720489377,3.328727741824155) -- (2.370599996549129,3.3362639071885756);
\draw[line width=1.2pt] (2.370599996549129,3.3362639071885756) -- (2.37982412104932,3.343757530316598);
\draw[line width=1.2pt] (2.37982412104932,3.343757530316598) -- (2.3890482455495112,3.3512086112082233);
\draw[line width=1.2pt] (2.3890482455495112,3.3512086112082233) -- (2.3982723700497024,3.3586171498634507);
\draw[line width=1.2pt] (2.3982723700497024,3.3586171498634507) -- (2.4074964945498936,3.3659831462822805);
\draw[line width=1.2pt] (2.4074964945498936,3.3659831462822805) -- (2.4167206190500847,3.373306600464713);
\draw[line width=1.2pt] (2.4167206190500847,3.373306600464713) -- (2.425944743550276,3.3805875124107483);
\draw[line width=1.2pt] (2.425944743550276,3.3805875124107483) -- (2.435168868050467,3.387825882120386);
\draw[line width=1.2pt] (2.435168868050467,3.387825882120386) -- (2.4443929925506582,3.395021709593626);
\draw[line width=1.2pt] (2.4443929925506582,3.395021709593626) -- (2.4536171170508494,3.4021749948304683);
\draw[line width=1.2pt] (2.4536171170508494,3.4021749948304683) -- (2.4628412415510406,3.4092857378309134);
\draw[line width=1.2pt] (2.4628412415510406,3.4092857378309134) -- (2.4720653660512317,3.416353938594961);
\draw[line width=1.2pt] (2.4720653660512317,3.416353938594961) -- (2.481289490551423,3.4233795971226106);
\draw[line width=1.2pt] (2.481289490551423,3.4233795971226106) -- (2.490513615051614,3.4303627134138632);
\draw[line width=1.2pt] (2.490513615051614,3.4303627134138632) -- (2.4997377395518052,3.437303287468718);
\draw[line width=1.2pt] (2.4997377395518052,3.437303287468718) -- (2.5089618640519964,3.444201319287176);
\draw[line width=1.2pt] (2.5089618640519964,3.444201319287176) -- (2.5181859885521876,3.4510568088692355);
\draw[line width=1.2pt] (2.5181859885521876,3.4510568088692355) -- (2.5274101130523787,3.457869756214898);
\draw[line width=1.2pt] (2.5274101130523787,3.457869756214898) -- (2.53663423755257,3.464640161324163);
\draw[line width=1.2pt] (2.53663423755257,3.464640161324163) -- (2.545858362052761,3.47136802419703);
\draw[line width=1.2pt] (2.545858362052761,3.47136802419703) -- (2.5550824865529522,3.4780533448335);
\draw[line width=1.2pt] (2.5550824865529522,3.4780533448335) -- (2.5643066110531434,3.4846961232335723);
\draw[line width=1.2pt] (2.5643066110531434,3.4846961232335723) -- (2.5735307355533346,3.4912963593972473);
\draw[line width=1.2pt] (2.5735307355533346,3.4912963593972473) -- (2.5827548600535257,3.4978540533245246);
\draw[line width=1.2pt] (2.5827548600535257,3.4978540533245246) -- (2.591978984553717,3.5043692050154043);
\draw[line width=1.2pt] (2.591978984553717,3.5043692050154043) -- (2.601203109053908,3.5108418144698867);
\draw[line width=1.2pt] (2.601203109053908,3.5108418144698867) -- (2.6104272335540992,3.5172718816879716);
\draw[line width=1.2pt] (2.6104272335540992,3.5172718816879716) -- (2.6196513580542904,3.5236594066696587);
\draw[line width=1.2pt] (2.6196513580542904,3.5236594066696587) -- (2.6288754825544816,3.5300043894149487);
\draw[line width=1.2pt] (2.6288754825544816,3.5300043894149487) -- (2.6380996070546727,3.5363068299238405);
\draw[line width=1.2pt] (2.6380996070546727,3.5363068299238405) -- (2.647323731554864,3.5425667281963356);
\draw[line width=1.2pt] (2.647323731554864,3.5425667281963356) -- (2.656547856055055,3.548784084232433);
\draw[line width=1.2pt] (2.656547856055055,3.548784084232433) -- (2.6657719805552462,3.5549588980321323);
\draw[line width=1.2pt] (2.6657719805552462,3.5549588980321323) -- (2.6749961050554374,3.561091169595435);
\draw[line width=1.2pt] (2.6749961050554374,3.561091169595435) -- (2.6842202295556286,3.5671808989223392);
\draw[line width=1.2pt] (2.6842202295556286,3.5671808989223392) -- (2.6934443540558197,3.5732280860128465);
\draw[line width=1.2pt] (2.6934443540558197,3.5732280860128465) -- (2.702668478556011,3.579232730866956);
\draw[line width=1.2pt] (2.702668478556011,3.579232730866956) -- (2.711892603056202,3.5851948334846684);
\draw[line width=1.2pt] (2.711892603056202,3.5851948334846684) -- (2.7211167275563932,3.5911143938659826);
\draw[line width=1.2pt] (2.7211167275563932,3.5911143938659826) -- (2.7303408520565844,3.5969914120109);
\draw[line width=1.2pt] (2.7303408520565844,3.5969914120109) -- (2.7395649765567756,3.6028258879194195);
\draw[line width=1.2pt] (2.7395649765567756,3.6028258879194195) -- (2.7487891010569667,3.6086178215915417);
\draw[line width=1.2pt] (2.7487891010569667,3.6086178215915417) -- (2.758013225557158,3.614367213027266);
\draw[line width=1.2pt] (2.758013225557158,3.614367213027266) -- (2.767237350057349,3.6200740622265934);
\draw[line width=1.2pt] (2.767237350057349,3.6200740622265934) -- (2.77646147455754,3.6257383691895226);
\draw[line width=1.2pt] (2.77646147455754,3.6257383691895226) -- (2.7856855990577314,3.631360133916055);
\draw[line width=1.2pt] (2.7856855990577314,3.631360133916055) -- (2.7949097235579226,3.6369393564061894);
\draw[line width=1.2pt] (2.7949097235579226,3.6369393564061894) -- (2.8041338480581137,3.642476036659926);
\draw[line width=1.2pt] (2.8041338480581137,3.642476036659926) -- (2.813357972558305,3.647970174677266);
\draw[line width=1.2pt] (2.813357972558305,3.647970174677266) -- (2.822582097058496,3.6534217704582077);
\draw[line width=1.2pt] (2.822582097058496,3.6534217704582077) -- (2.831806221558687,3.6588308240027523);
\draw[line width=1.2pt] (2.831806221558687,3.6588308240027523) -- (2.8410303460588784,3.664197335310899);
\draw[line width=1.2pt] (2.8410303460588784,3.664197335310899) -- (2.8502544705590696,3.6695213043826485);
\draw[line width=1.2pt] (2.8502544705590696,3.6695213043826485) -- (2.8594785950592607,3.6748027312180005);
\draw[line width=1.2pt] (2.8594785950592607,3.6748027312180005) -- (2.868702719559452,3.680041615816955);
\draw[line width=1.2pt] (2.868702719559452,3.680041615816955) -- (2.877926844059643,3.685237958179512);
\draw[line width=1.2pt] (2.877926844059643,3.685237958179512) -- (2.887150968559834,3.690391758305671);
\draw[line width=1.2pt] (2.887150968559834,3.690391758305671) -- (2.8963750930600254,3.695503016195433);
\draw[line width=1.2pt] (2.8963750930600254,3.695503016195433) -- (2.9055992175602166,3.7005717318487976);
\draw[line width=1.2pt] (2.9055992175602166,3.7005717318487976) -- (2.9148233420604077,3.705597905265764);
\draw[line width=1.2pt] (2.9148233420604077,3.705597905265764) -- (2.924047466560599,3.7105815364463335);
\draw[line width=1.2pt] (2.924047466560599,3.7105815364463335) -- (2.93327159106079,3.7155226253905056);
\draw[line width=1.2pt] (2.93327159106079,3.7155226253905056) -- (2.942495715560981,3.7204211720982796);
\draw[line width=1.2pt] (2.942495715560981,3.7204211720982796) -- (2.9517198400611724,3.7252771765696564);
\draw[line width=1.2pt] (2.9517198400611724,3.7252771765696564) -- (2.9609439645613636,3.730090638804636);
\draw[line width=1.2pt] (2.9609439645613636,3.730090638804636) -- (2.9701680890615547,3.7348615588032175);
\draw[line width=1.2pt] (2.9701680890615547,3.7348615588032175) -- (2.979392213561746,3.7395899365654017);
\draw[line width=1.2pt] (2.979392213561746,3.7395899365654017) -- (2.988616338061937,3.7442757720911883);
\draw[line width=1.2pt] (2.988616338061937,3.7442757720911883) -- (2.997840462562128,3.7489190653805777);
\draw[line width=1.2pt] (2.997840462562128,3.7489190653805777) -- (3.0070645870623194,3.7535198164335695);
\draw[line width=1.2pt] (3.0070645870623194,3.7535198164335695) -- (3.0162887115625105,3.7580780252501635);
\draw[line width=1.2pt] (3.0162887115625105,3.7580780252501635) -- (3.0255128360627017,3.7625936918303604);
\draw[line width=1.2pt] (3.0255128360627017,3.7625936918303604) -- (3.034736960562893,3.7670668161741596);
\draw[line width=1.2pt] (3.034736960562893,3.7670668161741596) -- (3.043961085063084,3.771497398281561);
\draw[line width=1.2pt] (3.043961085063084,3.771497398281561) -- (3.053185209563275,3.775885438152565);
\draw[line width=1.2pt] (3.053185209563275,3.775885438152565) -- (3.0624093340634664,3.7802309357871717);
\draw[line width=1.2pt] (3.0624093340634664,3.7802309357871717) -- (3.0716334585636575,3.7845338911853807);
\draw[line width=1.2pt] (3.0716334585636575,3.7845338911853807) -- (3.0808575830638487,3.7887943043471926);
\draw[line width=1.2pt] (3.0808575830638487,3.7887943043471926) -- (3.09008170756404,3.7930121752726067);
\draw[line width=1.2pt] (3.09008170756404,3.7930121752726067) -- (3.099305832064231,3.797187503961623);
\draw[line width=1.2pt] (3.099305832064231,3.797187503961623) -- (3.108529956564422,3.8013202904142425);
\draw[line width=1.2pt] (3.108529956564422,3.8013202904142425) -- (3.1177540810646134,3.8054105346304636);
\draw[line width=1.2pt] (3.1177540810646134,3.8054105346304636) -- (3.1269782055648045,3.809458236610288);
\draw[line width=1.2pt] (3.1269782055648045,3.809458236610288) -- (3.1362023300649957,3.8134633963537143);
\draw[line width=1.2pt] (3.1362023300649957,3.8134633963537143) -- (3.145426454565187,3.8174260138607434);
\draw[line width=1.2pt] (3.145426454565187,3.8174260138607434) -- (3.154650579065378,3.821346089131375);
\draw[line width=1.2pt] (3.154650579065378,3.821346089131375) -- (3.163874703565569,3.8252236221656086);
\draw[line width=1.2pt] (3.163874703565569,3.8252236221656086) -- (3.1730988280657604,3.829058612963445);
\draw[line width=1.2pt] (3.1730988280657604,3.829058612963445) -- (3.1823229525659515,3.832851061524884);
\draw[line width=1.2pt] (3.1823229525659515,3.832851061524884) -- (3.1915470770661427,3.8366009678499258);
\draw[line width=1.2pt] (3.1915470770661427,3.8366009678499258) -- (3.200771201566334,3.8403083319385694);
\draw[line width=1.2pt] (3.200771201566334,3.8403083319385694) -- (3.209995326066525,3.843973153790816);
\draw[line width=1.2pt] (3.209995326066525,3.843973153790816) -- (3.219219450566716,3.847595433406665);
\draw[line width=1.2pt] (3.219219450566716,3.847595433406665) -- (3.2284435750669074,3.8511751707861164);
\draw[line width=1.2pt] (3.2284435750669074,3.8511751707861164) -- (3.2376676995670985,3.85471236592917);
\draw[line width=1.2pt] (3.2376676995670985,3.85471236592917) -- (3.2468918240672897,3.8582070188358264);
\draw[line width=1.2pt] (3.2468918240672897,3.8582070188358264) -- (3.256115948567481,3.8616591295060854);
\draw[line width=1.2pt] (3.256115948567481,3.8616591295060854) -- (3.265340073067672,3.8650686979399467);
\draw[line width=1.2pt] (3.265340073067672,3.8650686979399467) -- (3.274564197567863,3.8684357241374103);
\draw[line width=1.2pt] (3.274564197567863,3.8684357241374103) -- (3.2837883220680544,3.8717602080984768);
\draw[line width=1.2pt] (3.2837883220680544,3.8717602080984768) -- (3.2930124465682455,3.8750421498231455);
\draw[line width=1.2pt] (3.2930124465682455,3.8750421498231455) -- (3.3022365710684367,3.8782815493114167);
\draw[line width=1.2pt] (3.3022365710684367,3.8782815493114167) -- (3.311460695568628,3.8814784065632906);
\draw[line width=1.2pt] (3.311460695568628,3.8814784065632906) -- (3.320684820068819,3.884632721578767);
\draw[line width=1.2pt] (3.320684820068819,3.884632721578767) -- (3.32990894456901,3.8877444943578454);
\draw[line width=1.2pt] (3.32990894456901,3.8877444943578454) -- (3.3391330690692014,3.890813724900527);
\draw[line width=1.2pt] (3.3391330690692014,3.890813724900527) -- (3.3483571935693925,3.8938404132068105);
\draw[line width=1.2pt] (3.3483571935693925,3.8938404132068105) -- (3.3575813180695837,3.8968245592766966);
\draw[line width=1.2pt] (3.3575813180695837,3.8968245592766966) -- (3.366805442569775,3.8997661631101854);
\draw[line width=1.2pt] (3.366805442569775,3.8997661631101854) -- (3.376029567069966,3.9026652247072766);
\draw[line width=1.2pt] (3.376029567069966,3.9026652247072766) -- (3.385253691570157,3.90552174406797);
\draw[line width=1.2pt] (3.385253691570157,3.90552174406797) -- (3.3944778160703484,3.9083357211922665);
\draw[line width=1.2pt] (3.3944778160703484,3.9083357211922665) -- (3.4037019405705395,3.911107156080165);
\draw[line width=1.2pt] (3.4037019405705395,3.911107156080165) -- (3.4129260650707307,3.913836048731666);
\draw[line width=1.2pt] (3.4129260650707307,3.913836048731666) -- (3.422150189570922,3.9165223991467695);
\draw[line width=1.2pt] (3.422150189570922,3.9165223991467695) -- (3.431374314071113,3.9191662073254756);
\draw[line width=1.2pt] (3.431374314071113,3.9191662073254756) -- (3.440598438571304,3.921767473267784);
\draw[line width=1.2pt] (3.440598438571304,3.921767473267784) -- (3.4498225630714954,3.9243261969736953);
\draw[line width=1.2pt] (3.4498225630714954,3.9243261969736953) -- (3.4590466875716865,3.926842378443209);
\draw[line width=1.2pt] (3.4590466875716865,3.926842378443209) -- (3.4682708120718777,3.929316017676325);
\draw[line width=1.2pt] (3.4682708120718777,3.929316017676325) -- (3.477494936572069,3.9317471146730436);
\draw[line width=1.2pt] (3.477494936572069,3.9317471146730436) -- (3.48671906107226,3.9341356694333642);
\draw[line width=1.2pt] (3.48671906107226,3.9341356694333642) -- (3.495943185572451,3.936481681957288);
\draw[line width=1.2pt] (3.495943185572451,3.936481681957288) -- (3.5051673100726424,3.938785152244814);
\draw[line width=1.2pt] (3.5051673100726424,3.938785152244814) -- (3.5143914345728335,3.9410460802959424);
\draw[line width=1.2pt] (3.5143914345728335,3.9410460802959424) -- (3.5236155590730247,3.9432644661106733);
\draw[line width=1.2pt] (3.5236155590730247,3.9432644661106733) -- (3.532839683573216,3.945440309689007);
\draw[line width=1.2pt] (3.532839683573216,3.945440309689007) -- (3.542063808073407,3.9475736110309425);
\draw[line width=1.2pt] (3.542063808073407,3.9475736110309425) -- (3.551287932573598,3.949664370136481);
\draw[line width=1.2pt] (3.551287932573598,3.949664370136481) -- (3.5605120570737894,3.951712587005622);
\draw[line width=1.2pt] (3.5605120570737894,3.951712587005622) -- (3.5697361815739805,3.9537182616383655);
\draw[line width=1.2pt] (3.5697361815739805,3.9537182616383655) -- (3.5789603060741717,3.9556813940347113);
\draw[line width=1.2pt] (3.5789603060741717,3.9556813940347113) -- (3.588184430574363,3.9576019841946595);
\draw[line width=1.2pt] (3.588184430574363,3.9576019841946595) -- (3.597408555074554,3.9594800321182104);
\draw[line width=1.2pt] (3.597408555074554,3.9594800321182104) -- (3.606632679574745,3.9613155378053637);
\draw[line width=1.2pt] (3.606632679574745,3.9613155378053637) -- (3.6158568040749364,3.9631085012561194);
\draw[line width=1.2pt] (3.6158568040749364,3.9631085012561194) -- (3.6250809285751275,3.9648589224704778);
\draw[line width=1.2pt] (3.6250809285751275,3.9648589224704778) -- (3.6343050530753187,3.9665668014484385);
\draw[line width=1.2pt] (3.6343050530753187,3.9665668014484385) -- (3.64352917757551,3.968232138190002);
\draw[line width=1.2pt] (3.64352917757551,3.968232138190002) -- (3.652753302075701,3.9698549326951675);
\draw[line width=1.2pt] (3.652753302075701,3.9698549326951675) -- (3.661977426575892,3.9714351849639358);
\draw[line width=1.2pt] (3.661977426575892,3.9714351849639358) -- (3.6712015510760834,3.972972894996307);
\draw[line width=1.2pt] (3.6712015510760834,3.972972894996307) -- (3.6804256755762745,3.9744680627922797);
\draw[line width=1.2pt] (3.6804256755762745,3.9744680627922797) -- (3.6896498000764657,3.9759206883518554);
\begin{scriptsize}
\draw [fill=black] (0.,0.) circle (1.5pt);
\end{scriptsize}
\end{tikzpicture}}
	\end{multipleChoice}
\end{exercise}

% Denkvragen over de valbeweging

\begin{exercise}
	Een lichaam wordt verticaal omhoog geworpen. De referentieas is omhoog gericht. Welke van de volgende $v(t)$-diagrammen geeft dan het juiste verloop van de snelheid weer?
	\begin{image}[.92\textwidth]
		\includegraphics[width=\textwidth]{vw}
	\end{image}
	\begin{oplossing}
		Het juist antwoord is (b). De versnelling is constant waardoor de snelheid lineair moet verlopen in de tijd. Aangezien de referentieas naar boven is gekozen, moet de snelheid in het naar boven bewegen positief zijn. Dat is het geval bij (b).
	\end{oplossing}
\end{exercise}

\begin{exercise}
	Vanaf een klif laat men vanop dezelfde hoogte twee identieke bollen vallen. Men laat de tweede bol \'e\'en seconde later vallen dan de eerste. De luchtwrijving is \textit{niet} te verwaarlozen. Dan
	\begin{multipleChoice}
		\choice{zal de tweede bol iets later dan \'e\'en seconde na de eerste neerkomen.}
		\choice{zal de tweede bol iets vroeger dan \'e\'en seconde na de eerste neerkomen.}
		\choice[correct]{zal de tweede bol exact \'e\'en seconde na de eerste neerkomen.}
		\choice{kunnen we hieromtrent geen uitspraak doen bij gebrek aan gegevens.}
	\end{multipleChoice}
	\begin{oplossing}
		Voor beide bollen is de omstandigheid waarin ze vallen gelijk.
	\end{oplossing}
\end{exercise}

\begin{exercise}
	Vanop een grote hoogte laat men achtereenvolgens twee stenen vallen met een tussentijd van 2 seconden. Op welke wijze verandert de afstand tussen beide stenen in de tijdsduur dat beide vallen?
	\begin{hint}
		Begin met een grafische voorstelling.
	\end{hint}
	\begin{oplossing}
		De afstand verandert lineair in functie van de tijd: $\Delta x(t)=gt_0\cdot t-\frac{1}{2}gt_0^2$ ($=\frac{1}{2}gt^2-\frac{1}{2}g(t-t_0)^2$ met $t\geq t_0=\SI{2}{s}$)
	\end{oplossing}
\end{exercise}

\end{document}
