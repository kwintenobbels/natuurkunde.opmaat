\documentclass{ximera}

\input{../preamble}
\addPrintStyle{..}

\begin{document}
	\author{Bart Lambregs}
	\xmtitle{De versnelling}{}
    \xmsource\xmuitleg


\subsection*{Gemiddelde versnelling}

Ook de snelheid kan veranderen in de tijd. Dit ken je uit het dagelijks leven als versnellen of vertragen.

\begin{definition}

De gemiddelde versnelling \(\overline{a}\) tussen twee tijdstippen wordt gedefiniëerd als
\[
\overline{a}=\frac{\Delta v}{\Delta t}=\frac{v_2-v_1}{t_2-t_1}
\]
De eenheid van versnelling is meter per seconde, per seconde -- wat meter per seconde in het kwadraat geeft $[a]=\rm\,m/s^2$.
\end{definition}

\begin{denkvraag*}{}
Kan je uitleggen wat de eenheid meter per seconde, per seconde betekent? 
\end{denkvraag*}

\subsection*{Ogenblikkelijke versnelling}

De gemiddelde versnelling \(\overline{a}\) geeft de verandering in snelheid \textit{tussen} twee tijdstippen \(t_1\) en \(t_2\).  Om de ogenblikkelijke versnelling \(a\) op één tijdstip \(t\) te bepalen wordt -net zoals bij de ogenblikkelijke snelheid- gebruik gemaakt van de afgeleide. 

\begin{definition}
De ogenblikkelijke snelheid wordt gedefinieerd als de afgeleide van de snelheidsfunctie \(v(t)\):
\[
a=\lim_{t\to t_0}\frac{v(t)-v(t_0)}{t-t_0} = \frac{dv}{dt}=\frac{d^2x}{dt^2}
\]
De notatie met een accent $a(t)=v'(t)$ of $a=v'$ wordt op dezelfde manier als in de wiskunde gebruikt. $a(t)$ is een functie die op elk moment de snelheid geeft. 
\end{definition}

\begin{exercise}
Verklaar onderstaande meme. 
\begin{image}
\includegraphics{meme_versnelling}
\end{image}

\end{exercise}
	
\end{document}
