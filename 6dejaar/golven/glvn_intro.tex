\documentclass{ximera}

\addPrintStyle{..}

\begin{document}
	\author{Bart Lambregs}
	\xmtitle{Inleiding}{}
    \xmsource\xmuitleg


	%% \chapter{Golven}

	Als je een steentje in een poel gooit, krijg je van die prachtige uitdijende cirkels in het wateroppervlak, die, al naargelang ze groter worden, ook in zichtbaarheid afnemen en meer en meer verdwijnen. De verstoring die door de steen in het wateroppervlak teweeg is gebracht, plant zich voort in het water. We hebben te maken met een golf.
	%%%\newline
	Een bijzondere eigenschap van de golf die in het wateroppervlak ontstaat is dat het de verstoring in het medium is die zich verplaatst, niet het water zelf. Het water blijft ter plaatse. Zo zal een dobber van een vislijn enkel op en neer bewegen wanneer een golf voorbijkomt. Ook bij een Mexican wave is dit duidelijk; het zijn niet de supporters die rondlopen!
	%%%\newline
	Golven hebben echter niet altijd een medium nodig om zich in voort te planten. Denk maar aan licht of in het algemeen aan een elektromagnetische golf. De wisselende elektrische en magnetische velden vormen hier de golf.
	
	%een golf heeft een trilling als bron. Als de bron een harmonische trilling uitvoert, ontstaat een golf die zowel in de ruimte als in de tijd sinusoöidaal is.
	
	%golf = def
	
	
\end{document}
