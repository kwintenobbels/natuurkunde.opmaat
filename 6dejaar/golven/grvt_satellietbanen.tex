\documentclass{ximera}

\addPrintStyle{..}

\begin{document}
	\author{Bart Lambregs}
	\xmtitle{Satellietbanen}{}
    \xmsource\xmuitleg


	%%%\section{Satellietbanen}

	Hoe komt het dat de maan door de aarde wordt aangetrokken en er toch niet naartoe valt? Dat komt omdat de neiging van de maan om door de traagheid weg te vliegen volgens de richting waarin hij beweegt, de valbeweging naar de aarde toe compenseert. De gravitatiekracht zorgt voor de middelpuntzoekende kracht van de nagenoeg cirkelvormige beweging die de maan maakt rond de aarde.\footnote{Maar welke kracht duwt de maan dan voort om zijn snelheid te onderhouden? Geen kracht! Herinner je dat de wet van de traagheid zegt dat je geen kracht nodig hebt om een eenmaal verworven snelheid aan te houden. Er is ook geen directe maar een indirecte relatie tussen kracht en snelheid. De tweede wet van Newton relateert kracht aan \textit{versnelling}, niet aan snelheid! Omdat er in de ruimte geen wrijving is, is er niemand om de maan tegen te houden en blijft hij voortgaan.}
	
	Net zoals de maan worden ook kunstsatellieten door de gravitatiekracht in een baan rond de aarde gehouden. Denk maar aan GPS-satellieten en het internationaal ruimtestation. Op de website van de NASA\footnote{\href{http://science.nasa.gov/iSat/}{iSat: Interactive Satellite Viewer (science.nasa.gov/iSat/)}} kan je real-time gegevens van verschillende satellieten terugvinden.
	
	\subsection{Parkeerbaan}
	
	Met een satelliet is het net zoals met de maan. Willen we ervoor zorgen dat de satelliet steeds in eenzelfde baan ronde de aarde blijft cirkelen, dan moet hij een welbepaalde snelheid meekrijgen. Voor een cirkelvormige baan met de aarde als middelpunt fungeert de gravitatiekracht als middelpuntzoekende kracht. De versnelling is dan die van een ECB, wat we kunnen gebruiken om de juiste snelheid af te leiden:
	\begin{eqnarray*}
	F&=&ma\\
	G\frac{m_am}{r^2}&=&m\frac{v^2}{r}\\
	v&=&\sqrt{\frac{Gm_a}{r}}
	\end{eqnarray*}
	Hierin is $m$ de massa van de satelliet, $m_a$ de massa van de aarde en $r$ de straal van de cirkelbeweging die de satelliet maakt. De massa van de satelliet speelt duidelijk geen rol. 
	
	Passen we deze formule toe op het internationaal ruimtestation, dat ongeveer op een hoogte van $412\rm\,km$ vliegt, dan krijgen we, met een gemiddelde aardstraal van $6370\rm\,km$, volgende schatting voor de snelheid:
	\begin{eqnarray*}
		v=\sqrt{\frac{6,67\cdot10^{-11}\rm\,\frac{Nm^2}{kg^2}\cdot5,9721986\cdot10^{24}\rm\,kg}{6370\cdot10^3\rm\,m+412\cdot10^3\rm\,m}}=7,66\rm\,km/s
	\end{eqnarray*}
	Dit ligt erg dicht bij de eigenlijke snelheid -- en dat voor zo'n `simpele' berekening. De mechanica van Newton is duidelijk heel krachtig!
	
	\subsection{Geostationaire baan}
	
	Er is \'e\'en satellietbaan die een speciale eigenschap heeft: de snelheid van de satelliet en de afstand tot de aarde zijn zodanig dat de hoeksnelheid van de satelliet gelijk is aan die van de aarde. De baan ligt bovendien in het equatoriaal vlak zodat voor een waarnemer op de evenaar de satelliet steeds zichtbaar is in het zenit -- loodrecht boven de waarnemer. Vandaar dat over een geostationaire baan wordt gesproken.
	
	Als we eisen dat de hoeksnelheid van de satelliet gelijk is aan die van de aarde (de periode moet bijgevolg 24 uur zijn) en dat de baan in het evenaarsvlak ligt, kunnen we de straal van de baan afleiden:
	\begin{eqnarray*}
	F&=&ma\\
	&\Downarrow&\\
	G\frac{m_am}{r^2}&=&mr\omega^2\\
	r&=&\sqrt[3]{\frac{Gm_a}{\omega^2}}=\sqrt[3]{\frac{Gm_aT^2}{4\pi^2}}\\
	&=&\sqrt[3]{\frac{6,67\cdot10^{-11}\rm\,\frac{Nm^2}{kg^2}\cdot5,9721986\cdot10^{24}\rm\,kg\cdot(24\cdot60\cdot60\rm\,s)^2}{4\pi^2}}\\
	&=&42\,232\rm\,km
	\end{eqnarray*}
	Als we hier de straal van de aarde nog afhalen, vinden we dat de satellietbaan zich $42\,232\rm\,km-6\,370\rm\,km=35\,862\rm\,km$ boven het aardoppervlak bevindt. Dat is ongeveer een tiende van de afstand tot de maan.
	
	\textit{Opmerking}: waarom kan er geen geostationaire baan boven België worden ge\"installeerd? Omdat alle parkeerbanen als middelpunt het middelpunt van de aarde moeten hebben. Het is immers de gravitatiekracht die voor de middelpuntzoekende kracht zorgt. Moest je een baan boven België willen maken, dan zou het middelpunt op de rotatieas ergens boven de noordpool liggen. Vanuit dat middelpunt wordt echter geen kracht uitgeoefend.
	
	
	
	
	
	
	
	
	
	

\end{document}
