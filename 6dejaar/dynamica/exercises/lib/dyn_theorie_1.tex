
\begin{exercise}

 Waarom valt in het luchtledige een massa van \SI{2}{kg} niet twee keer zo snel als een massa van \SI{1}{kg}?

\begin{oplossing}
In het vacu\"um werkt op een vrije massa enkel de zwaartekracht in. Dat is dan ook de resulterende kracht op de massa. Die kracht is inderdaad twee keer zo groot voor een twee keer zo grote massa, maar een twee keer zo grote massa verzet zich ook twee keer zo hard tegen het het veranderen van beweging; de traagheid is twee keer zo groot. Het resultaat is dat elk object met dezelfde versnelling naar de aarde valt.

Die hierboven eerder kwalitatieve redenering is kwantitatief uit te leggen met de tweede wet van Newton, $\vec{F}=m\vec{a}$:
\begin{equation*}
F_z=ma%\\
%&\Downarrow&\\
\end{equation*}
Met de formule $F_z=mg$ voor de zwaartekracht vinden we
\begin{equation*}
mg=ma%\\
%&\Downarrow&\\
\end{equation*}
Zodat, na de massa's te hebben geschrapt
\begin{equation*}
a=g%\\
%&\Downarrow&\\
\end{equation*}
De massa van het object heeft m.a.w. geen invloed op de versnelling waarmee het valt. Die versnelling is constant en in waarde gelijk aan de waarde van de veldsterkte. Omdat ook de eenheden overeenkomen (uit de tweede wet van Newton volgt dat \SI{}{N}=\SI{}{kg\cdot m/s^2}) wordt het symbool $g$ voor zowel de veldsterkte als de valversnelling gebruikt. Bij ons heeft die de waarde \SI{9,81}{m/s^2}.
\end{oplossing}

\end{exercise}
