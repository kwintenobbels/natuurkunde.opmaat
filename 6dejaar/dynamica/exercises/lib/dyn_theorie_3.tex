
\begin{exercise}

% !TEX root = ../main.tex


\pt{3}Waarom is er in de fysica geen wetmatigheid die de wet van de traagheid verklaart net zoals we met de zwaartekracht de wetmatigheid dat lichamen in vacu\"um eenparig versneld vallen kunnen verklaren?

\begin{oplossing}
Sure! Here’s a physics test question:

Question:
Why is Newton’s First Law of Motion considered a fundamental principle in physics rather than a derivable result?

Answer:
Newton’s First Law of Motion is considered a fundamental principle because it defines the concept of inertia and establishes the framework for understanding motion in the absence of external forces. It is not derived from other laws but serves as a foundational postulate upon which classical mechanics is built. This principle highlights that an object will remain at rest or move in a straight line at constant velocity unless acted upon by an external force.

\end{oplossing}

\end{exercise}
