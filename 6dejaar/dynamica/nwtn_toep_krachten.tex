\documentclass{ximera}
\input{../../preamble}

\addPrintStyle{..}

\newcommand{\ul}[1]{#1}

\begin{document}
	\author{Bart Lambregs}
	\xmtitle{De derde wet van Newton}{}
    \xmsource\xmuitleg

\textbf{\ul{Herhaling en uitbreiding soorten krachten}}

Een kracht is een interactie tussen twee voorwerpen. Wanneer een eerste
voorwerp A een kracht uitoefent op een ander voorwerp B, dan is een
bepaalde uitwerking het mogelijk gevolg. Deze uitwerking kan
\textbf{statisch (elastisch en/of plastisch)} zijn (vervorming), maar
ook \textbf{dynamisch} (verandering van bewegingstoestand, in één woord:
versnelling).

De meeste krachten zijn \textbf{contactkrachten}, dan is er rechtstreeks
contact tussen de twee voorwerpen. Bij \textbf{veldkrachten} is geen
rechtstreeks contact nodig en is er sprake van een krachtveld. De
veldkrachten zijn: gravitatiekracht (of zwaartekracht) tussen massa's,
elektromagnetische kracht tussen ladingen (Coulomb- en Lorentzkracht),
sterke kernkrachten tussen quarks en de zwakke wisselwerking die in
theorie mogelijk is tussen alle fundamentele deeltjes. Alle andere
krachten zijn bijgevolg contactkrachten.

De grootheid kracht is een vector, ze heeft een richting, zin en
grootte. De grootte van de kracht heeft als eenheid de Newton (N). Bij
een contactkracht grijpt de kracht aan op de plaats(en) waar er contact
is. Bij een veldkracht grijpen er in feite vele kleine veldkrachten aan
op de betrokken deeltjes van gans het voorwerp. Voor de eenvoud vatten
we die samen tot één kracht. Tenslotte laten we voor het gebruiksgemak
meestal \textbf{alle krachten op een voorwerp aangrijpen in het
zwaartepunt van het voorwerp}.

\includegraphics[width=2.31181in,height=2.52986in]{media/image1.png}Hoe
kan je achterhalen welke krachten op een voorwerp aangrijpen? Stel
jezelf de vraag: ``Wat duwt of trekt eraan?'' Bepaal de contactkrachten
door na te gaan met welke voorwerpen er contact is en overloop dan de
eventuele veldkrachten.

\begin{enumerate}
\def\labelenumi{\arabic{enumi}.}
\item
  \textbf{De normaalkracht (contactkracht)}
\end{enumerate}

De normaalkracht is een kracht van een (meestal vlakke, maar soms ook
gebogen) ondergrond op een voorwerp, meestal met de bedoeling het
voorwerp te ondersteunen. De normaalkracht \({\overrightarrow{F}}_{n}\)
staat altijd loodrecht op het grondvlak (net zoals de normale
versnelling loodrecht staat op de snelheidsvector; ``normaal'' betekent
in deze context nu eenmaal ``loodrecht''). Er is geen formule om de
grootte van de normaalkracht uit te rekenen. De grootte ervan dient
altijd bepaald te worden uit de toepassing van en het redeneren met de
wetten van Newton. In sommige eenvoudige situaties kan het zijn dat
\(F_{n} = F_{z}\), \textbf{maar dit is zeker niet altijd het geval!
Enkel met behulp van de wetten van Newton is er uitsluitsel!}

\begin{enumerate}
\def\labelenumi{\arabic{enumi}.}
\setcounter{enumi}{1}
\item
  \textbf{De veerkracht (contactkracht)}
\end{enumerate}

De veerkracht is een kracht die een uitgerekte of ingedrukte veer
uitoefent op een voorwerp die met één van de uiteindes verbonden is.
Wanneer de veer onbelast is (m.a.w. niet uitgerekt of ingedrukt, maar
gewoon ontspannen) dan heeft ze een rustlengte \(\mathcal{l}_{r}\). Bij
belasting is de lengte \(\mathcal{l}\) verschillend van rustlengte
\(\mathcal{l}_{r}\) en is er een uitrekking/indrukking, hetgeen met
\(\mathcal{\mathrm{\Delta}l = l -}\mathcal{l}_{r}\) wordt weergegeven.
Bij zuiver elastische vervormingen (dit zijn niet al te grote
vervormingen zodat bij ontspanning de veer uit zichzelf terug tot aan de
rustlengte ontspant) geldt de wet van Hooke als formule voor de grootte
van de veerkracht:
\(\mathbf{\ F}_{\mathbf{v}}\mathbf{= k \bullet}\left| \mathcal{\mathrm{\Delta}l} \right|\mathbf{\ }\).
Hierin is \(k\) de veerconstante uitgedrukt in \(\frac{N}{m}\). De
waarde ervan is specifiek voor elke veer en is ook constant indien de
veer enkel elastisch vervormt. Ze drukt uit hoeveel Newton kracht de
veer zet per meter dat de veer is verlengd/verkort t.o.v. de rustlengte.
Ze zegt dus iets over de soepelheid of stijfheid van een veer.

\includegraphics[width=3.23125in,height=2.56319in]{media/image2.png}

Het gebeurt dat voor het symbool \(\mathcal{\mathrm{\Delta}l}\) het
symbool \(x\) gebruikt wordt. Dit is korter en dus eenvoudiger omdat de
lengte van de veer ons meestal niet interesseert, enkel het verschil met
de rustlengte. Dit zal in het vervolg van de cursus ook zo zijn. Meer
nog, men kan voor de veerkracht ook een vectoriële formule opstellen
indien men de uitrekking/indrukking van de veer als een
uitwijkingsvector \(\overrightarrow{x}\) ziet die aangrijpt op de plaats
van de rustlengte. Bijgevoegde figuren tonen dit voor een ingedrukte en
uitgerekte veer. Er valt op dat de veerkracht telkens de tegengestelde
zin heeft van de uitwijkingsvector, waardoor de vectoriële formule
wordt: :
\(\ {\ \overrightarrow{F}}_{v} = - k \bullet \overrightarrow{x}\ \). De
scalaire getalcomponent (waarin met behulp van een gekozen x-as rekening
wordt gehouden met de zin van de kracht) wordt dan:
\({\ F}_{v} = - k \bullet x\ \).

\[\mathcal{l}_{r}\]

\begin{enumerate}
\def\labelenumi{\arabic{enumi}.}
\setcounter{enumi}{2}
\item
  \textbf{De opwaartse stuwkracht in een vloeistof of Archimedeskracht
  (contactkracht)}
\end{enumerate}

\includegraphics[width=1.99722in,height=1.84028in]{media/image3.png}Een
voorwerp dat volledig of gedeeltelijk ondergedompeld is in een vloeistof
ondervindt vanwege de vloeistof een opwaartse stuwkracht verticaal
omhoog. Deze kracht (van vloeistof op voorwerp) noemt men ook de
Archimedeskracht. De verklaring ligt in het feit dat de hydrostatische
druk onder het voorwerp groter is dan de druk boven het voorwerp. In
feite is de Archimedeskracht dus de nettokracht van alle drukkrachten
die de vloeistof op het voorwerp zet. Er kan proefondervindelijk en
theoretisch aangetoond worden dat de grootte van deze kracht gelijk is
aan die van de zwaartekracht op de verplaatste vloeistof (= wet van
Archimedes). Dit komt neer op:

\({\ F}_{s} = {\ F}_{A} = m_{vl} \bullet g = \rho_{vl} \bullet V_{verplaatst} \bullet g\)

Omdat de verplaatste vloeistof gelijk is aan het volume van het stuk
voorwerp dat zich onder het vloeistofoppervlak bevindt, kan eveneens
gezegd worden dat:

\({\mathbf{\ \ }\mathbf{F}}_{\mathbf{s}}\mathbf{=}{\mathbf{\ }\mathbf{F}}_{\mathbf{A}}\mathbf{=}\mathbf{\rho}_{\mathbf{vl}}\mathbf{\bullet}\mathbf{V}_{\mathbf{ond}}\mathbf{\bullet g\ }\)
met \(V_{ond}\) het ondergedompeld deel van het voorwerp

Om de grootte, richting en zin van de versnelling van een ondergedompeld
voorwerp te bepalen, dient men de wetten van Newton toe te passen.
Hierin onderscheidt men een aantal gevallen:

\begin{itemize}
\item
  Zinken: \({\overrightarrow{a}}_{y}\) is naar beneden gericht (en
  \({\overrightarrow{F}}_{ry}\) ook, de neerwaartse krachten zijn samen
  sterker dan de opwaartse)
\item
  Stijgen: \({\overrightarrow{a}}_{y}\) is naar boven gericht (en
  \({\overrightarrow{F}}_{ry}\) ook, de opwaartse krachten zijn samen
  sterker dan de neerwaartse)
\item
  Zweven: voorwerp volledig ondergedompeld en \(a_{y} = 0\) (en dus is
  \(F_{ry} = 0\))
\item
  Drijven: voorwerp gedeeltelijk ondergedompeld en \(a_{y} = 0\) (en dus
  is \(F_{ry} = 0\))
\end{itemize}

\ul{Speciaal geval:} Heel vaak werken op een ondergedompeld voorwerp
slechts twee krachten, de Archimedeskracht (naar boven) en de
zwaartekracht (naar onder) met bijbehorende formules:

\({\ F}_{A} = \rho_{vl} \bullet V_{ond} \bullet g\ \ \ \ \ \ \ \ \ \ \ \ \ \ \ \ \ \ \ \ \ \ \ \ \ \ \ \ \ \ \ \ \ \ \ \ \ \ \ \ \ \ \ \ \ {\ F}_{z} = m_{vw} \bullet g = \rho_{vw} \bullet V_{vw} \bullet g\)

\begin{itemize}
\item
  Indien in dit geval het voorwerp \ul{volledig ondergedompeld} is, is
  \(V_{ond} = V_{vw}\) waardoor enkel de massadichtheden een verschil in
  grootte tussen de twee krachten kunnen opleveren, met als gevolg:
\end{itemize}

\[\rho_{vw} > \rho_{vl}\overset{}{\Rightarrow}zinken\ \ \ \ \ \ \rho_{vw} = \rho_{vl}\overset{}{\Rightarrow}zweven\ \ \ \ \ \ \rho_{vw} < \rho_{vl}\overset{}{\Rightarrow}stijgen\ (om\ nadien\ vermoedelijk\ te\ gaan\ drijven)\]

\begin{itemize}
\item
  Als het voorwerp \ul{gedeeltelijk ondergedompeld} is, dan is
  \(V_{ond} < V_{vw}\) en moet men ook met de volumes rekening houden om
  het gedrag te bepalen. Merk op dat in deze situatie bij drijven (=
  rustsituatie waarbij slechts een gedeelte van het voorwerp
  ondergedompeld is) moet gelden dat \({\ F}_{A} = {\ F}_{z}\).
\end{itemize}

\begin{quote}
\includegraphics[width=5.72917in,height=2.50694in]{media/image4.png}
\end{quote}

\begin{enumerate}
\def\labelenumi{\arabic{enumi}.}
\item
  Speciaal geval: ondergedompeld voorwerp ondervindt enkel
  \({\ \overrightarrow{F}}_{A}\) en \({\ \overrightarrow{F}}_{z}\). Het
  ondergedompeld volume is ingekleurd.
\end{enumerate}

\begin{enumerate}
\def\labelenumi{\arabic{enumi}.}
\setcounter{enumi}{3}
\item
  \textbf{De gravitatiekracht of zwaartekracht (veldkracht)}
\end{enumerate}

= aantrekkingskracht tussen massa's

\includegraphics[width=2.63194in,height=1.16597in,alt={Afbeelding met lijn, diagram, schermopname Automatisch gegenereerde beschrijving}]{media/image5.png}

De \textbf{grootte} van de gravitatiekracht \({\overrightarrow{F}}_{G}\)
tussen twee puntmassa's \(m_{1}\) en \(m_{2}\) met een afstand
\(r\ \)tussenbeiden is gelijk is aan:

\[\mathbf{F}_{\mathbf{G}}\mathbf{= G \bullet}\frac{\mathbf{m}_{\mathbf{1}}{\mathbf{.}\mathbf{m}}_{\mathbf{2}}}{\mathbf{r}^{\mathbf{2}}}\ met\ G = 6,67 \bullet 10^{- 11}\frac{N \bullet m²}{kg²} = gravitatieconstante\]

In realistische situaties bij voorwerpen met afmetingen is r de afstand
tussen de zwaartepunten van die voorwerpen. Vaak valt het zwaartepunt
samen met het midden van het voorwerp (zeker bij voorwerpen met homogene
massaverdeling).

In de context van hemellichamen wordt gravitatiekracht vaak
zwaartekracht genoemd. Dit verschilt enkel in benaming, maar beide
benamingen weerspiegelen identiek hetzelfde krachtfenomeen. In het
Engels of Frans bestaat er geen woord voor ``zwaartekracht'', zij kennen
enkel \emph{gravity} of \emph{gravité}.

In een gravitatieveld \(\overrightarrow{g}\) is er een eenvoudigere
formule:
\(\ {\overrightarrow{\mathbf{F}}}_{\mathbf{G}}\mathbf{=}{\overrightarrow{\mathbf{F}}}_{\mathbf{z}}\mathbf{= m \bullet}\overrightarrow{\mathbf{g}}\mathbf{\ }\)

Rond een hemellichaam met massa \(M\) geldt:
\(g = \frac{G \bullet M}{r²}\) Op het aardoppervlak, levert dit:
\(g_{A} = 9,81\frac{N}{kg} \approx 10\frac{N}{kg}\)

\begin{enumerate}
\def\labelenumi{\arabic{enumi}.}
\setcounter{enumi}{4}
\item
  \textbf{De elektrische kracht of Coulombkracht (veldkracht)}
\end{enumerate}

= aantrekking- of afstotingskracht tussen elektrische ladingen

De \textbf{grootte} van de Coulombkracht \({\overrightarrow{F}}_{C}\)
(of elektrische kracht \({\overrightarrow{F}}_{e}\)) tussen twee
puntladingen \(q_{1}\) en \(q_{2}\) met een afstand \(r\ \)tussenbeiden
is gelijk is aan:

\[{\mathbf{F}_{\mathbf{C}}\mathbf{=}\mathbf{F}}_{\mathbf{e}}\mathbf{= k \bullet}\frac{\left| \mathbf{q}_{\mathbf{1}} \right|\mathbf{\bullet}\left| \mathbf{q}_{\mathbf{2}} \right|}{\mathbf{r}^{\mathbf{2}}}\ met\ k = 8,99 \bullet 10^{9}\frac{N \bullet m²}{C²} \approx 9,0 \bullet 10^{9}\frac{N \bullet m²}{C²}\]

\includegraphics[width=5.97032in,height=1.4754in,alt={Afbeelding met lijn, diagram, cirkel Automatisch gegenereerde beschrijving}]{media/image6.png}

In realistische situaties bij ladingen met afmetingen is r de afstand
tussen de middelpunten van die ladingen.

In een elektrisch veld \(\overrightarrow{E}\) is er een eenvoudigere
formule:
\(\ {\overrightarrow{\mathbf{F}}}_{\mathbf{C}}\mathbf{=}{\overrightarrow{\mathbf{F}}}_{\mathbf{e}}\mathbf{= q \bullet}\overrightarrow{\mathbf{E}}\mathbf{\ }\)
De grootte van \(\overrightarrow{E}\) kan in verschillende situaties
bepaald worden naargelang het type veld. Meer details: zie leerstof
5\textsuperscript{de} jaar.

\begin{enumerate}
\def\labelenumi{\arabic{enumi}.}
\setcounter{enumi}{5}
\item
  \textbf{De magnetische kracht of Lorentzkracht (veldkracht)}
\end{enumerate}

= kracht op bewegende lading in een magnetisch veld (dat veroorzaakt
wordt door andere bewegende ladingen)

De \textbf{grootte} van de Lorentzkracht \({\overrightarrow{F}}_{L}\)
(of magnetische kracht \({\overrightarrow{F}}_{m}\)) op een lading \(q\)
met snelheid \(\overrightarrow{v}\) die een hoek \(\alpha\) maakt met de
magnetische veldsterkte \(\overrightarrow{B}\) waar de lading door
vliegt is te berekenen met:

\[{\mathbf{F}_{\mathbf{L}}\mathbf{= F}}_{\mathbf{m}}\mathbf{= B \bullet}\left| \mathbf{q} \right|\mathbf{\bullet v \bullet}\mathbf{\sin}\mathbf{\alpha}\ met\ \alpha = hoek\ tussen\ \overrightarrow{B}\ en\ \overrightarrow{v}\]

Als vele ladingen samen een gemeenschappelijke driftsnelheid hebben door
een magneetveld hebben (vaak door een geleidende draad met lengte
\(\mathcal{l}\)) beschouwt men dit als een stroom \(I\) in een
magneetveld \(\overrightarrow{B}\). Dan wordt de formule:

\[{\mathbf{F}_{\mathbf{L}}\mathbf{= F}}_{\mathbf{m}}\mathbf{= B \bullet I}\mathcal{\bullet l \bullet}\mathbf{\sin}\mathbf{\alpha}\ met\ \alpha = hoek\ tussen\ \overrightarrow{B}\ en\ I\ \ (ook\ Laplacekracht\ genoemd)\]

\includegraphics[width=2.14955in,height=1.85102in,alt={Afbeelding met tekst Automatisch gegenereerde beschrijving}]{media/image7.png}De
richting en zin van \({\overrightarrow{F}}_{L}\) kan bepaald worden met
de derde rechterhandregel waarin geredeneerd wordt met de conventionele
stroomzin van de ladingen.

\emph{\textbf{Uitbreiding:} De Coulombkracht en de Lorentzkracht kunnen
samengevat worden tot één elektromagnetische kracht, want beide gevallen
gaan over kracht op een lading.}

\begin{quote}
\[{{\overrightarrow{\mathbf{F}}}_{\mathbf{em}}\mathbf{=}{\overrightarrow{\mathbf{F}}}_{\mathbf{C}}\mathbf{+}{\overrightarrow{\mathbf{F}}}_{\mathbf{L}}\mathbf{= q}\overrightarrow{\mathbf{E}}\mathbf{+ q}\overrightarrow{\mathbf{v}}\mathbf{\times}\overrightarrow{\mathbf{B}}\mathbf{= q}\left( \overrightarrow{\mathbf{E}}\mathbf{+}\overrightarrow{\mathbf{v}}\mathbf{\times}\overrightarrow{\mathbf{B}} \right)\mathbf{\ \ \ \ 
}}
\](\(\times is\ het\ symbool\ voor\ vectorieel\ product,\ zie\ externe\ cursussen\))
\end{quote}

\begin{enumerate}
\def\labelenumi{\arabic{enumi}.}
\setcounter{enumi}{6}
\item
  \textbf{De sterke kernkracht en zwakke wisselwerking (veldkrachten)}
\end{enumerate}

De sterke kernkrachten tussen quarks zijn heel relevant voor de
stabiliteit en samenhang van nucleonen in atoomkernen. Ze hebben echter
geen relevantie in de stabiele macroscopische wereld. Enkel bij zeer
kleine afstand tussen nucleonen (orde femtometer =
10\textsuperscript{-15} m) is deze kernkracht niet te verwaarlozen,
binnen één atoomkern dus. De sterke kernkracht neemt met de afstand veel
feller af in vergelijking met andere veldkrachten.

Bij de zwakke wisselwerking (of kernkracht) tussen fundamentele deeltjes
is er een andere uitwerking dan gebruikelijk (niet statisch of
dynamisch). In dit geval veranderen de deeltjes van aard. Ook dit
fenomeen is in onze stabiele macroscopische wereld irrelevant omdat het
net enkel gebeurt bij onstabiele deeltjes of atoomkernen.

\begin{enumerate}
\def\labelenumi{\arabic{enumi}.}
\setcounter{enumi}{7}
\item
  \textbf{De wrijvingskracht (contactkracht) (zie uitgebreid verhaal in
  boek FV p113-115)}
\end{enumerate}

= weerstandskracht \({\overrightarrow{F}}_{w}\) op een voorwerp van een
medium of middenstof waarin of waartegen het zich bevindt

% \includegraphics[width=2.08889in,height=1.81111in,alt={E:\textbackslash School updated tot en met 27\_08\_2012\textbackslash Fysica 6de jaar\textbackslash Afbeeldingen\textbackslash wrijving.gif}]{media/image8.gif}\textbf{Schuifwrijvingskracht}
= weerstandskracht van een vlakke ondergrond op een voorwerp dat erop
steunt, evenwijdig met het vlak (en vaak tegen de zin van de beweging)

Verklaring schuifwrijving: een vlakke ondergrond is nooit perfect vlak!
Zie figuur.

Voor de \ul{schuif}wrijvingskracht \({\overrightarrow{F}}_{w}\) wordt
vastgesteld dat\ldots{}

\begin{enumerate}
\def\labelenumi{\alph{enumi})}
\item
  \ldots indien het voorwerp in rust is: \({\overrightarrow{F}}_{w}\)
  heeft die grootte en zin zodat \(F_{r} = 0\)
\item
  \ldots indien het voorwerp een rechtlijnige beweging maakt:
\end{enumerate}

\begin{quote}
De zin van \({\overrightarrow{F}}_{w}\) is tegen de zin van de beweging
en de grootte van \({\overrightarrow{F}}_{w}\) is recht evenredig is met
de grootte van de normaalkracht en tegelijk afhankelijk van hoe goed of
slecht de materialen van het voorwerp en vlakke ondergrond over mekaar
schuiven.
\end{quote}

Er geldt dan: \(F_{w} = \mu \bullet F_{n}\) met \(\mu\) de
schuifwrijvingsfactor of schuifwrijvingscoëfficiënt (op te zoeken in
tabellen)

Bovenstaande omkaderde formule is dus de maximale waarde van de
schuifwrijvingskracht, waardoor de formule beter te schrijven is als:
\(F_{w,max} = \mu \bullet F_{n}\). Men kan overigens aantonen dat deze
laatste formule ook klopt bij bewegingen die niet rechtlijnig zijn (zie
verder voor voorbeelden). \emph{(zie ook animatie Hans Bekaert:
``wrijving'')}

\begin{enumerate}
\def\labelenumi{\alph{enumi})}
\setcounter{enumi}{2}
\item
  \ldots indien het voorwerp een ECB maakt: \({\overrightarrow{F}}_{w}\)
  heeft die grootte zodat
  \(F_{r} = m \bullet a = m \bullet \frac{v^{2}}{r} = m \bullet \omega ² \bullet r\)
\end{enumerate}

Bij het maken van een bocht (bijvoorbeeld ECB) is er in sommige gevallen
schuifwrijvingskracht die bijdraagt tot de middelpuntzoekende kracht.
Bij bijna alle vervoerswijzen over een vlakke weg doet dit zich voor
(zoals met de wagen, met de fiets en zelfs te voet!). Er dient wel
rekening gehouden te worden met het feit dat de schuifwrijvingskracht in
grootte beperkt is tot \(F_{w,max} = \mu \bullet F_{n}\) waardoor er een
gevaar bestaat dat de beoogde bocht niet gemaakt kan worden! In dit
geval spreekt men van slippen en ``uit de bocht vliegen''. Zie
oefeningen voor concrete gevallen.

%%TODO%% \includegraphics[width=3.08403in,height=2.31111in,alt={E:\textbackslash School updated tot en met 27\_08\_2012\textbackslash Fysica 6de jaar\textbackslash Afbeeldingen\textbackslash wrijving, grafiek.jpg}]{media/image9.jpeg}

\begin{itemize}
\item
  Opmerking 1: Soms wordt er een onderscheid gemaakt tussen een
  statische en kinetische wrijvingsfactor omdat in werkelijkheid de
  maximale schuifwrijvingskracht in rust een beetje groter is dan
  wanneer het voorwerp in beweging is. Dit maakt nevenstaande grafiek
  duidelijk. Niettemin zijn de verschillen meestal klein, waardoor we
  dit verschil voortaan zullen verwaarlozen.
\item
  Opmerking 2: Bovenvermelde formules zijn enkel van toepassing voor
  \ul{schuif}wrijving! \textbf{Rolwrijvingskracht en
  fluïdumwrijvingskracht} (zoals luchtwrijving) zijn
  snelheidsafhankelijke weerstandskrachten waardoor er hiervoor andere
  formules gelden. Je kan ter info p115 uit je boek FV hierover
  raadplegen (maar hoef je niet te kennen).
\end{itemize}

\begin{enumerate}
\def\labelenumi{\arabic{enumi}.}
\setcounter{enumi}{8}
\item
  \textbf{Trekkracht, touwkracht, andere steunkrachten}
\end{enumerate}

Om een touw op de spannen moet er aan beide uiteindes een kracht gezet
worden, \textbf{trekkracht}
\({\overrightarrow{\mathbf{F}}}_{\mathbf{t}}\) genaamd. Deze staat
altijd evenwijdig met het touw en met zin weg van het touw. Zolang het
touw niet knapt, is elke grootte voor \({\overrightarrow{F}}_{t}\)
mogelijk, afhankelijk van de situatie. Uiteraard heeft elk touw een
maximale trekkracht alvorens het knapt (maar in onze oefeningen
veronderstellen we sterke touwen die niet knappen).

\textbf{Touwkracht} is de kracht die een touw zet op hetgeen eraan
vasthangt, de reactiekracht van de trekkracht dus. Voor de eenvoud
noemen we dit ook \({\overrightarrow{\mathbf{F}}}_{\mathbf{t}}\). Uit de
derde wet van Newton zijn deze toch even groot.

Indien een touw zeer klein in massa is (of massaloos verondersteld
wordt), dan wordt er aan beide uiteindes altijd even hard getrokken. De
touwkracht is aan beide uiteindes dus ook even groot. Zie oefeningen.

Naast normaalkracht, veerkracht, Archimedeskracht, trekkracht,
touwkracht, \ldots{} zijn er nog steunkrachten. Een voorbeeld is de
steunkracht van een stang die meestal onvervormbaar is. Dergelijke
krachten kunnen ook hun rol spelen binnen de wetten van Newton, maar
hebben geen vaste formule.

\begin{enumerate}
\def\labelenumi{\arabic{enumi}.}
\setcounter{enumi}{9}
\item
  \textbf{Het gewicht (contactkracht) (zie uitgebreid verhaal in boek FV
  p126-128)}
\end{enumerate}

= kracht \(\overrightarrow{G}\) die een voorwerp uitoefent op zijn steun
(meestal tgv de zwaartekracht)

Let op! Het gewicht van een voorwerp grijpt niet aan op het voorwerp
zelf, maar op hetgeen waar het voorwerp contact mee maakt waardoor het
ondersteund wordt (bijvoorbeeld: oppervlak, veer, vloeistof, \ldots)!
Het gewicht is vaak de reactiekracht van de steunkracht op het voorwerp
(\({\overrightarrow{F}}_{n},{\overrightarrow{F}}_{v},{\overrightarrow{F}}_{A},{\overrightarrow{F}}_{t},\ldots)\)
en is daarom volgens de derde wet van Newton in grootte gelijk aan die
steunkracht.

Er is geen vaste formule voor gewicht, maar de grootte moet je bepalen
met de wetten van Newton (vaak combinatie 3\textsuperscript{de} wet met
1\textsuperscript{ste} of 2\textsuperscript{de}). In eenvoudige
situaties draait het vaak uit dat \(G = F_{z}\) , maar dit mag je niet
algemeen aannemen! De wetten van Newton brengen altijd uitsluitsel!


\end{document}
