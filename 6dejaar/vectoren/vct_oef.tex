\documentclass{ximera}
\input{../../preamble}

\addPrintStyle{../..}

\begin{document}
	\author{Bart Lambregs en Vincent Gellens}
	\xmtitle{Oefeningen vectoren}{}
    \xmsource\xmuitleg


% Bepaal grafisch en kwantitatief de resultante van de gegeven vectoren. 
\begin{exercise}

\begin{question}
\begin{image}[0.2\textwidth]
\begin{tikzpicture}
	\pgfmathsetmacro{\ang}{40}

	\coordinate (O) at (0,0); 
	\coordinate (X) at (2,0); 
	\coordinate (A) at (90:3); 
	\coordinate (B) at (-\ang :2); 


	\draw[dotted] (O)--(4,0);
	\draw[->, very thick, red] (O)--(A) node[midway, left]{$\vec{a}$};
	\draw[->, very thick, green] (O)--(B) node[midway, below left]{$\vec{b}$};

	\draw pic[ pic text= \(\alpha\), draw,  angle radius=0.7cm]{angle = B--O--X};
	\draw pic[ angle radius=0.7cm]{right angle = X--O--A};
	
\end{tikzpicture}
\end{image}
\end{question}

\begin{question}
\begin{image}[0.2\textwidth]
	\begin{tikzpicture}
		\pgfmathsetmacro{\ang}{15}
	
		\coordinate (O) at (0,0); 
		\coordinate (A) at (180:3); 
		\coordinate (B) at (90-\ang :2); 
		\coordinate (C) at (90:2.5); 
	
		\draw[->, very thick, red] (O)--(A) node[midway, below]{$\vec{a}$};
		\draw[->, very thick, green] (O)--(B) node[midway, below right]{$\vec{b}$};
		\draw[->, very thick, blue] (O)--(C) node[pos=0.8, left]{$\vec{c}$};
	
		\draw pic[ pic text= \(\alpha\), draw,  angle radius=0.7cm, angle eccentricity=1.5]{angle = B--O--C};
		\draw pic[ draw, angle radius=0.4cm]{right angle = C--O--A};
		% \draw pic[ draw, pic text= \LARGE{$\llcorner$},  angle radius=0.4cm]{right angle = C--O--A};
		
	\end{tikzpicture}
	\end{image}
\end{question}

\begin{question}
	\begin{image}[0.2\textwidth]
		\begin{tikzpicture}
			\pgfmathsetmacro{\ang}{60}
		
			\coordinate (O) at (0,0); 
			\coordinate (A) at (0:1); 
			\coordinate (B) at (90-\ang :2); 
			\coordinate (C) at (90:2.5); 
			\coordinate (D) at (270:1.5); 
		
		
			\draw[->, very thick, blue] (O)--(A) node[midway, below]{$\vec{a}$};
			\draw[->, very thick, green] (O)--(B) node[pos=0.7, below right]{$\vec{b}$};
			\draw[->, very thick, red] (O)--(C) node[midway, left]{$\vec{c}$};
			\draw[->, very thick, yellow] (O)--(D) node[midway, left]{$\vec{d}$};
		
			\draw pic[ pic text= \(\alpha\), draw,  angle radius=0.5cm]{angle = B--O--C};
			\draw pic[ draw, angle radius=0.2cm]{right angle = D--O--A};
			\draw pic[ pic text= \(\pi\), draw, angle radius=0.3cm]{ angle = C--O--D};
			
		\end{tikzpicture}
		\end{image}
\end{question}
\end{exercise}

\begin{exercise}
	\begin{image}[0.2\textwidth]
		\begin{tikzpicture}
			\pgfmathsetmacro{\ang}{50}
		
			\coordinate (O) at (0,0); 
			\coordinate (X) at (0:2); 
			\coordinate (F) at (\ang :1); 
		
			\draw[->, very thick, blue] (O)--(F) node[midway, above left]{$\vec{F}$};
			\draw[->, very thick, red] (O)--(X) node[midway, below]{$\vec{x}$};
		
			\draw pic[ pic text= \(\alpha\), draw,  angle radius=0.7cm]{angle = X--O--F};
			
		\end{tikzpicture}
		\end{image}
\end{exercise}

\begin{exercise}
	\begin{image}[0.2\textwidth]
		\begin{tikzpicture}
			\pgfmathsetmacro{\ang}{125}
			\pgfmathsetmacro{\angg}{30}
		
			\coordinate (O) at (0,0); 
			\coordinate (S) at (\ang + \angg :1); 
			\coordinate (F) at (\angg :2); 
		
			\draw[->, very thick, blue] (O)--(F) node[midway, below right]{$\vec{F}$};
			\draw[->, very thick, red] (O)--(S) node[pos=1, above]{$\vec{x}$};
		
			\draw pic[ pic text= \(\alpha\), draw,  angle radius=0.7cm]{angle = F--O--S};
			
		\end{tikzpicture}
		\end{image}
\end{exercise}



\begin{exercise}
	\begin{image}[0.2\textwidth]
		\begin{tikzpicture}
			\pgfmathsetmacro{\ang}{15}
		
			\coordinate (O) at (0,0); 
			\coordinate (A) at (180:2); 
			\coordinate (B) at (90-\ang :1.6); 
			\coordinate (C) at (90:1.8); 
		
			\draw[->, very thick, red] (O)--(A) node[midway, below]{$\vec{a}$};
			\draw[->, very thick, green] (O)--(B) node[midway, below right]{$\vec{b}$};
			\draw[->, very thick, blue] (O)--(C) node[pos=0.8, left]{$\vec{c}$};
		
			\draw pic[ pic text= \(\alpha\), draw,  angle radius=0.7cm, angle eccentricity=1.5]{angle = B--O--C};
			\draw pic[ draw, angle radius=0.4cm]{right angle = C--O--A};
			% \draw pic[ draw, pic text= \LARGE{$\llcorner$},  angle radius=0.4cm]{right angle = C--O--A};
			
		\end{tikzpicture}
	\end{image}
\end{exercise}


\begin{exercise}


	
Als \(\vec{F} \; \bot \; \vec{y}\), welk(e) van onderstaande uitspraken is dan juist? 
Meerdere antwoorden zijn mogelijk. 

\begin{question}
\(\vec{F} \cdot \vec{y} = \vec{0}\)
\end{question}

\begin{question}
\(\vec{F} \times \vec{y} = \vec{0}\)
\end{question}

\begin{question}
\(\vec{F} \cdot \vec{y} = \lVert \vec{F} \rVert \cdot \lVert \vec{y} \rVert\)	
\end{question}

\begin{question}
\(\lVert \vec{F} \times \vec{y} \rVert = \lVert \vec{F} \rVert \cdot \lVert \vec{y} \rVert\)	
\end{question}

\begin{question}
\(\vec{F} \times \vec{y} = \lVert \vec{F} \rVert \cdot \lVert \vec{y} \rVert\)	
\end{question}

\begin{question}
\(\vec{F} \cdot \vec{y} = 0\)
\end{question}

\begin{question}
\(\vec{F} \times \vec{y} = 0\)
\end{question}

\begin{question}
\(\lVert \vec{F} \times \vec{y} \rVert = 0\)
\end{question}

\end{exercise}



\begin{exercise}
	
Als \(\vec{F} \; \| \; y\), welk(e) van onderstaande uitspraken is dan juist? 
Meerdere antwoorden zijn mogelijk. 

\begin{question}
\(\vec{F} \cdot \vec{y} = \vec{0}\)
\end{question}

\begin{question}
\(\vec{F} \times \vec{y} = \vec{0}\)
\end{question}

\begin{question}
\(\vec{F} \cdot \vec{y} = \lVert \vec{F} \rVert \cdot \lVert \vec{y} \rVert\)	
\end{question}

\begin{question}
\(\lVert \vec{F} \times \vec{y} \rVert = \lVert \vec{F} \rVert \cdot \lVert \vec{y} \rVert\)	
\end{question}

\begin{question}
\(\vec{F} \times \vec{y} = \lVert \vec{F} \rVert \cdot \lVert \vec{y} \rVert\)	
\end{question}

\begin{question}
\(\vec{F} \cdot \vec{y} = 0\)
\end{question}

\begin{question}
\(\vec{F} \times \vec{y} = 0\)
\end{question}

\begin{question}
\(\lVert \vec{F} \times \vec{y} \rVert = 0\)
\end{question}
	
\end{exercise}
	


\begin{exercise}
Bij de opzet van een aanval loopt een voetballer eerst \(15\) m evenwijdig met de zijlijn om vervolgens onder een hoek van \(45\circ\) met de zijlijn \(18\) m naar binnen te snijden. 
Hoe ver van het vertrekpunt komt hij uit? Maak een schets met vectoren en voer ook hiermee je berekening uit.
\end{exercise}

\begin{exercise}
Vanop dezelfde middenstip vertrekken twee spelers, één wandelt \(9\) m evenwijdig met de zijlijn naar het ene doel en de ander wandelt \(17\) m in een richting die een hoek van \(35\circ\) maakt met de middellijn, naar het andere doel toe. Hoe ver komen de spelers van elkaar uit? Maak een schets met vectoren en voer ook hiermee je berekening uit.
\end{exercise}


\begin{exercise}
	Twee treinen vertrekken gelijktijdig uit Leuven station met constante snelheden van \(10\) m/s en \(20\) m/s. 
	De trage trein rijdt recht naar Mechelen en de andere recht naar Aarschot. 
	
	\begin{image}
	\begin{tikzpicture}
		\pgfmathsetmacro{\ang}{80}
	
		\coordinate (L) at (0,0); 
		\coordinate (A) at (70:1); 
		\coordinate (M) at (70 + \ang : 2); 
	
		\fill (L) circle (1pt);
		\fill (A) circle (1pt);
		\fill (M) circle (1pt);
	
		\draw[dotted, very thick] (L)--(A) node[pos=0, below]{\small{Leuven}} node[pos=1, right]{\small{Aarschot}} --(M) node[pos=1, below left]{\small{Mechelen}} --cycle;
	
		\draw pic[ pic text= \(80^\circ\), draw,  angle radius=0.7cm]{angle = A--L--M};
		
	\end{tikzpicture}
	\end{image}

	\begin{question}
		Bepaal de snelheid van de trage trein ten op zichte van de snelle trein. Werk met vectoren!
	\end{question}

	\begin{question}
		Heeft de snelheid van de snelle t.o.v. de trage trein dezelfde grootte, richting en/of zin?
	\end{question}
	\end{exercise}

\begin{exercise}
	Toon aan dat \( \| \vec{a} - \vec{b} = \|\vec{a}\|^2 + \|\vec{b}\|^2 - 2 \cdot \|\vec{a}\| \cdot \|\vec{b}\| \cdot \cos(\alpha)\) met \(\alpha\) de hoek tussen \(\vec{a}\) en \(\vec{b}\). 
\end{exercise}

\begin{exercise}
	Geldt er algemeen dat \(\vec{a} \cdot \vec{b} = \vec{b} \cdot \vec{a}\)? 
	Geldt er dat \(\vec{a} \times \vec{b} = \vec{b} \times \vec{a}\)? 
	Verklaar kort.
\end{exercise}

\begin{exercise}
	Kan er gelden dat \(\vec{a} \cdot \vec{b} = \| \vec{a} \times \vec{b} \|\)? 
	Zoja, geef de nodige voorwaarden en zoniet, verklaar. 
\end{exercise}
\end{document}
