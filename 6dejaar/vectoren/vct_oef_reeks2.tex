\documentclass{ximera}

\addPrintStyle{../..}

\begin{document}
	\author{Bart Lambregs en Vincent Gellens}
	\xmtitle{Oefeningen vectoren reeks 2}{}
    \xmsource\xmuitleg



\begin{exercise}\

Gegeven de drie waarvoor geldt \(\|\vec{a}\| = \SI{4}{N}\),\; \(\|\vec{b} \|= \SI{2.5}{N}\) \; \(\|\vec{c} \|= \SI{3}{N}\), \; \(\alpha = 15^\circ\)\;

Constureer en bepaal de groottes van: 
	\begin{image}[0.2\textwidth]
		\begin{tikzpicture}
			\pgfmathsetmacro{\ang}{15}
		
			\coordinate (O) at (0,0); 
			\coordinate (A) at (180:2); 
			\coordinate (B) at (90-\ang :1.6); 
			\coordinate (C) at (90:1.8); 
		
			\draw[->, very thick, red] (O)--(A) node[midway, below]{$\vec{a}$};
			\draw[->, very thick, xmgreen] (O)--(B) node[midway, below right]{$\vec{b}$};
			\draw[->, very thick, blue] (O)--(C) node[pos=0.8, left]{$\vec{c}$};
		
			\draw pic[ pic text= \(\alpha\), draw,  angle radius=0.7cm, angle eccentricity=1.5]{angle = B--O--C};
			\draw pic[ draw, angle radius=0.4cm]{right angle = C--O--A};
			% \draw pic[ draw, pic text= \LARGE{$\llcorner$},  angle radius=0.4cm]{right angle = C--O--A};
			
		\end{tikzpicture}
	\end{image}

\begin{question}
	\(\vec{a} - \vec{b}\)
\end{question}

\begin{question}
	\(\vec{b} - \vec{a}\)
\end{question}

\begin{question}
	\(\vec{a} + \vec{b} - \vec{c}\)
\end{question}

\begin{question}
	\(3\vec{a} - 2\vec{b}\)
\end{question}

\begin{question}
	\(4\vec{b} + \vec{c}\)
\end{question}

\begin{question}
	\(\vec{a} - 2\vec{b} + 3\vec{c}\)
\end{question}

\begin{question}
	\(3\vec{a} - 4\vec{c}\)
\end{question}

\end{exercise}


\begin{exercise}


	
Als \(\vec{F} \; \bot \; \vec{y}\), welk(e) van onderstaande uitspraken is dan juist? 
Meerdere antwoorden zijn mogelijk. 

\begin{question}
\(\vec{F} \cdot \vec{y} = \vec{0}\)
\end{question}

\begin{question}
\(\vec{F} \times \vec{y} = \vec{0}\)
\end{question}

\begin{question}
\(\vec{F} \cdot \vec{y} = \lVert \vec{F} \rVert \cdot \lVert \vec{y} \rVert\)	
\end{question}

\begin{question}
\(\lVert \vec{F} \times \vec{y} \rVert = \lVert \vec{F} \rVert \cdot \lVert \vec{y} \rVert\)	
\end{question}

\begin{question}
\(\vec{F} \times \vec{y} = \lVert \vec{F} \rVert \cdot \lVert \vec{y} \rVert\)	
\end{question}

\begin{question}
\(\vec{F} \cdot \vec{y} = 0\)
\end{question}

\begin{question}
\(\vec{F} \times \vec{y} = 0\)
\end{question}

\begin{question}
\(\lVert \vec{F} \times \vec{y} \rVert = 0\)
\end{question}

\end{exercise}



\begin{exercise}
	
Als \(\vec{F} \; \| \; \vec{y}\), welk(e) van onderstaande uitspraken is dan juist? 
Meerdere antwoorden zijn mogelijk. 

\begin{question}
\(\vec{F} \cdot \vec{y} = \vec{0}\)
\end{question}

\begin{question}
\(\vec{F} \times \vec{y} = \vec{0}\)
\end{question}

\begin{question}
\(\vec{F} \cdot \vec{y} = \lVert \vec{F} \rVert \cdot \lVert \vec{y} \rVert\)	
\end{question}

\begin{question}
\(\lVert \vec{F} \times \vec{y} \rVert = \lVert \vec{F} \rVert \cdot \lVert \vec{y} \rVert\)	
\end{question}

\begin{question}
\(\vec{F} \times \vec{y} = \lVert \vec{F} \rVert \cdot \lVert \vec{y} \rVert\)	
\end{question}

\begin{question}
\(\vec{F} \cdot \vec{y} = 0\)
\end{question}

\begin{question}
\(\vec{F} \times \vec{y} = 0\)
\end{question}

\begin{question}
\(\lVert \vec{F} \times \vec{y} \rVert = 0\)
\end{question}
	
\end{exercise}
	

\end{document}
