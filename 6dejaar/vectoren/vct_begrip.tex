\documentclass{ximera}
\input{../preamble}

\addPrintStyle{..}

\begin{document}
	\author{Bart Lambregs}
	\xmtitle{Het begrip vector}{}
    \xmsource\xmuitleg


	In de fysica wordt de natuur vaak beschreven met grootheden. Die kan men opsplitsen in twee categorieën: scalaire grootheden (scalars) en vectoriële grootheden (vectoren). 
	Het onderscheid ertussen en de omgang met beiden zijn ontzettend belangrijk in fysica.
	Grootheden die de vraag kunnen oproepen: “Naar waar gericht?” zijn vectoren, grootheden waarbij die vraag geen antwoord heeft, zijn scalars. 


	blablabla mijn aanapssing 

	
	Voorbeeld: een helikopter vliegt met een snelheid van 40 km/h. 
	Vraag: “Naar waar?” Antwoord: “Naar het zuiden, naar Brussel, naar omhoog, schuin naar onderen, …” 
	Er zijn vele betekenisvolle antwoorden mogelijk. 
	Snelheid is dus een vector. 
	Stel het zwembadwater heeft een temperatuur van 27°C. 
	Vraag: “Naar waar?” Antwoord: “Hier is geen zinnig antwoord op.” 
	Temperatuur is dus een scalar.

	Een vectoriële grootheid heeft drie variabele kenmerken: grootte, richting en zin. 
	Voorbeeld: de helikopter vliegt aan 40 km/h, horizontaal en naar het zuiden. 
	Een scalaire grootheid heeft slechts één kenmerk: de grootte (waarin soms ook een teken vervat zit). 
	Voorbeeld: een sneeuwbal heeft een temperatuur van -10°C.
	De plaats waarop de vector van toepassing is, noemt men het aangrijpingspunt van de vector.
	

	
\end{document}
