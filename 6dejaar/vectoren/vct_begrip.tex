\documentclass{ximera}
\input{../../preamble}

\addPrintStyle{..}

\begin{document}
	\author{Bart Lambregs}
	\xmtitle{Het begrip vector}{}
    \xmsource\xmuitleg


De natuurkunde beschrijft de natuur met grootheden die worden opsplitst in twee categorieën: scalaire grootheden (scalars) en vectoriële grootheden (vectoren). 
Grootheden die de vraag kunnen oproepen: “Naar waar gericht?” zijn vectoren, grootheden waarbij die vraag geen antwoord heeft, zijn scalars. 
Dit onderscheid en een correcte omgang met beiden zijn ontzettend belangrijk in fysica.


Stel dat \textit{een helikopter vliegt met een snelheid van 40 km/h.} 
Vraag: “Naar waar?” Antwoord: “Naar het zuiden, naar Brussel, naar omhoog, schuin naar onderen, …” 
Er zijn vele betekenisvolle antwoorden mogelijk. 
Snelheid is een vector. 
Als \textit{het zwembadwater een temperatuur van  \(\SI{27}{\celsius}\) heeft}, is er geen zinnig antwoord op de vraag \textit{naar waar?}. 
Temperatuur is een scalar.

Een vectoriële grootheid heeft vier kenmerken: grootte, richting, zin en een aangrijpingspunt. 
Zo kan de helikopter aan 40 km/h \textit{richting het zuiden} vliegen. % horizontaal?
Een scalaire grootheid heeft enkel een grootte met teken, zo kan je diepvries een temperatuur hebben van \(\SI{-10}{\celsius}\).


\begin{remark}
Het onderscheidt tussen scalar en vector is in de eerste plaats een verschil in \textit{naamgeving}.
Je kan zeggen dat de temperatuur in graden gelijk is aan de scalar \(27\) of gegeven wordt door de vector \(\vec{temp} = (27) \) met slechts één component. 
\end{remark}


% TODO: MOET ER NOG IETS GEZEGD WORDEN OVER VRIJE VECTOREN? 


	
\end{document}
