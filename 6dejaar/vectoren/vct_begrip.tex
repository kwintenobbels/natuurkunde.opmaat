\documentclass{ximera}
\input{../preamble}

\addPrintStyle{..}

\begin{document}
	\author{Bart Lambregs}
	\xmtitle{Het begrip vector}{}
    \xmsource\xmuitleg


In de fysica wordt de natuur beschreven met grootheden die men opsplitst in twee categorieën: scalaire grootheden (scalars) en vectoriële grootheden (vectoren). 
Grootheden die de vraag kunnen oproepen: “Naar waar gericht?” zijn vectoren, grootheden waarbij die vraag geen antwoord heeft, zijn scalars. 
Dit onderscheid en een correcte omgang met beiden zijn ontzettend belangrijk in fysica.


Voorbeeld: een helikopter vliegt met een snelheid van 40 km/h. 
Vraag: “Naar waar?” Antwoord: “Naar het zuiden, naar Brussel, naar omhoog, schuin naar onderen, …” 
Er zijn vele betekenisvolle antwoorden mogelijk. 
Snelheid is dus een vector. 
Stel het zwembadwater heeft een temperatuur van 27°C. 
Vraag: “Naar waar?” Antwoord: “Hier is geen zinnig antwoord op.” 
Temperatuur is dus een scalar.

Een vectoriële grootheid heeft drie variabele kenmerken: grootte, richting en zin. %aangrijpingspunt?
Voorbeeld: de helikopter vliegt aan 40 km/h, horizontaal en naar het zuiden. % horizontaal?
Een scalaire grootheid heeft slechts één kenmerk: de grootte (waarin soms ook een teken vervat zit). 
Voorbeeld: een sneeuwbal heeft een temperatuur van -10°C.
De plaats waar de vector aangrijpt, noemt men het aangrijpingspunt van de vector. %naar volgende activity? 


\begin{remark}
Het onderscheidt tussen scalar en vector is in de eerste plaats een verschil in \textit{naamgeving}.
Een scalar kan immers bekeken worden als een vector met slechts één component.

Zo kan men zeggen dat de temperatuur in graden gelijk is aan de scalar '\(27\)' of gegeven wordt door de vector \(\vec{t} = (27) \) met slechts één component. 
\end{remark}


% TODO: MOET ER NOG IETS GEZEGD WORDEN OVER VRIJE VECTOREN? 


	
\end{document}
