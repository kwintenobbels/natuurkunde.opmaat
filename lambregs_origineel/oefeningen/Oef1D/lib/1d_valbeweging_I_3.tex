% !TEX root = ../main.tex



\item Aan de rand van een afgrond laat men een steen vallen. Op hetzelfde ogenblik werpt men een steen op. Zou het kunnen dat, als de afgrond diep genoeg is, beide stenen elkaar nog ontmoeten?
\begin{oplossing}
\newline
\newline
Aangezien de opgeworpen steen later (en zelfs hoger) begint met vallen en beide stenen eenzelfde versnelling hebben, kan op geen enkel moment de opgeworpen steen een grotere snelheid hebben dan de steen die wordt losgelaten. Dat laatste zou op het moment van inhalen nochtans op zijn minst het geval moeten zijn.
\newline
\newline
In formules moet gelden, met $v_0$ een negatieve beginsnelheid:
\begin{eqnarray*}
\frac{1}{2}gt^2&=&v_0t+\frac{1}{2}gt^2.
\end{eqnarray*}
Dat is enkel het geval wanneer $t=0$.
\end{oplossing}