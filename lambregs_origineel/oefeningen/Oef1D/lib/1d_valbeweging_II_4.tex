% !TEX root = ../main.tex



 Wanneer de pelikaan naar vis duikt, trekt hij zijn vleugels in om als een steen loodrecht naar beneden te vallen. 

Stel een pelikaan duikt vanaf \SI{25}{m} hoogte en verandert onderweg dus niet meer van koers. Als het een vis \SI{0,15}{s} kost om te vluchten, wat is dan de hoogte waarop de vis de pelikaan minstens moet opmerken, wil de vis nog kans maken te ontsnappen? 

Neem aan dat de vis zich aan het wateroppervlak bevindt.

\begin{oplossing}
\item[Gegeven]$x_2=\SI{25}{m}$\newline$\Delta t=\SI{0,15}{s}$
\item[Gevraagd]$x_2-x_1$
\item[Oplossing]We kiezen de referentie-as naar beneden, met de oorsprong op de positie waar de pelikaan begint aan zijn duik. De versnelling is dan gelijk aan de valversnelling. 

We kennen de afstand waarover de pelikaan valt zodat we de tijd die de pelikaan nodig heeft om het wateroppervlak te bereiken, de valtijd, kunnen berekenen uit $x_2=\frac{1}{2}gt_2^2 $:
\begin{eqnarray*}
t_2=\sqrt{\frac{2x_2}{g}}
\end{eqnarray*}
De pelikaan heeft namelijk geen beginsnelheid.%De referentie-as is hierbij naar beneden gekozen, met de oorsprong daar waar de pelikaan begint te vallen.

Gedurende een tijd $t_1=t_2-\Delta t$ (15 honderdste van een seconde minder) mag de pelikaan vallen zonder door de vis te worden opgemerkt. De afstand boven het wateroppervlak is dan:
\begin{eqnarray*}
x_2-x_1&=&x_2-\frac{1}{2}gt_1^2\\
&=&x_2-\frac{1}{2}g\left(t_2-\Delta t\right)^2\\
%&=&x_2-\frac{1}{2}g\left(\sqrt{\frac{2x_2}{g}}-\Delta t\right)^2\\
%&=&\Delta t\sqrt{2gx_2}-\frac{1}{2}g\Delta t^2\\
&=&\SI{3,2}{m}
\end{eqnarray*}
\end{oplossing} 