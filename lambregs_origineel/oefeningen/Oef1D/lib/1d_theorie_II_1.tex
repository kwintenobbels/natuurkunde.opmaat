% !TEX root = ../main.tex



\item Beargumenteer het gebruik van het model van een eenparig versnelde beweging (EVB) voor de vrije beweging van een wagentje op een helling. Denk daarbij aan een proefneming die we in de klas deden. De notie kracht moet je hier even buiten beschouwing laten.

\begin{oplossing}

Het model beschrijft de meetgegevens accuraat.

M.a.w. zijn de meetgegevens van de positie van het wagentje op de helling in functie van de tijd, gemeten met een (ultrasone) positiesensor, accuraat te beschrijven met de plaatsfunctie van een eenparig versnelde beweging.

\emph{Toelichting}.
De vraag gaat over de relatie tussen de theorie en de realiteit. Het is maar door metingen te doen dat we kunnen nagaan of gevolgen van de theorie (in dit geval bijvoorbeeld dat de positie kwadratisch in de tijd verloopt voor een beweging met constante versnelling) overeenkomen met de realiteit. In het gegeven geval van een wagentje op een helling, is bijvoorbeeld een model van constante snelheid niet van toepassing. Het zou immers impliceren dat het wagentje niet van zin kan veranderen. Dat laatste wordt door metingen of waarnemingen weerlegd.

\end{oplossing} 