% !TEX root = ../main.tex


\pt{4}Laat zien dat voor een EVB waarvoor $x_0=0$ de volgende formule geldt:
\begin{eqnarray*}
v^2=v_0^2+2ax
%v^2=v_0^2+2a(x-x_0)
\end{eqnarray*}

\begin{oplossing}
Uit $v=v_0+at$ volgt:
\begin{eqnarray*}
v^2&=&v_0^2+2v_0at+a^2t^2\\
&=&v_0^2+2a(v_0t+\frac{1}{2}at^2)\\
&=&v_0^2+2ax
\end{eqnarray*}
\end{oplossing}
