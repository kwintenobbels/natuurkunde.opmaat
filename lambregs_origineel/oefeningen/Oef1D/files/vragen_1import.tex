% !TEX root = ../main.tex





\begin{enumerate}

\import{./lib/}{1d_theorie_II_1} % Model EVB wagentje op een helling


\import{./lib/}{FV_T1_18p40} % Meerkeuze Oriëntatie versnelling

\import{./lib/}{1d_EVRB_II_1} % Formules EVRB
\import{./lib/}{1d_EVRB_I_1}
\import{./lib/}{1d_EVRB_I_2} % Bewijs formule gemiddelde snelheid
\import{./lib/}{1d_EVRB_I_3} % Eenvoudige algebra met formules EVB
\import{./lib/}{1d_EVRB_II_3} % Vliegtuig, Afstand gedurende 12e seconde

\import{./lib/}{1d_theorie_I_1}
\import{./lib/}{1d_theorie_I_2}
\import{./lib/}{1d_theorie_I_3}
\import{./lib/}{1d_theorie_II_2} % Ordenen snelheid en versnelling

\import{./lib/}{1d_denkvraag_II_1} % Verloop afstandsverschil vallende lichamen

\import{./lib/}{1d_EVB_grafiek_II_1}
\import{./lib/}{1d_EVB_grafiek_II_2}
\import{./lib/}{1d_EVB_grafiek_II_3}

\import{./lib/}{FV_T1_7p47}

\import{./lib/}{1d_EVRB_II_2} % Remmende trein

\import{./lib/}{1d_wis_II_1} % Formule v^2-v_0^2=2ax bewijzen
\import{./lib/}{1d_wis_II_2} % mk


\import{./lib/}{1d_valbeweging_I_1} %Verticale worp, standaard
\import{./lib/}{1d_valbeweging_I_2} % hoogte
\import{./lib/}{1d_valbeweging_I_3} % Misconceptie ...
\import{./lib/}{1d_valbeweging_II_1} % Parachutist
\import{./lib/}{1d_valbeweging_II_2} % FV 8 p. 86
\import{./lib/}{1d_valbeweging_II_3} %FV 3 p. 90
\import{./lib/}{1d_valbeweging_II_4} % Pelikaan die naar vis duikt
\import{./lib/}{1d_valbeweging_II_5} % Maximale hoogte op de maan
\import{./lib/}{1d_valbeweging_III_1} % Laatste 100 m
\import{./lib/}{1d_valbeweging_III_2} % FV 2 p. 92 Empire State Building
\import{./lib/}{1d_valbeweging_III_3} % Ziek man voor een raam
\import{./lib/}{1d_valbeweging_III_4} % Student gooit een sleutelbos
\import{./lib/}{1d_valbeweging_III_5} % formule voor maximale hoogte





\end{enumerate}













