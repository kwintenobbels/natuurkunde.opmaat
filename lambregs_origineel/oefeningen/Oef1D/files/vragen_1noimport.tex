% !TEX root = ../main.tex





\begin{enumerate}



\item Toon aan dat de positie en de snelheid van een eendimensionale beweging met constante versnelling, worden gegeven door de volgende functies:
\begin{eqnarray*}
x&=&x_0+v_0t+\frac{1}{2}at^2\\
v&=&v_0+at 
\end{eqnarray*}

\item Bewijs dat de plaatsfunctie $x(t)$ van een EVRB met versnelling $a$ gegeven wordt door:
\begin{eqnarray*}
x(t)=x_0+v_0(t-t_0)+\frac{1}{2}a(t-t_0)^2
\end{eqnarray*}

\item Toon aan dat voor een EVRB de snelheid als functie van de tijd wordt gegeven door:
\[v=v_0+a(t-t_0)\]



\item Laat zien dat voor een EVRB de volgende formule geldt:
\begin{eqnarray*}
x-x_0&=&\left(\frac{v+v_0}{2}\right)(t-t_0)
\end{eqnarray*}

\item Bewijs voor een EVRB de volgende formule voor de gemiddelde snelheid:
\begin{eqnarray*}
\overline{v}=\frac{v_0+v}{2}
\end{eqnarray*}


\item Bewijs dat de remweg van een met constante versnelling remmende auto, evenredig is met het kwadraat van de beginsnelheid.
\begin{oplossing}
\item[Bewijs]Bij het tot stilstand komen is de snelheid van de auto nul, zodat de tijd die hij hiervoor nodig heeft als volgt te vinden is:
\begin{eqnarray*}
v&=&0\\
&\Updownarrow&\\
v_0+at&=&0\\
&\Updownarrow&\\
t&=&-\frac{v_0}{a}
\end{eqnarray*}
Door deze tijd in de plaatsfunctie in te vullen, weten we welke afstand de auto heeft afgelegd gedurende het remmen.
\begin{eqnarray*}
x&=&v_0t+\frac{1}{2}at^2\\
&=&v_0\left(-\frac{v_0}{a}\right)+\frac{1}{2}a\left(-\frac{v_0}{a}\right)^2\\
&=&-\frac{v_0^2}{a}+\frac{v_0^2}{2a}\\
&=&-\frac{v_0^2}{2a}\\
\end{eqnarray*}
De factor $-\frac{1}{2a}$ is een (positieve, de versnelling is negatief) constante. De afgele afstand $x$ en het kwadraat van de beginsnelheid $v_0^2$ zijn dus recht evenredig. 
\end{oplossing}

\item Vanaf welke hoogte $x$ moet een lichaam vallen om met een snelheid $v$ de grond te bereiken?

\begin{oplossing}
	$x=\frac{v^2}{2g}$
\end{oplossing}



\item Een voorwerp beweegt op een rechte baan en
voert een eenparig versnelde beweging uit. Twee seconden na zijn
doorkomst in een referentiepunt R is de snelheid verdubbeld ten
opzichte van deze in R.
\newline
\newline
Dan was \'e\'en seconde na zijn doorkomst in het referentiepunt R de
snelheid:
\begin{enumerate}
\item 3/2 maal zo groot als deze in R.
\item 1/2 maal zo groot als deze in R.
\item 2/3 maal zo groot als deze in R.
\item $\sqrt{2}$ maal zo groot als deze in R.
\end{enumerate}
\footnote{antw. a}



\end{enumerate}


