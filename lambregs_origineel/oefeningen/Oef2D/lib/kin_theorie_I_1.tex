% !TEX root = ../main.tex



\item Vergelijk de begrippen \emph{verplaatsing} en \emph{afgelegde weg} met elkaar. Geef dus enkele gelijkenissen en enkele verschillen. 
\begin{oplossing}
\newline
Zie p. 12 van het handboek en voorbeeld 3 op pagina 13.

De twee begrippen hebben gemeen dat ze beide 
\begin{enumerate}
	\item een fysische grootheid zijn;
	\item een verandering in de positie beschrijven;
	\item als eenheid de meter hebben;
	\item een gelijke numerieke waarde voor de grootte hebben als de beweging in \'e\'en dimensie volgens eenzelfde zin plaatsvindt.
\end{enumerate}

Een verschil tussen de begrippen is

\begin{enumerate}
	\item dat de verplaatsing een vectorie\"ele grootheid is, daar waar de afgelegde weg een scalaire grootheid is. 
	\item dat de verplaatsing gedefinieerd is als het verschil tussen de eind- en de beginpositie ($\vec{\Delta r}=\vec{r}_2-\vec{r}_1$) en de netto verandering in de ruimte weergeeft terwijl de afgelegde weg de totaal aantal afgelegde meters gemeten langs de baan weergeeft. Het is de lengte van de route.
	\item dat de numerieke waarde van de grootte kan verschillen, ook als de beweging in \'e\'en dimensie plaatsvindt.
\end{enumerate}

\end{oplossing}