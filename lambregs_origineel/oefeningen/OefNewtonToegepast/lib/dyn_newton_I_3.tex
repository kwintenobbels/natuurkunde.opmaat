% !TEX root = ../main.tex


\item\pt{5}Je springt vanuit een roeibootje naar de oever. Verklaar wat er kan gebeuren als het roeibootje niet is vastgemeerd. Wanneer je nu vanaf een groot binnenschip naar de oever springt, wat zou het resultaat dan zijn? Verklaar!
\begin{oplossing}

Bij een roeibootje dat niet vastgemeerd is, is er veel kans dat je in het water belandt. Bij het afzetten, schiet het bootje namelijk gemakkelijk onder je weg. Hoe komt dat? Wel, doordat je je afzet, oefen je een kracht uit op het bootje. De \textbf{derde wet van Newton} zegt dat het bootje dan een even grote kracht op jou uitoefent, in de tegengestelde richting. Het is die reactiekracht die je zou willen aanwenden om op de oever te geraken. De massa van het bootje is echter zo klein in vergelijking met jouw massa dat, volgens de \textbf{tweede wet van Newton} ($\vec{F}=m\vec{a}$), de versnelling die het bootje krijgt als gevolg van jouw actiekracht veel groter is dan de versnelling die je zelf krijgt door de afzet. In de korte tijd dat je jezelf kan afzetten, verwerft het bootje dus een grote snelheid waardoor het onder je wegschiet en krijg jij geen noemenswaardige snelheid opgebouwd.

%Als het bootje is vastgemeerd, lukt het je wel de oever te bereiken. Het bootje kan immers niet wegschieten waardoor je je voldoende lang kan afzetten (er werkt voldoende lang een kracht op jou) en zo de nodige snelheid kan verwerven (je krijgt immers een versnelling) om de sprong te kunnen maken.

Voor een groot binnenschip is de massa zo groot in vergelijking met die van jou, dat het binnenschip een verwaarloosbare versnelling weg van de oever krijgt. Je kan een voldoende grote versnelling opbouwen die lang genoeg aanhoudt om je op de oever te krijgen.
\end{oplossing}



%\item Je springt vanuit een roeibootje naar de oever. Verklaar wat er kan gebeuren als het roeibootje niet of wel vastgemeerd is. Wanneer je nu vanaf een groot binnenschip naar de oever springt, wat zou het resultaat dan zijn? Verklaar!