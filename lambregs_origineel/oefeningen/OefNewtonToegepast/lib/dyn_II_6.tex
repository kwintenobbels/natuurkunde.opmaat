% !TEX root = ../main.tex



\item\pt{1}Hoe kan een man die \SI{686}{N} weegt langs een touw naar beneden glijden, dat slechts \SI{600}{N} kan dragen zonder te breken?

Stel dat de man inschat dat hij, zonder zijn botten te breken, in staat is te springen van toch wel \SI{3,0}{m} hoog. Van hoe hoog zou hij dan met een dergelijk touw kunnen ontsnappen?



\begin{oplossing}
Door niet met zijn volle gewicht aan het touw te gaan hangen kan de man verhinderen dat het touw breekt. Het gevolg is wel dat hij een nettokracht naar beneden ondervindt waardoor hij toch naar beneden versnelt, al is het met een kleinere versnelling dan de valversnelling.

Door met \SI{600}{N} aan het touw te trekken, ondervindt hij een spankracht omhoog met diezelfde grootte. Met een referentieas naar beneden volgt uit $F_z-F_s=ma$ en uit $m=\frac{F_z}{g}$ voor de versnelling van de man:
\begin{equation}
	a=\left(1-\frac{F_s}{F_z}\right)g\label{versnelling_ontsnapper}
\end{equation}
wat gelijk is aan \SI{1,23}{m/s^2}.

Uit $v^2=v_0^2+2ax$ volgt de maximale snelheid die hij bij de impact op de grond aankan als we voor $a$ de valversnelling $g$ nemen en voor $x$ de gegeven \SI{3,0}{m}. Uit diezelfde formule vinden we de hoogte $h$ vanwaar de man kan ontsnappen als we nu de netto versnelling $a=\SI{1,23}{m/s^2}$ (\ref{versnelling_ontsnapper}) nemen: 
\begin{equation*}
	h=\frac{v^2}{2a}=\ldots=\frac{F_z}{F_z-F_s}x
\end{equation*}
wat gelijk is aan \SI{24}{m}.
\end{oplossing}