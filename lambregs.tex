%\documentclass[12pt,numbers,noauthor,nooutcomes,wordchoicegiven]{xourse}
\documentclass{xourse}
\input{./preamble.tex}

\addPrintStyle{.}

\pdfOnly{
    \renewcommand{\xmcursusnaam}{{\textsc{Veeltermen}}}
}


%  HIER NOG DE KORTE PREAMBLE VAN LAMBREGS ZETTEN 


\begin{document}
%	\setcounter{tocdepth}{2}
    \xmtitle{Veeltermen}{}  
 


\part{Basisbegrippen}


\activitychapter{veeltermen/vltrm_h1_eentermen.tex}
\activitychapter{veeltermen/vltrm_h1_veeltermen.tex}
\activitychapter{veeltermen/vltrm_h1_parameters.tex}
\activitychapter{veeltermen/vltrm_h1_basisbewerkingen.tex}
\activitychapter{veeltermen/vltrm_h1_nuldelers.tex}

\activitysection{veeltermen/vltrm_h1_zoef_reeks1.tex}
\activitysection{veeltermen/vltrm_h1_zoef_reeks2.tex}
\activitysection{veeltermen/vltrm_h1_zoef_reeks3.tex}





\part{Eendimensionale bewegingen}

\activitychapter{lambregs_ximera/knmtc_1dim_begrippen.tex}
\activitychapter{lambregs_ximera/knmtc_1dim_ERB.tex}
\activitychapter{lambregs_ximera/knmtc_1dim_EVRB.tex}
\activitychapter{lambregs_ximera/knmtc_1dim_verticale_worp.tex}



\part{Tweedimensionale bewegingen}


\activitychapter{lambregs_ximera/knmtx_2dim_onafhankelijkheidsbeginsel.tex}
\activitychapter{lambregs_ximera/knmtx_2dim_begrippen.tex}
\activitychapter{lambregs_ximera/knmtx_2dim_eenpariga_cirkelbeweging.tex}
\activitychapter{lambregs_ximera/knmtx_2dim_horizontale_worp.tex}


\part{De beginselen van Newton}

\activitychapter{lambregs_ximera/nwtn_traagheid.tex}
\activitychapter{lambregs_ximera/nwtn_tweede_beginsel.tex}
\activitychapter{lambregs_ximera/nwtn_derde_beginsel.tex}
\activitychapter{lambregs_ximera/nwtn_oplossingsstrategie.tex}
\activitychapter{lambregs_ximera/nwtn_historische_uitwijding.tex}


\part{De gravitatiekracht}

\activitychapter{lambregs_ximera/grvt_kepler.tex}
\activitychapter{lambregs_ximera/grvt_universele_gravitatiekracht.tex}
\activitychapter{lambregs_ximera/grvt_satellietbanen.tex}
\activitychapter{lambregs_ximera/grvt_gewicht.tex}
\activitychapter{lambregs_ximera/grvt_zwaartekracht.tex}


\part{Arbeid en Energie}

\activitychapter{lambregs_ximera/ae_constante_kracht.tex}
\activitychapter{lambregs_ximera/ae_niet_constante_kracht.tex}
\activitychapter{lambregs_ximera/ae_theorema.tex}
\activitychapter{lambregs_ximera/ae_pot_elastische.tex}
\activitychapter{lambregs_ximera/ae_pot_gravitationele.tex}
\activitychapter{lambregs_ximera/ae_pot_gravitationele_algemeen.tex}
\activitychapter{lambregs_ximera/ae_pot_referentiepunt.tex}
\activitychapter{lambregs_ximera/ae_pot_potentiële_energie.tex}
\activitychapter{lambregs_ximera/ae_behoud.tex}


\part{Harmonische trillingen}


\activitychapter{lambregs_ximera/hrmn_inleiding.tex}
\activitychapter{lambregs_ximera/hrmn_harmonische_oscillator.tex}
\activitychapter{lambregs_ximera/hrmn_relatie_cirkelbeweging.tex}
\activitychapter{lambregs_ximera/hrmn_mathematische_slinger.tex}
\activitychapter{lambregs_ximera/hrmn_resonantie.tex}

\part{Golven}

\activitychapter{lambregs_ximera/glvn_golven_kenmerken.tex}
\activitychapter{lambregs_ximera/glvn_periodieke_golven.tex}





\end{document}