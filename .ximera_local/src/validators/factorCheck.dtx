% \subsubsection{factorCheck}
%%%   This validators allows you to require a factored form for a given asnwer.
%%%   The validator parses the instructor's answer to determine:
%%%       1) How many (non-constant) factors the instructor supplies
%%%       2) What degree each factor is.
%%%       3) That each factor is indeed of the form: $ax^n$ for a in R and n in {0} cup N.
%%%   The validator then parses the student's answer and determines:
%%%       1) How many (non-constant) factors the instructor supplies
%%%       2) What degree each factor is.
%%%       3) That each factor is indeed of the form: $ax^n$ for a in R and n in {0} cup N.
%%%   The answer is validated as correct if the number of factors for each degree match AND 
%%%       The products of all the factors for the instructor and the student are equal
%%%       (i.e. the instructor and student submissions expand into the same polynomial).
%%%       Note that this means that the student can move constants around between factors
%%%           without any problem.
%%%       
%%%   Ex: If the instructor provides $(x^2-4)(3x+6)$ as an answer, then 
%%%       the following are some examples of what would be marked correct:
%%%           $(x^2-4)(3x+6), 3(x^2-4)(x+2), (3x^2-12)(x+2), 2(x^2-4)(\frac{3}{2}x+3), 3(x-2)(x^2+4x+4)$
%%%       The following would be marked as *incorrect*:
%%%           $(x-2)(x+2)(x+2), (x-2)(x+2)^2, 3x^3+6x^2-12x-24$
%%%
%    \begin{macrocode}
%<*classXimera>
%%% Apend code via \appendtoverbtoks

\appendtoverbtoks?
/*
HOW THIS SHOULD WORK:
    Initially check to make sure the submitted answer (and proposed answer) are at least in some kind of theoretically factored form.
    
    Next duplicate the raw trees so we can mess with them without worrying about changing the original.
    
    Fold up any exponents on both trees so that our root node is of the form: ['*',factor1,factor2,factor3,...]. 
        This includes killing off any leading negative signs (again we'll compare for equality using the original raw string)
    
    Now call a recursive function to deep-dive into each factor to find what degree that factor actually is.
        Will also identify if the factor isn't even a factor, in which case we will return a negative value to negate the answer.
        
    Once we have the degree for each factor, now we can compare the instructor degree list and student degree list as intended.
*/


// Subfunction to identify if something is a number:

function isNum(numb) {
    if ((typeof numb === 'number')||(numb=='e')||(numb=='pi'))
    {return true} else {return false}
}

// Subfunction to identify if something is a non-negative integer:

function isPosInt(numb) {
    if ((isNum(numb))*(numb>=0)*(numb%1==0))
    {return true} else {return false}
}


// This does a recursion through a factor to eventually find it's degree - assuming it's a polynomial.

function degreeHunt(tree,position,curDeg) {
    //(Re)set curDeg just in case:
    var curDeg=0;
    
    // First, let's figure out what to do about negative signs, since they can be annoying.
    //  I think there's three possibilities, it's a negative array, a negative x, or a negative number.
    
    if (tree[position][0]=='-') {
        debugText('Processing a minus sign.');
        if (tree[position][1]=='x') {
            // We found a ``-x'' term within our factor, so that's degree 1 I guess!
            
            debugText('Found a -x term!');
            curDeg = Math.max(curDeg,1);
            
        } else if (Array.isArray(tree[position][1])) {
            // Else if there's a negative outside of an array, just bypass the negative and keep digging for an exponent sign.
            
            debugText('Found a negative Array term!');
            let tempVal = degreeHunt(tree[position],1)
            if (tempVal<0) { return (-1)} else {
                curDeg = Math.max(curDeg,tempVal);
            }
            
            
        }// Note that, if it's a negative number, I don't care about it, so no need for an ``else''.
    } else if (tree[position][0]=='apply') {
        debugText('Processing an apply symbol.');
        // any 'apply' is inevitably a function that isn't a polynomial.
        //  Although, only if it actually has an 'x' in there - so I need to fix/update that at some point.
        
        // To find out if the apply is actually just a number or not, we need to recurs through every sub-node looking for an 'x'.
        //  If the degree ends up positive, then we have a variable inside the apply function it's not a polynomial.
        //  If the degree ends up zero, then it's ultimately some bizarre number formation and we're fine.
        
        var tempDeg = 0;
        
        for (var j = 1; j < tree[position].length; ++j) {
            // Walk the array to find any powers of x.
            if (tree[position][j]=='x') {
                // If the entry is just x, then we have pos deg and we are done.
                debugText('Found an x inside an apply function that suggests the factor is not actually a polynomial!');
                return (-1);
            } else if (Array.isArray(tree[position][j])) {
                // If the term is an array, then we need to recurs to find the degree.
                let tempVal=degreeHunt(tree[position],j,0);
                if (tempVal!==0) {
                    debugText('Found an issue inside an apply function that suggests the factor is not actually a polynomial!');
                    return (-1)
                }
            }// Any other option is degree 0, so no need for an else.
        }// End of for loop and end up 'apply' function
    } else if (tree[position][0]=='^') {
        // Now we process the exponent sign case, but be careful cause students do crazy shit,
        //  So it might be a x^N situation, but it might be some other shenanigans.
        
        debugText('Processing an exponential sign.');
        
        if ((tree[position][1]=='x')*(isPosInt(tree[position][2]))) {
            // We have something like x^N
            
            debugText('We have x^N');
            
            curDeg = Math.max(curDeg,tree[position][2]);
            
        } else if (tree[position][1]=='x') {
            // If the base is x but it's not being raised to an integer power, then it's not a proper monomial.
            //  Note that we are assuming here that students won't put shit like ``1+1'' as the power, if so, it will be marked wrong.
            
            debugText('We have x^(g(x)) but g(x) is not a positive integer.');
            
            return -1;
            
        } else if ((isNum(tree[position][1]))*(isNum(tree[position][2]))) {
            // This means we have a^b which is still just a number, so it's fine... but doesn't give a degree.
            
            debugText('We have a^b');
            
        } else if ((Array.isArray(tree[position][1]))*(isPosInt(tree[position][2]))) {
            // We have something of the form (f(x))^N which might be a part of a factor.
            //  We recurse on the array, and multiply the result by the N.
            
            debugText('We have f(x)^N');
            let tempVal = tree[position][2]*degreeHunt(tree[position],1,curDeg);
            if (tempVal<0) { return (-1)} else {
                curDeg = Math.max(curDeg,tempVal);
            }

            
        } else {
            // All other situations are inevitably not polynomials.
            
            return (-1);
        }
    } else if (tree[position][0]=='/') {
        // This is problematic, because the only way we can allow a division is if the denominator is just a number.
        //  But if the denominator has an 'x' anywhere in it, then we have a problem...
        
        debugText('Processing a division sign.');
        
        if (tree[position][2]=='x') {
            // We're dividing by x, which is bad.
            
            debugText('Dividing by x, naughty naughty!');
            
            return (-1);
        } else if (Array.isArray(tree[position][2])) {
            //  Let's try doing a recurs, and if we get a result that isn't 0, then that means there's an 'x' so we return -1.
            if (degreeHunt(tree[position],2,0)!==0) {
                // If it's not 0, then we found an 'x' or something that invalidates the polynomial.
                return (-1)
            } // otherwise we have some kind of 'dividing by a number' situation, so it's fine, and doesn't impact degree.
            
        }
        
        // Now that we've dealt with the possibilities that cause a non-polynomial function, we can proceed assuming it is just division by a number.
        
        debugText('The denominator is just a number, so we need to check the numerator.');
        
        if (tree[position][1]=='x') {
            // If we've made it past the first two hurtles, then the bottom is just a number of some form.
            //  So if the top is 'x', then we have something like ``x/a'' which is still a degree 1 factor.
            var curDeg = Math.max(curDeg,1);
            debugText('We found a fraction with just x in the top.');
        } else if (Array.isArray(tree[position][1])) {
            // If the top of the fraction is an array - but we've already reduced to case where denominator is a number.
            //  So we need to figure out if there is a degree in the top to count.
            
            debugText('We found a fraction with an array for the numerator.')
            var tempVal = degreeHunt(tree[position],1,0)
            debugText('I think the numerator degree from the array is: '+tempVal);
            if (tempVal<0) { return (-1)} else {
                 var curDeg = Math.max(curDeg,tempVal);
            }
            
        }// Note that if none of the above, then it's just a/b, which is fine and doesn't impact degree.

    } else if (tree[position][0]=='*') {
        // If we are multiplying, we could be multiplying a bunch of terms - maybe a bunch of x terms.
        
        debugText('Processing a product sign.');
        
        var tempDeg = 0;
        for (var j = 0; j < tree[position].length; ++j) {
            // Add the degree of each thing being multiplied, even though most are probably zero.
            if (tree[position][j]=='x') {
                // If the entry is just x, then we have deg 1.
                debugText('Found a solo x term inside the product sign, degree is at least 1.')
                tempDeg = Math.max(tempDeg,1);
            } else if (Array.isArray(tree[position][j])) {
                // If the term is an array, then we need to recurs to find the degree.
                // let tempVal=degreeHunt(tree[position],j,0);
                if (degreeHunt(tree[position],j,0)<0) { 
                    return (-1)
                } else if (degreeHunt(tree[position],j,0)==0){
                    debugText('No degree term found inside product sign, so we stay at degree: '+tempDeg);
                } else {
                    debugText('found a higher degree term inside the product sign, degree is at least: '+tempDeg);
                    tempDeg = Math.max(tempDeg,degreeHunt(tree[position],j,0));
                }
            }// Any other option is degree 0, so no need for an else.
        }
        // Once the for loop finishes, we convert the temp deg to the current deg.
        var curDeg = Math.max(curDeg,tempDeg);
        debugText('Inside the product sign we found the degree is: '+tempDeg);
        debugText('So we set the curDeg to: '+curDeg);
    } else {
        // If it's none of the above, then we should just recurs on any arrays we find.
        
        debugText('Processing an unknown sign... Specifically: '+tree[position][0]);
        
        for (var j = 0; j < tree[position].length; ++j) {
            // Walk the array to find any powers of x.
            if (tree[position][j]=='x') {
                // If the entry is just x, then we have deg 1.
                curDeg = Math.max(curDeg,1);
            } else if (Array.isArray(tree[position][j])) {
                // If the term is an array, then we need to recurs to find the degree.
                let tempVal=degreeHunt(tree[position],j,0);
                if (tempVal<0) { return (-1)} else {
                    curDeg = Math.max(curDeg,tempVal);
                }
            }// Any other option is degree 0, so no need for an else.
        }
        
    }
    debugText('Made it to the end of the degreeHunt function, which means we need to return a curDeg variable, which is: '+curDeg);
    return curDeg
}

// Subfunction just to make sure that the submitted function is in a legitimately factored form.

function JNFisFactored(factorTree) {
   
    // First we check to see if we have a negative factored out, which messes everything up in the tree.
    if ((factorTree[0]=='-')||(factorTree[0]=='*')||((factorTree[0]=='/')*(isNum(JNFoperation[1])))
    ) {return true} else {return false}
}


function factorCheck(f,g) {
    // This validator is designed to check that a student is submitting a factored polynomial. It works by:
    //  Checking that the degree of each factor matches between student submitted and instructor submitted answers (NOT order sensitive),
    //  Checking that the submitted answer and the expected answer are the same via real Ximera evaluation,
    //  Checking that the outer most (last to be computed when following order of operations) operation is multiplication.
    //  It ignores degree 0 terms for degree check, and now can ignore factored out negative signs.
    
    
    debugText(f.tree);
    debugText(g.tree);

    if (JNFisFactored(g.tree)==false) {
        console.log('Answer rejected, instructor answer not in a factored form. Bad instructor, no donut.');
        return false
    }
    
    // First we check to make sure it is in *a* factored form:
    if (JNFisFactored(f.tree)==true) {
        console.log('The student answer is at least in *a* factored form.');
        
        
    } else {
        console.log('Answer rejected, student answer not in a factored form.');
        return false
    }
    
    // Let's duplicate the trees to manipulate, so we keep the original correctly.
    var studentAns=f.tree
    var instructorAns=g.tree
    
    // Also, if there is a factored out negative, let's just kill that, since we aren't doing a funtion comparison on this part:
    while (studentAns[0]=='-'){studentAns = studentAns[1];};
    while (instructorAns[0]=='-'){instructorAns = instructorAns[1];};
    
    // Now we want to fold up any root-level exponents into duplicate children of the master tree,
    //  This lets us assume the top-level node has 1 child per factor.
    debugText('folding up external exponents of studentTree  so factors do not have exponents');
    for (var i = 0; i < studentAns.length; ++i) {
        if ((studentAns[i][0] == '^')*(isPosInt(studentAns[i][2]))) {
            studentAns=studentAns.concat(Array(studentAns[i][2]).fill(studentAns[i][1]));
            studentAns.splice(i,1);// This should theoretically remove the original term now that we've duplicated it.
            i=i-1;// since we shortened our array by 1, we should move the iteration value down 1 too.
            debugText('Ok, I folded up a term, so hopefully our student vector still makes sense. It is now: '+studentAns);
        } else if (studentAns[i][0] == '^') {
            //if we have a power, but not a positive integer power, then we have a non-polynomial factor, so we're done.
            console.log('I think I found a non-polynomial term, specifically some root term has a non natural number (or zero) power.');
            return false;
        }
    }
    debugText('After all preprocessing my studentAns vector is:');
    debugText(studentAns);

    // Now re repeat with instructor tree:
    debugText('folding up external exponents of instructorTree so factors do not have exponents')
    for (var i = 0; i < instructorAns.length; ++i) {
        if ((instructorAns[i][0] == '^')*(isPosInt(instructorAns[i][2]))) {
            instructorAns=instructorAns.concat(Array(instructorAns[i][2]).fill(instructorAns[i][1]));
            instructorAns.splice(i,1);// This should theoretically remove the original term now that we've duplicated it.
            i=i-1;// since we shortened our array by 1, we should move the iteration value down 1 too.
            debugText('Ok, I folded up a term, so hopefully our student vector still makes sense. It is now: '+instructorAns);
        } else if (instructorAns[i][0] == '^') {
            //if we have a power, but not a positive integer power, then we have a non-polynomial factor, so we're done.
            return false;
            console.log('Found a non-polynomial term in the instructor answer... huh... Check the code!');
        }
    }
    debugText('After all preprocessing my instructorAns vector is: '+instructorAns);
    
    /*
        ::NOW LETS PROCESS THE STUDENT ANSWER::
    */
    
    var studentDegList=[0]
    for (var i = 0; i < studentAns.length; ++i) {
        if (studentAns[i] == 'x') {
            // If the factor is simply 'x', then it's a degree 1 factor... yay...
            studentDegList.push(1);
            debugText('Found another factors degree, so now studentDegList is: '+studentDegList); 
        } else if (Array.isArray(studentAns[i])) {
            // Otherwise, if it is an array, we have something to go hunting in.
            studentDegList.push(degreeHunt(studentAns,i,0));
            debugText('Found another factors degree, so now studentDegList is: '+studentDegList); 
        }// Note that the only other possibility is it being a number, which we don't care about.
    }
    studentDegList = studentDegList.filter(x => x!==0);// Remove all zeros from the array to avoid stupid padded constant multipliers.
    studentDegList.sort();// Sort the result so that we can later compare it to the instructor version.
    debugText('The final List of Factor Degrees given by the student is: ' + studentDegList);
    
    if (studentDegList.some(elem => elem<0)) {
        console.log('I think one of the student factors is NOT a polynomial. So I am rejecting the answer.');
        return false
        }
    
    
    /*
        ::NOW LETS PROCESS THE INSTRUCTOR ANSWER::
    */
    
    var instructorDegList=[0]
    for (var i = 0; i < instructorAns.length; ++i) {
        if (instructorAns[i] == 'x') {
            // If the factor is simply 'x', then it's a degree 1 factor... yay...
            instructorDegList.push(1);
            debugText('Found another factors degree, so now instructorDegList is: '+instructorDegList); 
        } else if (Array.isArray(instructorAns[i])) {
            // Otherwise, if it is an array, we have something to go hunting in.
            instructorDegList.push(degreeHunt(instructorAns,i,0));
            debugText('Found another factors degree, so now instructorDegList is: '+instructorDegList); 
        }// Note that the only other possibility is it being a number, which we don't care about.
    }
    instructorDegList = instructorDegList.filter(x => x!==0);// Remove all zeros from the array to avoid stupid padded constant multipliers.
    instructorDegList.sort();// Sort the result so that we can later compare it to the instructor version.
    debugText('The final List of Factor Degrees given by the instructor is: ' + instructorDegList);
    
    if (instructorDegList.some(elem => elem<0)) {
        console.log('I think one of the instructor factors is NOT a polynomial. So I am rejecting the answer.');
        return false
        }
    
   /*
        ::NOW WE COMPARE::
   */
    
    if (studentDegList.length!=instructorDegList.length){
        console.log('Ans Rejected: Wrong number of factors.');
        return false
        }
    
    for (var i = 0; i < studentDegList.length; ++i) {
        if (studentDegList[i] !== instructorDegList[i]) {
        console.log('Ans Rejected: At least one factor is the wrong degree.');
        return false
        }
    }

    if (f.equals(g)){
        } else {
        console.log('Ans Rejected: Factors do not expand to original Polynomial.');
        }

    return (f.equals(g))
}
?
%</classXimera>
%    \end{macrocode}

