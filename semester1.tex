%\documentclass[12pt,numbers,noauthor,nooutcomes,wordchoicegiven]{xourse}
\documentclass{xourse}
\input{./preamble.tex}

\addPrintStyle{.}

\pdfOnly{
    \renewcommand{\xmcursusnaam}{{\textsc{Natuurkunde}}}
}

\logo{xmPictures/nomlogo.png}

\begin{document}
%	\setcounter{tocdepth}{2}
    \xmtitle{Eerste semester}{}  
 

    

\part{Inleiding}

\activitychapter{sem1/inleiding.tex}

\part{Vectoren}

\activitychapter{sem1/vct_inleiding.tex}
\activitychapter{sem1/vct_begrip.tex}
\activitychapter{sem1/vct_bewerkingen.tex}
\activitychapter{sem1/vct_oef.tex}



\part{Basisbegrippen van de kinematica}

\activitychapter{sem1/knmtc_bgrppn_intro.tex}
\activitychapter{sem1/knmtc_bgrppn_referentiestelsel.tex}
\activitychapter{sem1/knmtc_bgrppn_positie.tex}
\activitychapter{sem1/knmtc_bgrppn_snelheid.tex}
\activitychapter{sem1/knmtc_bgrppn_versnelling.tex}
\activitychapter{sem1/knmtc_bgrppn_oef.tex}


\part{Eendimensionale bewegingen}

\activitychapter{sem1/knmtc_1dim_intro.tex}
\activitychapter{sem1/knmtc_1dim_begrippen.tex}
\activitychapter{sem1/knmtc_1dim_ERB.tex}
\activitychapter{sem1/knmtc_1dim_EVRB_rechtstreeks.tex}
\activitychapter{sem1/knmtc_1dim_oplossingsstrategie.tex}
\activitychapter{sem1/knmtc_1dim_verticale_worp.tex}



\part{Tweedimensionale bewegingen}


\activitychapter{sem1/knmtc_2dim_intro.tex}
\activitychapter{sem1/knmtc_2dim_onafhankelijkheidsbeginsel.tex}
\activitychapter{sem1/knmtc_2dim_begrippen.tex}
\activitychapter{sem1/knmtc_2dim_eenparige_cirkelbeweging.tex}
\activitychapter{sem1/knmtc_2dim_horizontale_worp.tex}


\part{De beginselen van Newton}

\activitychapter{sem1/nwtn_intro.tex}
\activitychapter{sem1/nwtn_traagheid.tex}
\activitychapter{sem1/nwtn_tweede_beginsel.tex}
\activitychapter{sem1/nwtn_derde_beginsel.tex}
\activitychapter{sem1/nwtn_oef.tex}
% \activitychapter{sem1/nwtn_oplossingsstrategie.tex}
% \activitychapter{sem1/nwtn_historische_uitwijding.tex}


\part{Toepassing wetten van Newton}

\activitychapter{sem1/nwtn_toep_algemeen.tex}
\activitychapter{sem1/nwtn_toep_krachten.tex}
\activitychapter{sem1/nwtn_toep_dynamica_ECB.tex}




% \part{De gravitatiekracht}

% \activitychapter{sem1/grvt_kepler.tex}
% \activitychapter{sem1/grvt_universele_gravitatiekracht.tex}
% \activitychapter{sem1/grvt_satellietbanen.tex}
% \activitychapter{sem1/grvt_gewicht.tex}
% \activitychapter{sem1/grvt_zwaartekracht.tex}

% \part{Arbeid en Energie}

% \activitychapter{sem1/ae_intro.tex}
% \activitychapter{sem1/ae_constante_kracht.tex}
% \activitychapter{sem1/ae_niet_constante_kracht.tex}
% \activitychapter{sem1/ae_theorema.tex}
% \activitychapter{sem1/ae_pot_elastische.tex}
% \activitychapter{sem1/ae_pot_gravitationele.tex}
% \activitychapter{sem1/ae_pot_gravitationele_algemeen.tex}
% \activitychapter{sem1/ae_pot_referentiepunt.tex}
% \activitychapter{sem1/ae_pot_potentiele_energie.tex}
% \activitychapter{sem1/ae_behoud.tex}






\end{document}