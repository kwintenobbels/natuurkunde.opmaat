%\documentclass[12pt,numbers,noauthor,nooutcomes,wordchoicegiven]{xourse}
\documentclass{xourse}

%\addPrintStyle{.}



\pdfOnly{
    \renewcommand{\xmcursusnaam}{{\textsc{Natuurkunde}}}
}

\logo{xmPictures/nomlogo.png}

\handouttrue 
\nonewpagetrue

\begin{document}
%	\setcounter{tocdepth}{2}
    \xmtitle{Dynamica: Toepassingen van de wetten van Newton}{}  
 

\setcounter{part}{3}

\part{Dynamica: De wetten van Newton}

\activitychapter{../dynamica/dynamica_intro.tex}
\activitychapter{../dynamica/newton_eerste_wet.tex}
\activitychapter{../dynamica/newton_tweede_wet.tex}
\activitychapter{../dynamica/newton_derde_wet.tex}
\activitychapter{../dynamica/exercises/oefeningen.tex} 
\activitysection{../dynamica/exercises/denkvragen.tex}
\activitysection{../dynamica/exercises/vraagstukken.tex}

%%% \activitychapter{6dejaar/nwtn_oplossingsstrategie.tex}
%%% \activitychapter{6dejaar/nwtn_historische_uitwijding.tex}


\part{De wetten van Newton toegepast}

\activitychapter{../dynamica/toep_intro.tex}
\activitychapter{../dynamica/toep_algemeen.tex}

% \activitychapter{../dynamica/toep_krachten.tex}
% \activitysection{../dynamica/toep_krachten_normaalkracht.tex}
% \activitysection{../dynamica/toep_krachten_veerkracht.tex}
% \activitysection{../dynamica/toep_krachten_Archimedeskracht.tex}
% \activitysection{../dynamica/toep_krachten_gravitatie.tex}
% \activitysection{../dynamica/toep_krachten_Coulomb.tex}
% \activitysection{../dynamica/toep_krachten_Lorentz.tex}
% \activitysection{../dynamica/toep_krachten_kernkrachten.tex}
% \activitysection{../dynamica/toep_krachten_wrijving.tex}
% \activitysection{../dynamica/toep_krachten_trekkracht.tex}
% \activitysection{../dynamica/toep_krachten_gewicht.tex}

\activitychapter{../dynamica/toep_valversnelling.tex}
\activitychapter{../dynamica/toep_dynamica_ECB.tex}
\activitychapter{../dynamica/toep_satellietbanen.tex}

\activitychapter{../dynamica/exercises/toep_oefeningen.tex}
\activitysection{../dynamica/exercises/toep_vraagstukken.tex}
\activitysection{../dynamica/exercises/toep_ECB.tex}

\end{document}