%\documentclass[12pt,numbers,noauthor,nooutcomes,wordchoicegiven]{xourse}
\documentclass{xourse}

%\addPrintStyle{.}

\pdfOnly{
    \renewcommand{\xmcursusnaam}{{\textsc{Natuurkunde}}}
}

\logo{xmPictures/nomlogo.png}

\begin{document}
%	\setcounter{tocdepth}{2}
    \xmtitle{Kinematica: Vectoren en Basisbegrippen}{}  
 

%\setcounter{part}{1}

\part{Inleiding}
\activitychapter{../kinematica/inleiding.tex}


\part{Vectoren}

\activitychapter{../vectoren/vectoren_begrip.tex}
\activitychapter{../vectoren/vectoren_notatie.tex}
\activitychapter{../vectoren/vectoren_bewerkingen.tex}
\activitysection{../vectoren/exercises/vectoren_oef_reeks1.tex}
\activitysection{../vectoren/exercises/vectoren_oef_reeks2.tex}
\activitysection{../vectoren/exercises/vectoren_oef_reeks3.tex}

    
\part{Basisbegrippen van de kinematica}

\activitychapter{../kinematica/basis_intro.tex}
\activitychapter{../kinematica/basis_referentiestelsel.tex}
\activitychapter{../kinematica/basis_positie.tex}
\activitychapter{../kinematica/basis_snelheid.tex}
\activitychapter{../kinematica/basis_versnelling.tex}
\activitychapter{../kinematica/basis_oef.tex}



\end{document}