\documentclass{xourse}

\begin{document}
%	\setcounter{tocdepth}{2}
    \xmtitle{Semester  1}{}  
 

\setcounter{part}{-1}

\part{Inleiding}
\activitychapter{../kinematica/inleiding.tex}


\part{Vectoren}

\activitychapter{../vectoren/vectoren_begrip.tex}
\activitychapter{../vectoren/vectoren_notatie.tex}
\activitychapter{../vectoren/vectoren_bewerkingen.tex}
\activitysection{../vectoren/exercises/vectoren_oef_reeks1.tex}
\activitysection{../vectoren/exercises/vectoren_oef_reeks2.tex}
\activitysection{../vectoren/exercises/vectoren_oef_reeks3.tex}

    
\part{Basisbegrippen van de kinematica}

\activitychapter{../kinematica/basis_intro.tex}
\activitychapter{../kinematica/basis_referentiestelsel.tex}
\activitychapter{../kinematica/basis_positie.tex}
\activitychapter{../kinematica/basis_snelheid.tex}
\activitychapter{../kinematica/basis_versnelling.tex}
\activitychapter{../kinematica/exercises/basis_oef.tex}
    
    
\part{Eendimensionale bewegingen}

\activitychapter{../kinematica/1dim_intro.tex}
\activitychapter{../kinematica/1dim_ERB.tex}
\activitychapter{../kinematica/1dim_EVRB_calculus.tex}
\activitychapter{../kinematica/1dim_oplossingsstrategie.tex}
\activitychapter{../kinematica/1dim_verticale_worp.tex}
\activitychapter{../kinematica/exercises/1dim_oef.tex}
\activitysection{../kinematica/exercises/1dim_oef_denkvragen.tex}
\activitysection{../kinematica/exercises/1dim_oef_vraagstukken.tex}
\activitysection{../kinematica/exercises/1dim_oef_vraagstukken_valbeweging.tex}


\part{Tweedimensionale bewegingen}

\activitychapter{../kinematica/2dim_intro.tex}
\activitychapter{../kinematica/2dim_onafhankelijkheidsbeginsel.tex}
\activitychapter{../kinematica/2dim_eenparige_cirkelbeweging.tex}   
\activitychapter{../kinematica/2dim_horizontale_worp.tex}


\part{Dynamica: De wetten van Newton}

\activitychapter{../dynamica/dynamica_intro.tex}
\activitychapter{../dynamica/newton_eerste_wet.tex}
\activitychapter{../dynamica/newton_tweede_wet.tex}
\activitychapter{../dynamica/newton_derde_wet.tex}
\activitychapter{../dynamica/exercises/oefeningen.tex} 
\activitysection{../dynamica/exercises/denkvragen.tex}
\activitysection{../dynamica/exercises/vraagstukken.tex}
% \activitychapter{6dejaar/nwtn_oplossingsstrategie.tex}
% \activitychapter{6dejaar/nwtn_historische_uitwijding.tex}


\part{Toepassing wetten van Newton}

\activitychapter{../dynamica/toep_intro.tex}
\activitychapter{../dynamica/toep_algemeen.tex}
\activitychapter{../dynamica/toep_krachten.tex}
\activitychapter{../dynamica/toep_dynamica_ECB.tex}


\part{De gravitatiekracht}

\activitychapter{../golven/gravitatie_kepler.tex}
\activitychapter{../golven/gravitatie_universele_gravitatiekracht.tex}
\activitychapter{../golven/gravitatie_satellietbanen.tex}
\activitychapter{../golven/gravitatie_gewicht.tex}
\activitychapter{../golven/gravitatie_zwaartekracht.tex}

\part{Arbeid en Energie}

\activitychapter{../arbeidenergie/intro.tex}
\activitychapter{../arbeidenergie/constante_kracht.tex}
\activitychapter{../arbeidenergie/niet_constante_kracht.tex}
\activitychapter{../arbeidenergie/theorema.tex}
\activitychapter{../arbeidenergie/pot_elastische.tex}
\activitychapter{../arbeidenergie/pot_gravitationele.tex}
\activitychapter{../arbeidenergie/pot_gravitationele_algemeen.tex}
\activitychapter{../arbeidenergie/pot_referentiepunt.tex}
\activitychapter{../arbeidenergie/pot_potentiele_energie.tex}
\activitychapter{../arbeidenergie/behoud.tex}



\end{document}