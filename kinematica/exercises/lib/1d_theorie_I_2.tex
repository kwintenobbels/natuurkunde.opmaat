
\begin{exercise}
% !TEX root = ../main.tex


\pt{4}Kan de bewegingsrichting van een voorwerp omkeren terwijl de versnelling gelijk blijft? Zo ja, geef dan een voorbeeld. Zo nee, leg uit waarom dat niet kan.

\begin{oplossing}
Ja, dat kan. Als je een bal opwerpt zal op het hoogste punt de bewegingszin omdraaien terwijl de versnelling gelijk blijft. We kunnen immers een verticale worp modelleren als een EVB. Kiezen we de referentieas om de beweging te beschrijven omhoog, dan is de snelheid van de bal positief bij het naar boven bewegen en negatief wanneer hij naar beneden komt, terwijl de verandering van de snelheid in de tijd (de versnelling) systematisch gelijk is aan de negatieve valversnelling.


ChatGPT:

Ja, de bewegingsrichting van een voorwerp kan omkeren terwijl de versnelling gelijk blijft. Dit gebeurt wanneer de versnelling tegengesteld is aan de bewegingsrichting van het voorwerp. In zo'n geval zal het voorwerp eerst vertragen totdat het stopt, en daarna versnellen in de tegenovergestelde richting.

\textbf{Voorbeeld:}

Stel je voor dat je een bal recht omhoog gooit. De zwaartekracht zorgt voor een constante versnelling naar beneden (ongeveer \SI{9,8}{m/s^2}). Terwijl de bal omhoog beweegt, is de snelheid van de bal naar boven gericht, maar de versnelling door de zwaartekracht blijft naar beneden gericht.

\begin{itemize}
\item Op het hoogste punt komt de bal kort tot stilstand (de snelheid is nul), maar de versnelling naar beneden verandert niet.
\item Na dat punt keert de bal om en begint naar beneden te vallen, met een toenemende snelheid in de richting van de versnelling (naar beneden).
\end{itemize}

In dit scenario keert de bewegingsrichting van de bal om van omhoog naar omlaag, terwijl de versnelling constant naar beneden blijft gericht.


\end{oplossing}


\end{exercise}
