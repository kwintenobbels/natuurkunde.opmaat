
\begin{exercise}
% !TEX root = ../main.tex


[\SI{80}{\percent} (b)] Een vliegtuig start vanuit rust en versnelt met een constante versnelling langs de grond alvorens op te stijgen. Het legt \SI{600}{m} af in \SI{12}{s}. Bepaal de versnelling, de snelheid na \SI{12}{s} en de afstand afgelegd gedurende de twaalfde seconde.

\begin{oplossing}
	$a=\frac{2x}{t^2}=\SI{8,33}{m/s^2}$,
	$v=\frac{2x}{t}=\SI{100}{m/s}$,
	$x(t=12)-x(t=11)=\frac{1}{2}a(t_{12}^2-t_{11}^2)=\SI{95,8}{m}$
\end{oplossing}


\end{exercise}
