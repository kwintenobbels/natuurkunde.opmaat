\documentclass{ximera}

%\addPrintStyle{..}

\begin{document}
	\author{Bart Lambregs}
	\xmtitle{Eenparige rechtlijnige beweging}{}
    \xmsource\xmuitleg




Een eenvoudige beweging om te bestuderen is de \emph{eenparig rechtlijnige beweging} (afgekort ERB). De snelheid van de beweging is eenparig of gelijkmatig verdeeld wat betekent dat de snelheid steeds gelijk blijft. M.a.w. is de snelheid constant en dus de versnelling gelijk aan nul. Aangezien de snelheid niet verandert is de ogenblikkelijke snelheid gelijk aan de gemiddelde snelheid. Met deze observatie kan je eenvoudig een functie opstellen voor de plaatsfunctie van deze beweging:

\[
\begin{array}{rcl}

v=\frac{\Delta x}{\Delta t} & \Leftrightarrow & \Delta x=v\Delta t \\
&\Leftrightarrow & x-x_0=v(t-t_0) \\
&\Leftrightarrow & x=x_0+v(t-t_0)

\end{array}
\]

\begin{theorem}
De plaatsfunctie $x(t)$ van een ERB met snelheid $v$ is gegeven door:

\[
x(t)=x_0+v(t-t_0)
\]
waarbij $x_0=x(t_0)$ de coördinaat op het tijdstip $t_0$ is. Als $t_0=0$ dan vereenvoudigt de plaatsfunctie tot
\begin{eqnarray}%
x(t)&=&x_0+vt
\end{eqnarray}

\end{theorem}

De snelheid $v$ is gelijk aan de beginsnelheid $v_0=v(t_0)$ omdat in een ERB de snelheid constant is. De snelheidsfunctie is $v(t)=v$ en de versnellingsfunctie is $a(t)=0$.


\begin{image}
	
    \begin{tikzpicture}
		% ====== position vs Time (v > 0) ======
		\begin{scope}[shift={(0,4)}]
			% axes
			\draw[->] (0,0) -- (4,0) node[right] {$t$};
			\draw[->] (0,-1.5) -- (0,2.5) node[above] {$x$};
			% lines
			\draw[dashed] (0,1) -- (3.5,2.5);
			\draw[thick]  (0,0) -- (3.5,1.5);
			\draw[dashed] (0,-1) -- (3.5,0.5);
			% labels
			\node at (0.5,1.3) {$x_0>0$};
			\node at (0.5,0.0) {$x_0=0$};
			\node at (0.5,-1.0) {$x_0<0$};
		\end{scope}
			
		% ====== position vs Time (v < 0) ======
		\begin{scope}[shift={(6,5)}]
				% axes
				\draw[->] (0,0) -- (4,0) node[right] {$t$};
				\draw[->] (0,-2.5) -- (0,1.5) node[above] {$x$};
				% lines
				\draw[dashed] (0,1) -- (3.5,-0.5);
				\draw[thick]  (0,0) -- (3.5,-1.5);
				\draw[dashed] (0,-1) -- (3.5,-2.5);
				% labels
				\node at (0.5,1.0) {$x_0>0$};
				\node at (0.5,0.0) {$x_0=0$};
				\node at (0.5,-1.0) {$x_0<0$};
		\end{scope}
	
		% ====== velocity vs Time (v > 0) ======
		\begin{scope}[shift={(0,0)}]
			% axes
			\draw[->] (0,0) -- (4,0) node[right] {$t$};
			\draw[->] (0,-1) -- (0,1.2) node[above] {$v$};
			% constant acceleration
			\draw[thick] (0,0.6) -- (3.5,0.6);
			% label
			\node at (2,0.8) {$v>0$};
		\end{scope}
		
		% ====== velocity vs Time (v < 0) ======
		\begin{scope}[shift={(6,0)}]
			% axes
			\draw[->] (0,0) -- (4,0) node[right] {$t$};
			\draw[->] (0,-1.2) -- (0,1) node[above] {$v$};
			% constant acceleration
			\draw[thick] (0,-0.6) -- (3.5,-0.6);
			% label
			\node at (2,-0.8) {$v<0$};
		\end{scope}
		
	\end{tikzpicture}

\end{image}
\captionof{figure}{Grafieken van een ERB}




% \begin{image}
% \includegraphics[width=.7\textwidth]{ERB_grafieken}
% \end{image}
% \captionof{figure}{Grafieken van een ERB}



\end{document}
