\documentclass{ximera}

%\addPrintStyle{..}


\begin{document}
	\author{Bart Lambregs}
	\xmtitle{De sterke kernkracht en zwakke wisselwerking (veldkrachten)}{}
    \xmsource\xmuitleg

De sterke kernkrachten tussen quarks zijn heel relevant voor de stabiliteit en samenhang van nucleonen in atoomkernen. 
Ze hebben echter geen relevantie in de stabiele macroscopische wereld. 
Enkel bij zeer kleine afstand tussen nucleonen (orde femtometer = 10\textsuperscript{-15} m) is deze kernkracht niet te verwaarlozen, binnen één atoomkern dus. 
De sterke kernkracht neemt met de afstand veel feller af in vergelijking met andere veldkrachten.

Bij de zwakke wisselwerking (of kernkracht) tussen fundamentele deeltjes is er een andere uitwerking dan gebruikelijk (niet statisch of dynamisch). 
In dit geval veranderen de deeltjes van aard. 
Ook dit fenomeen is in onze stabiele macroscopische wereld irrelevant omdat het net enkel gebeurt bij onstabiele deeltjes of atoomkernen.


\end{document}
