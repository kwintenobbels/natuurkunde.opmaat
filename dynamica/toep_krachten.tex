\documentclass{ximera}

%\addPrintStyle{..}


\begin{document}
	\author{Bart Lambregs}
	\xmtitle{Verschillende soorten krachten}{}
    \xmsource\xmuitleg

% hier blijft enkel de eerste sectie met uitleg over; dit is het origineelDe gravitatiekracht of zwaartekracht (veldkracht)

\section*{Herhaling en uitbreiding soorten krachten}

Een kracht is een interactie tussen twee voorwerpen. 
Wanneer een eerste voorwerp A een kracht uitoefent op een ander voorwerp B, dan is een bepaalde uitwerking het mogelijk gevolg. 
Deze uitwerking kan \textbf{statisch (elastisch en/of plastisch)} zijn (vervorming), maar ook \textbf{dynamisch} (verandering van bewegingstoestand, in één woord: versnelling).

De meeste krachten zijn \textbf{contactkrachten}, dan is er rechtstreeks contact tussen de twee voorwerpen. 
Bij \textbf{veldkrachten} is geen rechtstreeks contact nodig en is er sprake van een krachtveld. 
De veldkrachten zijn: gravitatiekracht (of zwaartekracht) tussen massa's, elektromagnetische kracht tussen ladingen (Coulomb- en Lorentzkracht), sterke kernkrachten tussen quarks en de zwakke wisselwerking die in theorie mogelijk is tussen alle fundamentele deeltjes. 
Alle andere krachten zijn bijgevolg contactkrachten.

De grootheid kracht is een vector, ze heeft een richting, zin en grootte. 
De grootte van de kracht heeft als eenheid de Newton \SI{}{\newton}. 
Bij een contactkracht grijpt de kracht aan op de plaats(en) waar er contact is. 
Bij een veldkracht grijpen er in feite vele kleine veldkrachten aan op de betrokken deeltjes van gans het voorwerp. 
Voor de eenvoud vatten we die samen tot één kracht. 
Tenslotte laten we voor het gebruiksgemak meestal \textbf{alle krachten op een voorwerp aangrijpen in het zwaartepunt van het voorwerp}.


Hoekan je achterhalen welke krachten op een voorwerp aangrijpen? 
Stel jezelf de vraag: ``Wat duwt of trekt eraan?'' 
Bepaal de contactkrachten door na te gaan met welke voorwerpen er contact is en overloop dan de eventuele veldkrachten.

\end{document}