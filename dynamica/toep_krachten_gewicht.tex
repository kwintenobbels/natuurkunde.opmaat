\documentclass{ximera}

%\addPrintStyle{..}


\begin{document}
	\author{Bart Lambregs}
	\xmtitle{Het gewicht (contactkracht)}{}
    \xmsource\xmuitleg


= kracht \(\overrightarrow{G}\) die een voorwerp uitoefent op zijn steun
(meestal tgv de zwaartekracht)

Let op! Het gewicht van een voorwerp grijpt niet aan op het voorwerp zelf, maar op hetgeen waar het voorwerp contact mee maakt waardoor het ondersteund wordt (bijvoorbeeld: oppervlak, veer, vloeistof, \ldots)! Het gewicht is vaak de reactiekracht van de steunkracht op het voorwerp (\({\overrightarrow{F}}_{n},{\overrightarrow{F}}_{v},{\overrightarrow{F}}_{A},{\overrightarrow{F}}_{t},\ldots)\) en is daarom volgens de derde wet van Newton in grootte gelijk aan die steunkracht.

Er is geen vaste formule voor gewicht, maar de grootte moet je bepalen met de wetten van Newton (vaak combinatie 3\textsuperscript{de} wet met 1\textsuperscript{ste} of 2\textsuperscript{de}). 
In eenvoudige situaties draait het vaak uit dat \(G = F_{z}\) , maar dit mag je niet algemeen aannemen! De wetten van Newton brengen altijd uitsluitsel!

\begin{image}
	\begin{tikzpicture}
		\def\W{2.0} % ground width
		\def\D{0.2} % ground depth
		\def\h{0.6} % mass height
		\def\w{0.8} % mass width
		
		% Ground
		\draw[ground] (-\W/2,0) rectangle++ (\W,-\D);
		\draw (-\W/2,0) --++ (\W,0);
		
		% Mass
		\draw[mass] (-\w/2,0) rectangle++ (\w,\h) node[midway] {$m$};
		
		% Forces
		% Normal force (up) on mass
		\draw[force] (-0.2*\w,0) --++ (0, 1.3*\h) node[left] {$\vbF_n$};
		% Gravity (down) on mass
		\draw[force] (-0.2*\w,0.5*\h) --++ (0, -1.3*\h) node[right, pos=0.8] {$\vbF_z$};
		
		% Weight (down) on ground
		\draw[force] (0.2*\w,0) --++ (0, -1.3*\h) node[right] {$\vb{G}$};
	\end{tikzpicture}
\end{image}
\captionof{figure}{Het gewicht $\vb{G}$ grijpt aan op de steun, de normaalkracht $\vbF_n$ en zwaartekracht $\vbF_z$ op het voorwerp.}




% MATERIAAL BART; COPY VAN BESTAND GRAVITIE_GEWICHT.TEX 

%%%\section{Gewicht}

%- insteek via gewichtloos zijn in de ruimte

%- gewicht is hoeveel het weegt, de impact. Weegschaal?... Afwegen? Belang van iets, gewichtig? > voetnoot

%\begin{center}
\framebox{
\begin{minipage}[t]{\textwidth}
Het gewicht van een lichaam is de grootte van de kracht die door het lichaam op haar steun wordt uitgeoefend.
\end{minipage}}
%\end{center}

Let op het feit dat het om de \textit{grootte} van een kracht gaat en dat gewicht dus wordt uitgedrukt in newton. In de omgangstaal zijn we hierin niet correct. Men zegt zoveel kilogram te wegen terwijl kilogram de eenheid van massa is.

%- voorbeeld met een lift.
	




\end{document}
