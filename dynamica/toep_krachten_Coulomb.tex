\documentclass{ximera}

%\addPrintStyle{..}


\begin{document}
	\author{Bart Lambregs}
	\xmtitle{De elektrische kracht of Coulombkracht (veldkracht)}{}
    \xmsource\xmuitleg



= aantrekking- of afstotingskracht tussen elektrische ladingen

De \textbf{grootte} van de Coulombkracht \({\overrightarrow{F}}_{C}\)
(of elektrische kracht \({\overrightarrow{F}}_{e}\)) tussen twee
puntladingen \(q_{1}\) en \(q_{2}\) met een afstand \(r\ \)tussenbeiden
is gelijk is aan:

\[\vec{F}_{C}=\vec{F}_{e}=k\,\frac{|q_{1}|\,|q_{2}|}{r^{2}}\ \text{met}\ k=\SI{8,99e9}{\newton\meter\squared\per\coulomb\squared}\approx\SI{9,0e9}{\newton\meter\squared\per\coulomb\squared}\]

\begin{image}
  \includegraphics{toep_coulomb2}
\end{image}

In realistische situaties bij ladingen met afmetingen is r de afstand tussen de middelpunten van die ladingen.

In een elektrisch veld \(\vec{E}\) is er een eenvoudigere formule:
\(\ \vec{F}_{C}=\vec{F}_{e}=q\,\vec{E}\ \)
De grootte van \(\overrightarrow{E}\) kan in verschillende situaties bepaald worden naargelang het type veld. 
Meer details: zie leerstof 5\textsuperscript{de} jaar.



\end{document}
