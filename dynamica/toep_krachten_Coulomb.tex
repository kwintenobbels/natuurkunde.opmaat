\documentclass{ximera}

%\addPrintStyle{..}


\begin{document}
	\author{Bart Lambregs}
	\xmtitle{De elektrische kracht of Coulombkracht (veldkracht)}{}
    \xmsource\xmuitleg



= aantrekking- of afstotingskracht tussen elektrische ladingen

De \textbf{grootte} van de Coulombkracht \({\overrightarrow{F}}_{C}\)
(of elektrische kracht \({\overrightarrow{F}}_{e}\)) tussen twee
puntladingen \(q_{1}\) en \(q_{2}\) met een afstand \(r\ \)tussenbeiden
is gelijk is aan:

\[{\mathbf{F}_{\mathbf{C}}\mathbf{=}\mathbf{F}}_{\mathbf{e}}\mathbf{= k \bullet}\frac{\left| \mathbf{q}_{\mathbf{1}} \right|\mathbf{\bullet}\left| \mathbf{q}_{\mathbf{2}} \right|}{\mathbf{r}^{\mathbf{2}}}\ met\ k = 8,99 \bullet 10^{9}\frac{N \bullet m²}{C²} \approx 9,0 \bullet 10^{9}\frac{N \bullet m²}{C²}\]

\begin{image}
  \includegraphics{toep_coulomb2}
\end{image}

In realistische situaties bij ladingen met afmetingen is r de afstand tussen de middelpunten van die ladingen.

In een elektrisch veld \(\overrightarrow{E}\) is er een eenvoudigere formule:
\(\ {\overrightarrow{\mathbf{F}}}_{\mathbf{C}}\mathbf{=}{\overrightarrow{\mathbf{F}}}_{\mathbf{e}}\mathbf{= q \bullet}\overrightarrow{\mathbf{E}}\mathbf{\ }\)
De grootte van \(\overrightarrow{E}\) kan in verschillende situaties bepaald worden naargelang het type veld. 
Meer details: zie leerstof 5\textsuperscript{de} jaar.



\end{document}
