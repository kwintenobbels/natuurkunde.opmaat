
\begin{exercise}

% !TEX root = ../main.tex




 De plaatsvector van een deeltje wordt (voor $t\geq0$) gegeven door
\begin{equation*}
	\vec{r}=-t\vec{e}_x+(t-1)^2\vec{e}_y
\end{equation*}
\begin{enumerate}
\item Geef de baanvergelijking.

\begin{oplossing}
	$y(x)=(x+1)^2$
\end{oplossing}

\item Wanneer is \'e\'en van de snelheidscomponenten nul?

\begin{oplossing}
	De snelheidscomponent volgens de $x$-as is nooit nul. Die volgens de $y$-as is nul wanneer: $2(t-1)=0$ oftewel wanneer $t=1$.
\end{oplossing}

\item Hoe groot is dan de snelheid?

\begin{oplossing}
	$v(1)=\sqrt{v_x^2(1)+v_y^2(1)}=\sqrt{(-1)^2+0^2}=1\,\text{s}$
\end{oplossing}

\item Waar is het deeltje dan?

\begin{oplossing}
	$\vec{r}(1)=-\vec{e}_x$
\begin{center}
\begin{tikzpicture}[line cap=round,line join=round,>=triangle 45,x=1cm,y=1cm]
\begin{axis}[
x=1cm,y=1cm,
axis lines=middle,
xmin=-3.2773087503908354,
xmax=1.9347255210454852,
ymin=-0.5305920286290815,
ymax=4.946070914425411,
xtick={-3,-2,...,1},
ytick={0,1,...,4},]
\clip(-3.2773087503908354,-0.5305920286290815) rectangle (1.9347255210454852,4.946070914425411);
\draw [samples=50,rotate around={0:(-1,0)},xshift=-1cm,yshift=0cm,line width=2pt,domain=-4:4)] plot (\x,{(\x)^2/2/0.5});
\end{axis}
\end{tikzpicture}
\end{center}
\end{oplossing}

\item Geef de versnellingsvector.

\begin{oplossing}
	$\vec{a}=2\vec{e}_y$, ($\vec{v}=-\vec{e}_x+2(t-1)\vec{e}_y$)
\end{oplossing}

\item Raakt de versnellingsvector aan de baan? Licht toe.

\begin{oplossing}
	Nee, de versnellingsvector is niet rakend aan de baan. Hij is altijd verticaal ge\"ori\"enteerd terwijl de afgeleide van de baanvergelijking ($y'=2(x+1)$) overal bestaat en dus nergens een verticale helling heeft.
\end{oplossing}

\end{enumerate}

\end{exercise}
