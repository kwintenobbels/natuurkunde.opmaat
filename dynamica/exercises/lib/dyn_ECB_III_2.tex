\begin{exercise}

Een cirkelvormige renbaan is onder een helling van \SI{30}{\degree} gebouwd. De straal van de cirkel is \SI{50}{m}. Met welke snelheid moet een auto rijden om in de baan te blijven? Veronderstel dat de baan spekglad is.

\begin{oplossing}
De krachten die op de auto aangrijpen, zijn de zwaartekracht en de normaalkracht. In de $y$-richting is er geen versnelling omdat de auto in een horizontaal vlak beweegt. We kiezen dan ook een assenstelsel met de $y$-as verticaal georiënteerd. De $x$-as kunnen we in de richting van het centrum van de cirkel nemen. 
\begin{image}
    \centering\includegraphics[width=0.6\textwidth]{renbaan}
\end{image}
De $y$-component van de normaalkracht moet dus even groot zijn als de zwaartekracht. De $x$-component van de normaalkracht is dan ook de resulterende kracht en levert de middelpuntzoekende kracht. De $x$-component van de normaalkracht kunnen we m.b.v. de hoek en de zwaartekracht schrijven.

De $x$-component van de normaalkracht:
\begin{eqnarray*}
    \tan{\varphi}&=&\frac{F_{n,x}}{F_{n,y}}\\
    &\Updownarrow&\\
    F_{n,x}&=&F_{n,y}\tan{\varphi}\\
    &=&mg\tan{\varphi}
\end{eqnarray*}
We passen de tweede wet van Newton toe:
\begin{eqnarray*}
    \vec{F}&=&m\vec{a}\\
    &\Downarrow&\\
    F_{n,x}&=&ma\\
    mg\tan{\varphi}&=&\frac{mv^2}{r}\\
    %&\Downarrow&\\
    v&=&\sqrt{gr\tan{\varphi}}
\end{eqnarray*}
De gegevens invullen levert een snelheid van \SI{17}{m/s}.
\end{oplossing}
\end{exercise}
