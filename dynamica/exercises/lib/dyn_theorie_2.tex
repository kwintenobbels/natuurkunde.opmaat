
\begin{exercise}

% !TEX root = ../main.tex


\pt{3}We hebben een eenparig veranderlijke rechtlijnige beweging (EVRB) in eerste instantie binnen de kinematica behandeld. Binnen de dynamica kan je vervolgens de voorwaarden voor zo'n beweging in termen van krachten geven. 

\begin{enumerate}
\item Doe dat (wat moet er dus gelden voor de krachten om een EVRB te krijgen?), 
\item licht kort toe,
\item en geef een voorbeeld.
\end{enumerate}

\begin{oplossing}
Zie in het handboek de inleiding p. 7, de definitie van een EVRB op p. 23, de oorzaak van een versnelling op p. 29 (en eventueel ook p. 50) en de tweede wet van Newton onderaan p. 54.
\end{oplossing}

\end{exercise}
