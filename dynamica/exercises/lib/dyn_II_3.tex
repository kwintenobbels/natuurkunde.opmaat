\begin{exercise}
Een blok van \SI{4,0}{kg} heeft een beginsnelheid van \SI{8,0}{m/s} aan de voet van een helling van \SI{30,0}{\degree}. De wrijvingskracht die de beweging afremt is \SI{15}{N} groot.
\begin{enumerate}
    \item Teken en benoem de krachten die op het blok aangrijpen.
    \item Welke afstand zal het blok afleggen eer het tot rust komt?
    \item Zal het daarna terug naar beneden glijden?
    \item Hoe groot is de wrijvingsfactor?
\end{enumerate}
\begin{oplossing}
    De krachtendiagrammen/het systeem vrijgemaakt:
    \begin{image}
        \includegraphics[width=0.49\textwidth]{blok_helling_2}
        \includegraphics[width=0.49\textwidth]{blok_helling_2componenten}
    \end{image}

    $a=-\frac{F_w+mg\sin\varphi}{m}=\SI{-8,66}{m/s^2}$
    \newline
    $x=-\frac{v_0^2}{2a}=\frac{mv_0^2}{2(F_w+mg\sin\varphi)}=\SI{3,70}{m}$
    \newline
    $F_{zx}>F_w$ zodat het blok terug naar beneden komt. 
    \newline
    $\mu=\frac{F_w}{mg\cos\varphi}=0,44$
\end{oplossing}
\end{exercise}