\begin{exercise}

 Een hockeyschijf of puck krijgt op een bevroren vijver een horizontale slag waardoor hij met een snelheid van \SI{20,5}{m/s} vertrekt. Na hoeveel meter komt de puck tot rust als de wrijvingsfactor tussen de puck en het ijs \SI{0,18}{} bedraagt?

\begin{oplossing}
    De wrijvingskracht is de enige kracht volgens de horizontale richting en vormt dus de resulterende kracht. We kiezen de referentieas horizontaal en met de beginsnelheid mee.
    \begin{image}
        \includegraphics[width=0.7\textwidth]{hockeypuck}
    \end{image}
    Omdat de wrijvingskracht tegengesteld is aan de $x$-as, geldt $F_{w,x}=-F_w$. Met $F_w=\mu F_n$ vinden we voor de versnelling:
    \begin{eqnarray}
    -F_w &=& ma\nonumber\\
    &\Downarrow& \nonumber\\
    -\mu mg &=&ma\nonumber\\
    &\Updownarrow& \nonumber\\
    a &=& -\mu g\label{a}
    \end{eqnarray}
    De tijd uit de vergelijkingen voor een EVRB elimineren, waarbij de eindsnelheid $v$ en de beginpositie $x_0$ nul zijn, levert:
    \begin{eqnarray}
    v&=&v_0+at\nonumber\\
    x&=&x_0+v_0t+\frac{1}{2}at^2\nonumber\\
    &\Downarrow&\nonumber\\
    0&=&v_0+at\Leftrightarrow t=-\frac{v_0}{a}\nonumber\\
    x&=&v_0\left(-\frac{v_0}{a}\right)+\frac{1}{2}a\left(-\frac{v_0}{a}\right)^2\nonumber\\
    &\Downarrow&\nonumber\\
    a&=&-\frac{v_0^2}{2x}\label{-v_0^2/2x}
    \end{eqnarray}
    Vergelijking (\ref{-v_0^2/2x}) samen met (\ref{a}) levert dan:
    \begin{eqnarray}
    \mu &=& \frac{v_0^2}{2gx}\label{remvgl}\\
    &&\nonumber\\
    &=& 0,178\nonumber
    \end{eqnarray}
\end{oplossing}

\end{exercise}
