
\begin{exercise}

 Twee ploegen zijn aan het touwtrekken. Volgens het derde
beginsel van Newton oefenen de twee ploegen steeds even grote maar tegengestelde krachten op elkaar uit. Hoe is het dan mogelijk dat er toch een winnende ploeg is?
\begin{oplossing}
\newline
Er is een winnende ploeg mogelijk omdat het samenstellen van krachten op elke ploeg afzonderlijk gebeurt. Het zijn niet de actie- en reactiekracht (derde wet van Newton) die worden samengesteld. 

De winnende ploeg slaagt erin zich beter af te zetten dan de verliezende ploeg. Dat wil zeggen dat de reactiekracht van de kracht die ze op de grond uitoefenen (de weerstandskracht dus) groter is dan de kracht tussen de twee ploegen. De resulterende kracht van de weerstandskracht en de spankracht op de ploeg, zorgt volgens de tweede wet van Newton voor een versnelling; de ploeg komt in beweging. Voor de verliezende groep is dat net omgekeerd.  
\end{oplossing}

\end{exercise}
