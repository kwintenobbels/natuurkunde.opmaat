\begin{exercise}

Een speelgoedwagentje beweegt in een horizontale cirkel met straal $2l$ en heeft een tijd $T$ nodig om een volledige cirkel te beschrijven. Dit kan omdat aan het wagentje een veer vastgemaakt is. De lengte van de veer in niet uitgerekte toestand is $l$. Het wagentje versnelt waarbij de straal van de beschreven cirkel gelijk wordt aan $3l$.

De tijd die het wagentje nu nodig heeft om een volledige cirkel te beschrijven is dan gelijk aan:
\begin{multipleChoice}
    \choice{$T$}
    \choice{$\frac{3}{4}T$}
    \choice[correct]{$\sqrt{\frac{3}{4}}T$}
    \choice{$\sqrt{\frac{4}{3}}T$}
\end{multipleChoice}

\begin{oplossing}
    De veerkracht zorgt voor de middelpuntzoekende kracht. We leiden hieruit een uitdrukking af voor de periode:
    \begin{eqnarray*}
        F&=&ma\\
        &\Downarrow&\\
        k\Delta l &=& mr\omega^2\\
        &=&  mr\left(\frac{2\pi}{T}\right)^2\\
        &\Updownarrow&\\
        T^2&=&\frac{4\pi^2mr}{k\Delta l}
    \end{eqnarray*}
    De verhouding van het kwadraat van de periodes wordt dan:
    \begin{eqnarray*}
        \frac{T'^2}{T^2}&=&\frac{4\pi^2mr'}{k\Delta l'}\cdot\frac{k\Delta l}{4\pi^2mr}\\
        &&\\
        &=&\frac{3l(2l-l)}{(3l-l)2l}\\
        &&\\
        &=&\frac{3}{4}\\
        &\Downarrow&\\
        T'&=&\sqrt{\frac{3}{4}}\,T
    \end{eqnarray*}
\end{oplossing}
\end{exercise}

% Bron: ...?
