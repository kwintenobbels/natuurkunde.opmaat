\begin{exercise}

Op een draaitafel draait met een constante hoeksnelheid een grammofoonplaat. Twee muntstukken A en B zijn op zo'n plaats van het middelpunt van de draaitafel geplaatst dat zij nog net niet wegschuiven. Voor muntstuk A bedraagt de afstand tot de rotatieas dan \SI{6}{cm} en voor B is het dan \SI{12}{cm}. $m_a$ en $m_b$ zijn de massa's van respectievelijk de muntstukken A en B. $\mu_a$ en $\mu_b$ zijn de wrijvingsfactoren tussen de muntstukken en de grammofoonplaat.

Welke gevolgtrekking m.b.t. de massa's en de wrijvingsfactoren is juist?

\begin{multipleChoice}
    \choice[correct]{$\mu_a=\frac{\mu_b}{2}$}
    \choice{$ m_a=2m_b$}
    \choice{$ m_a=\frac{m_b}{2}$}
    \choice{$\frac{m_a}{\mu_a}=2\frac{m_b}{\mu_b}$}
\end{multipleChoice}

\begin{oplossing}
    De wrijvingskracht tussen de muntjes en de draaitafel moet voor de middelpuntzoekende kracht op de muntjes zorgen. Als de muntjes nog nét niet wegschuiven, mogen we de formule $F_w=\mu F_n$ voor de wrijvingskracht gebruiken. Dit levert, met $F_n=F_z=mg$:
    \begin{eqnarray*}
        F&=&ma\\
        &\Downarrow&\\
        \mu mg&=&mr\omega^2\\
        &\Downarrow&\\
        \frac{\mu_a}{\mu_b}&=&\frac{r_a\omega^2}{g}\cdot\frac{g}{r_b\omega^2}\\
        &=&\frac{r_a}{r_b}\\
        &=&\frac{1}{2}
    \end{eqnarray*}
\end{oplossing}
\end{exercise}

% Bron: ...?
