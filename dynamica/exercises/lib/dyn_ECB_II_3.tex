\begin{exercise}

Een bestuurder van een auto met een massa van \SI{1000}{kg} rijdt aan een snelheid in grootte gelijk aan \SI{10}{m/s}. Hij probeert een horizontale bocht, met een straal van \SI{100}{m} te nemen. De maximale wrijvingskracht tussen de banden en de baan is \SI{900}{N}. Kan de auto deze bocht nemen of zal hij beginnen slippen?

\begin{oplossing}
    De auto zal slippen in de bocht. Omdat we de snelheid en de straal kennen, kunnen we de versnelling van de gewenste cirkelbeweging berekenen. Met de tweede wet van Newton vinden we de (middelpuntzoekende) kracht nodig om deze versnelling te kunnen veroorzaken:
    \begin{eqnarray*}
        F&=&ma\\
        &=&\frac{mv^2}{r}\\
        &=&\SI{1000}{N}
    \end{eqnarray*}
    Dit is meer dan wat de grond maximaal op de wielen kan uitoefenen. De auto zal dus beginnen slippen.

    Realiseer je dat de wrijvingskracht door de grond op de auto wordt uitgeoefend en de resulterende kracht vormt. Het is dan ook de middelpuntzoekende kracht. 
    \begin{image}
        \includegraphics[width=0.5\textwidth]{auto_bocht_horizontaal}
    \end{image}
    De auto duwt met zijn wielen dwars ten opzichte van de snelheid tegen de grond en de grond duwt terug. In de figuur beweegt de auto het vlak van de tekening in en neemt de auto een bocht naar links (in de richting van de wrijvingskracht).
\end{oplossing}
\end{exercise}
