
\begin{exercise}

% !TEX root = ../main.tex



\pt{5}Een blok van $\SI{10}{kg}$ glijdt vanop een hoogte van $\SI{2,0}{m}$ en vanuit rust langs een gladde helling naar beneden. Onderaan botst het blok tegen een veer met veerconstante $\SI{100}{N/m}$ en drukt de veer in.
\begin{figure}[h]
\centering
\includegraphics[width=0.6\textwidth]{dyn/exercises/FV_5p188}
\end{figure}
\begin{enumerate}
\item Hoe groot is die indrukking?
\item Hoe groot is die indrukking als er op het (horizontale) stuk onder de veer wel wrijving is en de wrijvingsco\"effici\"ent $\mu$ tussen het blok en de ondergrond gelijk is aan 0,30?
\end{enumerate}

\begin{oplossing}
2+3 punten

\begin{enumerate}
	\item $x=\SI{1,98}{m}$
	\item $x=\SI{1,71}{m}$
\end{enumerate}
$\displaystyle\frac{- g m \mu + \sqrt{g m \left(g m \mu^{2} + 2 h k\right)}}{k}$
\end{oplossing}

\end{exercise}
