\documentclass{ximera}

%\addPrintStyle{..}

\begin{document}
	\author{Bart Lambregs}
	\xmtitle{Vraagstukken Wetten van Newton toegepast}{}
    \xmsource\xmuitleg

% Lukt nog niet met relatieve verwijzing. Zou handiger zijn ... 
% \begin{exercise}

 Een hockeyschijf of puck krijgt op een bevroren vijver een horizontale slag waardoor hij met een snelheid van \SI{20,5}{m/s} vertrekt. Na hoeveel meter komt de puck tot rust als de wrijvingsfactor tussen de puck en het ijs \SI{0,18}{} bedraagt?

\begin{oplossing}
    De wrijvingskracht is de enige kracht volgens de horizontale richting en vormt dus de resulterende kracht. We kiezen de referentieas horizontaal en met de beginsnelheid mee.
    \begin{image}
        \includegraphics[width=0.7\textwidth]{hockeypuck}
    \end{image}
    Omdat de wrijvingskracht tegengesteld is aan de $x$-as, geldt $F_{w,x}=-F_w$. Met $F_w=\mu F_n$ vinden we voor de versnelling:
    \begin{eqnarray}
    -F_w &=& ma\nonumber\\
    &\Downarrow& \nonumber\\
    -\mu mg &=&ma\nonumber\\
    &\Updownarrow& \nonumber\\
    a &=& -\mu g\label{a}
    \end{eqnarray}
    De tijd uit de vergelijkingen voor een EVRB elimineren, waarbij de eindsnelheid $v$ en de beginpositie $x_0$ nul zijn, levert:
    \begin{eqnarray}
    v&=&v_0+at\nonumber\\
    x&=&x_0+v_0t+\frac{1}{2}at^2\nonumber\\
    &\Downarrow&\nonumber\\
    0&=&v_0+at\Leftrightarrow t=-\frac{v_0}{a}\nonumber\\
    x&=&v_0\left(-\frac{v_0}{a}\right)+\frac{1}{2}a\left(-\frac{v_0}{a}\right)^2\nonumber\\
    &\Downarrow&\nonumber\\
    a&=&-\frac{v_0^2}{2x}\label{-v_0^2/2x}
    \end{eqnarray}
    Vergelijking (\ref{-v_0^2/2x}) samen met (\ref{a}) levert dan:
    \begin{eqnarray}
    \mu &=& \frac{v_0^2}{2gx}\label{remvgl}\\
    &&\nonumber\\
    &=& 0,178\nonumber
    \end{eqnarray}
\end{oplossing}

\end{exercise}

% 
\begin{exercise}
 Een doos van \SI{12}{kg} wordt losgelaten op een helling van \SI{20}{\degree} en glijdt met een versnelling van \SI{0,30}{m/s^2} hellingafwaarts. 
\begin{enumerate}
	\item\pt{1}Teken alle krachten die op de doos aangrijpen.
	\item\pt{4}Hoe groot is de wrijvingskracht die de doos afremt?
  	\item\pt{1}Hoe groot is de glijdende wrijvingscoëfficiënt?
\end{enumerate}
\begin{oplossing}
	$F_w=m(g\sin{\alpha}-a)=\SI{36,7}{N}$ 
	\newline
	$\mu=\frac{g\sin{\alpha}-a}{g\cos{\alpha}}=\SI{0,33}{}$
\end{oplossing}
\end{exercise}

% % !TEX root = ../main.tex



\item Drie voorwerpen zijn aan elkaar verbonden met touwtjes. De wrij\-vings\-factor tussen de tafel en het erop liggend voorwerp $V$ is $\mu$.
\begin{figure}[h]
\begin{center}
\includegraphics[width=0.6\textwidth ,angle=0]{drie_voorwerpen}
\end{center}
\end{figure}
Veronderstel dat er geen wrijving is in de katrollen en dat de massa van de katrollen te verwaarlozen is. Bepaal de versnelling van het voorwerp. (Bron: 16de VFO 2004)
\begin{oplossing}
\newline
Om de versnelling van het voorwerp te vinden kunnen we de tweede wet van Newton toepassen op de drie massa's. We tekenen de krachtendiagrammen op elk van de massa's (zie figuur). 
\begin{figure}[h]
\begin{center}
\includegraphics[width=0.73\textwidth ,angle=0]{drie_voorwerpen_krachten}
\end{center}
\end{figure}
\newline
Voor $m_1$ vinden we, met de keuze van de $x$-as verticaal naar beneden:
\begin{eqnarray}
F_{z,1}-F_{s,21}=m_1a\label{m_1}
\end{eqnarray}
Voor $m_3$ vinden we, met nu de keuze van de $x$-as verticaal naar boven:
\begin{eqnarray}
F_{s,23}-F_{z,3}=m_3a\label{m_3}
\end{eqnarray}
Voor $m_2$ vinden we, met de keuze van de $x$-as horizontaal naar links:
\begin{eqnarray}
F_{s,12}-F_w-F_{s,32}=m_2a\label{m_2}
\end{eqnarray}
Volgens de derde wet van Newton kunnen we de overeenkomstige spankrachten aan mekaar gelijk stellen: $F_{s,21}=F_{s,12}$ en $F_{s,32}=F_{s,23}$. Samen met $F_w=\mu F_n$ en $F_z=ma$ hebben we drie vergelijkingen en drie onbekenden. Oplossen naar de versnelling levert:
\begin{eqnarray*}
a=\frac{m_1-m_2-\mu m_3}{m_1+m_2+m_3}g=1,2\rm\,m/s^2
\end{eqnarray*}
\newline
\newline
Realiseer je dat de keuze van de $x$-as bij het bepalen van de componenten van de krachten de tekens bepalen van die componenten -- en ook die van de versnelling. Als je bijvoorbeeld tweemaal $a$ schrijft (in vergelijking (\ref{m_1}) en (\ref{m_3})) moet het ook effectief over dezelfde versnelling gaan, en niet over versnellingen die elkaars tegengestelde zijn. Met de $x$-as verticaal naar beneden geori\"enteerd voor $m_1$, zal de versnelling voor $m_1$ positief zijn als de massa naar beneden versnelt (wat hij doet; $m_1>m_3$). Voor $m_3$ moet je dan de $x$-as verticaal omhoog kiezen, wil je dat $a$ evenzeer positief is of dus dezelfde betekenis heeft.
\newline
\newline
Zie ook dat uit vergelijking (\ref{m_1}) volgt dat $m_1$ niet met de zwaartekracht aan $m_2$ trekt! De massa $m_1$ versnelt, waarvoor een resulterende kracht nodig is. 
\end{oplossing}
% \begin{exercise}
Een blok van \SI{4,0}{kg} heeft een beginsnelheid van \SI{8,0}{m/s} aan de voet van een helling van \SI{30,0}{\degree}. De wrijvingskracht die de beweging afremt is \SI{15}{N} groot.
\begin{enumerate}
    \item Teken en benoem de krachten die op het blok aangrijpen.
    \item Welke afstand zal het blok afleggen eer het tot rust komt?
    \item Zal het daarna terug naar beneden glijden?
    \item Hoe groot is de wrijvingsfactor?
\end{enumerate}
\begin{oplossing}
    De krachtendiagrammen/het systeem vrijgemaakt:
    \begin{image}
        \includegraphics[width=0.49\textwidth]{blok_helling_2}
        \includegraphics[width=0.49\textwidth]{blok_helling_2componenten}
    \end{image}

    $a=-\frac{F_w+mg\sin\varphi}{m}=\SI{-8,66}{m/s^2}$
    \newline
    $x=-\frac{v_0^2}{2a}=\frac{mv_0^2}{2(F_w+mg\sin\varphi)}=\SI{3,70}{m}$
    \newline
    $F_{zx}>F_w$ zodat het blok terug naar beneden komt. 
    \newline
    $\mu=\frac{F_w}{mg\cos\varphi}=0,44$
\end{oplossing}
\end{exercise}

\begin{exercise}
     Waarom kan de wrijvingswet bij glijdende wrijving zeker niet geschreven worden als $\vec{F}_w=\mu\vec{F}_n$?
\begin{oplossing}
    De formule geldt niet vectorieel. Ze geeft enkel een relatie tussen de groottes van de krachten, $F_w=\mu F_n$ (zonder pijltjes dus). De normaalkracht staat (per definitie) loodrecht op het ondersteunend oppervlak, de wrijvingskracht is (per definitie) evenwijdig met het ondersteunend oppervlak. De richtingen zijn dus niet gelijk.
\end{oplossing}
\end{exercise}

\begin{exercise}

 Een hockeyschijf of puck krijgt op een bevroren vijver een horizontale slag waardoor hij met een snelheid van \SI{20,5}{m/s} vertrekt. Na hoeveel meter komt de puck tot rust als de wrijvingsfactor tussen de puck en het ijs \SI{0,18}{} bedraagt?

\begin{oplossing}
    De wrijvingskracht is de enige kracht volgens de horizontale richting en vormt dus de resulterende kracht. We kiezen de referentieas horizontaal en met de beginsnelheid mee.
    \begin{image}
        \includegraphics[width=0.7\textwidth]{hockeypuck}
    \end{image}
    Omdat de wrijvingskracht tegengesteld is aan de $x$-as, geldt $F_{w,x}=-F_w$. Met $F_w=\mu F_n$ vinden we voor de versnelling:
    \begin{eqnarray}
    -F_w &=& ma\nonumber\\
    &\Downarrow& \nonumber\\
    -\mu mg &=&ma\nonumber\\
    &\Updownarrow& \nonumber\\
    a &=& -\mu g\label{a}
    \end{eqnarray}
    De tijd uit de vergelijkingen voor een EVRB elimineren, waarbij de eindsnelheid $v$ en de beginpositie $x_0$ nul zijn, levert:
    \begin{eqnarray}
    v&=&v_0+at\nonumber\\
    x&=&x_0+v_0t+\frac{1}{2}at^2\nonumber\\
    &\Downarrow&\nonumber\\
    0&=&v_0+at\Leftrightarrow t=-\frac{v_0}{a}\nonumber\\
    x&=&v_0\left(-\frac{v_0}{a}\right)+\frac{1}{2}a\left(-\frac{v_0}{a}\right)^2\nonumber\\
    &\Downarrow&\nonumber\\
    a&=&-\frac{v_0^2}{2x}\label{-v_0^2/2x}
    \end{eqnarray}
    Vergelijking (\ref{-v_0^2/2x}) samen met (\ref{a}) levert dan:
    \begin{eqnarray}
    \mu &=& \frac{v_0^2}{2gx}\label{remvgl}\\
    &&\nonumber\\
    &=& 0,178\nonumber
    \end{eqnarray}
\end{oplossing}

\end{exercise}


\begin{exercise}
 Een doos van \SI{12}{kg} wordt losgelaten op een helling van \SI{20}{\degree} en glijdt met een versnelling van \SI{0,30}{m/s^2} hellingafwaarts. 
\begin{enumerate}
	\item\pt{1}Teken alle krachten die op de doos aangrijpen.
	\item\pt{4}Hoe groot is de wrijvingskracht die de doos afremt?
  	\item\pt{1}Hoe groot is de glijdende wrijvingscoëfficiënt?
\end{enumerate}
\begin{oplossing}
	$F_w=m(g\sin{\alpha}-a)=\SI{36,7}{N}$ 
	\newline
	$\mu=\frac{g\sin{\alpha}-a}{g\cos{\alpha}}=\SI{0,33}{}$
\end{oplossing}
\end{exercise}


\begin{exercise}
 Drie voorwerpen zijn aan elkaar verbonden met touwtjes. De wrijvingsfactor tussen de tafel en het erop liggend voorwerp $V$ is $\mu$.
\begin{image}
    \includegraphics[width=0.6\textwidth]{drie_voorwerpen}
\end{image}
Veronderstel dat er geen wrijving is in de katrollen en dat de massa van de katrollen te verwaarlozen is. Bepaal de versnelling van het voorwerp. (Bron: 16de VFO 2004)

\begin{oplossing}
Om de versnelling van het voorwerp te vinden kunnen we de tweede wet van Newton toepassen op de drie massa's. We tekenen de krachtendiagrammen op elk van de massa's (zie figuur). 
\begin{image}
    \includegraphics[width=0.6\textwidth]{drie_voorwerpen_krachten}
\end{image}
Voor $m_1$ vinden we, met de keuze van de $x$-as verticaal naar beneden:
\begin{eqnarray}
F_{z,1}-F_{s,21}=m_1a\label{m_1}
\end{eqnarray}
Voor $m_3$ vinden we, met nu de keuze van de $x$-as verticaal naar boven:
\begin{eqnarray}
F_{s,23}-F_{z,3}=m_3a\label{m_3}
\end{eqnarray}
Voor $m_2$ vinden we, met de keuze van de $x$-as horizontaal naar links:
\begin{eqnarray}
F_{s,12}-F_w-F_{s,32}=m_2a\label{m_2}
\end{eqnarray}
Volgens de derde wet van Newton kunnen we de overeenkomstige spankrachten aan mekaar gelijk stellen: $F_{s,21}=F_{s,12}$ en $F_{s,32}=F_{s,23}$. Samen met $F_w=\mu F_n$ en $F_z=ma$ hebben we drie vergelijkingen en drie onbekenden. Oplossen naar de versnelling levert:
\begin{eqnarray*}
a=\frac{m_1-m_2-\mu m_3}{m_1+m_2+m_3}g=1,2\rm\,m/s^2
\end{eqnarray*}

Realiseer je dat de keuze van de $x$-as bij het bepalen van de componenten van de krachten de tekens bepalen van die componenten -- en ook die van de versnelling. Als je bijvoorbeeld tweemaal $a$ schrijft (in vergelijking (\ref{m_1}) en (\ref{m_3})) moet het ook effectief over dezelfde versnelling gaan, en niet over versnellingen die elkaars tegengestelde zijn. Met de $x$-as verticaal naar beneden georiënteerd voor $m_1$, zal de versnelling voor $m_1$ positief zijn als de massa naar beneden versnelt (wat hij doet; $m_1>m_3$). Voor $m_3$ moet je dan de $x$-as verticaal omhoog kiezen, wil je dat $a$ evenzeer positief is of dus dezelfde betekenis heeft.

Merk ook op dat uit vergelijking (\ref{m_1}) volgt dat $m_1$ niet met de zwaartekracht aan $m_2$ trekt! De massa $m_1$ versnelt, waarvoor een resulterende kracht nodig is. 
\end{oplossing}

\end{exercise}

\begin{exercise}
Een blok van \SI{4,0}{kg} heeft een beginsnelheid van \SI{8,0}{m/s} aan de voet van een helling van \SI{30,0}{\degree}. De wrijvingskracht die de beweging afremt is \SI{15}{N} groot.
\begin{enumerate}
    \item Teken en benoem de krachten die op het blok aangrijpen.
    \item Welke afstand zal het blok afleggen eer het tot rust komt?
    \item Zal het daarna terug naar beneden glijden?
    \item Hoe groot is de wrijvingsfactor?
\end{enumerate}
\begin{oplossing}
    De krachtendiagrammen/het systeem vrijgemaakt:
    \begin{image}
        \includegraphics[width=0.49\textwidth]{blok_helling_2}
        \includegraphics[width=0.49\textwidth]{blok_helling_2componenten}
    \end{image}

    $a=-\frac{F_w+mg\sin\varphi}{m}=\SI{-8,66}{m/s^2}$
    \newline
    $x=-\frac{v_0^2}{2a}=\frac{mv_0^2}{2(F_w+mg\sin\varphi)}=\SI{3,70}{m}$
    \newline
    $F_{zx}>F_w$ zodat het blok terug naar beneden komt. 
    \newline
    $\mu=\frac{F_w}{mg\cos\varphi}=0,44$
\end{oplossing}
\end{exercise}


\end{document}