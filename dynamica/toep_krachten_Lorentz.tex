\documentclass{ximera}

%\addPrintStyle{..}


\begin{document}
	\author{Bart Lambregs}
	\xmtitle{De magnetische kracht of Lorentzkracht (veldkracht)}{}
    \xmsource\xmuitleg


= kracht op bewegende lading in een magnetisch veld (dat veroorzaakt wordt door andere bewegende ladingen)

De \textbf{grootte} van de Lorentzkracht \({\overrightarrow{F}}_{L}\) (of magnetische kracht \({\overrightarrow{F}}_{m}\)) op een lading \(q\) met snelheid \(\overrightarrow{v}\) die een hoek \(\alpha\) maakt met de magnetische veldsterkte \(\overrightarrow{B}\) waar de lading door vliegt is te berekenen met:

\[{\mathbf{F}_{\mathbf{L}}\mathbf{= F}}_{\mathbf{m}}\mathbf{= B \bullet}\left| \mathbf{q} \right|\mathbf{\bullet v \bullet}\mathbf{\sin}\mathbf{\alpha}\ met\ \alpha = hoek\ tussen\ \overrightarrow{B}\ en\ \overrightarrow{v}\]

Als vele ladingen samen een gemeenschappelijke driftsnelheid hebben door
een magneetveld hebben (vaak door een geleidende draad met lengte \(\mathcal{l}\)) beschouwt men dit als een stroom \(I\) in een magneetveld \(\overrightarrow{B}\). 
Dan wordt de formule:

\[{\mathbf{F}_{\mathbf{L}}\mathbf{= F}}_{\mathbf{m}}\mathbf{= B \bullet I}\mathcal{\bullet l \bullet}\mathbf{\sin}\mathbf{\alpha}\ met\ \alpha = hoek\ tussen\ \overrightarrow{B}\ en\ I\ \ (ook\ Laplacekracht\ genoemd)\]

\begin{image}
  % \includegraphics{toep_rechterhand}
  \includegraphics{rechterhandregel}
\end{image}

De richting en zin van \({\overrightarrow{F}}_{L}\) kan bepaald worden met de derde rechterhandregel waarin geredeneerd wordt met de conventionele stroomzin van de ladingen.

\emph{\textbf{Uitbreiding:} De Coulombkracht en de Lorentzkracht kunnen samengevat worden tot één elektromagnetische kracht, want beide gevallen gaan over kracht op een lading.}

\begin{quote}
\[{{\overrightarrow{\mathbf{F}}}_{\mathbf{em}}\mathbf{=}{\overrightarrow{\mathbf{F}}}_{\mathbf{C}}\mathbf{+}{\overrightarrow{\mathbf{F}}}_{\mathbf{L}}\mathbf{= q}\overrightarrow{\mathbf{E}}\mathbf{+ q}\overrightarrow{\mathbf{v}}\mathbf{\times}\overrightarrow{\mathbf{B}}\mathbf{= q}\left( \overrightarrow{\mathbf{E}}\mathbf{+}\overrightarrow{\mathbf{v}}\mathbf{\times}\overrightarrow{\mathbf{B}} \right)\mathbf{\ \ \ \ 
}}
\](\(\times is\ het\ symbool\ voor\ vectorieel\ product,\ zie\ externe\ cursussen\))
\end{quote}


\end{document}
