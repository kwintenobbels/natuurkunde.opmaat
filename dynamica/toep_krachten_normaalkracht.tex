\documentclass{ximera}

%\addPrintStyle{..}


\begin{document}
	\author{Bart Lambregs}
	\xmtitle{De normaalkracht (contactkracht)}{}
    \xmsource\xmuitleg


De normaalkracht is een kracht van een (meestal vlakke, maar soms ook gebogen) ondergrond op een voorwerp, meestal met de bedoeling het voorwerp te ondersteunen. 
De normaalkracht \({\overrightarrow{F}}_{n}\) staat altijd loodrecht op het grondvlak (net zoals de normale versnelling loodrecht staat op de snelheidsvector; ``normaal'' betekent in deze context nu eenmaal ``loodrecht''). 
Er is geen formule om de grootte van de normaalkracht uit te rekenen. 
De grootte ervan dient altijd bepaald te worden uit de toepassing van en het redeneren met de wetten van Newton. 
In sommige eenvoudige situaties kan het zijn dat \(F_{n} = F_{z}\), \textbf{maar dit is zeker niet altijd het geval! Enkel met behulp van de wetten van Newton is er uitsluitsel!}

\end{document}
