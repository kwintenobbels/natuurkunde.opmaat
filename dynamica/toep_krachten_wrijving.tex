\documentclass{ximera}

%\addPrintStyle{..}


\begin{document}
	\author{Bart Lambregs}
	\xmtitle{De wrijvingskracht (contactkracht)}{}
    \xmsource\xmuitleg


= weerstandskracht \(\vec{F}_{w}\) op een voorwerp van een
medium of middenstof waarin of waartegen het zich bevindt

% \includegraphics[width=2.08889in,height=1.81111in,alt={E:\textbackslash School updated tot en met 27\_08\_2012\textbackslash Fysica 6de jaar\textbackslash Afbeeldingen\textbackslash wrijving.gif}]{media/image8.gif}\textbf{Schuifwrijvingskracht}
= weerstandskracht van een vlakke ondergrond op een voorwerp dat erop
steunt, evenwijdig met het vlak (en vaak tegen de zin van de beweging)

Verklaring schuifwrijving: een vlakke ondergrond is nooit perfect vlak!
Zie figuur.

Voor de schuifwrijvingskracht \({\overrightarrow{F}}_{w}\) wordt vastgesteld dat\ldots{}

\begin{enumerate}
\item\ldots indien het voorwerp in rust is: \({\overrightarrow{F}}_{w}\) heeft die grootte en zin zodat \(F_{r} = 0\)
\item\ldots indien het voorwerp een rechtlijnige beweging maakt:
\end{enumerate}

De zin van \(\vec{F}_{w}\) is tegen de zin van de beweging en de grootte van \(\vec{F}_{w}\) is recht evenredig is met de grootte van de normaalkracht en tegelijk afhankelijk van hoe goed of slecht de materialen van het voorwerp en vlakke ondergrond over mekaar schuiven.

Er geldt dan: \(F_{w} = \mu \cdot F_{n}\) met \(\mu\) de schuifwrijvingsfactor of schuifwrijvingscoëfficiënt (op te zoeken in tabellen)

Bovenstaande omkaderde formule is dus de maximale waarde van de schuifwrijvingskracht, waardoor de formule beter te schrijven is als: \(F_{w,max} = \mu \cdot F_{n}\). 
Men kan overigens aantonen dat deze laatste formule ook klopt bij bewegingen die niet rechtlijnig zijn (zie verder voor voorbeelden). 
\textit{(zie ook animatie Hans Bekaert: ``wrijving'')}

\begin{enumerate}
\def\labelenumi{\alph{enumi})}
\setcounter{enumi}{2}
\item
  \ldots indien het voorwerp een ECB maakt: \(\vec{F}_{w}\)
  heeft die grootte zodat
  \(F_{r} = m \cdot a = m \cdot \frac{v^{2}}{r} = m \cdot \omega^{2} \cdot r\)
\end{enumerate}

Bij het maken van een bocht (bijvoorbeeld ECB) is er in sommige gevallen schuifwrijvingskracht die bijdraagt tot de middelpuntzoekende kracht. 
Bij bijna alle vervoerswijzen over een vlakke weg doet dit zich voor (zoals met de wagen, met de fiets en zelfs te voet!). 
Er dient wel rekening gehouden te worden met het feit dat de schuifwrijvingskracht in grootte beperkt is tot \(F_{w,max} = \mu \bullet F_{n}\) waardoor er een gevaar bestaat dat de beoogde bocht niet gemaakt kan worden! 
In dit geval spreekt men van slippen en ``uit de bocht vliegen''. 
Zie oefeningen voor concrete gevallen.


%%TODO%% \includegraphics[width=3.08403in,height=2.31111in,alt={E:\textbackslash School updated tot en met 27\_08\_2012\textbackslash Fysica 6de jaar\textbackslash Afbeeldingen\textbackslash wrijving, grafiek.jpg}]{media/image9.jpeg}

\begin{remark}

Soms wordt er een onderscheid gemaakt tussen een statische en kinetische wrijvingsfactor omdat in werkelijkheid de maximale schuifwrijvingskracht in rust een beetje groter is dan wanneer het voorwerp in beweging is.
Dit maakt nevenstaande grafiek duidelijk. 
Niettemin zijn de verschillen meestal klein, waardoor we dit verschil voortaan zullen verwaarlozen.
\end{remark}

\begin{remark}
Bovenvermelde formules zijn enkel van toepassing voor schuifwrijving! \textbf{Rolwrijvingskracht en fluïdumwrijvingskracht} (zoals luchtwrijving) zijn snelheidsafhankelijke weerstandskrachten waardoor er hiervoor andere formules gelden. 
\end{remark}
  


\end{document}
